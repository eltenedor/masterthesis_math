% Folgende Pakete sind Minimalvoraussetzung.

% Anpassung von LaTeX an die deutsche Sprache. Zum Beispiel wird
% \chapter{} direkt als "`Kapitel"' uebersetzt usw. Ausserdem
% wird die Silbentrennung nach deutscher Rechtschreibung vorgenommen.
% Sonderzeichen, z.B. Umlaute, koennen direkt eingegeben werden.

%\usepackage[ngerman]{babel}
\usepackage[ngerman,english]{babel}

% Kodierung fuer unixoide Systeme und Windowssysteme.

\usepackage[utf8]{inputenc}

% Schriftart Times New Roman.

%\usepackage{times}
%\usepackage{lmodern}
%\usepackage{mathptmx}
%\usepackage{stix}
\usepackage{newtxtext}
\usepackage[vvarbb]{newtxmath}
\let\openbox\relax

% Erweitert den Zeichenvorrat, so dass z.B. auch Umlaute im PDF-Dokument
% gefunden werden.

\usepackage[T1]{fontenc}

% Zur Einbindung von Bildern.

\usepackage{graphicx}

% Erweiterte enumerate-Umgebung.

\usepackage{enumerate}

% Verschiedene Pakete, die nuetzlich sind um Mathematik in LaTeX zu setzen.

\usepackage{mathtools}
\usepackage{amsmath, amssymb, amsthm, dsfont}
\usepackage{relsize}
\usepackage{hyperref}
%\usepackage[a-1b]{pdfx}

%%%%%%%%%%%%%%%%
% Seitenlayout %
%%%%%%%%%%%%%%%%

% DIV# gibt den Divisor f�r die Layoutberechnung an.
% Vergr��ern des Divisors vergr��ert den Textbereich.
% BCOR#cm gibt die Breite des Bundstegs an.
\usepackage[DIV14,BCOR2cm]{typearea}

% Abstand obere Blattkante zur Kopfzeile ist 2.54cm - 15mm
%\setlength{\topmargin}{-15mm}

% Keine Einrueckung nach einem Absatz.

%\parindent 0pt

% Abstand zwischen zwei Abs\"atzen.

%\parskip 12pt

% Zeilenabstand.
\usepackage{bibgerm}

\linespread{1.3}

% Inhaltsverzeichnis erstellen.

\usepackage{makeidx}
\makeindex

\usepackage[inline]{showlabels}
\allowdisplaybreaks
