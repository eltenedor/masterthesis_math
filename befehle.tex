% Makros

\newcommand{\N}{\mathbb{N}} % natuerliche Zahlen
\newcommand{\Z}{\mathbb{Z}} % ganze Zahlen
\newcommand{\Q}{\mathbb{Q}} % rationale Zahlen
\newcommand{\R}{\mathbb{R}} % reelle Zahlen
\newcommand{\C}{\mathbb{C}}
\DeclareMathOperator{\Ret}{Re} % Guter Realteil

%%Mengen
\newcommand{\BB}{\mathrm{B}} 
\newcommand{\Dom}{\mathrm{D}} 
\newcommand{\EE}{\mathrm{E}} 
\newcommand{\verCone}{\Gamma_\mathrm{V}} 
\newcommand{\ShenCone}{\Gamma} 

%%Funktionenr�ume
\newcommand{\CC}{\mathrm{C}} 
\newcommand{\Ell}{\mathrm{L}} 
\newcommand{\Li}{\mathfrak{L}}
\newcommand{\WW}{\mathrm{W}}
\newcommand{\HH}{\mathrm{H}}

%%Mathematische Operatoren

\newcommand{\diam}{\operatorname{diam}}
\newcommand{\K}{\mathcal{K}}
\newcommand{\slp}{\mathcal{S}}
\newcommand{\MM}{\mathcal{M}}
\newcommand{\supp}{\operatorname{supp}}
\newcommand{\dlp}{\mathcal{D}}
\DeclareMathOperator{\pv}{p. v.}
\renewcommand{\Im}{\operatorname{Im}}
\renewcommand{\Re}{\operatorname{Re}}

%Differentialoperatoren
\let\divsymb=\div % rename builtin command \div to \divsymb
\renewcommand{\div}[1]{\mathrm{div\,} #1} % for divergence
\renewcommand{\d}[1]{\ensuremath\, {\operatorname{d}\!{#1}}}
\newcommand{\DD}{\ensuremath\,{\operatorname{D}}}

\newcommand{\ii}{\mathrm{i}}
\newcommand{\e}{\mathrm{e}}

% Umgebungen f�r Definitionen, S�tze, usw.

\theoremstyle{definition}
\newtheorem{defn}{Definition}[chapter]
\newtheorem{Beispiel}{Beispiel}[chapter]
\newtheorem*{RBsp}{Regelbeispiel}

\theoremstyle{plain}
\newtheorem{thm}{Theorem}[chapter]
\newtheorem{lem}[thm]{Lemma}
\newtheorem{cor}[thm]{Corollary}
\newtheorem{prop}[thm]{Proposition}

\theoremstyle{remark}
\newtheorem{rem}[thm]{Remark}

\def\Xint#1{\mathchoice
  {\XXint\displaystyle\textstyle{#1}}%
  {\XXint\textstyle\scriptstyle{#1}}%
  {\XXint\scriptstyle\scriptscriptstyle{#1}}%
  {\XXint\scriptscriptstyle\scriptscriptstyle{#1}}%
\!\int}
\def\XXint#1#2#3{{\setbox0=\hbox{$#1{#2#3}{\int}$ }
\vcenter{\hbox{$#2#3$ }}\kern-.6\wd0}}
\def\ddashint{\Xint=}
\def\dashint{\Xint-}
