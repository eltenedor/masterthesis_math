\chapter*{Introduction}\index{Introduction}
\addcontentsline{toc}{chapter}{Introduction}\index{Introduction}

Hier wird die Einleitung stehen. Dabei sollten Sie einen Kurz�berblick des Inhalts Ihrer Arbeit geben. Es ist sinnvoll, eingangs die bearbeitete Fragestellung zu erl�utern, welches Ziel Sie und Ihr Betreuer hatten und welche Ergebnisse Sie schlie�lich erzielt haben.

Es ist hilfreich, auf die Struktur Ihrer Arbeit einzugehen, zum Beispiel indem Sie die einzelnen Kapitel kurz zusammenfassen. Stellen Sie auch klar heraus, welche Resultate schon aus der Literatur bekannt sind und welche Ergebnisse eigene Beitr�ge darstellen.

\section*{Danksagung}\index{Danksagung}

Sie sind dazu verpflichtet (vgl. auch die Erkl�rung auf Seite 27), alle verwendeten Quellen und Hilfsmittel zu nennen. Viele dieser Quellen werden �blicherweise im Literaturverzeichnis aufgelistet, vgl. Seite 23. Zu den Quellen Ihrer Arbeit geh�rt zumindest auch der Betreuer, weil er nicht nur das Thema vergibt (in der Regel basierend auf seiner Idee), sondern Sie auch inhaltlich bei der Bearbeitung unterst�tzt. Die Danksagung stellt den nat�rlichen Rahmen daf�r dar, den Betreuer namentlich zu erw�hnen.

Au\ss erdem kann es auch sein, dass ein wissenschaftlicher Mitarbeiter Sie (mit)be\-treut. In diesem Sinne ist dies auch eine Quelle oder ein Hilfsmittel f\"ur Ihre Arbeit. Daher sollten alle an der Betreuung beteiligten Personen erw\"ahnt werden.

Es steht Ihnen frei, noch weitere Personen zu nennen:
\begin{itemize}
 \item Kommilitonen, mit denen Sie \"uber das Thema diskutieren konnten;
 \item Freunde, welche die Arbeit Korrektur gelesen haben.
\end{itemize}
