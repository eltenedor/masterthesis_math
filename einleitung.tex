\phantomsection
\chapter*{Introduction}
\addcontentsline{toc}{chapter}{Introduction}

In the solution theory for nonlinear partial differential equations, an integral part of the solution process is often to develop a semigroup theory for the linearization of the equation.
In the case of the famous \emph{Navier-Stokes equations} which for a given domain $\Omega \subseteq \R^d$, $d \geq 2$, describe the behavior of a Newtonian fluid over time, the linearization is given by the \emph{Stokes equation}s
\begin{align*}
  \partial_t u - \Delta u + \nabla \pi &= 0 \quad\text{in } \Omega, t > 0,\\
  \div(u) &= 0 \quad\text{in } \Omega, t > 0, \\
  u(0) &= a \quad\text{in } \Omega, \\
  u &= 0 \quad\text{on } \partial\Omega, t > 0,
\end{align*}
where $u \colon \R^+ \times \Omega \to \R^d$ stands for the velocity field and $\pi \colon \R^+ \times \Omega \to \R$ represents the pressure of the fluid.
The so called \emph{Stokes semigroup} $(e^{tA})_{t \geq 0}$ describes the evolution of the velocity $u$ and the \emph{Stokes operator} $A$ corresponds to the term ``$-\Delta u + \nabla \pi$'' in the Stokes equations. 

Having a semigroup makes it possible to look for \emph{mild solutions} to the Navier-Stokes equations using a variation of constants formula to construct an iteration method.
This approach was introduced by Fujita and Kato \cite{fujitaKato,katoFujita} and builds mainly on resolvent estimates for the Stokes operator $A$ and the analyticity property of the Stokes semigroup.
Fujita and Kato applied their methods to find solutions in $\Ell^2$ spaces, given that the initial values $a$ were to be found in domains of fractional powers of the Stokes operator.
Fractional power domains are in general difficult to describe which is why Fujita and Kato suggested that an $\Ell^p$ theory for the Navier-Stokes equations could overcome this problem, see \cite[p.\@~313]{fujitaKato}.
%The interest in an $\Ell^p$ theory of the Stokes equations became emminent as they were looking to find spaces with pertinence conditions that are easier to verify.
Resolvent estimates in $\Ell^p$ for the Stokes operator and the analyticity of the Stokes semigroup are crucial ingredients of this classical functional analytic approach to the solution theory of the Navier-Stokes equations.

On smooth domains, the first milestone in the direction of an $\Ell^p$ theory was laid by Giga \cite{giga} who proved that on bounded smooth domains the Stokes operator generates an analytic semigroup in $\Ell^p(\Omega)$ for $1 < p < \infty$.
This result was then used by Giga and Miyakawa \cite{gigaMiyakawa} to prove the existence of mild solutions given $\Ell^3$ initial data.
Kato \cite{katoExtend} then extended this theory to the whole space and later Giga \cite{gigaAdjust} adjusted the approach for bounded smooth domains.

A natural assumption in real world applications is that the boundary of the domain $\Omega$ is not smooth but merely Lipschitz continuous. 
On Lipschitz domains, the analysis of partial differential equations is more complicated as classical localization techniques fail due to the lack of smoothness of the boundary.
%In order to extend the advances of the $\Ell^p$ theory on smooth domains to Lipschitz domains, the first existence and unique results were given by Mitrea and Monniaux.
In order to overcome these problems, new techniques had to be developed which gave rise to the work of Fabes, Kenig and Verchota \cite{fabesKenigVerchota} who constructed solutions to the $\Ell^p$ Dirichlet problem of the Stokes equation by the method of layer potentials.

As on smooth domains, the further expansion of the $\Ell^p$ theory by means of the classical iteration method was reliant on the analyticity of the Stokes semigroup.
Due to the boundedness properties of the Helmholtz projection, Taylor conjectured in his paper \cite{taylor} from 1999 that the analyticity of the Stokes semigroup would only hold in a neighborhood of $p = 2$, namely for $p \in  (3/2 - \varepsilon, 3 + \varepsilon)$ in the three dimensional case.
His conjecture was supported by the fact that, not much later, Deuring showed in \cite{deuring} the existence of a three dimensional Lipschitz domain such that the needed resolvent estimates would fail for sufficiently large $p$.

Twelve years passed until a positive result could be given to Taylor's conjecture:  Shen showed in his seminal paper \cite{Shen2012} that in Lipschitz domains $\Omega \subseteq \R^d$ , $d \geq 3$, for all $p \in (2d/( d + 1) - \varepsilon , 2d/(d - 1) + \varepsilon)$ the Stokes operator generates an analytic semigroup on $\Ell^p_\sigma(\Omega)$.
Shen's result on the analyticity of the semigroup thus made the Fujita-Kato approach available for the study of the Navier-Stokes equations on Lipschitz domains which, together with other methods, was used by Tolksdorf to prove the existence of mild solutions in $\Ell^3$ to the Navier-Stokes equations \cite{tolksdorf}.

In the present thesis, we will study Shen's approach to the analyticity problem of the Stokes semigroup in $\Ell^p$ for bounded Lipschitz domains in $\R^d$.
While Shen's result was only formulated for $d \geq 3$, we will show that his approach can be extended to the two dimensional case by proving suitable estimates on the fundamental solutions of the Stokes resolvent problem.

Except for the first chapter, in which we will gather the fundamentals and formulate the problem under consideration, the rest of this thesis will closely follow the structure of Shen's work \cite{Shen2012}. 

In Chapter \ref{chap:2}, we will derive the central estimates of the fundamental solutions to the Stokes resolvent problem on $\R^d$ by taking advantage of their explicit representation formula.

Chapter \ref{chap:3} will introduce the method of boundary layer potentials for the solution of the Stokes resolvent problem on bounded Lipschitz domains.
We will discover the central relation between boundary values that are attained nontangentially and the singular integral operators that provide a representation formula for solutions to the Stokes resolvent problem. 

In Chapter \ref{chap:4}, we take a more general look on solutions to the Stokes resolvent problem and derive Rellich-type estimates for these solutions.

Based on the results of Chapter 3 and 4, in Chapter \ref{chap:5}, we will study the $\Ell^2$ Dirichlet problem of the Stokes resolvent system. In this chapter, we will see that for given boundary values in $\Ell^2(\partial\Omega; \C^d)$, there exists a unique solution to the $\Ell^2$ Dirichlet problem which is given via a double layer potential.

Finally, in Chapter \ref{chap:6}, we will tackle the problem of analyticity of the Stokes operator. In this chapter, we will derive the necessary resolvent estimates on $\Ell^p$ by making use of Shen's extrapolation theorem which can be considered a refined version of the Calder\'on-Zygmund Lemma.

%This last paragraph is reserved to mention some of the persons that were involved in the course of this thesis.
%I thank Prof. Dr. Robert Haller-Dintelmann for taking the responsibility of supervising this thesis.
%Furthermore, I want to express my gratitude to Dr. Patrick Tolksdorf. 
%This thesis would not have been possible without his help in innumerable occasions.
%I also want to thank Sebastian Bechtel for proofreading parts of this thesis. 
%Last but not least, I want to thank my family for supporting me during my studies.


