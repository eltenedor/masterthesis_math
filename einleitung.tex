\phantomsection
\chapter*{Introduction}
\addcontentsline{toc}{chapter}{Introduction}

In the solution theory for nonlinear partial differential equations, an integral part of the solution process is often to understand the linearization of the equation.
In the case of the famous Navier-Stokes equations which for a given domain $\Omega \subset \R^d$ describe the behavior of a Newtonian fluid over time, the linearization is given by the Stokes equations
\begin{align*}
  \partial_t u - \Delta u + \nabla \pi &= 0 \quad\text{in } \Omega, t > 0\\
  \div(u) &= 0 \quad\text{in } \Omega, t > 0 \\
  u(0) &= a \quad\text{in } \Omega \\
  u &= 0 \quad\text{on } \partial\Omega, t > 0,
\end{align*}
where $u \colon \R^+ \times \Omega \to \R^d$ stands for the velocity field and $\pi \colon \R^+ \times \Omega \to \R$ represents the pressure of the fluid.
If the evolution of the velocity $u$ can be described by means of a semigroup, the so called Stokes semigroup, this makes available a universal interation procedure which in the setting of the Navier-Stokes equations was first applied by Kato and Fujita in REF.
The idea consists in transforming the Navier-Stokes equations into an integral equation and the applicability of this procedure builds on the analyticity of the semigroup generated by the Stokes operator which realizes the expression $\Delta u + \nabla \pi$ in the Stokes equations.

Fujita and Kato applied their methods to find solutions in $\Ell^2$ spaces, given that the initial values $a$ were to be found in domains of fractional powers of the Stokes operator.
The interest in an $\Ell^p$ theory of the Stokes equations became emminent as they were looking to find spaces with pertinence conditions that are easier to verify.

On smooth domains, the first milestone in the direction of an $\Ell^p$ theory was laid by Giga who proved that the Stokes operator generatres an analytic semigroup in $\Ell^p(\Omega)$ for $1 < p < \infty$.
This result was then used by Miyakawa to give an $\Ell^p$ theory of the Stokes equations.

A natural assumption in real world applications is that the boundary of the domain $\Omega$ is not smooth but merely Lipschitz continuous. 
On Lipschitz domains the analysis of partial differential equations is more complicated as classical localization techniques fail due to the lack of smoothness of the boundary.
%In order to extend the advances of the $\Ell^p$ theory on smooth domains to Lipschitz domains, the first existence and unique results were given by Mitrea and Monniaux.
In order to overcome these problems, new techniques had to be depeloped which gave rise to the work of Fabes, Kenig and Verchota \cite{fabesKenigVerchota} which constructed solutions to the $\Ell^p$ Dirichlet problems by the method of layer potentials.
As on smooth domains, the further expansion of the $\Ell^p$ theory by means of the classical iteration method was reliant on the analyticity of the Stokes semigroup.

In 19 , Deuring showed that this would bdd more complicated as he proved the existence of a tree dimensional Lipschitz domain such that the needed resolvent estimates would fail for sufficiently large $p$.

In 1999, Taylor conjectured that the analyticity of the Stokes semigroup would only hold in a neighborhood of $p = 2$ and it took 12 years until a positive result could be given to his's conjecture:  Shen showed in his seminal paper \cite{Shen2012} that for all $2d/( d + 1) - \varepsilon < p < 2d/(d - 1) + \varepsilon$ the Stokes operator generates an analytic semigroup on $\Ell^p_\sigma(\Omega)$.
Shen's result on the analyticity of the semigroup thus made the Fujita-Kato approach available in the study of the Navier-Stokes equations on Lipschitz domains. 
That this approach would bear fruits was shown by Tolksdorf in \cite{paper}, where he proved the existence of mild solutions in $\Ell^3$ to the Navier-Stokes equations.

In the present thesis, we will study Shen's approach to the analyticity problem of the Stokes semigroup on $\Ell^p_\sigma(\Omega)$ for Lipschitz domains in $\R^d$.
While Shen's result was only formulated for $d \geq 3$, we will show that his approach can be extended to the two dimensinal case by prooving suitable estimates on the fundamental solutions of the Stokes resolvent problem.

Except for the first chapter, in which we will gather the fundamentals, the rest of this thesis will closely follow the structure of Shen's work. 

In the second chapter we will derive the central estimates of the fundamental solutions to the Stokes resolvent problem on $\R^d$ by taking advantage of their explicit representation formula.

The third chapter will introduce the method of boundary layer potentials for the solution of the Stokes resolvent problem on bounded Lipschitz domains.
We will discover the central relation between boundary values that are attained nontangentially and the singular integral operators that provide a representation formula for solutions to the stokes resolvent problem. 

In Chapter 4, we take a more general look on solutions to the stokes resolvent problem and derive Rellich-type estimates on these solutions.

Based on the results of Chapter 3 and 4, in Chapter 5 we will study the $\Ell^2$ Dirichlet problem of the Stokes resolvent system. In this chapter, we will see that for given boundary values in $\Ell^2(\partial\Omega)$, there exists a unique solution to the $\Ell^2$ dirichlet problem which is given via a double layer potential.

Finally, in Chapter 6 we will tackle the problem of analyticity of the Stokes operator. In this chapter, we will derive the necessary resolvent estimates on $\Ell^p$ by using a refined version of the Calder\'on-Zygmund lemma, Shen's extrapolation theorem, in order to bring the known results from the $\Ell^2$ theory into the game.

%This last paragraph is reserved to mention the persons who were involved in the course of this thesis.
%I thank Professor Haller-Dintelmann for taking the responsibility of supervising this thesis.
%Furthermore, I want to express my gratitude to Dr. Patrick Tolksdorf. 
%This thesis would not have been possible without his help in innumerable occasion.
%I also want to thank Sebastian Bechtel for proofreading parts of this thesis. 
%Last but not least, I want to thank my family for supporting me during my studies.


