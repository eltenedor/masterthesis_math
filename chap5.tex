\chapter{Solving the $\Ell^2$-Dirichlet Problem}

This section is all about the application of the method of layer potentials to solve the $\Ell^2$ Dirichlet problem for the Stokes resolvent system.
Furthermore we will establish a uniform $\Ell^p$ estimate for the nontangential-maximal-function which will be important for the proof of our central theorem.

For the remainder of this chapter let $\Omega$ always denote a bounded Lipschitz domain in $\R^d$, $d \geq 3$ with connected boundary.
We will use $\Ell^2_n(\partial\Omega)$ to denote the function space
\begin{align*}
  \Ell^2_n(\partial\Omega) \coloneqq \Big\{ f \in \Ell^2(\partial\Omega; \C^d) \colon \int_{\partial\Omega} f \cdot n \d \sigma = 0 \Big\},
\end{align*}
and $\Ell_0^2(\partial\Omega; \C^d)$  to denote the function space of $\Ell^2$ functions with mean value zero.
As before $\| \cdot \|_\partial$ stands for the norm of $\Ell^2(\partial\Omega)$.

We will first derive bounds on the inverse operator of $(1/2) I + \K_\lambda$ from Chapter 3.

\begin{lem}
  Let $\lambda \in \Sigma_\theta$ and $|\lambda| \geq \tau$, where $\tau \in (0,1)$.
  Suppose that $|\partial\Omega| = 1$.
  Then $(1/2)I + \K_\lambda$ is an isomorphism on $\Ell^2(\partial\Omega; \C^d)$ and
  \begin{align}
    \label{eq:inverseEstimate}
    \| f\|_\partial \leq C \|\big( (1/2) I + \K_\lambda \big) f \|_\partial \quad\text{for any } f \in \Ell^2(\partial\Omega; \C^d),
  \end{align}
  where $C$ depends only on $d$, $\theta$, $\tau$ and the Lipschitz character of $\Omega$.
\end{lem}

\begin{proof}
  We start with $f \in \Ell^2(\partial\Omega; \C^d)$ and the corresponding single layer potentials $u = \slp_\lambda(f)$ and $\phi = \slp_\Phi(f)$ given by \eqref{eq:defSingleLayer} and \eqref{eq:defSingleLayerPressure}.
  We saw in Chapter 3 that $(u,\phi)$ solves the Stokes resolvent problem in $\R^d \setminus \partial\Omega$ and got from Lemma \ref{lem:nontangentialMaximalFunctions} with $p = 2$ for the nontangential maximal functions that $(\nabla u)^*$, $(\phi)^* \in \Ell^2(\partial\Omega)$.
  We furthermore saw in Lemma \ref{lem:traceFormulas} that $\nabla u$ and $\phi$ have nontangential limits almost everywhere on $\partial\Omega$. 
  Finally in Theorem \ref{thm:jumpConditions} we saw that $\nabla_{\mathrm{tan}} u_+ = \nabla_{\mathrm{tan}} u_-$ and derived the jump condition $\big(\frac{\partial u}{\partial \nu}\big)_\pm = (\pm (1/2) I + \K_\lambda) f$.

  Our next step will be to show the estimate
  \begin{align}
    \label{eq:negNablaPhi}
    \| \nabla u_- \|_\partial + \| \phi_- \|_\partial \leq C \| \big( \frac{\partial u}{\partial \nu} \big) \|_\partial .
  \end{align}
  Assuming that \eqref{eq:negNablaPhi} holds we can prove \eqref{eq:inverseEstimate}:
  Set $f = \big(\frac{\partial u}{\partial \nu} \big)_+ -  \big( \frac{\partial u}{\partial \nu} \big)_-$.
  Then this gives with the definition of the conormal derivative and estimate \eqref{eq:negNablaPhi} that
  \begin{align*}
    \| f\|_\partial 
    &\leq  \| \big( \frac{\partial u}{\partial \nu} \big)_+ \|_\partial + \| \big( \frac{\partial u}{\partial \nu} \big)_- \|_\partial  \\
    &\leq \| \big( \frac{\partial u}{\partial \nu} \big)_+  \|_\partial + \| \big( \frac{\partial u}{\partial n} \big)_- \|_\partial + \| \phi_- n \|_\partial  \\
    &\leq C \| \big( \frac{\partial u}{\partial \nu} \big)_+ \|_\partial
    = C \| (1/2)I + K_\lambda) f\|_\partial.
  \end{align*}

  In order to prove \eqref{eq:negNablaPhi}, note that since $|u(x)| + |\nabla u(x)| = O(|x|^{-N})$ for all $N > 0$ and $\phi(x) = O(|x|^{1 - d})$ as $|x| \to \infty$ we can use Theorem \ref{thm:rellichExterior} to derive
  \begin{align}
    \|\nabla u_-\|_\partial + \|\phi_-\|_\partial
    &\leq C \Big\{ \|\nabla_{\mathrm{tan}} u_- \|_\partial + |\lambda|^{1/2} \| u_-\|_\partial + |\lambda| \| n \cdot u_-\|_{\HH^{-1}(\partial\Omega)} \Big\} \nonumber\\
    \label{eq:nablauminus}
    &= C \Big\{ \|\nabla_{\mathrm{tan}} u_+ \|_\partial + |\lambda|^{1/2} \| u_+ \|_\partial + |\lambda| \|n \cdot u_+ \|_{\HH^{-1}(\partial\Omega)} \Big\},
  \end{align}
  where we used the fact that $u_+ = u_-$ and $\nabla_{\mathrm{tan}} u_+ = \nabla_{\mathrm{tan}} u_-$ on $\partial\Omega$.
  Inequality \eqref{eq:rellich2} of Theorem \ref{thm:rellich} now allows us to estimate the right hand side of \eqref{nablauminus} by $C \|(\frac{\partial u}{\partial \nu} )_+ \|_\partial$ and thus the desired estimate \eqref{eq:neqNablaPhi} follows.

  Let's now work on the invertibility of $(1/2) I + \K_\lambda$. In the case $\lambda = 0$, Fabes, Kenig and Verchota showed in \cite{fabesKenigVerchota} that $(1/2) I + \K_0$ as an operator on $\Ell^2(\partial\Omega; \R^d)$ has a one dimensional null space and as range the space $\Ell_0^2(\partial\Omega; \R^d)$.
  Thus $(1/2) I + \K_0$ has Fredholm index $0$.
  This remains true if we replace $\Ell^2(\partial\Omega; \R^d)$ by $\Ell^2(\partial\Omega; \C^d)$ as this just corresponds to a complexification of the vector space and the operator. 
  Since the operator $\K_\lambda - \K_0$ is compact on $\Ell^2(\partial\Omega; \C^d)$ (see Toksdorf \cite{tolksdorfDiss}) we deduce that the operator
  \begin{align*}
    (1/2)I + \K_\lambda = (1/2)I + \K_0 + (\K_\lambda - \K_0)
  \end{align*}
  has the Fredholm index zero as well for all $\lambda \in \Sigma_\theta$.
  Now inequality \eqref{eq:inverseEstimate} gives that $(1/2)I + \K_\lambda$ is injective and thus the Fredholm index of zero implies that it is also surjective and hence an isomorphism.
\end{proof}

The next lemma works with the counterpart of $(1/2)I + \K_\lambda$.

\begin{lem}
  \label{lem:inverseEstimate}
  Let $\lambda \in \Sigma_\theta$.
  Then $-(1/2)I + \K_\lambda$ is a Fredholm operator on $\Ell^2(\partial\Omega; \C^d)$ with index zero and
  \begin{align}
    \label{eq:inverseEstimate2}
    \| f\|_\partial \leq C \| \big( - (1/2) I + \K_\lambda \big) f \|_\partial \quad\text{for all } f \in \Ell^2_n(\partial\Omega).
  \end{align}
\end{lem}

\begin{proof}
  In the case $\lambda = 0$, Fabes Kenig and Verchota proved in \cite{fabesKenigVerchota} that the Fredholm index of the operator $-(1/2)I + \K_0$ on $\Ell^2(\partial\Omega; \R^d)$ is zero and estimate \eqref{eq:inverseEstimate2} holds.
  As in the previous proof, this still remains true if we complexify the operator making it a Fredholm operator with index zero on  $\Ell^2(\partial\Omega; \C^d)$.
  Since $\K_\lambda - \K_0$ is compact on $\Ell^2(\partial\Omega; \C^d)$ and the Fredholm index remains unchanges under compact perturbations, we know that the Fredholm index of $-(1/2)I + \K_\lambda$ on $\Ell^2(\partial\Omega; \C^d)$ is zero for all $\lambda \in \Sigma_\theta$.
  This proves the first claim of the lemma.

  Now let $\tau < \frac{1}{2 \diam(\Omega)^2 + 1}$ and $|\lambda| < \tau$.
  Then
  \begin{align*}
    \| (\K_\lambda - \K_0 f\|_\partial \leq C |\lambda|^{1/2} \|f\|_\partial.
  \end{align*}
  In order to prove this inequality we once again apply Young's inequality, i.e. we start by estimating
  \begin{align*}
    \| (\K_\lambda - \K_0 f\|_\partial \leq \sup_{p \in \partial\Omega} \| \nabla_x \{ \Gamma(p - \cdot; \lambda) - \Gamma(p - \cdot; 0) \} \|_{\Ell^1(\partial\Omega)} \| f\|_{\Ell^2(\partial\Omega)}.
  \end{align*}
  In the next step we prove that for $p \in \partial\Omega$ the integral over the gradients of $\Gamma$ can be estimated independent of $p$.
  This is straightforward using Lemma \ref{lem:central} as Corollary \ref{cor:differenceFundamentalSolutionStokes} gives us
  \begin{align*}
    &\int_{\partial\Omega} | \nabla_x \{ \Gamma(p - y; \lambda) - \Gamma(p - y; 0) \} | \d\sigma(y)\\
    &\quad\leq C |\lambda|^{1/2} \int_{\partial\Omega} \frac{1}{|p - y|^{d - 2}} \d \sigma(y) \\
    &\quad= |\lambda|^{1/2} C \int_{\partial\Omega \cap \BB(p, r_0/4)} \frac{1}{|p - y|^{d - 2}} \d\sigma(y) + |\lambda|^{1/2} C \int_{\partial\Omega \setminus \BB(p, r_0/4)} \frac{1}{|p - y|^{d - 2}} \d\sigma(y) \\
    &\quad\leq C |\lambda|^{1/2} ( r_0/4 + 4^{2 - d} r_0^{d - 2} |\partial\Omega| ),
  \end{align*}
  where $r_0$ is the radius from the definition of Lipschitz domains. 
  Note that by the choice of $\tau$ the estimate from Corollary \ref{cor:differenceFundamentalSolutionStokes} applies on the whole domain of integration.

  For $f \in \Ell^2_n(\partial\Omega)$ we can now estimate
  \begin{align*}
    \| f\|_\partial 
    &\leq C \|(-(1/2)I + \K_0) f \|_\partial  \\
    &\leq C \|(-(1/2)I + \K_\lambda)f \|_\partial + \|(\K_\lambda - \K_0)f\|_\partial \\
    &\leq C \|(-(1/2)I + \K_\lambda)f\|_\partial + C |\lambda|^{1/2} \|f\|_\partial,
  \end{align*}
  with a constant $C$ depending only on $d$, $\theta$ and the Lipschitz character of $\partial\Omega$.
  Choosing $\tau$ even smaller allows us to rearrange the terms in the above estimate such that estimate \eqref{eq:inverseEstimate2} holds for $\lambda \in \Sigma_\theta$ and $|\lambda| < \tau$, with $\tau$ depending on $d$, $\theta$ and the Lipschitz character of $\Omega$.

  Now leave $\tau$ fixed and consider the case $|\lambda| \geq \tau$.
  This case will be handled using the Rellich estimates from Section 4.
  We use the facts that for $\nabla_{\mathrm{tan}} u$ and $u$ the inner and outer nontangential limits coincide and apply Theorems \ref{thm:rellich} and \ref{thm:rellichExterior} to conclude that
  \begin{align*}
    &\|\nabla u_+ \|_\partial + \| \phi_+ - \dashint_{\partial\Omega} \phi_+ \| \\
    &\quad\leq C \{ \|( \nabla_{\mathrm{tan}} u)_+ \|_\partial + |\lambda|^{1/2} \|u_+\|_\partial + |\lambda| \| u_+\cdot n\|_{\HH^{-1}(\partial\Omega)} \} \\
    &\quad = C \{ \|( \nabla_{\mathrm{tan}} u)_- \|_\partial + |\lambda|^{1/2} \|u_-\|_\partial + |\lambda| \| u_-\cdot n\|_{\HH^{-1}(\partial\Omega)} \} \\
    &\quad\leq C \| \big(\frac{\partial u}{\partial \nu} \big)_- \|_\partial.
  \end{align*}
  We can now use this inequality to estimate $\| \big( \frac{\partial u}{\partial \nu} \big)_+ \|_\partial$ since
  \begin{align*}
    \| \big( \frac{\partial u}{\partial \nu} \big)_+ \|_\partial
    &\leq \|\big(\frac{\partial u}{\partial n} \big)_+ \|_\partial + C \| \phi_+ \|_\partial \\
    &\leq C \| (\nabla u)_+ \|_\partial + C \|\phi_+ - \dashint_{\partial\Omega} \phi_+ \d\sigma \|_\partial + C \big| \dashint_{\partial\Omega} \phi_+ \d \sigma \big| \\
    &\leq C \| \big(\frac{\partial u}{\partial \nu} \big)_- \|_\partial + C \big| \int_{\partial\Omega} \phi_+ \d\sigma \big|
  \end{align*}
  Considering the jump relation \eqref{eq:nontangentialConormalDerivative} and the previous estimate we get that
  \begin{align}
    \|f \|_\partial 
    &\leq \|\big( \frac{\partial u}{\partial \nu} \big)_+ \|_\partial + \| \big( \frac{\partial u}{\partial \nu} \big)_- \|_\partial  \nonumber\\
    &\leq C \| \big( \frac{\partial u}{\partial \nu} \big)_- \|_\partial + C \big| \int_{\partial\Omega} \phi_+ \d\sigma \big| \nonumber\\
    \label{eq:estimatef}
    &\leq C \| (-(1/2)I + \K_\lambda) f \|_\partial + C \big| \int_{\partial\Omega} \phi_+ \d\sigma \big|.
  \end{align}
  We now are left with the term $\int_{\partial\Omega} \phi_+ \d\sigma$ that needs to be estimated.
  To this end, note that multiplying the conormal derivatives of $u$ by $n$ gives
  \begin{align*}
    \big( \frac{\partial u}{\partial \nu} \big)_+ \cdot n
    = \big( \frac{\partial u_i}{\partial x_j}\big)_+ n_i n_j - \phi_+
    = n_j \big( n_i \frac{\partial}{\partial x_j} - n_j \big( \frac{\partial}{\partial x_i} \big) u_i \big)_+ - \phi_+,
  \end{align*}
  where for the second equality we used that $\div(u) = 0$ in $\Omega$ and thus this also holds for the nontangential limit.
  This identity now implies
  \begin{align}
    \big| \int_{\partial\Omega} \phi_+ \d\sigma \big|
    &\leq \big| \int_{\partial\Omega} \big( \frac{\partial u}{\partial \nu} \big)_+ \cdot n \d\sigma \big|  + C \|\nabla_{\mathrm{tan}} u \|_{\partial} \nonumber\\
    &\leq \big| \int_{\partial\Omega} \big( \frac{\partial u}{\partial \nu} \big)_- \cdot n \d\sigma \big| + C \| \nabla_{\mathrm{tan}} u\|_\partial \nonumber\\
    \label{eq:estimatephiplus}
    &\leq C \| \big( \frac{\partial u}{\partial \nu} \big)_- \|_\partial,
  \end{align}
  where in the second step, we used the jump relation to exchange $\big(\frac{\partial u}{\partial \nu}\big)_+ \cdot n$ by $\big(\frac{\partial u}{\partial \nu} \big)_- + f \cdot n$ and then used the fact $f \in \Ell^2_n(\partial\Omega)$.
  The third step follows from Theorem \ref{thm:rellichExterior} considering that $\|\nabla_{\mathrm{tan}} u \|_\partial \leq C \|\nabla u \|_{\partial}$.
  Now extending estimate \eqref{eq:estimatef} by \eqref{eq:estimatephiplus} gives
  \begin{align*}
    \| f\|_\partial 
    \leq C \| \big( - (1/2) I + \K_\lambda) f \|_\partial + C \| \big(\frac{\partial u}{\partial \nu} \big)_- \|_\partial 
    \leq C \| \big( - (1/2) I + \K_\lambda) f\|_{\partial},
  \end{align*}
  where we used the jump relation \eqref{eq:nontangentialConormalDerivative} again.
  This proves estimate \eqref{eq:inverseEstimate2} in the case $|\lambda| \geq \tau$ and thus concludes the proof.
\end{proof}

In the following lemma we will show the uniqueness of solutions to the $\Ell^2$ Dirichlet problem to the Stokes resolvent system.

\begin{lem}
  \label{lem:l2unique}
  Let $\lambda \in \Sigma_\theta$ and $(u,\phi)$ be a solution to the Stokes resolvent problem in $\Omega$.
  Furthermore suppose that the nontangential limit of $u$ exists almost everywhere on $\partial\Omega$ and that $(u)^* \in \Ell^2(\partial\Omega)$.
  Then 
  \begin{align}
    \label{eq:OmegaBoundaryEstimate}
    \int_\Omega |u|^2 \d x \leq C \int_{\partial\Omega} |u|^2 \d\sigma,
  \end{align}
  where $C$ depends only on $d$, $\theta$ and the Lipschitz character of $\Omega$.
\end{lem}

\begin{proof}
  We use Verchota's approximation theorem \cite{verchotaDiss} and approximate $\Omega$ by a sequence of smooth domains with uniform Lipschitz characters from inside.
  As a consequence we may assume that $\Omega$ is smooth and $u$, $\phi$ are smooth in $\overline\Omega$.
  %The fact that $(u)^* \in \Ell^2(\partial\Omega)$ is necessary if we want to take the limit of the approximating sequence of domains, as it enables us to apply the dominated convergence theorem.
  Let $(w,\psi) \in \HH_0^1(\Omega; \C^d) \times \HH^1(\Omega)$ be a solution to the inhomogenous system
  \begin{align}
    \label{eq:inhomogenousStokes}
    \begin{cases}
      - \Delta w + \lambda w + \nabla \psi = \bar u &\text{ in } \Omega, \\
      \div(w) = 0 &\text{ in } \Omega.
    \end{cases}
  \end{align}
  In fact the regularity theory for the Stokes equation gives us that $w$ and $\psi$ are smooth.
  It follows from testing \eqref{eq:inhomogenousStokes} against $u$ that
  \begin{align}
    \label{eq:testingInhomogenousStokes}
    \int_\Omega |u|^2 \d x
    &= \int_\Omega u \cdot \{ - \Delta w + \lambda w + \nabla \psi \} \d x.
    %&= - \int_{\partial\Omega} u \cdot \big\{ \frac{\partial w}{\partial n} - \psi n \big\} \d \sigma
  \end{align}
  Using one of Green's identities on the first summand and the fact that $u$ is the solution to the Stokes resolvent problem gives that
  \begin{align*}
    \int_\Omega -u \cdot \Delta w \d x
    &= \int_\Omega -w \cdot \Delta u \d x - \int_{\partial\Omega} u \cdot \frac{\partial w}{\partial n} \d \sigma, \\
    &= \int_\Omega w \cdot ( -\lambda u - \nabla \phi) \d x - \int_{\partial\Omega} u \cdot \frac{\partial w}{\partial n} \d \sigma \\
    &= \int_\Omega -\lambda w \cdot u \d x - \int_{\partial\Omega} u \cdot \frac{\partial w}{\partial n} \d \sigma, 
  \end{align*}
  where in the last step we used partial integration and the fact that $w$ vanishes on $\partial\Omega$ and is divergence free:
  \begin{align*}
    \int_\Omega w \cdot \nabla\phi \d x = -\int_\Omega \div(w) \phi \d x + \int_{\partial\Omega} \phi w \cdot n \d\sigma = 0.
  \end{align*}
  For the third summand in \eqref{eq:testingInhomogenousStokes} we do the same with the only difference that the second integral does not vanish.
  Putting everything together gives
  \begin{align}
    \int_\Omega |u|^2 \d x 
    &= - \int_{\partial\Omega} u \cdot \big\{ \frac{\partial w}{\partial n} - \psi n \big\} \d\sigma \nonumber\\
    &\leq \| u\|_\partial \{ \|\nabla w\|_\partial + \|\psi\|_\partial \} \label{eq:u2estimate}
  \end{align}
  by the Cauchy-Schwartz inequality.
  As the pressure $\psi$ is only specified modulo additive constants, we may as well assume that $\int_{\partial\Omega} \psi \d\sigma = 0$.
  Furthermore by the Schwartz theorem we see from \eqref{eq:inhomogenousStokes} that $\Delta \psi = \div(\bar u) = 0$ in $\Omega$. 
  As stated in Remark \ref{rem:harmonicEstimate}, this allows us to use the results from the proof of \eqref{eq:nablaPhin} by setting $\phi = \psi$ to conclude that
  \begin{align}
    \|\psi\|_\partial 
    &\leq C \| \nabla \psi \cdot n\|_{\HH^{-1}(\partial\Omega)} \nonumber
    \intertext{and since $w$ has vanishing trace on $\partial\Omega$ we can use that fact that $(w,\psi)$ solves \eqref{eq:inhomogenousStokes} to further estimate}
    &\leq C \{ \| \Delta w \cdot n \|_{\HH^{-1}(\partial\Omega)} + \| u \cdot n \|_{\HH^{-1}(\partial\Omega)} \} \nonumber\\
    &\leq C \{ \|  \nabla w\|_\partial + \|u\|_\partial  \}, \label{eq:psiEstimate}
  \end{align}
  where for the last estimate we used \eqref{eq:deltaun} which is applicable since $\div w = 0$ on $\Omega$.
  If we combine inequalities \eqref{eq:u2estimate} and \eqref{eq:psiEstimate}, we get
  \begin{align}
    \int_\Omega |u|^2 \d x \leq C \|u\|_\partial \|\nabla w \|_\partial + C \|u\|_\partial^2. \label{eq:u2estimate2}
  \end{align}
  We are left with estimating the first term in \eqref{eq:u2estimate2}.
  In fact it will suffice to show the following inequality
  \begin{align}
    \label{eq:w2estimate}
    \int_{\partial\Omega} |\nabla w|^2 \d\sigma \leq C \int_\Omega |u|^2 \d x + C \int_{\partial\Omega} |u|^2 \d \sigma
  \end{align}
  since by the weighted Young inequality this would make the estimate
  \begin{align*}
    C \| u\|_\partial \|\nabla w\|_\partial \leq \frac{1}{2} \int_\Omega |u|^2 \d x + C \int_{\partial\Omega} |u|^2 \d \sigma
  \end{align*}
  available which after rearranging terms yields \eqref{eq:OmegaBoundaryEstimate}.

  To this end, we will first prove the Rellich type identity
  \begin{align}
    \int_{\partial\Omega} h_k n_k |\nabla w|^2 \d \sigma
    &= \int_\Omega \div(h) |\nabla w|^2 \d x + 2 \Re \int_{\Omega} h_j \frac{\partial \psi}{\partial x_j} \frac{\bar w_j}{\partial x_k}\d x  \nonumber \\
    &\quad + 2 \Re \int_\Omega h_k \lambda w_j \frac{\partial\bar w_j}{\partial x_k} + 2 \Re \int_\Omega h_k \bar u_j \frac{\partial \bar w_j}{\partial x_k} \d x. \label{eq:rellichIdentity3}
  \end{align} 
  Note that since all involved quantities are smooth up to the boundary, integration by parts is allowed and the proof the stated  Rellich identity boils down to a formal calculation.
  Let $h \in C_0^1(\R^d, \R^d)$ with $h_k n_k \geq c > 0$ on $\partial \Omega$, see Verchota \cite{verchotaDiss}.
  Then the divergence theorem gives
  \begin{align*}
    \int_{\partial\Omega} h_k n_k |\nabla w|^2 \d\sigma
    = \int_{\Omega} \div(h |\nabla w|^2) \d x = \int_\Omega \div(h) |\nabla w|^2 \d x + \int_\Omega h_k \frac{\partial}{\partial x_k} \big( |\nabla w|^2 \big) \d x
  \end{align*}
  and we can rewrite the second summand as
  \begin{align*}
    \int_{\Omega} h_k \frac{\partial}{\partial x_k} (|\nabla w|^2) \d x
    &= \int_{\Omega} h_k \frac{\partial}{\partial x_k} \big( \frac{\partial w_j}{\partial x_i} \frac{\partial \bar w_j}{\partial x_i}\big) \d x \\
    &= 2 \Re \int_\Omega h_k \frac{\partial^2 w_j}{\partial x_k \partial x_i} \frac{\partial \bar w_j}{\partial x_i} \d x \\
    &= - 2 \Re \int_\Omega h_k \frac{\partial}{\partial x_k} \big( \frac{\partial^2}{\partial x_i^2} w_j \big) \bar w_j \d x + \int_{\partial\Omega} h_k \big( \frac{\partial^2}{\partial x_k \partial x_i} w_j \big) \bar w_j \d x\\
    &= 2 \Re \int_\Omega h_k ( \Delta w_j ) \frac{\partial \bar w_j}{\partial x_k} \d x + 0 \\
    &= 2 \Re \int_\Omega h_k \big( \frac{\partial \psi}{\partial x_j} + \lambda w_j - \bar u_j \big) \frac{\partial \bar w_j}{\partial x_k} \d x,
  \end{align*}
  where in addition to partial integration we used \eqref{eq:inhomogenousStokes} and the fact that $w = 0$ on $\partial\Omega$.
  Now we can apply the triangle inequality to the Rellich type identity \eqref{eq:rellichIdentity3} to obtain
  \begin{align}
    \int_{\partial\Omega} |\nabla w|^2 \d \sigma 
    &\leq C \Big\{ \int_\Omega |\nabla w|^2 \d x + \int_\Omega |\nabla w | |\psi| \d x \nonumber\\
    &\qquad + |\lambda| \int_\Omega |\nabla w| |w| \d x + \int_\Omega |\nabla w| |u| \d x \Big\}. \label{eq:rellichInequality3}
  \end{align}

  The next step consists in deriving estimates which are compatible with the right hand side of \eqref{eq:rellichInequality3}.
  Testing \eqref{eq:inhomogenousStokes} with $\bar w$, itegration by parts gives us as in the proof of Lemma \ref{lem:laxMilgramIneq}
  \begin{align*}
    \int_\Omega |\nabla w|^2 \d x + |\lambda| \int_\Omega |w|^2 \d x \leq C \int_\Omega |w| |u| \d x.
  \end{align*}
  The next step consists in using the previous inequality and the Poincar\'{e} inequality to estimate
  \begin{align*}
    \int_\Omega |\nabla w|^2 \d x + (1 + |\lambda| ) \int_\Omega |w|^2 \d x 
    &\leq  (1 + C) \int_\Omega |\nabla w|^2 \d x + |\lambda| \int_\Omega |w|^2 \d x \\
    &\leq C \int_\Omega |w| |u| \d x \\
    &\leq C \| w\|_\partial \|u\|_\partial,
    \intertext{where for the last step we used the Cauchy-Schwartz inequality. The weighted Young inequality allows us to further estimate}
    &\leq \frac{C}{4\varepsilon} \int_\Omega |u|^2 \d x + C \varepsilon \int_\Omega |w|^2 \d x \\
    &= \frac{\tilde C}{1 + |\lambda|} \int_\Omega |u|^2 \d x + \frac{1}{2} ( 1 + |\lambda| ) \int_\Omega |w|^2 \d x
  \end{align*}
  if we set $\varepsilon = \frac{(1 + |\lambda|)}{2 C}$.
  Rearranging terms, we can produce our next estimate
  \begin{align}
    \int_\Omega |\nabla w|^2 \d x + ( 1 + |\lambda| ) \int_\Omega |w|^2 \d x
    &\leq \frac{C}{1 + |\lambda|} \int_\Omega |u|^2 \d x.\label{eq:w2lambdaEstimate}
  \end{align}
  Now it's time to harvest: 
  Using estimate \eqref{eq:rellichInequality3} together with \eqref{eq:w2lambdaEstimate} gives
  \begin{align*}
    &\int_{\partial\Omega} |\nabla w|^2 \d\sigma \\
    &\quad\leq C \Big\{ \int_\Omega |\nabla w|^2 \d x + \int_\Omega |\nabla w| |\psi| \d x + |\lambda| \int_\Omega |\nabla w| |w| \d x + \int_\Omega |\nabla w| |u| \d x \Big\}.
  \end{align*}
  Using the weighted Young inequality, we see that we can simplify the right hand to
  \begin{align*}
    &C_\varepsilon (1 + |\lambda|) \int_{\Omega} |\nabla w|^2 \d x + C |\lambda| \int_\Omega |w|^2 \d x + C \int_\Omega |u|^2 \d x + \varepsilon \int_\Omega |\psi|^2 \d x, \\
    \intertext{which with \eqref{eq:w2lambdaEstimate} can be bounded in this way}
    &\quad \leq  C_\varepsilon \int_\Omega |u|^2 \d x + \varepsilon \int_\Omega |\psi|^2 \d x.
  \end{align*}
  Using the estimate $\|\psi\|_{\Ell^2(\partial\Omega)} \leq C \|\psi\|_\partial$ and inequality \eqref{eq:psiEstimate} gives
  \begin{align*}
    \varepsilon \int_\Omega |\psi|^2 \d x \leq \varepsilon C \int_{\partial\Omega} |\nabla w|^2 \d \sigma + C_\varepsilon \int_{\partial\Omega} |u|^2 \d\sigma.
  \end{align*}
  Choosing $\varepsilon = \frac{1}{2 C}$ and rearranging gives the desired estimate 
  \eqref{eq:w2estimate}.
  This concludes our proof.
\end{proof}

The next Theorem states the important fact that in $\Ell^2$ the Dirichlet Stokes resolvent problem has a unique solution.

\begin{thm}
  \label{thm:exAndUniqueSolution}
  Let $\Omega$ be a bounded Lipschitz domain in $\R^d$, $d \geq 3$ with connected boundary and let $\lambda \in \Sigma_\theta$.
  For all $g \in \Ell^2_n(\partial\Omega)$ there exists a unique $u$ and harmonic function $\phi$ which is unique up to constants such that $(u,\phi)$ satisfies \eqref{eq:stokesResolvent}, $(u)^* \in \Ell^2(\partial\Omega)$ and $u = g$ on $\partial\Omega$ in the sense of nontangential convergence.
  Moreover the estimate $ \| (u)^* \|_\partial \leq C \| g\|_\partial$ holds and $u$ may be represented by the double layer potential $\dlp_\lambda(f)$ with $\|f\|_\partial \leq C \|g\|_\partial$, where $C$ depends only on $d$, $\theta$ and the Lipschitz character of $\Omega$.
\end{thm}

\begin{proof}
  By Lemma \ref{lem:l2unique} we already now that the problem under consideration admits at most one solution.
  Therefore we only have to worry about the existence of a solution.
  To this end we want to apply Lemma \ref{lem:inverseEstimate}.
  We first note that since $T \coloneqq -(1/2) I + \K_{\bar \lambda}$ is a Fredholm operator on $\Ell^2(\partial\Omega; \C^d)$ with index $0$ the same is true for its adjoint $T^* \coloneqq (-1/2)I + \K_{\bar\lambda}^*$.
  We know that for all $f \in \Ell^2(\partial\Omega; \C^d)$ we have $\div(\dlp_\lambda f) = 0$ and therefore
  \begin{align*}
    \int_{\partial\Omega} T^* f \cdot n \d\sigma =\int_{\partial\Omega} u_+ \cdot n \d\sigma =  0, 
  \end{align*}
  where for the first inequality we applied Theorem \ref{thm:nontangentialLimitDoubleLayer}. The second equality uses Verchota's approximation scheme in order to apply the divergence theorem and the fact that $(u)^*$ is integrable together with dominated convergence.
  This gives $\Im(T^*) \subseteq \Ell_n^2(\partial\Omega)$.
  This gives us that we have 
  \begin{align*}
    \operatorname{span}(n)
    = \Ell_n^2(\partial\Omega)^\perp
    \subseteq \Im(T^*)^\perp
    = \ker(T)
  \end{align*}
  on the one hand and on the other hand, as $T$ is injective on $\Ell^2_n(\partial\Omega)$ by \eqref{eq:inverseEstimate2}, we have that $ \operatorname{span}(n) \supseteq \ker(T)$.
  This yields $\operatorname{span}(n) = \ker(T)$.
  We can use this equality and show that
  \begin{align*}
    \Ell^2_n(\partial\Omega) = \ker(T)^\perp = \overline{\Im(T^*)} = \Im(T^*),
  \end{align*}
  where for the last equality we used the fact that the range of $T^*$ is closed, as usual for Fredholm operators.
  With the same argument we can show for $T$ that
  \begin{align*}
    \ker(T^*)^\perp = \overline{\Im(T)} = \Im(T).
  \end{align*}
  Consequently the operator
  \begin{align*}
    T^* \colon \Im(T) \to \Ell^2_n(\partial\Omega)
  \end{align*}
  is invertible by the continuous inverse theorem.
  Considering once again estimate \eqref{eq:inverseEstimate2} and a duality argument we get that
  \begin{align}
    \label{eq:dualityArgument}
    \|f\|_\partial \leq C \| T^* f\|_\partial
  \end{align}
  for all $f \in \Im(-(1/2)I + \K_\lambda)$.

  We are now in position to derive the missing estimates which were stated in the theorem.
  For $g \in \Ell^2_n(\partial\Omega)$ let $f \in \Im(T)$ with $T^* f = g$.
  Furthermore let $(u,\phi)$ be the double layer potential defined in equations \eqref{eq:defDoubleLayer} and \eqref{eq:defDoubleLayerPressure}.
  Then $u_+ = T^* f =  g$ on $\partial\Omega$ by Theorem \ref{thm:nontangentialLimitDoubleLayer}.
  Additionally we have that
  \begin{align*}
    \|(u)^* \|_\partial \leq C \|f\|_\partial \leq C \| g\|_\partial
  \end{align*}
  where we used inequality \eqref{eq:lpBoundednessUNontangentialMax} and \eqref{eq:dualityArgument}.
\end{proof}

The next theorem can in some sense be regarded as a reverse trace theorem and will play an important role for the proof of the needed reverse H\"{o}lder inequality.

\begin{thm}
  Let $\Omega$ be a bounded Lipschitz domain in $\R^d$, $d \geq 3$ with connected boundary.
  Let $u \in \HH^1(\Omega; \C^d)$ and $\pi \in \Ell^2(\Omega)$ satisfy the Stokes resolvent problem in $\Omega$ for some $\lambda \in \Sigma_\theta$.
  Then
  \begin{align}
    \label{eq:reverseTrace}
    \Big( \int_\Omega |u|^p \d x \Big)^{1/p} \leq C \Big( \int_{\partial\Omega} |u|^2 \d\sigma \Big)^{1/2},
  \end{align}
  where $p = \frac{2d}{d - 1}$ and $C$ depends only on $d$, $\theta$ and the Lipschitz character of $\Omega$.
\end{thm}

\begin{proof}
  Let us denote the trace of $u$ on $\partial\Omega$ by $f$ and let furthermore $u = \dlp(f)$ be the solution of the $\Ell^2$ Dirichlet problem given by Theorem \ref{thm:exAndUniqueSolution}.
  For the sequence $(\Omega_j)_{j \in \N}$ of smooth domains that approximates $\Omega$ from inside as described by Verchota \cite{verchotaDiss} an application of Lemma \ref{lem:l2unique} shows
  \begin{align}
    \int_{\Omega_j} |u - w|^2 \d x \leq C \int_{\partial\Omega_j} |u - w|^2 \d \sigma,
  \end{align}
  where $C$ does not depend on $j$ but on the Lipschitz character of $\Omega$.
  Now let $\varepsilon > 0$ be given. 
  Then there exists $\varphi_\varepsilon \in \CC^\infty(\overline\Omega)$ such that $\|\varphi_\varepsilon - u\|_{\HH^1(\Omega)}^2 \leq \frac{\varepsilon}{2}$.
  This gives that
  \begin{align*}
    \int_{\partial\Omega_j} |u - w|^2 \d\sigma 
    &\leq \int_{\partial\Omega_j} |u - \varphi_\varepsilon |^2 \d\sigma + \int_{\partial\Omega_j} |\varphi_\varepsilon - w|^2 \d\sigma \\
    &\leq C \| u - \varphi_\varepsilon \|_{\HH^1(\Omega)}^2 + \int_{\partial\Omega_j} |\varphi_\varepsilon - w|^2 \d \sigma,
  \end{align*}
  where once again $C$ only depends on the Lipschitz character of $\Omega$ and is thus independent of $j$.
\end{proof}


