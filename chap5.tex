\chapter{Solving the $\Ell^2$ Dirichlet Problem}
\label{chap:5}

This section is all about the application of the method of layer potentials to solve the $\Ell^2$ Dirichlet problem \hyperref[eq:dirProblem]{$(\mathrm{Dir}_\lambda)$} for the Stokes resolvent system.
The results from Chapter~\ref{chap:3} on the jump relations and the corresponding operators, see Theorem~\ref{thm:jumpConditions} and Theorem~\ref{thm:nontangentialLimitDoubleLayer}, together with the Rellich estimates from Chapter~\ref{chap:4} will be the essential ingredients.

In the first part of this chapter, we will show that the operators $\pm (1/2)I + \K_\lambda$ and their adjoints are isomorphisms on suitable Hilbert spaces.
Then, in the second part of this chapter, we will build on this information to construct solutions to $(\mathrm{Dir}_\lambda)$ via double layer potentials.
%Furthermore we will establish a uniform $\Ell^p$ estimate for the nontangential maximal function which will be important for the proof of our central theorem.

For the remainder of this chapter, let $\Omega$ always denote a bounded Lipschitz domain in $\R^d$, $d \geq 2$, with connected boundary.
We will use $\Ell^2_n(\partial\Omega)$ to denote the function space
\begin{align*}
  \Ell^2_n(\partial\Omega) \coloneqq \Big\{ f \in \Ell^2(\partial\Omega; \C^d) \colon \int_{\partial\Omega} f \cdot n \d \sigma = 0 \Big\},
\end{align*}
and $\Ell_0^2(\partial\Omega; \C^d)$  to denote the function space of $\Ell^2$ functions with mean value zero.
As before, $\| \cdot \|^{}_\partial$ stands for the norms of the spaces $\Ell^2(\partial\Omega; \C^k), k \in \N$.

The following results will build on the application of Theorem~\ref{thm:rellichExterior}.
Therefore, the next lemma shows that solutions to the Stokes resolvent problem \eqref{eq:stokesResolventProblem} that are given by single layer potentials fulfill the requirements of the theorem.

\begin{lem}
  \label{lem:requirements}
  Let $\lambda \in \Sigma_\theta$ and $(u,\phi)$ be given by \eqref{eq:defSingleLayer} and \eqref{eq:defSingleLayerPressure}, respectively.
  Then the following holds for $|x| \to \infty\,$:
  \begin{align*}
    |\phi(x)| + |\nabla u(x)| = O(|x|^{1 - d}) \quad\text{and}\quad 
    u(x) = \begin{cases} O(|x|^{2 - d}) &\quad\text{if } d \geq 3, \\ o(1) &\quad\text{if } d = 2. \end{cases}
  \end{align*}
\end{lem}

\begin{proof}
  If $d \geq 2$, then an application of the dominated convergence theorem gives that $|\phi(x)| + |\nabla u(x)| = O(|x|^{1 - d})$ as $|\phi(x)| = O(|\Phi(x)|) = O(|x|^{1 - d})$ by \eqref{eq:fundamentalVectorPressure}.
 Furthermore, we have $|\nabla u(x)| = O(|x|^{1 - d})$ by estimate \eqref{eq:fundamentalMatrixEstimate}.

  If $d \geq 3$, then the first part of Lemma~\ref{lem:estimateHelmholtzDerivatives} and an application of the dominated convergence theorem give $u(x) = O(|x|^{2 - d})$.
  If $d = 2$, consider the definition of the fundamental matrix \eqref{eq:fundamentalMatrixStokes}. According to the first part of Lemma~\ref{lem:estimateHelmholtzDerivatives} and the asymptotic behavior of the second derivative of $\log(|x|)$, we only have to worry about the first summand in \eqref{eq:fundamentalMatrixStokes}. 
  But the asymptotic behavior of the fundamental solution to the scalar Helmholtz equation is already available thanks to the second part of Lemma~\ref{lem:estimateHelmholtzDerivatives}. 
  Through dominated convergence, the same asymptotic behavior holds for $u(x)$.
\end{proof}

As announced in the introduction to this chapter, we will study the invertibility of the operators $\pm(1/2) I + \K_\lambda$ from Chapter~\ref{chap:3}, starting with the one corresponding to $+$.
We will furthermore be concerned about bounds on the inverse of the operator $(1/2) I + \K_\lambda$.

\begin{lem}
  \label{lem:inverseEstimate1}
  Let $\lambda \in \Sigma_\theta$ and $|\lambda| \geq \tau$, where $\tau \in (0,1)$.
  Suppose that $|\partial\Omega| = 1$.
  Then $(1/2)I + \K_\lambda$ is an isomorphism on $\Ell^2(\partial\Omega; \C^d)$ and
  \begin{align}
    \label{eq:inverseEstimate}
    \| f\|^{}_\partial \leq C \, \big\|\, \big( (1/2) I + \K_\lambda \big) f\, \, \big\|^{}_\partial \quad\text{for any } f \in \Ell^2(\partial\Omega; \C^d),
  \end{align}
  where $C > 0$ depends only on $d$, $\theta$, $\tau$ and the Lipschitz character of $\Omega$.
\end{lem}

\begin{proof}
  We start with $f \in \Ell^2(\partial\Omega; \C^d)$ and the corresponding single layer potentials $u = \slp_\lambda(f)$ and $\phi = \slp_\Phi(f)$ given by \eqref{eq:defSingleLayer} and \eqref{eq:defSingleLayerPressure}.
  We saw in Chapter 3 that $(u,\phi)$ solves the Stokes resolvent problem in $\R^d \setminus \partial\Omega$ and got from Lemma~\ref{lem:nontangentialMaximalFunctions} with $p = 2$ for the nontangential maximal functions that $(\nabla u)^*$, $(\phi)^* \in \Ell^2(\partial\Omega)$.
  We furthermore saw in Lemma~\ref{lem:traceFormulas} that $\nabla u$ and $\phi$ have nontangential limits almost everywhere on $\partial\Omega$. 
  Finally, in Theorem~\ref{thm:jumpConditions} we saw that $\nabla_{\mathrm{tan}} u_+ = \nabla_{\mathrm{tan}} u_-$ and derived the jump relation $\big(\frac{\partial u}{\partial \nu}\big)_\pm = (\pm (1/2) I + \K_\lambda) f$.

  Let us assume for a moment that the estimate
  \begin{align}
    \label{eq:negNablaPhi}
    \| \nabla u_- \|^{}_\partial + \| \phi_- \|^{}_\partial \leq C \, \Big\| \bigg( \frac{\partial u}{\partial \nu} \bigg)_+ \Big\|^{}_\partial .
  \end{align}
  holds for a constant $C > 0$ depending only on $d$, $\theta$, $\tau$ and the Lipschitz character of $\Omega$. 
  Using \eqref{eq:negNablaPhi}, we can prove \eqref{eq:inverseEstimate}:
  Note that the jump relation \eqref{eq:nontangentialConormalDerivative} gives us $f = \big(\frac{\partial u}{\partial \nu} \big)_+ -  \big( \frac{\partial u}{\partial \nu} \big)_-$.
  With the definition of the conormal derivative and estimate \eqref{eq:negNablaPhi}, we calculate that
  \begin{align*}
    \| f\|^{}_\partial 
    &\leq  \Big\|    \bigg( \frac{\partial u}{\partial \nu} \bigg)_+ \Big\|^{}_\partial + \Big\|  \bigg( \frac{\partial u}{\partial \nu} \bigg)_- \Big\|^{}_\partial  \\[0.5em]
    &\leq \Big\|     \bigg( \frac{\partial u}{\partial \nu} \bigg)_+  \Big\|^{}_\partial + \Big\| \bigg( \frac{\partial u}{\partial n}   \bigg)_- \Big\|^{}_\partial + \| \phi_- \|^{}_\partial  \\[0.5em]
    &\leq C\, \Big\| \bigg( \frac{\partial u}{\partial \nu} \bigg)_+ \Big\|^{}_\partial
    = C\, \big\| \, \big((1/2)I + \K_\lambda\big) f\, \big\|^{}_\partial.
  \end{align*}

  In order to prove \eqref{eq:negNablaPhi}, note that due to Lemma~\ref{lem:requirements} we can use Theorem~\ref{thm:rellichExterior} to derive the following inequality for the outer nontangential limit:
  \begin{align}
    \|\nabla u_-\|^{}_\partial + \|\phi_-\|^{}_\partial
    &\leq C\, \Big\{ \big\|\nabla_{\mathrm{tan}} u_- \big\|^{}_\partial + |\lambda|^{1/2} \| u_-\|^{}_\partial + |\lambda|\, \| n \cdot u_-\|_{\HH^{-1}(\partial\Omega)} \Big\} \nonumber\\[0.5em]
    \label{eq:nablauminus}
    &= C \, \Big\{ \big\|\nabla_{\mathrm{tan}} u_+ \big\|^{}_\partial + |\lambda|^{1/2} \| u_+ \|^{}_\partial + |\lambda|\, \|n \cdot u_+ \|_{\HH^{-1}(\partial\Omega)} \Big\},
  \end{align}
  where we used the fact that $u_+ = u_-$ and $\nabla_{\mathrm{tan}} u_+ = \nabla_{\mathrm{tan}} u_-$ on $\partial\Omega$. The former fact is a consequence of the continuity of the single layer potential across $\partial\Omega$, see Mitrea and Wright \cite[Prop.\@~4.7]{mitreaWright}.
  Inequality \eqref{eq:rellich2} of Theorem~\ref{thm:rellich} now allows us to estimate the right-hand side of \eqref{eq:nablauminus} by $C \,\|\big(\frac{\partial u}{\partial \nu} \big)_+ \|^{}_\partial$ and thus the desired estimate \eqref{eq:negNablaPhi} follows.

  Let us now work on the invertibility of $(1/2) I + \K_\lambda$. In the case $\lambda = 0$, Mitrea and Wright showed in \cite[Eq.\@~(5.166)]{mitreaWright} that $(1/2) I + \K_0$ as an operator on $\Ell^2(\partial\Omega; \C^d)$ has a one dimensional null space and as range the space $\Ell_0^2(\partial\Omega; \C^d)$.
  In other words, $(1/2) I + \K_0$ has Fredholm index $0$ as the orthogonal complement of $\Ell^2_0(\partial\Omega; \C^d)$ is the span of the normal vector $n$ which is one dimensional.
  %This remains true if we replace $\Ell^2(\partial\Omega; \R^d)$ by $\Ell^2(\partial\Omega; \C^d)$ as this just corresponds to a complexification of the vector space and the operator. 
  Since the operator $\K_\lambda - \K_0$ is compact on $\Ell^2(\partial\Omega; \C^d)$ by Lemma~\ref{lem:compactness}, we deduce that for all $\lambda \in \Sigma_\theta$ the operator
  \begin{align*}
    (1/2)I + \K_\lambda = (1/2)I + \K_0 + (\K_\lambda - \K_0)
  \end{align*}
  has the Fredholm index zero as well.
  Now inequality \eqref{eq:inverseEstimate} gives that $(1/2)I + \K_\lambda$ is injective and thus the Fredholm index of zero implies that it is also surjective and hence an isomorphism.
\end{proof}

The next lemma is the counterpart to Lemma~\ref{lem:inverseEstimate1} and proves a similar result for the operator $-(1/2)I + \K_\lambda$ on the slightly smaller space $\Ell^2_n(\partial\Omega)$.

\begin{lem}
  \label{lem:inverseEstimate}
  Let $\lambda \in \Sigma_\theta$.
  Then $-(1/2)I + \K_\lambda$ is a Fredholm operator on $\Ell^2(\partial\Omega; \C^d)$ with index zero and
  \begin{align}
    \label{eq:inverseEstimate2}
    \| f\|^{}_\partial \leq C \, \big\| \, \big( - (1/2) I + \K_\lambda \big) f\, \big\|^{}_\partial \quad\text{for all } f \in \Ell^2_n(\partial\Omega),
  \end{align}
  where $C > 0$ depends only on $d$, $\theta$, the Lipschitz character of $\Omega$ and $\operatorname{diam}(\Omega)$.
\end{lem}

\begin{proof}
  Let us assume without loss of generality that $\sigma( \partial\Omega ) = 1$.
  In the case $\lambda = 0$, Mitrea and Wright showed in \cite[Eq.\@~(5.166)]{mitreaWright} that the Fredholm index of the operator $-(1/2)I + \K_0$ on $\Ell^2(\partial\Omega; \C^d)$ is zero and estimate \eqref{eq:inverseEstimate2} holds.
  %As in the previous proof, this still remains true if we complexify the operator making it a Fredholm operator with index zero on  $\Ell^2(\partial\Omega; \C^d)$.
  Since $\K_\lambda - \K_0$ is compact on $\Ell^2(\partial\Omega; \C^d)$ and the Fredholm index remains unchanged under compact perturbations, we know that the Fredholm index of $-(1/2)I + \K_\lambda$ on $\Ell^2(\partial\Omega; \C^d)$ is zero for all $\lambda \in \Sigma_\theta$.
  This proves the first claim of the lemma.

  Now let $\tau < (2 \diam(\Omega)^2 + 1)^{-1}$ and $|\lambda| < \tau$.
  We claim that
  \begin{align*}
  \big\|\, (\K_\lambda - \K_0) f\, \big\|^{}_\partial \leq C \, |\lambda|^{1/2} \|f\|^{}_\partial.
  \end{align*}
  In order to prove this inequality, we want to apply Young's inequality from Lemma~\ref{lem:young}. To this end, we start by estimating
  \begin{align*}
      \big\|\,  (\K_\lambda - \K_0) f\, \big\|^{}_\partial 
    \leq \sup_{\substack {p \in \partial\Omega \\ i,j = 1,\dots,d}} \Big\|\, \nabla_x \Big\{ \Gamma_{ij}(p - \cdot\,; \lambda) - \Gamma_{ij}(p - \cdot\,; 0) \Big\} \, \Big\|_{\Ell^1(\partial\Omega;\C^d)} \| f\|_\partial^{}.
  \end{align*}
  In the next step, we prove that for $p \in \partial\Omega$ the integral over the gradients of $\Gamma$ can be estimated independently of $p$ and of course $i$ and $j$.
  This is straightforward using Lemma~\ref{lem:central} as Corollary~\ref{cor:differenceFundamentalSolutionStokes} gives us
  \begin{align*}
    &\int_{\partial\Omega} \Big| \nabla_x \big\{ \Gamma_{ij}(p - y; \lambda) - \Gamma_{ij}(p - y; 0) \big\} \Big| \d\sigma(y)\\[0.5em]
      &\quad\leq C \, |\lambda|^{1/2} \int_{\partial\Omega} \frac{1}{|p - y|^{d - 2}} \d \sigma(y) \\[0.5em]
      &\quad= C \, |\lambda|^{1/2} \, \int_{\partial\Omega \cap \BB(p,\, r_0/4)} \frac{1}{|p - y|^{d - 2}} \d\sigma(y) + C |\lambda|^{1/2}\, \int_{\partial\Omega \setminus \BB(p,\, r_0/4)} \frac{1}{|p - y|^{d - 2}} \d\sigma(y) \\[0.5em]
      &\quad\leq C \, |\lambda|^{1/2} \,\big( r_0/4 + 4^{2 - d} r_0^{d - 2} \sigma( \partial\Omega ) \big),
  \end{align*}
  where $r_0$ is the radius from the definition of Lipschitz domains.
  But as $\sigma( \partial\Omega )  = 1$ and $r_0$ can be related to $|\partial\Omega|$ by Lemma~\ref{lem:compareBoundaryWithBall}, we get
  \begin{align*}
    \int_{\partial\Omega} \Big|\nabla_x \big\{\Gamma_{ij}(p - y; \lambda) - \Gamma_{ij}(p - y; 0) \big\} \Big| \d\sigma(y) \leq C \, |\lambda|^{1/2}
  \end{align*}
  with a constant $C > 0$ that only depends on $d$, $\theta$ and the Lipschitz character of $\Omega$.
  Note that by the choice of $\tau$ the estimate from Corollary~\ref{cor:differenceFundamentalSolutionStokes} applies on the whole domain of integration.

  For $f \in \Ell^2_n(\partial\Omega)$, we now estimate
  \begin{align*}
    \| f\|^{}_\partial 
    &\leq C\,  \big\|\,(-(1/2)I + \K_0) f \, \big\|^{}_\partial  \\
    &\leq C\,  \big\|\,(-(1/2)I + \K_\lambda)f\, \big\|^{}_\partial + \big\|\, (\K_\lambda - \K_0)f\, \big\|^{}_\partial \\
    &\leq C\,  \big\|\,(-(1/2)I + \K_\lambda)f\, \big\|^{}_\partial + C \, |\lambda|^{1/2} \|f\|^{}_\partial,
  \end{align*}
  with a constant $C > 0$ which depends only on $d$, $\theta$ and the Lipschitz character of $\partial\Omega$.
  Choosing $\tau$ smaller than $(2C)^{-2}$ allows us to rearrange the terms in the estimate above such that estimate \eqref{eq:inverseEstimate2} holds for $\lambda \in \Sigma_\theta$ and $|\lambda| < \tau$, with $\tau$ depending on $d$, $\theta$, the Lipschitz character of $\Omega$ and $\operatorname{diam}(\Omega)$ by the first choice of $\tau$.

  Now leave $\tau$ fixed and consider the case $|\lambda| \geq \tau$.
  This case will be handled using the Rellich estimates from Section 4.
  We use the facts that for $\nabla_{\mathrm{tan}} u$ and $u$ the inner and outer nontangential limits coincide and apply Theorems~\ref{thm:rellich} and~\ref{thm:rellichExterior} to conclude that
  \begin{align*}
    &\|\nabla u_+ \|^{}_\partial + \Big\| \, \phi_+ - \frac{1}{r_0^{d - 1}}\int_{\partial\Omega} \phi_+ \d\sigma \, \Big\|^{}_\partial \\[0.5em]
      &\qquad\leq C\, \Big\{ \big\|( \nabla_{\mathrm{tan}} u)_+ \big\|_\partial^{} + |\lambda|^{1/2} \|u_+\|^{}_\partial + |\lambda|\,  \| u_+\cdot n\|_{\HH^{-1}(\partial\Omega)} \Big\} \\[0.5em]
      &\qquad = C\, \Big\{ \big\|( \nabla_{\mathrm{tan}} u)_- \big\|_\partial^{} + |\lambda|^{1/2} \|u_-\|^{}_\partial + |\lambda| \, \| u_-\cdot n\|_{\HH^{-1}(\partial\Omega)} \Big\}
      \leq C\, \Big\| \bigg(\frac{\partial u}{\partial \nu} \bigg)_- \Big\|^{}_\partial.
  \end{align*}
  We can now use this inequality to estimate $\| \big( \frac{\partial u}{\partial \nu} \big)_+ \|^{}_\partial$ via
  \begin{align*}
      \Big\| \bigg( \frac{\partial u}{\partial \nu} \bigg)_+ \Big\|^{}_\partial
      &\leq \Big\|\bigg(\frac{\partial u}{\partial n} \bigg)_+ \Big\|^{}_\partial + C\, \| \phi_+ \|^{}_\partial \\[0.5em]
      &\leq C\,  \Big\{  \| (\nabla u)_+ \|^{}_\partial + \Big\|\, \phi_+ - \frac{1}{r_0^{d - 1}} \int_{\partial\Omega} \phi_+ \d\sigma \, \Big\|^{}_\partial + \Big| \, \frac{1}{r_0^{d - 1}} \int_{\partial\Omega} \phi_+ \d \sigma \, \Big| \; \Big\}\\[0.5em]
      &\leq C\,  \Big\{ \, \Big\| \bigg(\frac{\partial u}{\partial \nu} \bigg)_- \Big\|^{}_\partial + \Big| \int_{\partial\Omega} \phi_+ \d\sigma \, \Big|\,\Big\}.
  \end{align*}
  Furthermore, considering the jump relation \eqref{eq:nontangentialConormalDerivative} and the previous estimate, we get that
  \begin{align}
      \|f \|^{}_\partial 
      &\leq \Big\|\Big( \frac{\partial u}{\partial \nu} \Big)_+ \Big\|^{}_\partial + \Big\| \Big( \frac{\partial u}{\partial \nu} \Big)_- \Big\|^{}_\partial  \nonumber\\[0.5em]
      &\leq C \,\Big\{ \,  \Big\| \Big( \frac{\partial u}{\partial \nu} \Big)_- \Big\|^{}_\partial + \Big| \int_{\partial\Omega} \phi_+ \d\sigma \, \Big|\, \Big\} \nonumber\\[0.5em]
    \label{eq:estimatef}
    &\leq C \, \Big\{ \big\| (-(1/2)I + \K_\lambda) f \big\|^{}_\partial +  \Big| \int_{\partial\Omega} \phi_+ \d\sigma \, \Big|\,\Big\}.
  \end{align}
  Now we are left with the term $\int_{\partial\Omega} \phi_+ \d\sigma$ that needs to be estimated.
  To this end, note that multiplying the conormal derivatives of $u$ by $n$ gives
  \begin{align*}
    \bigg( \frac{\partial u}{\partial \nu} \bigg)_+ \cdot n
    = \bigg( \frac{\partial u_i}{\partial x_j}\bigg)_+ n_i n_j - \phi_+
    = n_j \bigg( n_i \frac{\partial u_i}{\partial x_j} - n_j \bigg( \frac{\partial u_i}{\partial x_i} \bigg) \bigg)_+ - \phi_+\, ,
  \end{align*}
  where for the second equality we used that $\div(u) = 0$ in $\Omega$ and thus this also holds for the nontangential limit.
  Note that the expression on the right-hand side involves a first-order tangential derivative operator, see \eqref{eq:defnTangDerivative}, and thus we can also write
  \begin{align*}
    \bigg( \frac{\partial u}{\partial \nu} \bigg)_+ \cdot n 
    = n_j \big( \partial_{\tau_{ij}}u_i \big)_+ - \phi_+.
  \end{align*}
  Using the above identity and relation \eqref{eq:relGradTan} to bring the tangential gradient into the game, it follows that
  \begin{align}
    \Big| \int_{\partial\Omega} \phi_+ \d\sigma \, \Big|
    &\leq \Big| \int_{\partial\Omega} \bigg( \frac{\partial u}{\partial \nu} \bigg)_+ \cdot n \d\sigma \, \Big|  + C\, \big\|(\nabla_{\mathrm{tan}} u)_+ \big\|^{}_\partial \nonumber\\[0.5em]
    &\leq \Big| \int_{\partial\Omega} \bigg( \frac{\partial u}{\partial \nu} \bigg)_- \cdot n \d\sigma \, \Big| +  C\, \big\| (\nabla_{\mathrm{tan}} u)_-\big\|^{}_\partial \nonumber\\[0.5em]
    \label{eq:estimatephiplus}
    &\leq C\, \Big\| \bigg( \frac{\partial u}{\partial \nu} \bigg)_- \Big\|^{}_\partial,
  \end{align}
  where in the second step, we used the jump relation to exchange $\big(\frac{\partial u}{\partial \nu}\big)_+ \cdot n$ by $\big(\frac{\partial u}{\partial \nu} \big)_- + f \cdot n$ and then used the fact $f \in \Ell^2_n(\partial\Omega)$.
  The third step follows from Theorem~\ref{thm:rellichExterior} considering that $\big\|(\nabla_{\mathrm{tan}} u)_- \big\|^{}_\partial \leq C\, \|(\nabla u)_- \|^{}_{\partial}$ with a constant that only depends on $d$.
  Now, extending estimate \eqref{eq:estimatef} by \eqref{eq:estimatephiplus} gives
  \begin{align*}
      \| f\|^{}_\partial 
      \leq C \, \big\| \, \big( - (1/2) I + \K_\lambda\big) f\, \big\|^{}_\partial + C\, \Big\| \bigg(\frac{\partial u}{\partial \nu} \bigg)_- \Big\|^{}_\partial 
    \leq C \, \big\| \, \big( - (1/2) I + \K_\lambda\big) f \,\big\|^{}_{\partial},
  \end{align*}
  where we used the jump relation \eqref{eq:nontangentialConormalDerivative} again.
  This proves estimate \eqref{eq:inverseEstimate2} in the case $|\lambda| \geq \tau$ and thus concludes the proof.
\end{proof}

With the following lemma that looks like a reverse trace theorem, we will later show the uniqueness of solutions to the $\Ell^2$ Dirichlet problem \hyperref[eq:dirProblem]{$(\mathrm{Dir}_\lambda)$} for the Stokes resolvent system.

\begin{lem}
  \label{lem:l2unique}
  Let $\lambda \in \Sigma_\theta$ and $(u,\phi)$ be a solution to the Stokes resolvent problem \eqref{eq:stokesResolventProblem} in $\Omega$.
  Furthermore, suppose that the nontangential limit of $u$ exists almost everywhere on $\partial\Omega$ and that $(u)^* \in \Ell^2(\partial\Omega)$.
  Then,
  \begin{align}
    \label{eq:OmegaBoundaryEstimate}
    \int_\Omega |u|^2 \d x \leq C \int_{\partial\Omega} |u|^2 \d\sigma,
  \end{align}
  where $C > 0$ depends only on $d$, $\theta$ and the Lipschitz character of $\Omega$.
\end{lem}

\begin{proof}
  We use the approximation theorem, Theorem~\ref{thm:smoothApproximation}, and approximate $\Omega$ by a sequence of smooth domains with uniform Lipschitz characters from inside.
  It suffices to prove \eqref{eq:OmegaBoundaryEstimate} for elements of this sequence of domains as $(u)^* \in \Ell^2(\partial\Omega)$.
  As a consequence, we will assume for the rest of the proof that $\Omega$ is smooth and that $u$, $\phi$ are smooth in $\overline\Omega$.
  %The fact that $(u)^* \in \Ell^2(\partial\Omega)$ is necessary if we want to take the limit of the approximating sequence of domains, as it enables us to apply the dominated convergence theorem.
  Let $(w,\psi) \in \HH_0^1(\Omega; \C^d) \times \HH^1(\Omega)$ be a solution to the inhomogeneous system
  \begin{align}
    \label{eq:inhomogenousStokes}
    \begin{alignedat}{1}
      - \Delta w + \lambda w + \nabla \psi = \overline u &\quad\text{ in } \Omega\,, \\
      \div(w) = 0 &\quad\text{ in } \Omega\,.
    \end{alignedat}
  \end{align}
  In fact, the regularity theory for the Stokes equation gives us that $w$ and $\psi$ are even smooth in $\Omega$ as $\overline u$ is smooth.
  It follows from testing \eqref{eq:inhomogenousStokes} against $u$ that
  \begin{align}
    \label{eq:testingInhomogenousStokes}
    \int_\Omega |u|^2 \d x
    &= \int_\Omega u \cdot \big\{ - \Delta w + \lambda w + \nabla \psi \big\} \d x.
    %&= - \int_{\partial\Omega} u \cdot \big\{ \frac{\partial w}{\partial n} - \psi n \big\} \d \sigma
  \end{align}
  The left-hand side of \eqref{eq:testingInhomogenousStokes} gives the starting point for the proof of inequality \eqref{eq:OmegaBoundaryEstimate}.
  
  Using one of Green's identities, see \cite[Thm.\@~3, App.\@~C.2]{evans}, on the first summand and the fact that $u$ is the solution to the Stokes resolvent problem gives that
  \begin{align*}
    \int_\Omega -u \cdot \Delta w \d x
    &= \int_\Omega -w \cdot \Delta u \d x - \int_{\partial\Omega} u \cdot \frac{\partial w}{\partial n} \d \sigma, \\[0.5em]
    &= \int_\Omega w \cdot ( -\lambda u - \nabla \phi) \d x - \int_{\partial\Omega} u \cdot \frac{\partial w}{\partial n} \d \sigma \\[0.5em]
    &= \int_\Omega -\lambda w \cdot u \d x - \int_{\partial\Omega} u \cdot \frac{\partial w}{\partial n} \d \sigma, 
  \end{align*}
  where in the last step we used integration by parts and the fact that $w$ vanishes on $\partial\Omega$ and is divergence free:
  \begin{align*}
    \int_\Omega w \cdot \nabla\phi \d x = -\int_\Omega \div(w) \, \phi \d x + \int_{\partial\Omega} \phi \, w \cdot n \d\sigma = 0.
  \end{align*}
  For the third summand in \eqref{eq:testingInhomogenousStokes}, we do the same with the only difference that the second integral does not vanish:
  \begin{align*}
    \int_\Omega u \cdot \nabla\psi \d x 
    = -\int_\Omega \div(u)\, \psi \d x + \int_{\partial\Omega} u \cdot n\, \psi \d\sigma 
    = \int_{\partial\Omega} u \cdot n  \,\psi \d\sigma.
  \end{align*}
  Putting everything together gives
  \begin{align}
    \int_\Omega |u|^2 \d x 
    &= \Big| \int_{\partial\Omega} u \cdot \Big\{ - \frac{\partial w}{\partial n} + n\, \psi \Big\} \d\sigma\, \Big|
    \leq \| u\|^{}_\partial \, \Big\{ \|\nabla w\|^{}_\partial + \|\psi\|^{}_\partial \Big\} \label{eq:u2estimate}
  \end{align}
  by the Cauchy-Schwarz inequality.
  As the pressure $\psi$ is only specified modulo additive constants, we may as well assume that $\int_{\partial\Omega} \psi \d\sigma = 0$.
  Furthermore, by the Schwarz theorem, we see from \eqref{eq:inhomogenousStokes} that $\Delta \psi = \div(\overline u) = 0$ in $\Omega$. 
  As stated in Remark~\ref{rem:harmonicEstimate}, this allows us to use the results from the proof of \eqref{eq:nablaPhin} with $\phi = \psi$ to conclude that
  \begin{align}
      \|\psi\|^{}_\partial 
    &\leq C\, \| \nabla \psi \cdot n\|_{\HH^{-1}(\partial\Omega)} \nonumber
    \intertext{and since $w$ has vanishing trace on $\partial\Omega$ we can use that property together with the fact that  $(w,\psi)$ solves \eqref{eq:inhomogenousStokes} to further estimate}
    &\leq C \, \Big\{ \| \Delta w \cdot n \|_{\HH^{-1}(\partial\Omega)} + \| \overline u \cdot n \|_{\HH^{-1}(\partial\Omega)} \Big\} \nonumber\\[0.5em]
    &\leq C \, \Big\{ \|  \nabla w\|^{}_\partial + \|u\|^{}_\partial  \Big\}, \label{eq:psiEstimate}
  \end{align}
  where for the last estimate we used \eqref{eq:deltaun} which is applicable since $\div w = 0$ on $\Omega$, see Remark~\ref{rem:harmonicEstimate}.
  If we combine inequalities \eqref{eq:u2estimate} and \eqref{eq:psiEstimate}, we get
  \begin{align}
    \int_\Omega |u|^2 \d x \leq C \, \Big\{ \|u\|^{}_\partial \|\nabla w \|^{}_\partial + \|u\|_\partial^2\Big\}. \label{eq:u2estimate2}
  \end{align}
  We are left with the task to estimate the first term on the right-hand side of \eqref{eq:u2estimate2}.
  To this end, it will suffice to show the following inequality
  \begin{align}
    \label{eq:w2estimate}
    \int_{\partial\Omega} |\nabla w|^2 \d\sigma \leq C \, \bigg\{ \int_\Omega |u|^2 \d x + \int_{\partial\Omega} |u|^2 \d \sigma \bigg\}
  \end{align}
  with a constant $C > 0$ depending only on $d$, $\theta$ and the Lipschitz character of $\Omega$ since by the weighted Young inequality for real numbers this would make the estimate
  \begin{align*}
    C \, \| u\|^{}_\partial \|\nabla w\|^{}_\partial \leq \frac{1}{2} \int_\Omega |u|^2 \d x + C \int_{\partial\Omega} |u|^2 \d \sigma
  \end{align*}
  available which after rearranging terms in \eqref{eq:u2estimate2} yields \eqref{eq:OmegaBoundaryEstimate}.

  In order to derive estimate \eqref{eq:w2estimate}, we will need the Rellich-type identity
%  \begin{align}
%    \label{eq:rellichIdentity3}
%    \begin{alignedat}{1}
%    \int_{\partial\Omega} h_k n_k |\nabla w|^2 \d \sigma
%    &= \int_\Omega \div(h)\, |\nabla w|^2 \d x + 2 \Re \int_{\Omega} h_j \frac{\partial \psi}{\partial x_j} \frac{\overline w_j}{\partial x_k}\d x   \\[0.5em]
%    &\quad + 2 \Re \int_\Omega h_k \lambda w_j \frac{\partial\overline w_j}{\partial x_k} + 2 \Re \int_\Omega h_k \overline u_j \frac{\partial \overline w_j}{\partial x_k} \d x. 
%    \end{alignedat}
%  \end{align} 
  \begin{align}
    \int_{\partial\Omega} h_k n_k |\nabla w|^2 \d\sigma
    &= - \int_\Omega \div(h)\, |\nabla w|^2 \d x 
    + 2 \Ret \int_\Omega \frac{\partial h_k}{\partial x_j} \cdot \frac{\partial w_i}{\partial x_k} \cdot \frac{\partial \overline w_i}{\partial x_j} \d x  \nonumber\\[0.5em]
    &\quad - 2 \Ret \int_\Omega \frac{\partial h_k}{\partial x_i} \cdot \frac{\partial w_i}{\partial x_k} \, \overline \psi \d x 
    + 2 \Ret \int_\Omega h_k \frac{\partial w_i}{\partial x_k} \cdot \overline{ \lambda w_i} \d x  \nonumber\\[0.5em]
    &\quad + 2 \Ret \int_{\Omega} h_k \frac{\partial \overline w_i}{\partial x_k} \, \overline u_i \d x\,, \label{eq:rellichIdentity3}
  \end{align}
  where $h = (h_1, \dots,h_d) \in \CC_0^1(\R^d; \R^d)$.
  Note that since all involved quantities are smooth up to the boundary, integration by parts is allowed and the proof of the stated  Rellich identity boils down to a formal calculation.
  The proof is analogous to the proof of Rellich identity \eqref{eq:rellichIdentity2} and one can prove that for $(w,\psi)$ an equality like \eqref{eq:rellichIdentity2} holds with one extra term, namely
  \begin{align*}
      - 2\Re \int_\Omega h_k \frac{\partial \overline w_i}{\partial x_k} \, \overline u_i \d x.
  \end{align*}
  This term will show up in the proof of Lemma~\ref{lem:rellichIdentity} in the calculation of $I_5$ when one uses the fact that $(w,\psi)$ solves problem \eqref{eq:inhomogenousStokes}.
  Finally, the first two terms on the right-hand side of \eqref{eq:rellichIdentity2} vanish due to the integration by parts rule for tangential derivatives, see \eqref{eq:intByPartsTang}, and the fact that $w$ is equal to $0$ on  $\partial\Omega$.

  Now, let $h \in \CC_0^1(\R^d; \R^d)$ with $h_k n_k \geq c > 0$ on $\partial \Omega$ be the function from Theorem~\ref{thm:smoothApproximation}.
%  Then the divergence theorem gives
%  \begin{align*}
%    \int_{\partial\Omega} h_k n_k |\nabla w|^2 \d\sigma
%    = \int_{\Omega} \div(h\, |\nabla w|^2) \d x 
%    = \int_\Omega \div(h)\, |\nabla w|^2 \d x + \int_\Omega h_k \frac{\partial}{\partial x_k} \Big\{ |\nabla w|^2 \Big\} \d x
%  \end{align*}
%  and we can rewrite the second summand as
%  \begin{align*}
%    \int_{\Omega} h_k \frac{\partial}{\partial x_k} \Big\{ |\nabla w|^2 \Big\} \d x
%    &= \int_{\Omega} h_k \frac{\partial}{\partial x_k} \Big\{ \frac{\partial w_j}{\partial x_i} \cdot \frac{\partial \overline w_j}{\partial x_i}\Big\} \d x \\[0.5em]
%    &= 2 \Re \int_\Omega h_k \frac{\partial^2 w_j}{\partial x_k \partial x_i} \cdot  \frac{\partial \overline w_j}{\partial x_i} \d x \\[0.5em]
%    + 2 \Re \int_
%    &= - 2 \Re \int_\Omega h_k \frac{\partial}{\partial x_k} \Big\{ \frac{\partial^2}{\partial x_i^2} w_j \Big\}\, \overline w_j \d x 
%    + \int_{\partial\Omega} \frac{\partial h_k}{\partial x_i} \cdot \frac{\partial^2 w_j}{\partial x_k \partial x_i} \, \overline w_j \d x\\[0.5em]
%    &= 2 \Re \int_\Omega h_k ( \Delta w_j ) \frac{\partial \overline w_j}{\partial x_k} \d x + 0 \\[0.5em]
%    &= 2 \Re \int_\Omega h_k \big( \frac{\partial \psi}{\partial x_j} + \lambda w_j - \overline u_j \big) \frac{\partial \overline w_j}{\partial x_k} \d x,
%  \end{align*}
%  where in addition to integration by parts we used \eqref{eq:inhomogenousStokes} and the fact that $w = 0$ on $\partial\Omega$.
  We apply the triangle inequality to the Rellich-type identity \eqref{eq:rellichIdentity3} to obtain
  \begin{align}
\label{eq:rellichInequality3}
      \begin{alignedat}{2}
    \int_{\partial\Omega} |\nabla w|^2 \d \sigma 
    &\leq C \, \Big\{ \int_\Omega |\nabla w|^2 \d x + \int_\Omega |\nabla w | |\psi| \d x \\[0.5em]
    &\qquad + |\lambda| \int_\Omega |\nabla w| |w| \d x + \int_\Omega |\nabla w| |u| \d x \Big\}
\end{alignedat}
  \end{align}
  with a constant $C > 0$ that only depends on $d$ and the Lipschitz character of $\Omega$.
  The left-hand side of this estimate establishes the starting point for the proof of inequality~\eqref{eq:w2estimate}.

  The next step consists in deriving estimates which are compatible with the right-hand side of \eqref{eq:rellichInequality3}.
  Testing the first equation of \eqref{eq:inhomogenousStokes} with $\overline w$, integration by parts and Lemma~\ref{lem:lambdaIneq} give us as in the proof of Lemma~\ref{lem:laxMilgramIneq} the estimate
  \begin{align*}
    \int_\Omega |\nabla w|^2 \d x + |\lambda| \int_\Omega |w|^2 \d x \leq C \int_\Omega |w| |u| \d x,
  \end{align*}
  where $C > 0$ depends only on $\theta$.
  The next step consists in using the previous inequality and the Poincar\'{e} inequality to estimate
  \begin{align*}
    \int_\Omega |\nabla w|^2 \d x + (1 + |\lambda| ) \int_\Omega |w|^2 \d x 
    &\leq  (1 + C) \int_\Omega |\nabla w|^2 \d x + |\lambda| \int_\Omega |w|^2 \d x \\[0.5em]
    &\leq C \int_\Omega |w| |u| \d x \\[0.5em]
    &\leq C \, \bigg( \int_\Omega |w|^2 \d x \bigg)^{1/2} \bigg( \int_\Omega |u|^2 \d x \bigg)^{1/2}\\[0.5em]
    \intertext{where for the last step we used the Cauchy-Schwarz inequality. The weighted Young inequality for real numbers allows us to further estimate}
    &\leq \frac{C}{4\varepsilon} \int_\Omega |u|^2 \d x + C \, \varepsilon \int_\Omega |w|^2 \d x \\[0.5em]
    &= \frac{\tilde C}{1 + |\lambda|} \int_\Omega |u|^2 \d x + \frac{1}{2} ( 1 + |\lambda| ) \int_\Omega |w|^2 \d x
  \end{align*}
  if we set $\varepsilon = \frac{(1 + |\lambda|)}{2 C}$.
  Rearranging terms, we can produce our next estimate
  \begin{align}
    \int_\Omega |\nabla w|^2 \d x + ( 1 + |\lambda| ) \int_\Omega |w|^2 \d x
    &\leq \frac{C}{1 + |\lambda|} \int_\Omega |u|^2 \d x,\label{eq:w2lambdaEstimate}
  \end{align}
  where $C > 0$ depends on $d$, $\theta$, the Lipschitz character of $\Omega$ and $\operatorname{diam}(\Omega)$.

  Now it's time to harvest: 
  %Using estimate \eqref{eq:rellichInequality3} together with \eqref{eq:w2lambdaEstimate} gives
  Using the weighted Young inequality, we see that we can simplify the right-hand side of \eqref{eq:rellichInequality3} via the chain of estimates
  \begin{align*}
    \int_{\partial\Omega} |\nabla w|^2 \d\sigma 
    &\leq C_\varepsilon\,  (1 + |\lambda|) \int_{\Omega} |\nabla w|^2 \d x + C\,  |\lambda| \int_\Omega |w|^2 \d x \\[0.5em]
    &\qquad\qquad+ C \int_\Omega |u|^2 \d x + \varepsilon \int_\Omega |\psi|^2 \d x  \\
    &\leq C_\varepsilon \int_\Omega |u|^2 \d x + \varepsilon \int_\Omega |\psi|^2 \d x,
  \end{align*}
  where the second inequality is thanks to estimate \eqref{eq:w2lambdaEstimate}.
  The first term on the right-hand side is already fine for \eqref{eq:w2estimate}.
  For the second one, we use the estimate $\|\psi\|^{}_{\Ell^2(\Omega)} \leq C\, \|\psi\|^{}_\partial$ and inequality \eqref{eq:psiEstimate} and arrive at
  \begin{align*}
    \varepsilon \int_\Omega |\psi|^2 \d x \leq \varepsilon \, C \int_{\partial\Omega} |\nabla w|^2 \d \sigma + C_\varepsilon \int_{\partial\Omega} |u|^2 \d\sigma.
  \end{align*}
  Choosing $\varepsilon = \frac{1}{2 C}$ and rearranging finally gives the desired estimate \eqref{eq:w2estimate}.
  This concludes our proof.
\end{proof}

The next theorem states the important fact that, in $\Ell^2$, the Dirichlet Stokes resolvent problem has a unique solution.

\begin{thm}
  \label{thm:exAndUniqueSolution}
  Let $\Omega$ be a bounded Lipschitz domain in $\R^d$, $d \geq 2$, with connected boundary and let $\lambda \in \Sigma_\theta$.
  For all $g \in \Ell^2_n(\partial\Omega)$ there exists a unique vectorfield $u$ and harmonic function $\phi$ which is unique up to constants such that $(u,\phi)$ satisfies \eqref{eq:stokesResolventProblem}, $(u)^* \in \Ell^2(\partial\Omega)$ and $u = g$ on $\partial\Omega$ in the sense of nontangential convergence.
  Moreover, the estimate $ \| (u)^* \|^{}_\partial \leq C \, \| g\|^{}_\partial$ holds and $u$ may be represented by the double layer potential $\dlp_\lambda(f)$ with $\|f\|^{}_\partial \leq C \, \|g\|^{}_\partial$, where in both cases $C > 0$ depends only on $d$, $\theta$ and the Lipschitz character of $\Omega$.
\end{thm}

\begin{proof}
  By Lemma~\ref{lem:l2unique}, we already know that the problem under consideration admits at most one solution.
  Therefore, we only have to worry about the existence of a solution.
  In Chapter~\ref{chap:3}, it was already established that a solution to the Stokes resolvent problem is given by the double layer potentials $u \coloneqq \dlp_\lambda(f)$ and $\phi \coloneqq \dlp_\Phi(f)$, $f \in \Ell^2(\partial\Omega; \C^d)$.
  It was also shown that in this way one also solves the $\Ell^2$ Dirichlet problem with boundary data $\dlp_\lambda(f)_+$.
  From Theorem~\ref{thm:nontangentialLimitDoubleLayer} we know that we find this nontangential limit as $\big((-1/2)I + \K_{\bar\lambda}^*\big)f$.
  Thus, the central idea of this proof will be to invert the operator $(-1/2)I + \K_{\bar\lambda}^*$ in order to find the \emph{right} $f$ to plug into the double layer potentials in order to attain the given boundary data $g \in \Ell^2_n(\partial\Omega)$ as a nontangential limit.
  
  We first note that due to Lemma~\ref{lem:inverseEstimate} the operator
  \begin{align*}
    T \colon \Ell^2(\partial\Omega; \C^d) \to \Ell^2(\partial\Omega; \C^d), \quad\text x \mapsto -(1/2)x  + \K_{\bar \lambda}x,
  \end{align*}
  is a Fredholm operator on $\Ell^2(\partial\Omega; \C^d)$ with index $0$ and thus the same is true for its adjoint 
  \begin{align*}
    T^* \colon \Ell^2(\partial\Omega; \C^d) \to \Ell^2(\partial\Omega; \C^d),\quad x \mapsto  -(1/2)x + \K_{\bar\lambda}^*x.
  \end{align*}
  In the following paragraphs, we will show that $T^*$ has a bounded inverse.

  We know that for all $f \in \Ell^2(\partial\Omega; \C^d)$ we have $\div(\dlp_\lambda (f)) = 0$ and therefore 
  \begin{align*}
    \int_{\partial\Omega} T^* f \cdot n \d\sigma =\int_{\partial\Omega} u_+ \cdot n \d\sigma =  0 
  \end{align*}
  holds, where for the first equality we applied Theorem~\ref{thm:nontangentialLimitDoubleLayer}. The second equality uses the fact that since $(u)^*$ is integrable, Proposition~\ref{prop:approximationArgument} and hence the divergence theorem are available.
  This gives $\Im(T^*) \subseteq \Ell_n^2(\partial\Omega)$.
  Now, on the one hand we have
  \begin{align*}
    \operatorname{span}(n)
    = \Ell_n^2(\partial\Omega)^\perp
    &\subseteq \Im(T^*)^\perp
    = \ker(T)
    \intertext{and on the other hand, as $T$ is injective on $\Ell^2_n(\partial\Omega)$ by \eqref{eq:inverseEstimate2}, we have that }
     \operatorname{span}(n) 
     &\supseteq \ker(T).
  \end{align*}
  This yields $\operatorname{span}(n) = \ker(T)$.
  We can use this equality and show that
  \begin{align*}
    \Ell^2_n(\partial\Omega) = \ker(T)^\perp = \overline{\Im(T^*)} = \Im(T^*),
  \end{align*}
  where for the last equality we used the fact that the range of $T^*$ is closed, as usual for Fredholm operators.
  With the same argument we can show for $T$ that
  \begin{align*}
    \ker(T^*)^\perp = \overline{\Im(T)} = \Im(T).
  \end{align*}

  Now we want to consider restrictions of the operators $T$ and $T^*$ and derive estimates on the operator norms of their inverses.
  To make the following proof more readable let 
  \begin{align*}
    X = \Ell^2_n(\partial\Omega)\quad\text{ and }\quad Y = \Im(T).
  \end{align*}
  Both spaces are closed subspaces of the Hilbert space $\Ell^2(\partial\Omega; \C^d)$ and therefore again Hilbert spaces.
  Consequently, the operator
  \begin{align*}
    K_{Y,X}' \colon Y \to X, \quad x \mapsto T^* x
  \end{align*}
  is invertible by the continuous inverse theorem.
  At this point we could already establish the solvability of the $\Ell^2$ Dirichlet problem.
  But before we do this, in order to derive the additional estimates which were stated in the theorem, we want to bound the operator norm of $\big(K_{Y,X}'\big)^{-1}$ by a constant that does not depend on $\lambda$ but on the sectoriality parameter $\theta$.
  To this end, let us introduce the operator
  \begin{align*}
    K_{X,Y} \colon X \to Y, \quad x \mapsto T x \,.
  \end{align*}
  Now, for $x \in X$ and $y \in Y$ we have that
  \begin{align*}
    \Big\langle x, K_{X,Y}^* y \Big\rangle_X
    & =\Big\langle K_{X,Y}^{} x, y  \Big\rangle_Y
      =\Big\langle T x, y        \Big\rangle_Y
      =\Big\langle x, T^*y       \Big\rangle_X
      =\Big\langle x, K_{Y,X}' y \Big\rangle_X
  \end{align*}
  which shows that $K_{Y,X}' = K_{X,Y}^*$ on $Y$.
  With the above definitions at hand, Lemma~\ref{lem:inverseEstimate} states that $K_{X,Y}^{}$ is an invertible operator with operator norm of the inverse bounded from above by some $C > 0$ and $C$ depends at most on $d$, $\theta$, the Lipschitz character of $\Omega$ and $\operatorname{diam}(\Omega)$.
  As $\| K_{X,Y}^* \|_{\Li(Y,X)} = \|K_{X,Y}\|_{\Li(X, Y)}$ the same holds for the adjoint operator $K_{X,Y}^*$.
  In particular, we have that $\|(K_{X,Y}^*)^{-1} \|_{\Li(X,Y)} = \|K_{X,Y}^{-1}\|_{\Li(Y,X)}$.
  Therefore, for all $f \in Y = \Im(T) = \Im \big((1/2)I + \K_{\bar \lambda}\big)$ we have
  \begin{align}
    \label{eq:dualityArgument}
    \begin{alignedat}{1}
    \| f\|_\partial^{}
    &= \Big\| (K_{Y,X}')^{-1} K_{Y,X}' f \Big\|_\partial^{} \\
      &\leq  \Big\| (K_{X,Y}^*)^{-1} \Big\|_{\Li(X,Y)}^{}\, \Big\| K_{X,Y}^* f\Big\|_\partial^{} %\nonumber\\
    \leq C \, \big\|(-(1/2)I + \K_{\bar \lambda}^*)f \big\|_\partial^{}.
    \end{alignedat}
  \end{align}
%  Considering once again estimate \eqref{eq:inverseEstimate2} and a duality argument we get that
%  \begin{align}
%    \|f\|_\partial \leq C \| T^* f\|_\partial
%  \end{align}
%  for all $f \in \Im(-(1/2)I + \K_\lambda)$, as
%  \begin{align*}
%    \| (T^*)^{-1} \|_{ \Ell^2_n(\partial\Omega), R(-(1/2)I + \K_{\bar\lambda}),} = \| T^{-1} \|_{R(-(1/2) I + \K_{\bar\lambda}), \Ell^2_n(\partial\Omega)}.
%  \end{align*}

  We are now in position to derive the missing estimates which were stated in the theorem.
  For $g \in \Ell^2_n(\partial\Omega)$, let $f \in \Im(T)$ with $T^* f = g$.
  Fix this $f$ and let $(u,\phi)$ be the respective double layer potentials which were defined in equations \eqref{eq:defDoubleLayer} and \eqref{eq:defDoubleLayerPressure}.
  Then $u_+ = T^* f =  g$ on $\partial\Omega$ by Theorem~\ref{thm:nontangentialLimitDoubleLayer}.
  Additionally, we have that
  \begin{align*}
    \|(u)^* \|_\partial^{} \leq C \, \|f\|_\partial^{} \leq C\, \| g\|_\partial^{}
  \end{align*}
  where we used inequality \eqref{eq:lpBoundednessUNontangentialMax} and \eqref{eq:dualityArgument}. In particular, this gives $(u)^* \in \Ell^2(\partial\Omega)$.
  Consequently, all claims of the theorem have been proven.
\end{proof}

The next theorem can in some sense be regarded as a reverse trace theorem and will play an important role for the proof of the needed reverse Hölder inequality in the forthcoming chapter.

\begin{thm}
  \label{thm:reverseTrace}
  Let $\Omega$ be a bounded Lipschitz domain in $\R^d$, $d \geq 2$, with connected boundary.
  Let $u \in \HH^1(\Omega; \C^d)$ and $\pi \in \Ell^2(\Omega)$ satisfy the Stokes resolvent problem in $\Omega$ for some $\lambda \in \Sigma_\theta$.
  Then
  \begin{align}
    \label{eq:reverseTrace}
    \bigg( \int_\Omega |u|^{p_d} \d x \bigg)^{1/{p_d}} \leq C\, \bigg( \int_{\partial\Omega} |u|^2 \d\sigma \bigg)^{1/2},
  \end{align}
  where $p_d = \frac{2d}{d - 1}$ and $C > 0$ depends only on $d$, $\theta$ and the Lipschitz character of $\Omega$.
\end{thm}

\begin{proof}
  We start our proof of \eqref{eq:reverseTrace} on the left-hand side by using inequality \eqref{eq:weizhangestimate}:
  \begin{align}
    \label{eq:appWeiZhang}
    \bigg( \int_\Omega |u|^{p_d} \d x \bigg)^{1/{p_d}} \leq C\, \bigg( \int_{\partial\Omega} |(u)^*|^2 \d \sigma \bigg)^{1/2},
  \end{align}
  where $C>0$ only depends on $d$ and the Lipschitz character of $\Omega$.
  %The proof of \eqref{eq:weizhangestimate} was carried out by Wei and Zhang in \cite[Lem.\@~3.3]{weiZhang} and can also be found in Shen's paper \cite[p.\@~418f.]{Shen2012}.

  Up to now, we do not know whether the right-hand side of inequality \eqref{eq:appWeiZhang} equals infinity.
  We only know that $(u,\phi)$ solves the Stokes resolvent problem and that it implicitly solves a Dirichlet problem with boundary data given as $f = \operatorname{Tr}_{\partial\Omega} (u) \in \Ell^2(\partial\Omega; \C^d)$.
  The following part of this proof will show that $u$ coincides with the solution $w \coloneqq \dlp_\lambda(g)$, $g \in \Ell^2(\partial\Omega; \C^d),$  of the $\Ell^2$ Dirichlet problem \hyperref[eq:dirProblem]{$(\mathrm{Dir}_\lambda)$} with boundary data $f$ as given by Theorem~\ref{thm:exAndUniqueSolution}. 
  If this is the case, then the knowledge about $(w)^*$ that is contained in Theorem~\ref{thm:exAndUniqueSolution} will help us to complete estimate \eqref{eq:appWeiZhang}.

  In order to show that $u = w$ on $\Omega$, consider a sequence $(\Omega_j)_{j \in \N}$ of smooth domains that approximates $\Omega$ from inside as described by Theorem~\ref{thm:smoothApproximation}.
  Then, an application of Lemma~\ref{lem:l2unique} shows
  \begin{align}
    \label{eq:reverseTraceUnique}
    \int_{\Omega_j} |u - w|^2 \d x \leq C \int_{\partial\Omega_j} |u - w|^2 \d \sigma_j,
  \end{align}
  where $C> 0$ does not depend on $j$ but on the Lipschitz character of $\Omega$.
  Furthermore, the trace theorem on bounded Lipschitz domains gives us for all $h \in \HH^1(\Omega; \C^d)$ that
  \begin{align}
    \label{eq:traceLipschitzDomain}
     \|h\|_\partial^2 \leq C\, \|h\|_{\HH^1(\Omega; \C^d)}^2,
  \end{align}
  where $C> 0$ only depends on $d$ and the Lipschitz character of $\Omega$, see Wei and Zhang \cite[Lem.\@~2.2]{weiZhang}.
  Now, let $\varepsilon > 0$ be given. 
  From the theory of Sobolev spaces it is known that there exists $\varphi_\varepsilon \in \CC^\infty(\overline\Omega; \C^d)$ such that $\big\|\varphi_\varepsilon - u\big\|_{\HH^1(\Omega; \C^d)}^2 \leq {\varepsilon}/({6 \tilde C})$, see Adams and Fournier \cite[Thm.\@~3.18]{adams}.
  Here, $\tilde C > 0$ denotes the fixed constant from \eqref{eq:traceLipschitzDomain} which is uniform for all $\Omega_j$ and $\Omega$.
  Thanks to Theorem~\ref{thm:smoothApproximation}, we know that 
  \begin{align*}
    \int_{\partial\Omega_j} |\varphi_\varepsilon - w|^2 \d\sigma_j \to \int_{\partial\Omega} |\varphi_\varepsilon - u|^2 \d\sigma, \quad \text{ as } j \to \infty,
  \end{align*}
  since $w = f$ on $\partial\Omega$ in the sense of nontangential convergence.
  Therefore, we choose $J$ large enough such that for all $j \geq J$ we have
  \begin{align*}
    \int_{\partial\Omega_j} |\varphi_\varepsilon - w|^2 \d\sigma_j \leq \int_{\partial\Omega} |\varphi_\varepsilon - u|^2 \d\sigma + \frac{\varepsilon}{6}\,.
  \end{align*}
  Plugging everything together, this gives us the chain of estimates
  \begin{align*}
    &\int_{\partial\Omega_j} |u - w|^2 \d\sigma_j  \\
    &\qquad\leq 2 \, \bigg\{ \int_{\partial\Omega_j} |u - \varphi_\varepsilon |^2 \d\sigma_j + \int_{\partial\Omega_j} |\varphi_\varepsilon - w|^2 \d\sigma_j \bigg\}
    \leq 4 \, \tilde C \, \big\| u - \varphi_\varepsilon \big\|_{\HH^1(\Omega; \C^d)}^2 + \frac{\varepsilon}{3} 
    \leq \varepsilon.
  \end{align*}
  Gluing together this inequality and \eqref{eq:reverseTraceUnique}, we see that $w = u$ holds a.e.\@ in $\Omega$ once we take the limit $j \to \infty$ and remember that $\varepsilon$ was chosen arbitrarily.
  In particular, we have $(u)^* = (w)^*$ a.e.\@ on $\partial\Omega$.
  Now, Theorem~\ref{thm:exAndUniqueSolution} gives
  \begin{align*}
%    \| (u)^* \|^{}_\partial 
%    = \| (w)^* \|^{}_\partial \
%    \leq C\, \| f\|^{}_\partial 
%    = C\, \|u\|_\partial,
    \int_{\partial\Omega} |(u)^*|^2 \d\sigma 
    = \int_{\partial\Omega} |(w)^*|^2 \d \sigma
    \leq C \int_{\partial\Omega} |f|^2 \d\sigma 
    = C \int_{\partial\Omega} |u|^2 \d\sigma
  \end{align*}
  where $C> 0$ depends on $d$, $\theta$ the Lipschitz character of $\Omega$ and $\operatorname{diam}(\Omega)$.
  Using this inequality to continue estimate \eqref{eq:weizhangestimate} concludes our proof.
\end{proof}

\begin{rem}
  At this point, the choice $p = \frac{2d}{d - 1}$ in Theorem~\ref{thm:reverseTrace} may seem arbitrary.
  Taking a closer look at the proof of inequality \eqref{eq:appWeiZhang} which is part of Lemma \ref{lem:weiZhang} and appeared in Wei and Zhang \cite[Lem.\@~3.3]{weiZhang}, one sees that the choice of $p$ is due to two facts: (1) For the dual exponent we have $p' = \frac{2d}{d + 1}$. (2) It holds $\frac{1}{p'} - \frac{1}{p} = \frac{1}{d}$ and thus the \emph{Hardy-Littlewood-Sobolev theorem on fractional integration} may be applied to estimate the $p$-norm of the \emph{Riesz potential} $I_1(f)$ of a function $f \in \Ell^{p'}(\R^d)$ by the $p'$-norm of $f$, see Grafakos \cite[Thm.\@~6.1.3]{grafakos2009modern}.
\end{rem}

In the following remark and the forthcoming chapter, we will make use of an integration argument which can be considered an application of the following theorem on \emph{integration along slices}. A proof of this result can be found in Federer \cite[Thm.\@~3.2.12]{federer}.

\begin{thm}[Co-area formula]
  \label{thm:coarea}
  If $f \colon \R^d \to \R$, $d \geq 2$, is Lipschitz continuous and $\mathfrak{g}$ the representative of a function $g \in \Ell^2(\R^d)$, then
  \begin{align}
    \label{eq:coarea}
    \int_{\R^d} \mathfrak{g}(x) \bigg[ \, \sum_{i = 1}^d \Big|\frac{\partial f}{\partial x_i}(x)\Big|^2 \, \bigg]^{1/2} \d x 
    = \int_\R \int_{f^{-1}(y)} \mathfrak{g}(x) \d m_{d - 1}(x) \d y,
  \end{align}
  where $m_{d - 1}$ denotes the $(d - 1)$-dimensional Hausdorff measure on $\R^d$.
\end{thm}

Note that the $(d-1)$-dimensional Hausdorff measure on $\R^d$ is comparable to the surface measure $\sigma$ of a bounded Lipschitz domain $\Omega \subseteq \R^d$.
When using Theorem~\ref{thm:coarea} in estimations, we will therefore always be working with the surface measure only.

\begin{rem}
  Let $(u,\phi)$ be a solution of the Stokes resolvent system in the domain $\BB(x_0,\, r) \subseteq \R^d$.
  Then the interior estimate
  \begin{align}
    \label{eq:interiorEstimateDoubleLayer}
    |\nabla^l u(x_0) | \leq \frac{C_l}{r^{l}} \, \bigg( \, \frac{1}{r^d} \int_{\BB(x_0,\, r)} |u(x)|^2 \d x \bigg)^{1/2}
  \end{align}
  holds for all $l \in \N_0$, where $C_l > 0$ only depends on $d$, $l$ and $\theta$:
  Without loss of generality, we may rescale and translate and thus assume that $x_0 = 0$ and $r = 2$.
  Let $t \in (1,2)$. 
  By Theorem~\ref{thm:exAndUniqueSolution}, we know that a solution to the Stokes resolvent system on $\BB(0,\, t) \subsetneq \BB(0,\,2)$ with boundary values $g_t \coloneqq \operatorname{Tr}_{\partial\BB(0,\,t)} (u) \in \Ell^2(\partial\BB(0,\,t); \C^d)$ is given by a boundary layer potential $\dlp_\lambda(f_t)$, $f_t \in \Ell^2(\partial\BB(0,\,t); \C^d)$.
  We use this fact to derive the desired estimate via
  \begin{align*}
    |\nabla^l u(0)|^2 
    &\leq C\, \bigg( \int_{\partial \BB(0,\, t)} \Big\{ \, \big|\nabla_x^{l+1} \Gamma(y; \lambda)\big|  + \big|\nabla_x^l \Phi(y) \big| \, \Big\} \, |f_t(y)| \d\sigma(y)\bigg)^2 \\[0.5em]
    &\leq C\, \bigg( \int_{\partial \BB(0,\,t)} \frac{|f_t(y)|}{t^{d - 1 + l}}  \d\sigma(y)\bigg)^2 \\[0.5em]
    &\leq C\, \int_{\partial \BB(0,\,t)} |f_t(y)|^2  \d\sigma(y) \\[0.5em]
    &\leq C\, \int_{\partial \BB(0,\,t)} |u(y)|^2 \d\sigma(y), 
  \end{align*}
  where in the last step we used the estimate of $f_t$ against the ``data'' from Theorem~\ref{thm:exAndUniqueSolution}.
  Integrating this inequality in $t$ over the interval $(1,2)$ and using the co-area formula \eqref{eq:coarea} with Lipschitz function $|\cdot|$ gives 
  \begin{align*}
    |\nabla^l u(0)|^2 \leq C \int_{\BB(0,\,2)} |u(x)|^2 \d x.
  \end{align*}
  Note that for this argument to work it is crucial that $C > 0$ does only depend on $d$, the Lipschitz character and diameter of $\BB(x_0,2)$.
  All this quantities are comparable for the involved domains $\BB(x, t)$, $1 \leq t \leq 2$.
  Now, the claim follows readily.
\end{rem}

