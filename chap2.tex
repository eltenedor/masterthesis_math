\chapter{Estimating Fundamental Solutions}
\label{chap:2}

The purpose of this section is to study fundamental solutions of the Stokes resolvent problem and to deduce related estimates which will be crucial for  the next chapters.
Before working on the Stokes resolvent problem we will take a look at the atoms of the fundamental solution of this problem: the Hankel functions. 

As a basis for the subsequent sections and chapters let us fix recurring quantities regarding sectors in the complex plane $\C$.

Let $\theta \in (0, \pi/2)$ and $\lambda \in \Sigma_\theta$ as in Section \ref{sec:stokesOperator}.  
The polar form of $\lambda$ is given as $\lambda = r \e^{\ii \tau}$ with $0 < r < \infty$ and $-\pi + \theta < \tau < \pi - \theta$.
Now set 
\begin{align*}
  k = \sqrt{r} \, \e^{\ii(\pi + \tau)/2}.
\end{align*}
Then we have
\begin{align*}
  k^2 = -\lambda\quad\text{ and }\quad \frac{\theta}{2} < \arg(k) < \pi - \frac{\theta}{2}
\end{align*}
as it holds
\begin{align*}
  \arg(k) = \frac{\pi + \tau}{2} &> \frac{\pi}{2} + \frac{-\pi + \theta}{2} = \frac{\theta}{2} 
  \intertext{on the one hand and}
  &< \frac{\pi}{2} + \frac{\pi - \theta}{2} = \pi  - \frac{\theta}{2}
\end{align*}
on the other hand.
The preceding calculation gives rise to the following estimate:
\begin{align}
  \label{eq:imaginaryPartEstimate}
  \Im(k) > \sqrt{|\lambda|}  \sin(\theta/2) > 0.
\end{align}
Indeed, we have 
\begin{align*}
  \Im( k) = \sqrt{r} \sin\left( \frac{\pi + \tau}{2} \right) = \sqrt{|\lambda|} \sin\left( \frac{\pi + \tau}{2} \right)\quad\text{and}\quad \frac{\theta}{2} < \frac{\pi + \tau}{2} < \pi - \frac{\theta}{2}
\end{align*}
which gives for $\tau$ with $\frac{\pi + \tau}{2} \leq \frac{\pi}{2}$ that $\sin(\frac{\pi + \tau}{2}) \geq \sin(\frac{\theta}{2} )$ and for $\tau$ with $\frac{\pi + \tau}{2} > \frac{\pi}{2}$ that $\sin(\frac{\pi + \tau}{2}) > \sin(\pi - \frac{\theta}{2} ) = \sin(\frac{\theta}{2} )$.

\section{Hankel Functions and the Helmholtz equation}

Before diving into fundamental solutions of the Stokes resolvent problem, we will first consider a fundamental solution for the (scalar) Helmholtz equation in $\R^d$
\begin{align*}
-\Delta u + \lambda u = 0.
\end{align*}
One fundamental solution with pole at the origin is given by
\begin{align}
  \label{eq:definitionFundamentalHelmholtz}
  G(x; \lambda) = \frac{\ii}{4 ( 2\pi )^{\frac{d}{2} - 1}} \cdot \frac{1}{|x|^{d - 2}} \cdot (k |x|)^{\frac{d}{2} - 1} H_{\frac{d}{2} - 1}^{(1)} (k|x|),
\end{align}
see McLean \cite[Eq. (9.14)]{mclean}, where $H_{\nu}^{(1)}(z)$ is the Hankel function of the first kind which according to Lebedev \cite[Sec. 5.11]{lebedev} can be also be written as
\begin{align}
  \label{eq:integralRepresentationHankel}
  H_\nu^{(1)}(z) = \frac{2^{\nu + 1} \e^{\ii(z - \nu \pi)} z^\nu}{\ii \sqrt{\pi} \Gamma(\nu + \frac{1}{2})} \int_0^\infty \e^{2 z \ii s} s^{\nu - \frac{1}{2}} (1 + s)^{\nu - \frac{1}{2}} \d s.
\end{align}
This formula holds for $\nu > -\frac{1}{2}$ and $0 < \arg(z) < \pi$.
We will usually set 
\begin{align*}
  \nu = \nu_d = \frac{d}{2} - 1 \quad\text{and}\quad z = k|x|. 
\end{align*}
Note that by \eqref{eq:imaginaryPartEstimate} we will always have $\Im(z) > 0$. 
Since $\nu_d < \nu_{d + 1}$ for all $d \geq 2$ and $\nu_2 = 0$, formula \eqref{eq:integralRepresentationHankel} will hold for all dimensions $d \geq 2$ and all $x \in \R^d$.

In the case $d = 2$, formula \eqref{eq:definitionFundamentalHelmholtz} simplifies to 
\begin{align}
  \label{eq:2dDefinitionFundamentalHelmholtz}
  G(x;\lambda) = \frac{\ii}{4} H_{0}^{(1)}(k|x|).
\end{align}
In the case $d = 3$, one has an even easier formula, namely
\begin{align}
  \label{eq:3dDefinitionFundamentalHelmholtz}
  G(x; \lambda) = \frac{\ii}{4(2\pi)^{1/2}} \cdot \frac{1}{|x|} \cdot (k |x|)^{1/2} H_{1/2}^{(1)}(k|x|) =  \frac{\e^{\ii k |x|}}{4 \pi |x|},
\end{align}
which is due to an easy formula for $H_{1/2}^{(1)}(z)$ in Lebedev, see \cite[Eq. (5.8.4)]{lebedev} and \cite[Eq. (9.15)]{mclean}.

Our first estimate is concerned with estimates on the fundamental solution and its derivatives for the (scalar) Helmholtz equation.
The main concern of this lemma is with the asymptotic behavior of $G(\,\cdot\,, \lambda)$ for large values of $|x|$.

\begin{lem}
  \label{lem:estimateHelmholtzDerivatives}
  Let $\lambda \in \Sigma_\theta$.
  Then
  \begin{align}
    \label{eq:estimateHelmholtzDerivatives}
    |\nabla_x^l G(x; \lambda)| \leq \frac{ C_l \e^{-c \sqrt{|\lambda|} |x|}}{|x|^{d - 2 + l}}
  \end{align}
  for any integer $l \geq 0$ if $d \geq 3$ and for $l \geq 1$ if $d = 2$.
  Here, $c > 0$ depends only on $\theta$ and $C_l$ depends only on $d$, $l$ and $\theta$.

  Let $d = 2$. Then $|G(x; \lambda)| = o(1)$ as $|x| \to \infty$.
\end{lem}

\begin{proof}
  We start with the case $l = 0$ and $d \geq 3$.
  Let $\Im(z) > 0$ and $\nu - \frac{1}{2} \geq 0$.
  Then \eqref{eq:integralRepresentationHankel} gives
  \begin{align}
    \label{eq:firstEstimate}
    |H_\nu^{(1)}(z)|
    &\leq C_d \, \e^{-\Im(z)} |z|^\nu \int_0^\infty \e^{-2s \Im(z)} s^{\nu - \frac{1}{2}} (1 + s)^{\nu - \frac{1}{2}} \d s,
  \end{align}
  where $C_d > 0$ depends only on $d$.
  We apply the substitution rule with $t = s - (1/2)$ and calculate
  \begin{align*}
    \e^{\frac{-\Im(z)}{2}} \int_0^\infty \e^{-s \Im(z)} s^{\nu - \frac{1}{2}} ( 1 + s )^{\nu - \frac{1}{2}} \d s
    &= \int_0^\infty \e^{- (s + \frac{1}{2}) \Im(z)} s^{\nu - \frac{1}{2}} ( 1 + s )^{\nu - \frac{1}{2}} \d s \\
    &= \int_{\frac{1}{2}}^\infty \e^{-t \Im(z) } \Big(t^2 - \frac{1}{4} \Big) ^{\nu - \frac{1}{2}} \d t \\
    &\leq \int_0^\infty e^{-t \Im(z)} t^{2\nu - 1} \d t \\
    &= \int_0^\infty e^{-u}  u^{2 \nu - 1} \Im(z)^{1 - 2\nu} \Im(z)^{-1} \d u \\
    &= \Im(z)^{-2 \nu} \int_0^\infty \e^{-u} u^{2 \nu - 1} \d u \\
    &= C_\nu \Im(z)^{-2 \nu},
  \end{align*}
  where we also used the substitution rule with $u = t \Im(z)$.
  Now we multiply \eqref{eq:firstEstimate} by $|z|^{v}$ and reuse the previous estimate to arrive at
  \begin{align*}
    |z|^\nu |H_\nu^{(1)}(z)| \leq C_d \, C_\nu\, |z|^{2 \nu} \, |\Im(z)|^{-2\nu} \e^{-\frac{\Im(z)}{2}},
  \end{align*}
  which for $z = k|x|$ gives
  \begin{align}
    \label{eq:zHEstimate}
    |kx|^\nu |H_\nu^{(1)}(k |x|)| \leq C \sin(\theta/2)^{-2\nu} \, \e^{-\frac{1}{2} \sin(\theta/2) \sqrt{|\lambda|} |x|},
  \end{align}
  where $C > 0$ depends only on $d$ and we used \eqref{eq:imaginaryPartEstimate} to estimate
  \begin{align*}
    (|kx|)^{2\nu} \cdot |\Im(k|x|)|^{-2\nu} 
    = |\lambda|^\nu \cdot |\Im(k)|^{-2\nu} 
    \leq \sin(\theta/2)^{-2\nu}.
  \end{align*}
  Using \eqref{eq:definitionFundamentalHelmholtz}, we estimate for $d \geq 3$ setting $\nu = \frac{d}{2} - 1$
  \begin{align*}
    |G(x; \lambda)| 
    \leq C\,  |x|^{2 - d} e^{-c \sqrt{|\lambda|} |x|}
  \end{align*}
  and it is clear that the generic constants depends on $d$ and $\theta$.
  This gives the estimate for $l = 0$ and $d \geq 3$.

  Using the relation for the derivatives of Hankel functions which one finds in the book of Lebedev~\cite[Eq. (5.6.3)]{lebedev},
  \begin{align*}
    \frac{\d{}}{\d z} \Big\{ z^{-\nu} H_\nu^{(1)}(z) \Big\} = -z^{-\nu} H_{\nu + 1}^{(1)} (z),
  \end{align*}
  we inductively establish the estimate \eqref{eq:estimateHelmholtzDerivatives} for $l \geq 1$ and $d \geq 2$:
  For $1 \leq j \leq d$, we calculate using the product and chain rule 
  \begin{align}
    \label{eq:derivativeOfG}
    |\nabla_x^{} G(x; \lambda)|
    &\leq C  \cdot \Big\{|x|^{1 - d} \cdot (k|x|)^{\frac{d}{2} - 1} H_{\frac{d}{2} - 1}^{(1)}(k|x|) - |x|^{2 - d} (k|x|)^{\frac{d}{2} - 1} H_{\frac{d}{2}}^{(1)}(k|x|) \cdot k\Big\} \nonumber\\
    &\leq C \cdot |x|^{1 - d}\Big\{(k|x|)^{\frac{d}{2} - 1} H_{\frac{d}{2} - 1}^{(1)}(k|x|) -  (k|x|)^{\frac{d}{2}} H_{\frac{d}{2}}^{(1)}(k|x|) \Big\},
  \end{align}
  where $C > 0$ is a generic constant that depends on $d$. 
  Note that the first summand in \eqref{eq:derivativeOfG} does not arise in the case $d = 2$ as is easily seen from equation \eqref{eq:2dDefinitionFundamentalHelmholtz}.
  The terms in the bracket can now be estimated individually by \eqref{eq:zHEstimate}.
  The extension of this proof to orders of differentiation $l \geq 2$ is straightforward using the Leibniz product rule for higher derivatives.

  Now for the last part of the proof, let us verify the claim regarding the asymptotic behavior of $|G(x; \lambda)|$ if $d = 2$.
  Based on the integral representation \eqref{eq:integralRepresentationHankel}, Lebedev derived an asymptotic expansion for the Hankel function \cite[Sec. 5.11, Eq.  (5.11.3)]{lebedev}. 
  For $\nu = (d/2) - 1 = 0$ and $z = k|x|$ this expansion reads
  \begin{align*}
    H_0^{(1)}(z) = \left(\frac{2}{\pi z}\right)^{1/2} \e^{\ii (z - (1/4) \pi)} \big[ 1 + O(|z|^{-1})\big].
  \end{align*}
  As $\Im(z) > 0$ we see that $|H_0^{(1)}(k |x|)| = O(|x|^{-1/2})$.
  Due to the simple structure of $G(x; \lambda)$ for $d = 2$ as shown in equation \eqref{eq:2dDefinitionFundamentalHelmholtz}, the claim follows easily.
\end{proof}

In the derivation of the next estimates we will use the following useful interior estimate for solutions of Poisson's equation which we write down for further use.

\begin{lem}
  \label{lem:interiorEstimatePoisson}
  Let $r > 0$ and $x \in \R^d$, $d \geq 2$. 
  If $w \in \CC^{k}(\BB(x,r)) \cap \CC^0(\overline{\BB(x,r)})$ is a solution to $\Delta w = f$ in $\BB(x,r)$, then
  \begin{align}
    \label{eq:interiorEstimatePoisson}
    |\nabla^l w(x)| \leq C r^{-l} \sup_{\BB(x,r)} |w| + C \max_{0 \leq j \leq l - 1} \sup_{\BB(x,r)} r^{j - l + 2} |\nabla^j f|, \quad l \leq k,.
  \end{align}
  where $C > 0$ only depends on $d$ and $l$.
\end{lem}

\begin{proof}
  If $l = 1$, then estimate \eqref{eq:interiorEstimatePoisson} is a consequence of the \emph{comparison principle} and a proof of this fact can be found in the book of Gilbarg and Trudinger \cite[Sec. 3.4, Eq. (3.16)]{gilbarg}.
  We can now use this estimate to inductively deduce the estimates for higher derivatives.
  Note that by translating from $x$ to $0$ and rescaling like
  \begin{align*}
    u_r(x) \coloneqq u(rx) \quad\text{and}\quad f_r(x) \coloneqq r^2 f(rx)
  \end{align*}
  we may assume that $\Delta w = f$ in $\BB(0,1)$ and that it suffices to prove
  \begin{align}
    \label{eq:interiorEstimatePoissonSimple}
    |\nabla^l w(0) | \leq C \sup_{\BB(0,1)} |w| + C \max_{0 \leq j \leq l - 1} \sup_{\BB(0,1)} |\nabla^j f|.
  \end{align}
  for $l > 1$.
  By the Schwartz theorem we have that if $w$ solves Poisson's equation with right hand side $f$ and $w$ and $f$ are sufficiently regular, then $\nabla^l w$ solves Poisson's equation with right hand side $\nabla^l f$.
  We can thus estimate inductively
  \begin{align*}
    |\nabla^l w(0)|
    &\leq C_l \sup_{\BB(0,1/2^{l - 1})} |\nabla^{l - 1} w| + C_l \sup_{\BB(0,1/2^{l-1})} |\nabla^{l - 1} f| \\
    &\leq C_l \sup_{\BB(0,1/2^{l-2})} |\nabla^{l - 2} w| + C_l \bigg\{ \sup_{\BB(0,1)} |\nabla^{l - 2} f| + \sup_{\BB(0,1)} |\nabla^{l - 1} f| \bigg\} \\
    &\leq \dots \\
    &\leq C_l \; \sup_{\BB(0,1)} |w| + C_l \sum_{j = 0}^{l - 1} \sup_{B(0,1)} |\nabla^j f|
  \end{align*}
  which readily yields the desired estimate.
\end{proof}

We will need the following asymptotic expansions for the function $z^\nu H_\nu^{(1)}(z)$ in $\C \setminus (-\infty, 0]$.
The derivation of these asymptotic expansions is based on asymptotic expansions of the \emph{Bessel functions of the first} and \emph{the second kind} and can be found in Tolksdorf \cite[Sec. 4.2]{tolksdorf}:
\begin{alignat}{2}
  z^{\nu}H_\nu^{(1)}(z) &= \frac{2^\nu \Gamma(\nu)}{\ii \pi} + \frac{\ii}{\pi} z^2 \log(z) + \omega z^2 + O(|z|^4 |\log(z)) \quad&&\text{if } d = 4, \label{eq:asymptoticd4}\\
  z^{\nu} H_{\nu}^{(1)}(z) &= \frac{2^\nu \Gamma(\nu)}{\ii\pi} + \frac{2^\nu \Gamma(\nu - 1)}{4 \pi \ii} z^2 + \omega z^3 + O(|z|^4) &&\text{if } d = 5, \label{eq:asymptoticd5}\\
  z^\nu H_\nu^{(1)}(z) &= \frac{2^\nu \Gamma(\nu)}{\ii \pi} + \frac{2^\nu \Gamma(\nu - 1)}{4\pi \ii}z^2 + O(|z|^4 |\log z|) &&\text{if } d = 6,\label{eq:asymptoticd6} \\
  z^\nu H_\nu^{(1)}(z) &= \frac{2^\nu \Gamma(\nu)}{\ii \pi} + \frac{2^\nu \Gamma(\nu - 1)}{4 \pi \ii} z^2 + O(|z|^4) &&\text{if } d \geq 7.\label{eq:asymptoticd7}
\end{alignat}

The next Lemma will be concerned with estimating the difference $G(x;\lambda) - G(x; 0)$ (and derivatives of this difference) of the fundamental solution to the scalar Helmholtz equation and the fundamental solution for $-\Delta = 0$ in $\R^d$ which is given by
\begin{align}
  \label{eq:laplace}
  G(x;0) \coloneqq \begin{cases} -\frac{1}{2\pi} \log(|x|), &\quad\text{for } d = 2, \\c_d \frac{1}{|x|^{d - 2}}, &\quad\text{for } d > 2,  \end{cases}
\end{align}
where the coefficient $c_d$ is given as as a multiple of the surface measure of the $(d-1)$-dimensional sphere $\mathbb{S}^{d - 1}$
\begin{align}
  \label{eq:wd}
  c_d = \frac{1}{(d - 2) \omega_d}, \quad\text{with}\quad \omega_d = \frac{2\pi^{\frac{d}{2}}}{\Gamma(\frac{d}{2})} = |\mathbb{S}^{d - 1}|.
\end{align}
Note that $c_2 = (2 \pi)^{-1}$.
By rearranging terms and using the functional equation of the Gamma function 
\begin{align}
  \label{eq:functionalGamma}
  \begin{alignedat}{1}
    \Gamma(z + 1) &= z\, \Gamma(z), \quad z \in \C, \Re(z) > 0 \\
    \Gamma(1) &= 1,
  \end{alignedat}
\end{align}
we get
\begin{align*}
  (d - 2) \omega_d 
  = 2  \bigg( \frac{d}{2} - 1\bigg) \frac{2 \pi^{\frac{d}{2}}}{\Gamma(\frac{d}{2})}
  =  2 \bigg(\frac{d}{2} - 1\bigg) \frac{2 \pi^{\frac{d}{2}}}{\Gamma(\frac{d}{2} - 1)(\frac{d}{2} - 1)}
  = \frac{4 \pi^{\frac{d}{2}}}{\Gamma(\frac{d}{2} - 1)},
\end{align*}
and thus we will also sometimes use the equivalent definition
\begin{align}
  \label{eq:defncd}
  c_d \coloneqq \frac{\Gamma(\frac{d}{2} - 1)}{4 \pi^{\frac{d}{2}}}.
\end{align}
Furthermore, the leading coefficient of the asymptotic expansions of the Hankel functions \eqref{eq:asymptoticd4}-\eqref{eq:asymptoticd7} for $d \geq 3$ will be denoted as
\begin{align}
  \label{eq:Defnad}
  a_d \coloneqq \frac{2^{\frac{d}{2} - 1} \Gamma(\frac{d}{2} - 1)}{\ii \pi}.
\end{align}
The coefficients $a_d$ and $c_d$ are related in the following way:
\begin{align*}
  c_d 
  %= \frac{\Gamma(\frac{d}{2} - 1) 2^\nu}{\ii \pi} 
  %=\frac{4 (2 \pi)^{\frac{d}{2} - 1}}{\ii} \frac{\Gamma(\frac{d}{2} - 1)}{4 \pi^{\frac{d}{2}}} 
  =\frac{\ii} {4 (2 \pi)^{\frac{d}{2} - 1}}\; a_d .
\end{align*}
This allows us to write for $d \geq 3$
\begin{align}
  \label{eq:HelmholtzLaplaceDifference}
  G(x;\lambda) - G(x; 0) = \frac{i}{4(2\pi)^{\frac{d}{2} - 1}} \cdot \frac{1}{|x|^{d - 2}} \Big\{ (k|x|)^{\frac{d}{2} - 1} H_{\frac{d}{2} - 1}^{(1)}(k|x|) - a_d \Big\}.
\end{align}

The following lemma will help us to estimate derivatives of \eqref{eq:HelmholtzLaplaceDifference} and the $2$-dimensional counterpart.

\begin{lem}
  \label{lem:HelmholtzLaplaceDifference}
  Let $\lambda \in \Sigma_\theta$.
  Then
  \begin{align}
    \label{eq:HelmholtzLaplaceDifferenceEstimate}
    \Big|\nabla_x^l \Big\{ G(x; \lambda) - G(x; 0) \Big\}\Big| \leq C |\lambda| |x|^{4 - d - l}
  \end{align}
  if $d \geq 5$ and $l \geq 0$, where $C$ depends only on $d$, $l$ and $\theta$.
  If $d = 3$ or $4$, estimate \eqref{eq:HelmholtzLaplaceDifferenceEstimate} holds for $l \geq 1$ and if $d = 2$, the estimate holds for $l \geq 3$.
\end{lem}

\begin{proof}
  \begin{enumerate}[(a)]
    \item In this part we will show that the desired estimate \eqref{eq:HelmholtzLaplaceDifferenceEstimate} holds if we assume that $|\lambda| |x|^2 > ({1}/{2})$.
    In this case, Lemma \ref{lem:estimateHelmholtzDerivatives} gives
    \begin{align*}
      \Big|\nabla_x^l \Big\{ G(x; \lambda) - G(x; 0) \Big\} \Big|
      \leq C \Bigg\{ \frac{\e^{-c \sqrt{|\lambda|} |x|}}{|x|^{d - 2 + l}} + \frac{1}{|x|^{d - 2 + l}} \Bigg\} \leq C \frac{|\lambda|}{|x|^{d - 4 + l}},
    \end{align*}
    where $C$ depends only on $d$, $l$ and $\theta$.
      Therefore, for the remaining proof we will suppose $|\lambda||x|^2 \leq ({1}/{2})$.
  \item In this step we show that we can restrict ourselves to proving \eqref{eq:HelmholtzLaplaceDifferenceEstimate} in three cases: (1) $d \geq 5$ and $l = 0$; (2) $d = 3$ or $4$ and $l = 1$; (3) $d = 2$ and $l = 3$.
    
    Suppose \eqref{eq:HelmholtzLaplaceDifferenceEstimate} holds in case (1) and let $l > 1$.
    If we set $ w(x) = G(x;\lambda) - G(x;0) $, we have $\Delta_x w = \lambda G(x; \lambda)$ in $\R^d \setminus \{0\}$.
    For $f = \lambda G(x; \lambda)$, estimate \eqref{eq:interiorEstimatePoisson} now gives
    \begin{align*}
      |\nabla^l w(x) |
      &\leq C r^{-l} \sup_{\BB(x,r)} |w| + C \max_{0 \leq j \leq l - 1} \sup_{\BB(x,r)} r^{j - l + 2} |\nabla^j f| \\
      &\leq C r^{-l} \sup_{y \in \BB(x,r)} |\lambda| |y|^{4 - d} + C \sum_{j = 0}^{l - 1} \sup_{y \in \BB(x,r)} r^{j - l + 2} |\lambda| |y|^{2 - d - j} \\
      &= C r^{-l} |\lambda| \, \bigg|x - r \frac{x}{|x|} \bigg|^{4 - d} \! + C \sum_{j = 0}^{l - 1} r^{j - l + 2} |\lambda|\, \bigg|x - r \frac{x}{|x|} \bigg|^{2 - d - j}\! ,
      \intertext{
    for all $0 < r < |x|$, where we used \eqref{eq:HelmholtzLaplaceDifferenceEstimate} with $l = 1$ for the first summand and \eqref{eq:estimateHelmholtzDerivatives} to estimate the second summand.
    We choose $r = \frac{|x|}{2}$ and receive
  }
      |\nabla^l w(x)| 
      &\leq C |\lambda |x|^{-l} |x|^{4 - d} + C \sum_{j = 0}^{l - 1} |x|^{j - l + 2} |\lambda| |x|^{2 - d - j} \\
      &\leq C |\lambda| |x|^{4 - d - l}.
    \end{align*}

    The proof for case (2) is completely analogous if one sets 
      \begin{align*}
        w(x) = \nabla_x (G(x; \lambda) - G(x; 0))\quad\text{and}\quad f(x) = \lambda \nabla_x G(x; \lambda).
      \end{align*}
    Also case (3) is proven in a similar fashion. 
%      We will give the proof for the sake of completeness.
%
%    For $w$ and $f$ as in case (2) by \eqref{eq:HelmholtzLaplaceDifferenceEstimate} we get
%    \begin{align*}
%      |\nabla^l w(x) |
%      &\leq C r^{-l} \sup_{\BB(x,r)} |w| + C \max_{0 \leq j \leq l - 1} \sup_{\BB(x,r)} r^{j - l + 2} |\nabla^j f| \\
%      &\leq C r^{-l} \sup_{y \in \BB(x,r)} |\lambda| |y|(|\log |\lambda| |y|^2 | + 1 ) + C \sum_{j = 0}^{l - 1} \sup_{y \in \BB(x,r)} r^{j - l + 2} |\lambda| |y|^{- j - 1} \\
%      &\leq S_1 + S_2,
%      \intertext{wheras}
%      S_1 
%      &\leq C r^{-l} |\lambda| |x + r \frac{x}{|x|} | ( |\log |\lambda| |x - r \frac{x}{|x|}|^2 | + |\log( |\lambda||x + r \frac{x}{|x|}|^2 )|+ 1) \\
%      &\leq C  |\lambda| |x|^{1-l}(|\log( |\lambda| |x|^2)| + 1)
%      \intertext{
%        if we choose $r = \frac{|x|}{2}$. For $S_2$ we calculate as before, using estimate \eqref{eq:estimateHelmholtzDerivatives}
%  }
%      S_2
%      &\leq C \sum_{j = 0}^{l - 1} C |x|^{j - l + 2} |\lambda| |x|^{-j - 1} \\
%      &\leq C |\lambda| |x|^{1 - l}.
%    \end{align*}

  \item In this step we prove \eqref{eq:HelmholtzLaplaceDifferenceEstimate} for $d \geq 5$ and $l = 0$. 
    First, note that for the functions
    \begin{align*}
      &g(x) \coloneqq (k|x|)^{\frac{d}{2} - 1} H_{\frac{d}{2} - 1}^{(1)}(k|x|), \quad g(0) = a_d, \\
      &h(z) \coloneqq z^{\frac{d}{2} - 1} H_{\frac{d}{2} - 1}^{(1)}(z), \quad h(0) = a_d,
    \end{align*}
    the mean value theorem yields the estimate
    \begin{align*}
      |g(x) - g(0)| \leq |x| \sup_{y \in \BB(0,|x|)} |\nabla g(y)| \leq |x| |k| \sup_{y \in \BB(0,|x|)}  \Big| \Big(\frac{\d{}}{\d z} h \Big) (k|y|)\Big|.
    \end{align*}
    Using representation \eqref{eq:HelmholtzLaplaceDifference}, we estimate
    \begin{align}
      |G(x; \lambda) - G(x; 0) |
      &\leq C |x|^{2 - d} \cdot |k| |x| \max_{\substack{|z| \leq |k| |x| \\ \Im(z) > 0}} \Big| \frac{\d{}}{\d z} \Big\{ z^{\frac{d}{2} - 1} H_{\frac{d}{2} - 1}(z) \Big\} \Big| \nonumber\\
      &= C |x|^{2 - d} \cdot |k| |x| \max_{\substack{|z| \leq |k||x| \\ \Im(z) > 0}} \Big|z^{\frac{d}{2} - 1} H_{\frac{d}{2} - 2}^{(1)}(z) \Big| \, , \label{eq:HelmholtzLaplaceDifferenceGt5}
    \end{align}
    where for the last equality we used another useful relation that can be found in Lebedev \cite[Eq. (5.6.3)]{lebedev},
    \begin{align}
      \frac{\d{}}{\d z} \Big\{ z^\nu H_\nu^{(1)}(z) \Big\} = z^\nu H_{\nu - 1}^{(1)}(z).
    \end{align}
    Since the asymptotic expansions yield that $|z^{\nu} H_\nu^{(1)}(z)| \leq C_\nu$ for $\nu > 0$ and $|z| \leq 1$ with $\Im(z) > 0$, it follows from \eqref{eq:HelmholtzLaplaceDifferenceGt5} that
    \begin{align*}
      |G(x;\lambda) - G(x; 0)| \leq C |x|^{2 - d} \cdot |k| |x| \cdot |k| |x| \max_{\substack{|z| \leq |k||x| \\ \Im(z) > 0}} \Big|z^{\frac{d}{2} - 2} H_{\frac{d}{2} - 2}^{(1)}(z) \Big|
      \leq C |\lambda| |x|^{4 - d}.
    \end{align*}
    
  \item Now we consider the case $d = 4$ and $l = 1$.
    The asymptotic expansion \eqref{eq:asymptoticd4} gives that
    \begin{align}
      \label{eq:mwt4d}
      \bigg| \frac{\d{}}{\d z} \bigg\{ \frac{z H_1^{(1)}(z) - a_4}{z^2} \bigg\} \bigg| \leq C |z|^{-1}
    \end{align}
    for all $|z| \leq \frac{1}{2}$ with $\Im(z) > 0$.
    Since
    \begin{align*}
      \frac{G(x; \lambda) - G(x; 0)}{\lambda} = - \frac{C (z H_1^{(1)}(z) - a_4)}{z^2},
    \end{align*}
    where $z = k |x|$. With \eqref{eq:mwt4d} we conclude that
    \begin{align*}
      \bigg|\frac{\nabla_x \{ G(x; \lambda) - G(x; 0) \}}{\lambda} \bigg|
      \leq C |k| \; \bigg| \frac{\d{}}{\d z} \bigg\{ \frac{z H_1^{(1)}(z) - a_4}{z^2} \bigg\} \bigg|_{z = k|x|} \bigg|
      \leq C |k| |k|^{-1} |x|^{-1},
    \end{align*}
    which after rearrangement of the involved terms gives the claim.

  \item For the case $d = 3$ and $l = 1$, we get from equation \eqref{eq:defncd} and a well known fact of the Gamma function, $\Gamma(1/2) = \sqrt{\pi}$, the following identity:
    \begin{align*}
      G(x;\lambda) - G(x; 0) = \frac{\e^{\ii k |x|}}{4\pi |x|} - \frac{c_3}{|x|} = \frac{\e^{\ii k|x|} - 1}{4 \pi |x|}.
    \end{align*}

    Now we calculate
    \begin{align*}
      \frac{\partial}{\partial x_j} \bigg\{ \frac{\e^{\ii k|x|} - 1}{|x|} \bigg\} 
      &= \frac{\partial}{\partial x_j} \bigg\{ \frac{\e^{\ii k |x|} - 1 - \ii k |x|}{|x|} \bigg\} 
      = \frac{\partial}{\partial x_j} \bigg\{ \sum_{n = 2}^\infty \frac{(\ii k |x|)^n}{n!} \cdot \frac{1}{|x|} \bigg\} \\
      &= \sum_{n = 2}^\infty \frac{(\ii k)^n}{n!} (n - 1) \cdot \frac{x_j}{|x|} |x|^{n - 2} 
    \end{align*}
    which in turn implies 
    \begin{align*}
      \bigg| \frac{\partial}{\partial x_j} \bigg\{ \frac{\e^{\ii k |x|} - 1}{|x|} \Big\} \Big|
      &\leq |\lambda| \sum_{n = 2}^\infty \frac{n - 1}{n!} |k|^{n - 2} |x|^{n - 2} 
      \leq C \, |\lambda|
    \end{align*}
    since $|\lambda||x| \leq (1/2)$.

  \item For the last case $d = 2$ and $l = 3$, we will directly calculate the estimate using the asymptotic expansion of $H_0^{(1)}(z)$ with $z = k|x|$. The calculations are omitted from this chapter. Instead, they can be found in the appendix of this thesis. \qedhere
%    \begin{align*}
%      H_0^{(1)}(z) 
%      &= J_0(z) + \ii Y_0(z) \\
%      &= \sum_{l = 0}^\infty \frac{(-1)^l}{(l!)^2 4^l} z^{2l} \big( 1 - \frac{2\ii \log(2)}{\pi} \big) 
%      - \frac{2 \ii}{\pi} \sum_{l = 0}^\infty \frac{(-1)^l}{(l!)^2 4^l} \psi(l + 1) \cdot z^{2l} \\
%      &\quad + \frac{2\ii}{\pi} \sum_{l = 0}^\infty \frac{(-1)^l}{(l!)^2 4^l} z^{2l} \log(z)
%    \end{align*}
%    The first complex derivative of $H_0^{(1)}(z)$ reads
%    \begin{align*}
%      \frac{\d{}}{\d z} H_0^{(1)}(z) = 
%    \end{align*}
%
  \end{enumerate} 
\end{proof}

\begin{rem}
  \label{rem:HelmholtzLaplaceDifference}
  In the situation of Lemma \ref{lem:HelmholtzLaplaceDifference}, one can show for $|\lambda| |x|^2 \leq (1/2)$ by considering the asymptotic expansions that 
  \begin{align*}
    |G(x; \lambda) - G(x; 0) | \leq \begin{cases}
      C \sqrt{|\lambda|} \quad&\text{if } d = 3, \\
      C |\lambda| \{ |\log(|\lambda| |x|^2) | + 1 \} \quad&\text{if } d = 4.
    \end{cases}
  \end{align*}
  Also using the asymptotic expansions it can be shown that if $d = 2$, then
  \begin{align*}
    |\nabla_x^l \{ G(x; \lambda) - G(x; 0) \} | \leq C |\lambda| |x|^{2 - l} \{ |\log(|\lambda| |x|^2 ) | + 1 \},
  \end{align*}
  for $l \in \{1, 2\}$.  
\end{rem}

\section{The Stokes Resolvent Problem}
\label{sec:2.2}

We will now analyze fundamental solutions to the \emph{Stokes resolvent problem}
\begin{align}
  \label{eq:stokesResolventProblem}
  \begin{alignedat}{1}
  -\Delta u + \nabla \phi + \lambda u &= 0 \\
  \div u &= 0
  \end{alignedat}
\end{align}
in $\R^d$ with $\lambda \in \Sigma_\theta$ with the goal to deduce helpful estimates for the following chapters.
  The fundamental solutions to the (scalar) Helmholtz equation and the Laplace equation will form the main ingredients for the following matrix of fundamental solutions to the Stokes resolvent problem with pole at the origin:
  \begin{align}
    \label{eq:fundamentalMatrixStokes}
    \Gamma_{\alpha\beta}(x;\lambda) = G(x; \lambda) \delta_{\alpha\beta} - \frac{1}{\lambda} \frac{\partial^2}{\partial x_\alpha \partial x_\beta} \Big\{ G(x; \lambda) - G(x; 0) \Big\}, \quad \alpha,\beta = 1,\dots,d.
  \end{align}
  As the matrix of fundamental solutions $\Gamma(x; \lambda) = (\Gamma_{\alpha\beta}(x; \lambda))_{d \times d}$ carries two arguments it cannot be confused with the Gamma function.
  Having formula \eqref{eq:fundamentalMatrixStokes} at sight, the following observations are obvious:
  \begin{align*}
    \Gamma_{\alpha\beta}(x; \lambda) = \Gamma_{\beta\alpha}(x; \lambda), \quad 
    \overline{\Gamma_{\alpha\beta}(x; \lambda)} = \Gamma_{\alpha\beta}(x; \bar\lambda)
    \quad\text{and}\quad
    \Gamma_{\alpha\beta}(x; \lambda) = \Gamma_{\alpha\beta}(-x; \lambda).
  \end{align*}
  For the pressure, we define the vector of fundamental solutions
  \begin{align}
    \label{eq:fundamentalVectorPressure}
      \Phi_\beta(x) = -\frac{\partial}{\partial x_\beta} \Big\{ G(x; 0) \Big\} = \frac{x_\beta}{\omega_d |x|^d}, \quad \beta= 1,\dots,d.
  \end{align}
  We note that $\Phi_\beta(x) = \Phi_\beta(-x)$.

  Using the fact that $\Delta_x G(x; \lambda) = \lambda G(x; \lambda)$ in $\R^d \setminus \{0\}$, one can see that on $\R^d \setminus \{0\}$ and for all $1 \leq \beta \leq d$
  \begin{align}
    \label{eq:solutionStokesSystem}
    \begin{alignedat}{1}
      (-\Delta_x + \lambda) \Gamma_{\alpha\beta}(x;\lambda) + \frac{\partial}{\partial x_\alpha} \Big\{ \Phi_\beta(x) \Big\} &= 0, \\
      \frac{\partial}{\partial x_\alpha} \Big\{ \Gamma_{\alpha\beta} (x; \lambda) \Big\} &= 0, \quad\text{for } 1 \leq \alpha \leq d.
    \end{alignedat}
  \end{align}
  Note that in the last equation the summation convention was used.

  We now keep up to the spirit of this exhausting chapter by proving further estimates, this time for the fundamental solutions to the Stokes resolvent problem \eqref{eq:stokesResolventProblem}.

\begin{thm}
  \label{thm:fundamentalMatrixEstimate}
  Let $\lambda \in \Sigma_\theta$.
  Then for any $d \geq 3$ and $l \geq 0$
  \begin{align}
    \label{eq:fundamentalMatrixEstimate}
    \big| \nabla_x^l \Gamma(x; \lambda) \big| &\leq \frac{C}{(1 + |\lambda||x|^2) |x|^{d - 2 + l}} 
  \end{align}
    where $C$ depends only on $d$, $l$ and $\theta$. For $d = 2$ and $l \geq 1$ the same estimate holds.
\end{thm}

  \begin{proof}
    Let $|\lambda| |x|^2 > (1/2)$. 
    Then there exist constants $C_a$, $C_b$, $C_c$ such that
    \begin{align*}
      \e^{-c \sqrt{|\lambda|} |x|} ( 1 + |\lambda| |x|^2) &\leq C_a, \\
      1 &\leq \frac{C_b |\lambda| |x|^2}{1 + |\lambda| |x|^2}, \\
      \e^{-c \sqrt{|\lambda|} |x|} &\leq \frac{ C_c |\lambda| |x|^2}{1 + |\lambda| |x|^2},
    \end{align*}
    where $c$ is the constant from Lemma \ref{lem:estimateHelmholtzDerivatives}.
    Using these estimates and Lemma \ref{lem:estimateHelmholtzDerivatives} gives
    \begin{align*}
      \big|\nabla_x^l \Gamma(x; \lambda) \big|
      &\leq \big|\nabla_x^l G(x; \lambda)\big| + \frac{1}{|\lambda|} \big|\nabla_x^{l + 2} G(x; \lambda) \big| + \frac{1}{|\lambda|} \big|\nabla_x^{l + 2} G(x; 0)\big| \\
    &\leq\frac{C_l\, \e^{-c \sqrt{|\lambda|} |x|}}{|x|^{d - 2 + l}} + \frac{1}{|\lambda|} \frac{C_{l + 2}\, \e^{-c \sqrt{|\lambda|} |x|}}{|x|^2 |x|^{d - 2 + l}} + \frac{1}{|\lambda|} \frac{C}{|x|^2 |x|^{d - 2 + l}} \\
    %&\leq \frac{C}{(1 + |\lambda||x|^2)} \frac{1}{|x|^{d - 2 + l}} + \frac{C_{l + 2} |\lambda||x|^2}{(1 + |\lambda||x|^2)} \frac{1}{|\lambda| |x|^2} \frac{1}{|x|^{d - 2 + l}} + \frac{C}{1 + |\lambda||x|^2} \frac{1}{|x|^{d - 2 + l}}
    &\leq \frac{C}{1 + |\lambda| |x|^2}\; \frac{1}{|x|^{d - 2 + l}}.
    \end{align*}
    Now let $|\lambda| |x|^2 \leq ({1}/{2})$.
    Then by Lemma \ref{lem:estimateHelmholtzDerivatives} and Lemma \ref{lem:HelmholtzLaplaceDifference} we get
    \begin{align*}
      \big|\nabla_x^l \Gamma(x; \lambda) \big|
      &\leq \big|\nabla_x^l G(x; \lambda) \big| + \frac{1}{|\lambda|} \cdot \big|\nabla_x^{l + 2} ( G(x; \lambda) - G(x; 0))\big| \\
      &\leq \frac{C}{|x|^{d - 2 + l}} + \frac{1}{|\lambda|} \cdot C |\lambda| |x|^{4 - d - (l + 2)} \\
      &\leq \frac{C}{|x|^{d - 2 + l}} \frac{(1 + |\lambda| |x|^2)}{(1 + |\lambda| |x|^2)} \\
      &\leq \frac{C}{(1 + |\lambda| |x|^2) |x|^{d - 2 + l}}
    \end{align*}
    which gives the claim.
    %If $d = 2$ the steps are analogous considering the different structure of the estimate \eqref{eq:HelmholtzLaplaceDifferenceEstimate2d}.
  \end{proof}

%  \begin{rem}
%    It is important to note that using the information from Remark \ref{rem:HelmholtzLaplaceDifference} regarding the case $d = 2$, the above proof would also work using only minor modifications starting from $l = 1$.
%    As a consequence one would derive estimates of the form
%    \begin{align*}
%      \big|\nabla_x^l \Gamma(x; \lambda)| \leq \frac{C}{(1 + |\lambda| |x|^2) |x|^l} (1 + \big|\log(|\lambda||x|^2)\, \big|), \quad l \geq 1 ,
%    \end{align*}
%    which offer a different behavior as estimates of the form \ref{eq:fundamentalMatrixEstimate}.
%    It will be important in later chapters to have estimates on $\nabla^l_x \Gamma(x; \lambda)$ that do not depend on $\lambda$.
%    Affortunately this type of estimates will only be needed for orders of differentiation $l \geq 1$.
%  \end{rem}

 %Regarding the structure of the above proof, the derived estimate for the case $d = 2$ seems natural at first sight. 
 %One important difference compared to $d \geq 3$ is that it is not possible to derive an estimate of the form $|\nabla_x^l \Gamma(x; \lambda)| \leq C |x|^{-l}$.

  If $\lambda = 0$, the Stokes resolvent problem becomes just the Stokes problem in $\R^d$
\begin{align}
  \label{eq:stokesProblem}
  \begin{alignedat}{1}
  -\Delta u + \nabla \phi + \lambda u &= 0, \\
  \div u &= 0.
  \end{alignedat}
\end{align}
  Whereas the fundamental solution for the pressure is maintained, the matrix of fundamental solutions to the Stokes problem in $\R^d$ with pole at the origin is given by $\Gamma(x; 0) = (\Gamma_{\alpha\beta}(x; 0))_{d \times d}$, where
  \begin{align}
    \label{eq:fundamentalSolutionStokes}
    \Gamma_{\alpha\beta}(x; 0) &\coloneqq \frac{1}{2 \omega_d} \Big\{ \frac{\delta_{\alpha\beta}}{(d - 2) |x|^{d - 2}} + \frac{x_{\alpha} x_\beta}{|x|^d} \Big\}
    \intertext{if $d \geq 3$ and}
    \label{eq:fundamentalSolutionStokes2d}
    \Gamma_{\alpha\beta}(x; 0) &\coloneqq \frac{1}{2 \omega_2} \Big\{ - \delta_{\alpha\beta} \log(|x|) + \frac{x_\alpha x_\beta}{|x|^2} \Big\}  \end{align}
  for $d = 2$.
  Note that the given fundamental solution for the case $d = 2$ differs from the one given by Mitrea and Wright \cite[Sec. 4.2]{mitreaWright} by having summands with alternating signs.
  Considering the structure of the fundamental solution for $d \geq 3$, our choice seems more natural with regard to the structure of the fundamental solutions to the Laplace equation \eqref{eq:laplace}.
  The alternating sign is necessary for $\Gamma_{\alpha\beta}$ to be divergence free.

  One important technique in the following chapter will be to reduce problems formulated for $\Gamma(x; \lambda)$ to problems formulated in $\Gamma(x; 0)$ perturbed by the difference $\Gamma(x; \lambda) - \Gamma(x; 0)$.
  Under this aspect it seems reasonable to study estimates of the difference of fundamental solutions.
  To this end it is helpful to rewrite parts of the fundamental solution.
  Using the fact that for $d \geq 5$ or $d = 3$, we have
  \begin{align*}
    \frac{\partial^2}{\partial x_\alpha \partial x_\beta} \bigg( \frac{1}{|x|^{d - 4}} \bigg)
    = -(d - 4) \frac{\partial}{\partial x_\alpha} \frac{x_\beta}{|x|^{d - 2}}
    = -(d - 4) \frac{\delta_{\alpha\beta}}{|x|^{d - 2}} + \frac{(d - 4)(d - 2) x_\alpha x_\beta}{|x|^d}.
  \end{align*}
  This allows us to write
  \begin{align*}
    \frac{x_\alpha x_\beta}{|x|^d} = \frac{\delta_{\alpha \beta}}{(d - 2)|x|^{d - 2}} + \frac{1}{(d - 4)(d - 2)} \frac{\partial^2}{\partial x_\alpha \partial x_\beta} \bigg( \frac{1}{|x|^{d - 4}} \bigg),
  \end{align*}
  which, considering definition \eqref{eq:fundamentalSolutionStokes}, gives 
  \begin{align}
    \label{eq:fundamentalSolutionStokes35}
    \Gamma_{\alpha\beta}(x; 0) &= G(x; 0)\delta_{\alpha\beta} + \frac{1}{2\omega_d(d - 4)(d - 2)} \frac{\partial^2}{\partial x_\alpha \partial x_\beta} \bigg( \frac{1}{|x|^{d - 4}} \bigg).
  \intertext{A similar trick works for $d = 4$:
  Since $\omega_4 = 2 \pi^2$, we have}
    \Gamma_{\alpha \beta}(x; 0)
    &= \frac{1}{2 \omega_4} \frac{1}{|x|^2} \delta_{\alpha\beta} - \frac{1}{8\pi^2} \bigg( \frac{\delta_{\alpha\beta}}{|x|^2} - \frac{2 x_\alpha x_\beta}{|x|^4} \bigg) \nonumber\\
    &= G(x; 0) \delta_{\alpha\beta} - \frac{1}{8\pi^2} \frac{\partial^2}{\partial x_\alpha \partial x_\beta}( \log(|x|)) \nonumber \\
    \label{eq:fundamentalSolutionStokes4}
    &= G(x; 0) \delta_{\alpha\beta} - \frac{1}{4 \omega_4} \frac{\partial^2}{\partial x_\alpha \partial x_\beta}( \log(|x|))
  \intertext{In the case $d = 2$, we use
    $$
  \frac{1}{8\pi} \frac{\partial^2}{\partial_\alpha \partial_\beta} (|x|^2 \log(|x|))
  = \frac{\delta_{\alpha\beta}}{4\pi} \log(|x|) + \frac{1}{4\pi} \frac{x_\alpha x_\beta}{|x|^2} + \frac{\delta_{\alpha\beta}}{8\pi}
$$
to find the identity}
    \label{eq:fundamentalSolutionStokes2}
    \Gamma_{\alpha \beta}(x; 0) &= G(x; 0)\delta_{\alpha\beta} - \frac{\delta_{\alpha\beta}}{8\pi} - \frac{\partial^2}{\partial x_\alpha \partial x_\beta} (|x|^2 \log(|x|)).
  \end{align}
  This ends the preparatory step and brings us to the next theorem.
  
\begin{thm}
  \label{thm:differenceFundamentalSolutionStokes}
  Let $\lambda \in \Sigma_\theta$.
  Suppose that $|\lambda| |x|^2 \leq ({1}/{2})$.
  Then
  \begin{align}
    \big|\nabla_x \{ \Gamma(x; \lambda) - \Gamma(x; 0) \} \big|
    \leq 
    \begin{cases}
      C |\lambda| |x|^{3 - d} &\quad\text{if } d \geq 7 \text{ or } d = 5, \\
      C |\lambda| |x|^{3 - d} \, \big| \log(|\lambda| |x|^2)\, \big| &\quad\text{if } d = 4 \text{ or } 6, \\
      C \sqrt{|\lambda|} |x|^{-1} &\quad\text{if } d = 3,\\
      %C |\lambda| |x| ( |\log(|\lambda| |x|^2)| + 1 ) &\quad\text{if } d = 2,
      C |\lambda| |x| \, \big|\log(|\lambda| |x|^2)\,\big|  &\quad\text{if } d = 2,
    \end{cases}
  \end{align}
  where $C$ depends only on $d$ and $\theta$.
\end{thm}

\begin{proof}
  We will split the proof in several parts.
  According to the preparatory step, for $d \geq 2$ and all $\alpha, \beta = 1,\dots,d$, the difference $\partial_\gamma\{\Gamma_{\alpha\beta}(x; \lambda) - \Gamma_{\alpha\beta}(x; 0)\}, \gamma = 1,\dots,d$, is always of the form
  \begin{align*}
    &\frac{\partial}{\partial x_\gamma}\big\{\Gamma_{\alpha\beta}(x; \lambda) - \Gamma_{\alpha\beta}(x; 0)\big\} \\
    &    \qquad\quad= \frac{\partial}{\partial x_\gamma}\big\{ G(x; \lambda) - G(x; 0) \big\}\delta_{\alpha\beta} \
        - \frac{1}{\lambda} \frac{\partial^3}{\partial x_\gamma \partial x_\alpha \partial x_\beta} \bigg\{ G(x; \lambda) - G(x; 0) + \big[ \dots \big] \bigg\}.
  \end{align*}
  But the first term on the right hand side of the above expression is already under control thanks to Lemma \ref{lem:HelmholtzLaplaceDifference}.
  It thus suffices to estimate the second term on the right hand side.

  We start by considering the cases $d = 3$ and $d \geq 5$.
      Taking into account identity \eqref{eq:fundamentalSolutionStokes35}, we have for all $\alpha, \beta, \gamma = 1,\dots,d$:
      \begin{align*}
        %&\Gamma_{\alpha\beta}(x; \lambda) - \Gamma_{\alpha\beta}(x; 0) \\
        %&\frac{1}{\lambda} \frac{\partial^3}{\partial x_\gamma \partial x_\alpha \partial x_\beta} \bigg\{ G(x; \lambda) - G(x; 0) + \big[ \dots \big] \bigg\}\\
        & G(x; \lambda) - G(x; 0) + \big[ \dots \big]
        %= \frac{\partial}{\partial x_\gamma} \big\{ G(x; \lambda) - G(x; 0) \big\}\delta_{\alpha\beta}  \\
        %&\qquad\qquad\qquad = \frac{1}{\lambda} \frac{\partial^3}{\partial x_\gamma \partial x_\alpha \partial x_\beta} \bigg\{ G(x; \lambda) - G(x; 0) + \frac{\lambda}{2 \omega_d (d - 4) (d - 2) |x|^{d - 4}} \bigg\}.
        = \frac{1}{\lambda} \bigg\{ G(x; \lambda) - G(x; 0) + \frac{\lambda}{2 \omega_d (d - 4) (d - 2) |x|^{d - 4}} \bigg\}.
      \end{align*}
  %As the first term can already be estimated via Lemma \ref{lem:HelmholtzLaplaceDifference} in a satisfactory way, we will only be concerned about the second one.
  If $d = 3$, a direct calculation will then yield the desired result:
  We start by noting that $\omega_3 = 4 \pi$ gives
  \begin{align*}
    G(x; \lambda) - G(x; 0) - \frac{\lambda}{2\omega_3 {|x|}^{-1}}
    &= \frac{\e^{\ii k |x|}}{4\pi |x|} - \frac{1}{4 \pi |x|} - \frac{(\ii k)^2}{2 \omega_3 |x|^{-1}} \\
    &= \frac{1}{4 \pi |x|} \Big( \e^{\ii k |x|} - 1 - \frac{ (\ii k)^2 |x|^2}{2} \Big)\\
    &= \frac{1}{4 \pi |x|} \Big( \ii k |x| + \sum_{n = 3}^\infty \frac{(\ii k |x| )^n}{n!} \Big) \\
    &= \frac{1}{4\pi} \Big( \ii k + \sum_{n = 3}^\infty \frac{ (\ii k )^n |x|^{n - 1}}{n!} \Big).
  \end{align*}
  Taking the first derivative of this expression we get
  \begin{align*}
    \frac{\partial}{\partial x_\beta} \Big\{ \dots \Big\} = 
    \frac{x_\beta}{4 \pi} \sum_{n = 3}^\infty \frac{(ik)^n (n - 1) }{n!} |x|^{n - 3}
  \end{align*}
  and differentiating with respect to $x_\alpha$ yields
  \begin{align*}
    \frac{\partial}{\partial x_\alpha} \Big\{ \dots  \Big\}
    = \frac{\delta_{\alpha\beta}}{4 \pi} \sum_{n = 3}^\infty \frac{(ik)^n (n - 1) }{n!} |x|^{n - 3}
    + \frac{x_\beta x_\alpha}{4\pi} \sum_{n = 4}^\infty \frac{(\ii k)^n (n - 1) (n - 3)}{n!} |x|^{n - 5}.
  \end{align*}
  As we are interested in estimating the \emph{gradient} of the difference of $\Gamma(x; \lambda)$ and $\Gamma(x; 0)$, we have to consider one additional derivative. This leaves us with
\begin{align*}
  \frac{\partial}{\partial x_\gamma} \Big\{ \dots \Big\}
    &= \frac{\delta_{\alpha\beta} x_\gamma + \delta_{\beta\gamma} x_\alpha + \delta_{\alpha\gamma} x_\beta}{4\pi} \sum_{n = 4}^\infty \frac{(\ii k)^n (n - 1) (n - 3)}{n!}  |x|^{n - 5} \\
    &\qquad + \frac{x_\beta x_\alpha x_\gamma}{4\pi} \sum_{\substack{n = 4}}^\infty \frac{(\ii k)^n (n - 1) (n - 3) (n - 5)}{n!} |x|^{n - 7}.
\end{align*}
We can now prove the desired estimate via
\begin{align*}
  \bigg| \frac{1}{\lambda} \frac{\partial^3}{\partial x_\gamma \partial x_\alpha \partial x_\beta} \Big\{ \dots \Big\}  \bigg|
  &\leq \frac{1}{|k|^2\pi}  \sum_{n = 4}^\infty \frac{|k|^n (n - 1) (n - 3)(1 + (n - 5))}{n!} |x|^{n - 4} \\
  &\leq \frac{1}{|k|^2 |x|\pi} |k|^3 \sum_{k = 4}^\infty \frac{(n - 1)(n - 3)(1 + (n - 5))}{n!} |k|^{n - 3} |x|^{n - 3}\\
  &\leq C \frac{1}{|k| |x|}.
\end{align*}
This gives the claim for $d = 3$.
If $d \geq 5$, equation \eqref{eq:HelmholtzLaplaceDifference} gives
\begin{align}
  \label{eq:secondTerm}
  \begin{alignedat}{1}
  &G(x; \lambda)- G(x; 0) + \frac{\lambda}{2 \omega_2 (d - 4) (d - 2) |x|^{d - 4}} \\
  &\hspace{4cm}= \frac{\ii}{4 (2\pi)^{\frac{d}{2} - 1}} \frac{1}{|x|^{d - 2}} \Big\{ z^{\frac{d}{2} - 1} H_{\frac{d}{2} - 1}^{(1)}(z) - a_d - b_d z^2\Big\},
  \end{alignedat}
\end{align}
where $z = k|x|$, $a_d$ was calculated in \eqref{eq:Defnad} and $b_d$ is given by
\begin{align*}
 b_d 
 &= -\frac{2\ii (2\pi)^{\frac{d}{2} - 1}}{\omega_d (d - 4) (d - 2)}.
  \end{align*}
    Using relation \eqref{eq:wd} and the functional equation of the Gamma function \eqref{eq:functionalGamma} twice, we see that
    \begin{align*}
      b_d &= -\frac{2\ii (2\pi)^{\frac{d}{2} - 1} \Gamma(\frac{d}{2})}{2\pi^{\frac{d}{2}} (d - 2)(d - 4)}
 = \frac{2^{\frac{d}{2} - 1} }{\pi \ii (d - 4)} \frac{\Gamma(\frac{d}{2})}{(d - 2)} 
 = \frac{2^{\frac{d}{2} - 1} }{2\pi \ii } \frac{\Gamma(\frac{d}{2} - 1)}{(d - 4)} \\
      &= \frac{2^{\frac{d}{2} - 1}}{4 \pi \ii } \frac{\Gamma(\frac{d}{2} - 1)}{(\frac{d}{2} - 1 - 1)}
      = \frac{2^{\frac{d}{2} - 1} \Gamma(\frac{d}{2} - 1 - 1)}{4 \pi \ii } = \frac{2^{\nu_d} \Gamma(\nu_d - 1)}{4 \pi \ii}.
\end{align*}
This shows that for $d \geq 5$, $b_d$ is the second coefficient of the asymptotic expansions \eqref{eq:asymptoticd5}-\eqref{eq:asymptoticd7}, respectively. 
Now we split the proof for $d \geq 5$ into (1) $d \geq 7$, (2) $d = 6$ and (3) $d = 5$.
If $d \geq 7$, we use the asymptotic expansions \eqref{eq:asymptoticd7} to estimate the part of \eqref{eq:secondTerm} which involves the Hankel function as
\begin{align}
  \label{eq:estimateDerivativesd7}
  \bigg| \frac{\mathrm{d}^l}{\d z^l} \Big\{ z^{\frac{d}{2} - 1} H_{\frac{d}{2} - 1}^{(1)}(z) - a_d - b_d z^2 \Big\} \bigg| \leq C |z|^{4 - l}
\end{align}
for $0 \leq l \leq 3$ and $z \in \C \setminus (-\infty, 0]$.
For better readability we define the function
\begin{align*}
  g(z) \coloneqq z^{\frac{d}{2} - 1} H_{\frac{d}{2} - 1}^{(1)}(z) - a_d - b_d z^2
\end{align*}
and consider the function $f(x) \coloneqq g(k |x|)$ on $\R^d \setminus \{0\}$.
The derivatives of $f$ read
\begin{align*}
  \frac{\partial}{\partial x_\beta} f(x)
  &= \Big(\frac{\d{}}{\d z} g\Big)(k|x|) \cdot \frac{k x_\beta}{|x|}, \\
  %
  \frac{\partial^2}{\partial x_\alpha \partial x_\beta} f(x)
  &= \Big(\frac{\mathrm{d}^2}{\d z^2} g\Big)(k|x|) \cdot \frac{k^2 x_\alpha x_\beta}{|x|^2} 
  + \Big(\frac{\d{}}{\d z} g\Big)(k |x|) \cdot k \bigg\{ \frac{\delta_{\alpha\beta}}{|x|} - \frac{x_\beta x_\alpha}{|x|^3} \bigg\}, \\
  %
  \frac{\partial^3}{\partial x_\gamma \partial x_\alpha \partial x_\beta} f(x)
  &= \Big(\frac{\mathrm{d}^3}{\d z^3}g\Big)(k |x|) \cdot \frac{k^3 x_\alpha x_\beta x_\gamma}{|x|^3} \\
  &\quad + \Big(\frac{\mathrm{d}^2}{\d z^2}g\Big)(k|x|) 
  \cdot k^2 \bigg\{ \frac{x_\alpha \delta_{\beta\gamma} + x_\beta \delta_{\alpha\gamma} + x_\gamma \delta_{\alpha\beta}}{|x|^2} 
  - \frac{x_\alpha x_\beta x_\gamma}{|x|^4} \bigg\} \\
  &\quad + \Big(\frac{\d{}}{\d z}g\Big)(k|x|) \cdot k \bigg\{-\frac{ x_\gamma\delta_{\alpha\beta} + x_\alpha \delta_{\beta\gamma} + x_\beta \delta_{\alpha\gamma}}{|x|^3} + \frac{3 x_\alpha x_\beta x_\gamma}{|x|^5} \bigg\}.
\end{align*}
If we now look for estimates on the absolute value of the derivatives, we see that by \eqref{eq:estimateDerivativesd7}
\begin{align*}
% |\nabla_x f(x) | 
% &\leq C |k|^{4 - 1 + 1} |x|^{4 - 1} = C |k|^{4} |x|^{3}\\
% |\nabla_x^2 f(x) | 
% &\leq C \{ |k|^{4 - 2 + 2} |x|^{4 - 2} + |k|^{4 - 1 + 1} |x|^{4 -1 -1} \}  = C |k|^4 |x|^2 \\
% |\nabla^3_x f(x) | 
% &\leq C \{ |k|^{4 - 3 + 3} |x|^{4 - 3} + |k|^{4 - 2 + 2} |x|^{4 - 2 -1} + |k|^{4 - 1 + 1} |x|^{4 - 1 - 2}
  \big|\nabla^l f(x)\big| \leq C |k|^4 |x|^{4 - l}, \quad 1 \leq l \leq 3,
\end{align*}
where $C$ only depends on $l$.
We can now finally uncover the desired estimate via
\begin{align*}
  &\bigg|\frac{1}{\lambda} \nabla_x^3 \bigg\{ \frac{\ii}{4 (2\pi)^{\frac{d}{2} - 1}} \frac{1}{|x|^{d - 2}} \Big\{ z^{\frac{d}{2} - 1} H_{\frac{d}{2} - 1}^{(1)}(z) - a_d - b_d z^2\Big\} \bigg\} \bigg| \\
  &\quad\leq C \frac{1}{|k|^2} \sum_{l = 0}^3 \Big| \nabla^{3 - l}\Big(\frac{1}{|x|^{d - 2}}\Big)\Big|\; \big| \nabla^l f(x) \big| 
  \leq C \sum_{l = 0}^3 |x|^{-d + 2 - 3 + l} |k|^2 |x|^{4 - l} = C |\lambda| |x|^{3 - d},
\end{align*}
where $C$ is a constant only depending on $d$.

If $d = 6$, the asymptotic expansion \eqref{eq:asymptoticd6} gives us 
in analogy to \eqref{eq:estimateDerivativesd7} the estimate
\begin{align}
  \label{eq:estimateDerivativesd6}
  \bigg| \frac{\mathrm{d}^l}{\d z^l} \Big\{ z^{\frac{d}{2} - 1} H_{\frac{d}{2} - 1}^{(1)}(z) - a_d - b_d z^2 \Big\} \bigg| \leq C |z|^{4 - l} \, \big| \log(z)\, \big|\, ,
\end{align}
for $0 \leq l \leq 3$ and $z \in \C \setminus (-\infty, 0]$.
Using as before the expressions for the derivatives of $f$, we estimate their absolute values as
\begin{align*}
  \big|\nabla^l f(x) \big| \leq C |k|^4 |x|^{4 - l}\, \big|\log(|\lambda| |x|^2)\, \big|\, , 
\end{align*}
which, by a calculation analogous to the case $d \geq 7$, yields the claim.
%\begin{align*}
%  \big| \nabla_x \{ \Gamma(x; \lambda) - \Gamma(x; 0) \}\big| \leq C |\lambda| |x|^{3 - d} \, \big| \log(|\lambda| |x|^2)\, \big|\,.
%\end{align*}

For $d = 5$, we differentiate \eqref{eq:secondTerm} twice and use relation \eqref{eq:fundamentalSolutionStokes35} for the fundamental solution of the Stokes problem to  write
\begin{align*}
  &\frac{\partial^2}{\partial x_\alpha \partial x_\beta} \bigg\{ G(x; \lambda) - G(x; 0) + \frac{\lambda}{6\, \omega_5 |x|^{d - 4}} \bigg\} \\
  &\hspace{3cm}= \frac{\partial^2}{\partial x_\alpha x_\beta} \bigg\{ \frac{\ii}{4(2\pi)^{\frac{3}{2}}} \cdot \frac{1}{|x|^3} \Big[ z^{\frac{3}{2}} H_{\frac{3}{2}}^{(1)}(z) - a_5 - b_5 z^2 - w z^3 \Big] \bigg\},
\end{align*}
where $w\in\C$ can be an arbitrary constant if we set $z = k|x|$.
Now, for the appropriate choice of $w \in \C$ the asymptotic expansion \eqref{eq:asymptoticd5} gives the same estimate as \eqref{eq:estimateDerivativesd7} which, like for $d \geq 7$, proves the claim for $d = 5$.

In the case $d = 4$ we use the respective relation for the fundamental solution \eqref{eq:fundamentalSolutionStokes4} in order to simplify the difference
\begin{align*}
  %&\Gamma_{\alpha\beta}(x; \lambda) - \Gamma_{\alpha\beta}(x; 0) \\
  %&\quad= \big\{ G(x; \lambda) - G(x; 0) \big\}\delta_{\alpha\beta}  
  & \frac{\partial^2}{\partial x_\alpha \partial x_\beta} \bigg\{ G(x; \lambda) - G(x; 0) - \frac{\lambda \log(|x|)}{4\, \omega_4} \bigg\} \\
  &\qquad\qquad\qquad= \frac{\partial^2}{\partial x_\alpha \partial x_\beta} \bigg\{ \frac{\ii}{8 \pi |x|^2} \Big[ z H_1^{(1)}(z) - a_4 - w z^2 - b_4 z^2 \log(z)\,\Big] \bigg\},
\end{align*}
where $z = k |x|$, $b_4 = ({\ii}/{\pi})$ and $w \in \C$ is an arbitrary constant. Using the asymptotic expansion \eqref{eq:asymptoticd4} and the appropriate constant $w \in \C$ we get the estimate
\begin{align*}
  \bigg| \frac{\mathrm{d}^l}{\d z^l} \Big\{ z H_1^{(1)}(z) - a_4 - w z^2 - b_4 z^2 \log(z)\, \Big\} \bigg| \leq C |z|^{4 - l}\, \big|\log(z)\,\big|\,.
\end{align*}
The estimate has the same right hand side as \eqref{eq:estimateDerivativesd6} and the proof can be carried out just as in the previous cases.

For $d = 2$, the claimed estimate follows from a direct calculation which is postponed until the appendix of this thesis.
\end{proof}

We can now use the assumption $|\lambda||x|^2 \leq ({1}/{2})$  to unify the structure of the estimates from Theorem \ref{thm:differenceFundamentalSolutionStokes}.
\begin{cor}
  \label{cor:differenceFundamentalSolutionStokes}
  Let $\lambda \in \Sigma_\theta$. Suppose that $|\lambda||x|^2 \leq ({1}/{2})$.
  Then for all $d \geq 2$
  \begin{align*}
    \big|\nabla_x\{ \Gamma(x; \lambda) - \Gamma(x; 0)\} \big| \leq C \sqrt{|\lambda|} |x|^{2 - d},
  \end{align*}
  where $C$ depends only on $d$ and $\theta$.
\end{cor}

\begin{proof}
  We just extend the estimates given in Theorem \ref{thm:differenceFundamentalSolutionStokes}.
  Let $d \geq 7$ or $d = 5$. Since $\sqrt{|\lambda|} \leq C{|x|}^{-1}$, we have
  \begin{align*}
    C |\lambda| |x|^{3 - d} \leq C \sqrt{|\lambda|} |x|^{2 - d}.
  \end{align*}
  For $d = 2,4,6$, we have
  \begin{align*}
    |\lambda| |x|^{3 - d} \, \big| \log(|\lambda||x|^2)\,\big|
    = C \sqrt{|\lambda|} |x|^{2 - d} \cdot \sqrt{|\lambda|} |x|\, \big| \log(|\lambda| |x|^2)\, \big|
    \leq C \sqrt{|\lambda|} |x|^{2 - d},
  \end{align*}
  since $\sqrt{|\lambda|} |x| \,\big|\log(|\lambda| |x|^2)\,\big|$ is bounded for $|\lambda| |x|^2 \leq ({1}/{2})$.
  %For $d = 2$, the same argument applies to the expression $\sqrt{|\lambda|} |x| (|\log(|\lambda| |x|^2)| + 1)$.
\end{proof}

