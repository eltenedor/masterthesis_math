\chapter{Rellich Estimates}

In this section we will establish Rellich type estimates for the Stokes resolvent problem.
We will for this entire section always assume that $\Omega$ is a bounded Lipschitz domain in $\R^d$, $d \geq 2$ with connected boundary and $|\partial\Omega| = 1$.
Furthermore we will use the shorthand notation
\begin{align*}
  \| \cdot \|_{\partial} \coloneqq \| \cdot \|_{\Ell^2(\partial\Omega)}.
\end{align*}

The goal of this section is to derive an Rellich type inequality which will be used to prove the invertibility of the operators $\pm(1/2)I + \K_{\lambda}$. 
This inequality is part of the following theorem.

\begin{thm}
  \label{thm:rellich}
  Let $\lambda \in \Sigma_\theta$ and $|\lambda| \geq \tau$, where 
  $\tau \in (0,1)$.
  Let $(u,\phi)$ be a solution to the Stokes resolvent problem in $\Omega$ and suppose that $(\nabla u)^* \in \Ell^2(\partial\Omega)$ and $(\phi)^* \in \Ell^(\partial\Omega)$.
  Furthermore, assume that $\nabla u$, $\phi$ have nontangential limits almost everywhere on $\partial\Omega$.
  Then
  \begin{align}
    \label{eq:rellich1}
    \|\nabla u\|_\partial + \|\phi - \dashint_{\partial\Omega} \phi \|_\partial
    \leq C \big\{ \|\nabla_{\mathrm{tan}} u\|_\partial + |\lambda|^{1/2} \|u\|_\partial + |\lambda| \|u \cdot n\|_{\HH^{-1}(\partial\Omega)} \big\}
  \end{align}
  and
  \begin{align}
    \label{eq:rellich2}
    \|\nabla u\|_\partial + |\lambda|^{1/2} \|u\|_\partial + |\lambda| \|u \cdot n\|_{\HH^{-1}(\partial\Omega)} + \|\phi\|_\partial
    \leq C \|\frac{\partial u}{\partial \nu} \|_\partial,
  \end{align}
  where $\frac{\partial}{\partial \nu}$ denotes the conormal derivative, and $C$ depends only on $d$, $\tau$, $\theta$ and the Lipschitz character of $\Omega$.
\end{thm}

We will now prepare the proof of this theorem by proving several helpful lemmata.

\begin{lem}
  \label{lem:rellichIdentity}
  Under the same conditions on $(u,\phi)$ as in Theorem \ref{thm:rellich}, we have
  \begin{align}
    \int_{\partial\Omega} h_k n_k |\nabla u|^2 \d\sigma 
    &= 2 \Ret \int_{\partial\Omega} h_k \frac{\partial \bar u_i}{\partial x_k} \Big( \frac{\partial u}{\partial \nu} \Big)_i \d\sigma + \int_\Omega \div(h) |\nabla u|^2 \d x \nonumber\\
    &\quad - 2 \Ret \int_\Omega \frac{\partial h_k}{\partial x_j} \cdot \frac{\partial u_i}{\partial x_k} \cdot \frac{\partial \bar u_i}{\partial x_j} \d x + 2 \Ret \int_\Omega \frac{\partial h_k}{\partial x_i} \cdot \frac{\partial u_i}{\partial x_k} \bar \phi \d x \nonumber\\
    &\quad - 2 \Ret \int_\Omega h_k \frac{\partial u_i}{\partial x_k} \cdot \bar \lambda \bar u_i \d x \label{eq:rellichIdentity}
  \end{align}
  and
  \begin{align}
    \int_{\partial\Omega} h_k n_k |\nabla u|^2 \d\sigma
    &= 2\Ret \int_{\partial\Omega} h_k \frac{\partial \bar u_i}{\partial x_j} \Big\{ n_k \frac{\partial u_i}{\partial x_j} - n_j \frac{\partial u_i}{\partial x_k} \Big\} \d\sigma \nonumber\\
    &\quad + 2 \Ret \int_{\partial\Omega} h_k \bar \phi \Big\{ n_i \frac{\partial u_i}{\partial x_k} - n_k \frac{\partial u_i}{\partial x_i} \Big\} \d\sigma - \int_\Omega \div(h) |\nabla u|^2 \d x \nonumber\\
    &\quad + 2 \Ret \int_\Omega \frac{\partial h_k}{\partial x_j} \cdot \frac{\partial u_i}{\partial x_k} \cdot \frac{\partial \bar u_i}{\partial x_j} \d x - 2 \Ret \int_\Omega \frac{\partial h_k}{\partial x_i} \cdot \frac{\partial u_i}{\partial x_k} \bar \phi \d x \nonumber\\
    &\quad + 2 \Ret \int_\Omega h_k \frac{\partial u_i}{\partial x_k} \cdot \bar \lambda \bar u_i \d x, \label{eq:rellichIdentity2}
  \end{align}
  where $h = (h_1, \dots,h_d) \in \CC_0^1(\R^d, \R^d)$.
\end{lem}

\begin{proof}
  The proof of the stated identities reduces to several integrations by part  once we establish its applicability. 
  Approximating $\partial \Omega$ by a sequence of $\CC^\infty$ domains with uniform Lipschitz characters as described in Verchota \cite{verchotaDiss} and the facts that on the one hand $(\nabla u)^*, (\phi)^* \in \Ell^2(\partial\Omega)$ and on the other hand $\nabla u$ and $\phi$ have nontantential limits almost everywhere, the integration by parts is justified.
  Details on how the approximation argument works can be found in Brown \ref{brown}.
  For the sake of completeness we show the approximation argument once for a solid integral and once for a boundary integral.

  Let $(\Omega_l)_{l \in \N}$ denote the approximating sequence of $\CC^\infty$ domains with outer normal $n^{(l)}$.
  Then
  \begin{align*}
    \int_{\partial\Omega_l} h_k n_k^{(l)} |\nabla u|^2 \d \sigma_l
    = \int_{\partial \Omega} w_l(x) h_k(\Lambda_l(x)) n_k^{(l)}(\Lambda_l(x)) |\nabla u|^2(\Lambda_l(x)) \d\sigma
  \end{align*}
  Now we know that $\lim_{l \to \infty} w_l(x) = 1$ and $\lim_{l \to \infty} \Lambda_l(x) = x$ almost everywhere and $\Lambda_l(x) \in \Gamma(x)$ for all $l \in \N$.
  Furthermore we know that $\lim_{l \to \infty} n_k^{(l)} = n_k$ almost everywhere and that $\nabla u$ has a nontangential limit almost everywhere.
  This gives us that the integrand converges almost everywhere to $ h_k(x) n_k(x) |\nabla u|^2(x)$.
  Now furthermore we have that the integrand is dominated by $\delta \|h\|_\infty ((\nabla u)^*)^2$, where $\delta$ is a uniform bound to $w_l$. Since by assumption $(\nabla u)^* \in \Ell^2(\partial\Omega)$, the dominated convergence theorem gives us
  \begin{align*}
    \lim_{l \to \infty} \int_{\partial \Omega} w_l(x) h_k(\Lambda_l(x)) n_k^{(l)}(\Lambda_l(x)) |\nabla u|^2(\Lambda_l(x)) \d\sigma 
    &= 
    \int_{\partial\Omega} h_k n_k |\nabla u|^2 \d \sigma.
  \end{align*}
  It is easy to check, that the only differences in the approximation argument when applied to the other boundary integrals lie in the choice of the majorant.
  Now consider for instance the solid integral

  We now start a formal calculation on $\Omega$ and $\partial\Omega$ keeping in mind that the stated equalities sometimes hold only after the application of the former approximation argument.

  Let's expand the first summand in \eqref{eq:rellichIdentity} using the definition of conormal derivatives
  \begin{align*}
    2\Ret \int_{\partial\Omega} h_k \frac{\partial \bar u_i}{\partial x_k} \Big( \frac{\partial u}{\partial \nu} \Big)_i \d\sigma
    &= 2 \Ret \int_{\partial\Omega} h_k \frac{\partial \bar u_i}{\partial x_k} \big( \frac{\partial u_i}{\partial x_j} \big) n_j \d\sigma  - 2\Ret \int_\Omega h_k \frac{\partial \bar u_i}{\partial x_k} \phi n_i \d x \\
    &\eqqcolon I_1 - I_2.
  \end{align*}
  For $I_1$ we find using the divergence theorem
  \begin{align*}
    I_1
    &= 2\Ret \int_\Omega \frac{\partial}{\partial x_j} \big( h_k \frac{\partial}{\partial x_j} \frac{\partial \bar u_i}{\partial x_k} \frac{\partial u_i}{\partial x_j} \big) \d x \\
    &= 2\Ret \int_\Omega \frac{\partial h_k}{\partial x_j} \frac{\partial \bar u_i}{\partial x_k} \frac{\partial u_i}{\partial x_j} + h_k \frac{\partial^2 \bar u_i}{\partial x_j \partial x_k} \frac{\partial u_i}{\partial x_j} + h_k \frac{\partial \bar u_i}{\partial x_k} \frac{\partial^2 u_i}{\partial x_j^2} \d x \\
    &\eqqcolon I_3 + I_4 + I_5.
  \end{align*}
  For $I_5$ we use the fact that $u$ solves the Stokes resolvent problem which gives
  \begin{align*}
     I_5 
     &= 2 \Ret \int_\Omega h_k \frac{\partial \bar u_i}{\partial x_k} \frac{\partial \phi}{\partial x_i} \d x + 2\Ret \int_\Omega h_k \frac{\partial\bar u_i}{\partial x_k} \lambda u_i \d x \\
     &\eqqcolon I_6 + I_7 .
  \end{align*}
  Another application of the divergence theorem gives
  \begin{align*}
    I_2
    &= 2 \Ret \int_\Omega \frac{\partial}{\partial x_i} \big( h_k \frac{\partial \bar u_i}{\partial x_k} \phi \big) \d x \\
    &= 2 \Ret \int_\Omega \frac{\partial h_k}{\partial x_i} \frac{\partial \bar u_i}{\partial x_k} \phi + h_k \frac{\partial^2 \bar u_i}{\partial x_j \partial x_k} \phi + h_k \frac{\partial \bar u_i}{\partial x_k} \frac{\partial \phi}{\partial x_i} \d x\\
    &\eqqcolon I_8 + I_9 + I_6.
  \end{align*}
  One term that hasn't come up so far, the second summand of the right side in \eqref{eq:rellichIdentity}, will now be expanded
  \begin{align*}
    \div(h) |\nabla u|^2 \d x
    &= \int_\Omega \div( h|\nabla u|^2) \d x - \int_\Omega h_k \frac{\partial}{\partial x_i} |\nabla u|^2 \d x \\
    &\eqqcolon I_{10} - I_{11}.
  \end{align*}
  Expanding this further gives us
  \begin{align*}
    I_{11}
    &= \int_\Omega h_i \frac{\partial}{\partial x_i} \big( \frac{\partial u_k}{\partial x_j} \frac{\partial \bar u_k}{\partial x_j} \big) \d x
    = \int_\Omega h_i  \frac{\partial^2 u_k}{\partial x_i \partial x_j} \frac{\partial \bar u_k}{\partial x_j} + \frac{\partial u_k}{\partial x_j} \frac{\partial^2 \bar u_k}{\partial x_i \partial x_j} = I_4.
  \end{align*}
  If we now put everything together, the right side of \eqref{eq:rellichIdentity} reads
  \begin{align*}
    &(I_1 - I_2) + (I_{10} - I_{11}) - I_3 + I_8 - I_7 \\
    &\quad= (I_3 + I_4 + I_6 + I_7) - (I_8 + I_9 + I_6) + I_{10} - I_{11} - I_3 + I_8 - I_7 = I_{10}.
  \end{align*}
  Noting that by the divergence theorem we have
  \begin{align*}
    I_{10} = \int_{\partial\Omega} h_k n_k |\nabla u|^2 \d \sigma,
  \end{align*}
  the first identity is proven.

  In order to prove identity \eqref{eq:rellichIdentity2}, we show that the expression we get from considering (\eqref{eq:rellichIdentity} + \eqref{eq:rellichIdentity2}) holds, i.e. we show the identity
  \begin{align*}
    2 \int_{\partial\Omega} h_k n_k |\nabla u|^2 \d\sigma
    &= 2 \Ret \int_{\partial\Omega} h_k \frac{\partial \bar u_i}{\partial x_k} \big( \frac{\partial u}{\partial \nu} \big)_i \\
    &\quad + 2 \Ret \int_{\partial\Omega} h_k \frac{\partial\bar u_i}{\partial x_j} \Big\{ n_k \frac{\partial u_i}{\partial x_j} - n_j \frac{\partial u_i}{\partial x_k} \Big\} \d \sigma \\
    &\quad+ 2 \Ret \int_{\partial\Omega} h_k \bar \phi \Big\{ n_i \frac{\partial u_i}{\partial x_i} - n_k \frac{\partial u_i}{\partial x_i} \Big \} \d\sigma.
  \end{align*}
  To this end, note that the left side of the identity equals $2 I_10$, whereas the right side can be written as 
  \begin{align*}
    I_1 - I_2 + 2 I_{10} - I_1 + I_2 - 0,
  \end{align*}
  where we used the fact that $\div u = 0$.
\end{proof}

We note that the operators
\begin{align*}
  \Big\{ n_k \frac{\partial u_i}{\partial x_j} - n_j \frac{\partial u_i}{\partial x_k} \Big\}
  \quad\text{and}\quad
  \Big\{ n_i \frac{\partial u_i}{\partial x_k} - n_k \frac{\partial u_i}{\partial x_i} \Big\}
\end{align*}
are the \emph{first-order tangential derivative operators} which can be found in Mitrea and Wrigth \cite{mitreaWright}.

We make a quick detour that gives us the following lemma.
\begin{lem}
  \label{lem:lambdaIneq}
  Let $\theta \in (0,\pi/2)$.
  Then there exists $\alpha$ depending only on $\theta$ such that
  \begin{align*}
    \Re(\lambda) + \alpha |\Im(\lambda)| \geq |\lambda|
  \end{align*}
  for all $\lambda \in \Sigma_\theta$.
\end{lem}

\begin{proof}
  For $|\lambda| = 1$ we have $\Re(\lambda) = \cos(\varphi)$ and $\Im(\lambda) = \sin(\varphi)$ for some $\varphi \in (0,\pi - \theta)$.
  Set
  \begin{align*}
    \alpha = \frac{1 - \cos(\pi - \theta)}{\sin(\pi - \theta)} \geq \frac{1 - \cos(\varphi)}{\sin(\varphi)}.
  \end{align*}
  Then we have
  \begin{align*}
    \Re(\lambda) + \alpha |\Im(\lambda)| = \cos(\varphi) + \alpha \sin(\varphi) \geq 1.
  \end{align*}
  For arbitrary $\lambda$ the claim follows by considering the normalized value $\lambda / |\lambda|$.
\end{proof}

The next lemma enables us to handle the solid integrals in \eqref{eq:rellichIdentity} and \eqref{eq:rellichIdentity2}.

\begin{lem}
  \label{lem:laxMilgramIneq}
  Under the same assumptions on $(u,\phi)$ and $\lambda$ as in Theorem \ref{thm:rellich}, we have
  \begin{align}
    \label{eq:laxMilgramIneq}
    \int_\Omega |\nabla u|^2 \d x + |\lambda| \int_\Omega |u|^2 \leq C \| \frac{\partial u}{\partial \nu} \|_\partial \|u\|_\partial,
  \end{align}
  where $C$ depends only on $\theta$.
\end{lem}

\begin{proof}
  Testing the Stokes resolvent problem against the solution $u$ gives us
  \begin{align*}
    \int_\Omega -\Delta u \cdot \bar u \d x + \lambda \int_\Omega u \cdot \bar u \d x= \int_\Omega -\nabla \phi \cdot \bar u \d x
  \end{align*}
  Using integration by parts which may, as in the proof of the previous lemma, be justified by an approximation argument, we get
  \begin{align*}
    \int_\Omega |\nabla u|^2 \d x - \int_{\partial\Omega} \frac{\partial u}{\partial n} \cdot \bar u \d \sigma + \lambda \int_\Omega |u|^2 \d x = \int_{\partial\Omega}  \Phi n \cdot \bar u \d \sigma
  \end{align*}
  or with the definition of conormal derivatives
  \begin{align}
    \label{eq:testedStokes}
    \int_\Omega |\nabla u|^2 \d x + \lambda \int_\Omega |u|^2 \d x = \int_{\partial\Omega} \frac{\partial u}{\partial \nu} \cdot \bar u \d \sigma.
  \end{align}
  If we now take the real and imaginary part of \eqref{eq:testedStokes} and sum them up with a prefactor $\alpha > 0$, we get
  \begin{align*}
    \int_\Omega |\nabla u|^2 \d x + \{ \Re(\lambda) + \alpha |\Im(\lambda)| \} \int_\Omega |u|^2 \d x
    \leq (1 + \alpha) \big| \int_{\partial\Omega} \frac{\partial u}{\partial \nu} \cdot \bar u \d \sigma \big|.
  \end{align*}
  Lemma \ref{lem:lambdaIneq} now gives
  \begin{align*}
    \int_\Omega |\nabla u|^2 \d x + |\lambda| \int_\Omega |u|^2 \d x \leq C \big| \int_{\partial\Omega} \frac{\partial u}{\partial \nu} \cdot \bar u \d \sigma \big|,
  \end{align*}
  from which we ge estimate \eqref{eq:laxMilgramIneq} after applying the Cauchy-Schwartz inequality.
\end{proof}

The next lemma combines Rellich identities \eqref{eq:rellichIdentity} and \eqref{eq:rellichIdentity2} with estimate \eqref{eq:laxMilgramIneq}.

\begin{lem}
  Under the same assumptions on $(u,\phi)$ and $\lambda$ as in Theorem \ref{thm:rellich}, we have
  \begin{align}
    \label{eq:gradEstimateRellich}
    \| \nabla u\|_\partial \leq C_\varepsilon \| \frac{\partial u}{\partial \nu} \|_\partial + \varepsilon \Big\{ \|\nabla u\|_\partial + \|\phi\|_\partial + \| |\lambda|^{1/2} u\|_\partial \Big\}
  \end{align}
  and
  \begin{align}
    \label{eq:gradEstimateRellich2}
    \|\nabla u\|_\partial \leq C_\varepsilon \Big\{ \|\nabla_{\mathrm{tan}} u \|_\partial + \||\lambda|^{1/2} u\|_\partial \Big\} + \varepsilon \{ \|\nabla u\|_\partial + \|\phi\|_\partial \}
  \end{align}
  for all $\varepsilon \in (0,1)$, where $C_\varepsilon$ depends only on $d$, $\theta$, $\tau$, $\varepsilon$ and the Lipschitz character of $\Omega$.
\end{lem}

\begin{proof}
  Let $h = (h_1, \dots, h_d) \in \CC_0^1(\R^d, \R^d)$ with $h_k n_k \geq c > 0$ on $\partial \Omega$. 
  The existence of this vector field follows from Verchota.
  Now in view of identity \eqref{rellichIdentity}, we have
  \begin{align}
    \label{eq:normRellich}
    \|\nabla u\|_\partial^2
    \leq C \Big\{ \|\nabla u\|_\partial \| \frac{\partial u}{\partial \nu} \|_\partial + \int_\Omega |\nabla u|^2 \d x + \int_\Omega |\nabla u| |\phi| \d x + |\lambda \int_\Omega |\nabla u| |u| \d x \Big\},
  \end{align}
  where the first term follows from the Cauchy-Schwartz inequality.
  Since $\Delta \phi = 0$ and the nontangential maximal function $(\phi)^* \in \Ell^2(\partial\Omega)$ a result from Dahlberg \cite{dahlberg} gives
  \begin{align}
    \label{eq:dahlbergEstimate}
    \int_\Omega |\phi|^2 \d x \leq C \|(\phi)^* \|_\partial^2 \leq C \|\phi\|_\partial^2.
  \end{align}
  The last summand of \eqref{eq:normRellich} can be estimated as follows
  \begin{align}
    |\lambda| \int_\Omega |\nabla u| |u| \d x 
    &\leq |\lambda| \big\{ \frac{|\lambda|^{1/2}}{2} \int_\Omega |u|^2 \d x + \frac{1}{2 |\lambda|^{1/2}} \int_\Omega |\nabla u|^2 \d x \nonumber\\\label{eq:lambdaNablaU}
    &\leq C \| \frac{\partial u}{\partial \nu} \|_\partial \| |\lambda|^{1/2} u\|_\partial, 
  \end{align}
  where in the first step we used the weigthed Young inequality and in the second step we applied estimate \eqref{eq:laxMilgramIneq}.
  Similarly we calculate
  \begin{align}
    \int_\Omega |\nabla u| |\phi| \d x
    \leq \Big( \int_\Omega |\nabla u|^2 \d x \Big)^{1/2} \Big( \int_\Omega |\phi|^2 \d x \Big)^{1/2}
    \leq C \, \| \frac{\partial u}{\partial \nu} \|_\partial^{1/2} \|u\|_\partial^{1/2} \|\phi\|_\partial, \label{eq:nablaPhi}
  \end{align}
  where the first step is just the Cauchy-Schwartz inequality and the second step combines estimate \eqref{eq:laxMildgramIneq} with estimate \eqref{eq:dahlbergEstimate}.
  Putting everything together, we calculate
  \begin{align*}
    \| \nabla u\|_\partial^2 
    &\leq C \| \nabla u\|_\partial \|\frac{\partial u}{\partial \nu} \|_\partial + C \| \frac{\partial u}{\partial \nu} \|_\partial \|u\|_\partial + C \| \frac{\partial u}{\partial \nu} \|_\partial^{1/2} \|u\|_\partial^{1/2} \|\phi\|_\partial \\
    &\quad+ C \| \frac{\partial u}{\partial \nu} \|_\partial \| |\lambda|^{1/2} u\|_\partial.
  \end{align*}
  Note that we used the fact that $|\lambda| \geq \tau$ to bound $\|u\|_\partial$ as
  \begin{align*}
    \|u\|_\partial \leq \frac{|\lambda|^{1/2}}{\tau^{1/2}} \|u\|_\partial = C |\lambda|^{1/2} \|u\|_{\partial}.
  \end{align*}
  The desired estimate \eqref{eq:gradEstimateRellich} now follows applying Young's weighted inequality and the norm equivalence on finite dimensional vector spaces.

  For inequality \eqref{eq:gradEstimateRellich2} we use the identity \eqref{eq:rellichIdentity2} and obtain
  \begin{align*}
    \| \nabla u\|_\partial^2 
    &\leq C \|\nabla_{\mathrm{tan}} u\|_\partial \{ \|\nabla u\|_\partial + \|\phi\|_\partial \} + C \int_\Omega |\nabla u|^2 \d x \\
    &\quad+ C \int_\Omega |\nabla u| |\phi| \d x + C |\lambda| \int_\Omega |\nabla u| |u|\d x.
  \end{align*}
  Using estimates \eqref{eq:laxMilgramIneq}, \eqref{eq:dahlbergEstimate}, \eqref{eq:lambdaNablaU} and \eqref{eq:nablaPhi} together with the weighted Young inequality gives us
  \begin{align*}
    \|\nabla u\|_\partial^2 
    &\leq C_\varepsilon \{ \|\nabla_{\mathrm{tan}} u\|_\partial^2 + \| |\lambda|^{1/2} u\|_\partial^2 \}  + \varepsilon \{ \|\nabla u\|_\partial^2 + \|\phi\|_\partial^2 + \frac{1}{4}\|\frac{\partial u}{\partial \nu} \|_\partial^2 \}.
  \end{align*}
  The claim now follows if we use the definition of the conormal derivative and the norm equivalence on finite dimensional vector spaces.
\end{proof}

We prove one last lemma befor we tackle the central theorem of this chapter.

\begin{lem}
  Assume that $(u,\phi)$ satisfies the same conditions as in Theorem \ref{thm:rellich}.
  Then
  \begin{align}
    \label{eq:phiDashintPhi}
    \| \phi - \dashint_{\partial\Omega} \phi \|_\partial \leq C \{ \|\nabla u\|_\partial + |\lambda| \|u\cdot n\|_{\HH^{-1}(\partial\Omega}
  \end{align}
  and
  \begin{align}
    \label{eq:lambdaun}
    |\lambda| \| u\cdot n \|_{\HH^{-1}(\partial\Omega)} \leq C \{ \|\phi\|_\partial + \|\nabla u\|_\partial,
  \end{align}
  where $C$ depends only on $d$ and the Lipschitz character of $\Omega$.
\end{lem}

\begin{proof}
  By Verchota's approximation argument \cite{verchotaDiss} we may assume that $\Delta u = \nabla \phi + \lambda u$ on $\partial\Omega$.
  Multiplying the Stokes resolvent equation on $\partial\Omega$ with $n$ and using the triangle inequality gives
  \begin{align}
    \|\nabla \phi \cdot n\|_{\HH^{-1}(\partial\Omega)} 
    &\leq \|\Delta u \cdot n \|_{\HH^{-1}(\partial\Omega)} + |\lambda| \| u\cdot n\|_{\HH^{-1}(\partial\Omega)}, \nonumber\\
    \label{eq:stokesEquationH1}
    |\lambda| \| u\cdot n \|_{\HH^{-1}(\partial\Omega)} 
    &\leq \|\Delta u \cdot n \|_{\HH^{-1}(\partial\Omega)} + \|\nabla \phi \cdot n \|_{\HH^{-1}(\partial\Omega)}.
  \end{align}
  We will now show that
  \begin{align}
    \label{eq:deltaun}
    \|\Delta u \cdot n\|_{\HH^{-1}(\partial\Omega)}
    \leq C \|\nabla u\|_\partial
  \end{align}
  and 
  \begin{align}
    \label{eq:nablaPhin}
    c \| \phi - \dashint_{\partial\Omega} \phi \d\sigma \|_\partial
    \leq \|\nabla \phi \cdot n\|_{\HH^{-1}(\partial\Omega)}
    \leq C \|\phi\|_\partial
  \end{align}
  Using these two estimates applied to \eqref{eq:stokesEquationH1}, we can directly derive \eqref{eq:phiDashintPhi} and \eqref{eq:lambdaun}.

  In order to prove \eqref{eq:deltaun}, note that
  \begin{align*}
    \Delta u \cdot n = n_i \frac{\partial^2 u_i}{\partial x_j^2} = \Big( n_i \frac{\partial}{\partial x_j} - n_j \frac{\partial}{\partial x_i} \Big) \frac{\partial u_i}{\partial x_j}
  \end{align*}
  since $\div u = 0$ in $\overline \Omega$.
  As the expression in between the brackets is a tangential derivative we derive estimate \eqref{eq:deltaun} from
  \begin{align*}
    | \langle \Delta u \cdot n, u \rangle | = | \langle \nabla u, \nabla_{\mathrm{tan}} u \rangle| \leq \|\nabla u\|_\partial^2
  \end{align*}
  since this implies 
  \begin{align*}
    \|\nabla u \cdot  n\|_{\HH^{-1}(\partial\Omega)} \leq \|\nabla u\|_\partial.
  \end{align*}

  Now for the proof of estimate \eqref{eq:nablaPhin} we will use $\Ell^2$-estimates for the Neumann and regularity problems for the Laplace equation in Lipschitz domains.
  For $g \in \Ell^2(\partial\Omega)$ with mean value zero, by Jerison and Kenig \cite{jerisonKenig} the Neumann problem for Laplace's equation on the Lipschitz domain $\Omega$ has a solution $\psi$ with  $(\nabla \psi)^* \in \Ell^2(\partial\Omega)$ and $\frac{\partial \psi}{\partial n} = g$ on $\partial \Omega$.
  Green's identity we have that since $\phi$ and $\psi$ are harmonic
  \begin{align}
    \big| \int_{\partial\Omega} \phi g \d \sigma \big|
    &=  \big| \int_{\partial\Omega} \phi \frac{\partial \psi}{\partial n} \d \sigma \big|
    = \big| \int_{\partial\Omega} \frac{\partial \phi}{\partial n} \psi \d \sigma \big|\nonumber \\
    \label{eq:dualityPhi}
    &\leq \| \frac{\partial \phi}{\partial n} \|_{\HH^{-1}(\partial\Omega)} \| \psi \|_{\HH^1(\partial\Omega)} \leq C \| \frac{\partial \phi}{\partial n} \|_{\HH^{-1}(\partial\Omega)} \| g\|_\partial ,
  \end{align}
  where in the last step we used the estimate $\|\psi\|_{\HH^1(\partial\Omega)} \leq C \|g\|_\partial$ for the $\Ell^2$ Neumann problem which can be found in Jerison and Kenig \cite{jerisonKenig}.
  Now if we set $\bar g = \phi - \tilde \phi$, with $\tilde \phi = \dashint_{\partial\Omega} \phi \d \sigma$ and use that $\int_{\partial\Omega} (\phi - \tilde \phi) \overline{(\phi - \tilde\phi)} \d\sigma = \int_{\partial\Omega} \phi \overline{(\phi - \tilde\phi)} \d\sigma$, we get from \eqref{eq:dualityPhi}
  \begin{align*}
    \| \phi - \tilde\phi \|_\partial^2
    \leq C \|\frac{\partial\phi}{\partial n} \|_{\HH^{-1}(\partial\Omega)} \|\phi - \tilde \phi\|_\partial
  \end{align*}
  or, after rearranging and expanding
  \begin{align*}
    \| \phi - \dashint_{\partial\Omega} \phi \d\sigma \|_\partial \leq C \|\frac{\partial\phi}{\partial n} \|_{\HH^{-1}(\partial\Omega)}
  \end{align*}
  We work in a similar way with results from the regularity problem of Laplace's equation by Jerison and Kenig \cite{jerisonKenig2}.
  Given $f \in \HH^1(\partial\Omega)$, there exists a harmonic function $\psi$ in $\Omega$ such that $(\nabla\psi)^* \in \Ell^2(\partial\Omega)$ and $\psi = f$ on $\partial\Omega$.
  As for \eqref{eq:dualityPhi}, we calculate
  \begin{align*}
    \big| \int_{\partial\Omega} \frac{\partial\phi} f \d\sigma \big|
    &= \big| \int_{\partial\Omega} \frac{\partial\phi} \psi \d\sigma \big|
    = \big| \int_{\partial\Omega} \phi \frac{\partial \psi}{\partial n} \d \sigma \big| \\
    &\leq \|\phi\|_{\partial} \|\nabla \psi \|_\partial 
    \leq C \|\phi\|_\partial \|f\|_{\HH^1(\partial\Omega)},
  \end{align*}
  where in the last step we used the estimate $\|\nabla \psi\|_\partial \leq C \|f\|_{\HH^1(\partial\Omega)}$ for the $\Ell^2$ regularity problem.
  By duality this gives that
  \begin{align*}
    \|\frac{\partial\phi}{\partial n} \|_{\HH^{-1}(\partial\Omega)} \leq C \|\phi\|_\partial. 
  \end{align*}
\end{proof}

\begin{rem}
  \label{rem:harmonicEstimate}
  A careful look at the proof of inequality \eqref{eq:nablaPhin} reveals that the estimate
  \begin{align*}
    c \| \phi\phi \|_\partial \leq \| \nabla \phi \cdot n \|_{\HH^{-1}(\partial\Omega)},
  \end{align*}
  holds for all harmonic functions $\phi$ with vanishing mean on $\partial\Omega$.
\end{rem}

After all this preparation we are now able to prove Theorem \ref{thm:rellich}.

\begin{proof}[Proof of Theorem \ref{thm:rellich}]
  For the proof of estimate \eqref{eq:rellich1}, without loss of generality we can assume that $\int_{\partial\Omega} \phi \d \sigma = 0$.

  Using \eqref{eq:phiDashintPhi} for the second summand  in \eqref{eq:rellich1} and and then \eqref{eq:gradEstimateRellich2} for the terms involving $\nabla u$ we get
  \begin{align*}
    \|\nabla u\|_\partial + \|\phi\|_\partial
    &\leq C \{ \|\nabla u\|_\partial + |\lambda| \|u \cdot n \|_{\HH^1(\partial\Omega)} \} \\
    &\leq C_\varepsilon \Big\{ \|\nabla_{\mathrm{tan}} u \|_\partial + |\lambda|^{1/2} \|u\|_\partial + |\lambda| \| u\cdot n\|_{\HH^{-1}(\partial\Omega)} \Big\} \\
    &\quad + C \varepsilon \{ \|\nabla u\|_\partial + \|\phi\|_\partial \}
  \end{align*}
  for all $\varepsilon \in (0,1)$.
  Chosing $\varepsilon$ such that $C \varepsilon < (1/2)$ we can rearrange the above inequality and obtain estimate \eqref{eq:rellich1}.

  Estimate \eqref{eq:rellich2} will need more effort to be proven.
  We start with inequality \eqref{eq:lambdaun} and derive
  \begin{align*}
    \|\nabla u\|_\partial + \|\phi\|_\partial + |\lambda| \|u \cdot n\|_{\HH^{-1}(\partial\Omega)}
    \leq C \{ \|\nabla u\|_\partial + \| \phi\|_\partial \}
    \leq C \Big\{ \|\frac{\partial u}{\partial \nu} \|_\partial + \|\nabla u\|_\partial \Big\},
  \end{align*}
  where in the last step we used the definition of conormal derivatives.
  If we now apply \eqref{eq:gradEstimateRellich} we get
  \begin{align*}
    \|\nabla u\|_\partial + \|\phi\|_\partial + |\lambda| \|u \cdot n\|_{\HH^{-1}(\partial\Omega)}
    \leq C_\varepsilon \| \frac{\partial u}{\partial \nu} \|_\partial + \varepsilon \big\{ \|\nabla u\|_\partial + \|\phi\|_\partial + \| |\lambda|^{1/2} u \|_\partial \big\}
  \end{align*}
  for all $\varepsilon \in (0,1)$.
  Choosing $\varepsilon$ appropriately yields
  \begin{align}
    \label{eq:partOfRellich2}
    \|\nabla u\|_\partial + \|\phi\|_\partial + |\lambda| \|u \cdot n\|_{\HH^{-1}(\partial\Omega)}
    \leq C \|\frac{\partial u}{\partial \nu} \|_\partial + C |\lambda|^{1/2} \| u\|_\partial.
  \end{align}
  Now we need to estimate $|\lambda|^{1/2} \|u\|_\partial$.
  Green's identity yields
  \begin{align}
    \int_{\partial\Omega} h_k n_k |u|^2 \d \sigma
    = \int_{\Omega} \frac{\partial}{\partial x_k} \big( h_k |u|^2 ) \d x
    &= \int_{\Omega} \frac{\partial h_k}{\partial x_k} |u|^2 \d x + \int_{\Omega} h_k \frac{\partial |u|^2}{\partial x_k}  \d x \nonumber\\
    \label{eq:hknkgreen}
    &= \int_\Omega \div(h) |u|^2 \d x + 2 \Re \int_\Omega h_k \frac{\partial \bar u_i}{\partial x_k} u_i \d x.
  \end{align}
  We choose $h \in \CC_0^1(\R^d, \R^d)$ with $h_k n_k \geq c > 0$ on $\partial\Omega$. 
  The existence of such a function $h$ was proven by Verchota \cite{verchotaDiss}.
  Using this, we can continue the estimate \eqref{eq:hknkgreen} as
  \begin{align}
    \label{eq:estupartial}
    \|u\|_\partial^2 \leq C \int_\Omega |u|^2 \d x + C \int_\Omega |u| |\nabla u| \d x.
  \end{align}
  The next estimate uses \eqref{eq:estupartial} and \eqref{eq:laxMilgramIneq} which gives
  \begin{align*}
    |\lambda| \|u\|_\partial^2 
    &\leq |\lambda| C \int_\Omega |u|^2 \d x + |\lambda| C \int_\Omega |u| |\nabla u| \d x \nonumber\\
    &\leq C \| \frac{\partial u}{\partial \nu} \|_\partial \|u\|_\partial + |\lambda|^{1/2} C \int_\Omega (|\lambda|^{1/2} |u|) |\nabla u| \d x \\
    &\leq C \|\frac{\partial u}{\partial \nu} \|_\partial \|u\|_\partial + |\lambda|^{1/2} C \big( \int_\Omega |\lambda| |u|^2 \big)^{1/2} \big( \int_\Omega (|\nabla u|^2 \d x \big)^{1/2} \\
    &\leq C \|\frac{\partial u}{\partial \nu} \|_\partial \||\lambda|^{1/2} u\|_\partial.
  \end{align*}
  Note that for the last estimate we also used the fact that $|\lambda| \geq \tau$ helps us to bound $\|u\|_\partial$ by $C |\lambda|^{1/2} \|u\|_\partial$.
  Rearranging terms in the last estimate, we now derive
  \begin{align}
    \label{eq:lambda12u}
    \| |\lambda|^{1/2} u\|_\partial \leq C \| \frac{\partial u}{\partial \nu} \|_\partial.
  \end{align}
  Estimate \eqref{eq:rellich2} follows directly from \eqref{eq:partOfRellich2} in combination with \eqref{eq:lambda12u} and this concludes our proof.
\end{proof}

Shen proved that under reasonable assumptions a theorem similar to \ref{thm:rellich} also holds for exterior domains

\begin{thm}
  \label{thm:rellichExterior}
  Let $\lambda \in \Sigma_\theta$ and $|\lambda| \geq \tau$, where $\tau \in (0,1)$.
  Let $(u,\phi)$ be a solution of the Stokes resolvent Problem in $\Omega_- = \R^d \setminus \overline\Omega$.
  Suppose additionally that $(\nabla u)^*$, $(\phi)^* \in \Ell^2(\partial\Omega)$ and that $\nabla u$, $\phi$ have nontangential limits almost everywhere on $\partial\Omega$.
  Furthermore let for $|x| \to \infty$
  \begin{align*}
    |\phi(x)| + |\nabla u(x)| = O(|x|^{1 - d}) \quad\text{and}\quad 
    u(x) = \begin{cases} O(|x|^{2 - d}) \quad\text{if } d \geq 3 \\ o(1) \quad\text{if } d = 2. \end{cases}
  \end{align*}
  Then
  \begin{align}
    \label{eq:rellich1ext}
    \|\nabla u\|_\partial + \|\phi\|_\partial
    \leq C \big\{ \|\nabla_{\mathrm{tan}} u\|_\partial + |\lambda|^{1/2} \|u\|_\partial + |\lambda| \|u \cdot n\|_{\HH^{-1}(\partial\Omega)} \big\}
  \end{align}
  and
  \begin{align}
    \label{eq:rellich2ext}
    \|\nabla u\|_\partial + |\lambda|^{1/2} \|u\|_\partial + |\lambda| \|u \cdot n\|_{\HH^{-1}(\partial\Omega)} + \|\phi\|_\partial
    \leq C \|\frac{\partial u}{\partial \nu} \|_\partial,
  \end{align}
  where $C$ depends only on $d$, $\tau$, $\theta$ and the Lipschitz character of $\Omega$.
\end{thm}
