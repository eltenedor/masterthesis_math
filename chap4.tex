\chapter{Rellich Estimates}
\label{chap:4}

In this section, we will establish Rellich-type estimates for the Stokes resolvent problem which will be used in the following chapter to prove the invertibility of the operators $\pm(1/2)I + \K_{\lambda}$ and their adjoints from Theorems  \ref{thm:jumpConditions} and \ref{thm:nontangentialLimitDoubleLayer}.

We will for this entire section always assume that $\Omega$ is a bounded Lipschitz domain in $\R^d$, $d \geq 2$, with connected boundary.
Furthermore, we will use the shorthand notation
\begin{align*}
    \| \cdot \|^{}_{\partial} \coloneqq \| \cdot \|_{\Ell^2(\partial\Omega; \C^k)}, \quad k \in \N,
\end{align*}
and we will tacitly use the summation convention whenever it is applicable.

The following Theorem states formulates the aforementioned Rellich estimates and is the central result of this chapter:

\begin{thm}
  \label{thm:rellich}
  Let $\lambda \in \Sigma_\theta$ and $|\lambda| \geq \tau$, where 
  $\tau \in (0,1)$.
  Let $(u,\phi)$ be a smooth solution to the Stokes resolvent problem in $\Omega$ and suppose that $(\nabla u)^* \in \Ell^2(\partial\Omega)$ and $(\phi)^* \in \Ell^2(\partial\Omega)$.
  Furthermore, assume that $\nabla u$, $\phi$ have nontangential limits almost everywhere on $\partial\Omega$.
  Then
  \begin{align}
    \label{eq:rellich1}
    \begin{alignedat}{1}
        &\|\nabla u\|^{}_\partial + \bigg\|\phi - \bigg\{ \frac{1}{{r_0}^{d - 1}}\int_{\partial\Omega} \phi \d\sigma \bigg\} \bigg\|^{}_\partial \\[1em]
      &\hspace{2cm} \leq C\, \bigg\{ \big\|\nabla_{\mathrm{tan}} u\big\|^{}_\partial + |\lambda|^{1/2} \|u\|^{}_\partial + |\lambda| \, \|u \cdot n\|_{\HH^{-1}(\partial\Omega)} \bigg\}
    \end{alignedat}
  \end{align}
  and
  \begin{align}
    \label{eq:rellich2}
    \|\nabla u\|^{}_\partial + |\lambda|^{1/2} \|u\|^{}_\partial + |\lambda| \, \|u \cdot n\|_{\HH^{-1}(\partial\Omega)} + \|\phi\|^{}_\partial
    \leq C \, \Big\|\frac{\partial u}{\partial \nu} \Big\|^{}_\partial,
  \end{align}
  where $\frac{\partial}{\partial \nu}$ denotes the conormal derivative, and $C$ depends only on $d$, $\tau$, $\theta$ and the Lipschitz character of $\Omega$.
\end{thm}

\begin{rem}
  \label{rem:shenNontangential}
  The assumptions on $u$ in Theorem \ref{thm:rellich} are sufficient for $u$ to have a nontangential limit and a square integrable maximal function $(u)^*$. 
  Indeed for $d = 2$ we have $(u)^* \in \Ell^\infty(\partial\Omega)$, for $d = 3$ we have $(u)^* \in \Ell^p(\partial\Omega), p \in (1,\infty)$, and for $d \geq 3$ we have $(u)^* \in \Ell^p(\partial\Omega)$, $p \in \big(1, 2 (d - 1) / (d - 3) \big)$.
  A proof of these facts can be found in Shen's notes \cite[Prop. 7.1.3]{Shen2017}.
\end{rem}

We will now prepare the proof of Theorem \ref{thm:rellich} by proving several helpful lemmata.
The first lemma deals with so called \emph{Rellich identities} for solutions to the Stokes resolvent system.

\begin{lem}
  \label{lem:rellichIdentity}
  Under the same conditions on $(u,\phi)$ as in Theorem \ref{thm:rellich}, we have
  \begin{align}
    \int_{\partial\Omega} h_k n_k |\nabla u|^2 \d\sigma 
    &= 2 \Ret \int_{\partial\Omega} h_k \frac{\partial \overline u_i}{\partial x_k} \bigg( \frac{\partial u}{\partial \nu} \bigg)_i \d\sigma + \int_\Omega \div(h)\, |\nabla u|^2 \d x \nonumber\\[0.5em]
    &\quad - 2 \Ret \int_\Omega \frac{\partial h_k}{\partial x_j} \cdot \frac{\partial u_i}{\partial x_k} \cdot \frac{\partial \overline u_i}{\partial x_j} \d x + 2 \Ret \int_\Omega \frac{\partial h_k}{\partial x_i} \cdot \frac{\partial u_i}{\partial x_k} \, \overline \phi \d x \nonumber\\[0.5em]
    &\quad - 2 \Ret \int_\Omega h_k \frac{\partial u_i}{\partial x_k} \cdot \overline {\lambda u_i} \d x \label{eq:rellichIdentity}
  \end{align}
  and
  \begin{align}
    \int_{\partial\Omega} h_k n_k |\nabla u|^2 \d\sigma
    &= 2\Ret \int_{\partial\Omega} h_k \frac{\partial \overline u_i}{\partial x_j} \bigg\{ n_k \frac{\partial u_i}{\partial x_j} - n_j \frac{\partial u_i}{\partial x_k} \bigg\} \d\sigma \nonumber\\[0.5em]
    &\quad + 2 \Ret \int_{\partial\Omega} h_k \, \overline \phi \, \bigg\{ n_i \frac{\partial u_i}{\partial x_k} - n_k \frac{\partial u_i}{\partial x_i} \bigg\} \d\sigma - \int_\Omega \div(h)\, |\nabla u|^2 \d x \nonumber\\[0.5em]
    &\quad + 2 \Ret \int_\Omega \frac{\partial h_k}{\partial x_j} \cdot \frac{\partial u_i}{\partial x_k} \cdot \frac{\partial \overline u_i}{\partial x_j} \d x - 2 \Ret \int_\Omega \frac{\partial h_k}{\partial x_i} \cdot \frac{\partial u_i}{\partial x_k} \, \overline \phi \d x \nonumber\\[0.5em]
    &\quad + 2 \Ret \int_\Omega h_k \frac{\partial u_i}{\partial x_k} \cdot \overline{ \lambda u_i} \d x\,, \label{eq:rellichIdentity2}
  \end{align}
  where $h = (h_1, \dots,h_d) \in \CC_0^1(\R^d, \R^d)$.
\end{lem}

\begin{proof}
  The proof of the stated identities reduces to several applications of the divergence theorem once we establish its applicability.
  To this end, we want to make Proposition \ref{prop:approximationArgument} available. We note that the assumptions given in Theorem \ref{thm:rellich} are sufficient for this purpose and we will verify them, once they are used.

  Let's expand the first summand in \eqref{eq:rellichIdentity} using the definition of conormal derivatives
  \begin{align*}
    2\Ret \int_{\partial\Omega} h_k \frac{\partial \overline u_i}{\partial x_k} \bigg( \frac{\partial u}{\partial \nu} \bigg)_i \d\sigma
    &= 2 \Ret \int_{\partial\Omega} h_k \frac{\partial \overline u_i}{\partial x_k} \cdot \frac{\partial u_i}{\partial x_j}  n_j \d\sigma  - 2\Ret \int_{\partial\Omega} h_k \frac{\partial \overline u_i}{\partial x_k} \, \phi\, n_i \d x \\
    &\eqqcolon I_1 - I_2.
  \end{align*}
  The divergence theorem is applicable for $I_1$ as $h$ is bounded and defined everywhere and the integrand has nontangential limits that can be dominated by $|(\nabla u)^*|^2 \in \Ell^2(\partial\Omega)$.
  Therefore, we find using the divergence theorem and the product rule:
  \begin{align*}
    I_1
    &= 2\Ret \int_\Omega \frac{\partial}{\partial x_j} \bigg\{ h_k \frac{\partial \overline u_i}{\partial x_k} \cdot \frac{\partial u_i}{\partial x_j} \bigg\} \d x \\[0.5em]
    &= 2\Ret \int_\Omega \frac{\partial h_k}{\partial x_j} \cdot \frac{\partial \overline u_i}{\partial x_k} \cdot \frac{\partial u_i}{\partial x_j} \d x 
    + 2\Ret \int_{\Omega} h_k \frac{\partial^2 \overline u_i}{\partial x_j \partial x_k} \cdot \frac{\partial u_i}{\partial x_j} \d x  \\[0.5em]
    &\hphantom{= 2\Ret \int_\Omega \frac{\partial h_k}{\partial x_j} \cdot \frac{\partial \overline u_i}{\partial x_k} \cdot \frac{\partial u_i}{\partial x_j} \d x } \;
    + 2 \Ret \int_\Omega h_k \frac{\partial \overline u_i}{\partial x_k} \cdot \frac{\partial^2 u_i}{\partial x_j^2} \d x \\
    &\eqqcolon I_3 + I_4 + I_5.
  \end{align*}
  For $I_5$, we use the fact that $u$ solves the Stokes resolvent problem which gives
  \begin{align*}
     I_5 
     &= 2 \Ret \int_\Omega h_k \frac{\partial \overline u_i}{\partial x_k}\cdot  \frac{\partial \phi}{\partial x_i} \d x + 2\Ret \int_\Omega h_k \frac{\partial\overline u_i}{\partial x_k} \lambda u_i \d x \\
     &\eqqcolon I_6 + I_7 .
  \end{align*}
  Now we want to apply the divergence theorem, i.e. Proposition \ref{prop:approximationArgument} to integral $I_2$.
  This is possible since $h$ is defined everywhere and bounded, $(\partial_k u_i) \cdot \phi$ has a nontangential limit and can be bounded by $\big( |(\nabla u)^*| |(\phi)^*| \big)$ which is integrable due to Hölder's inequality as $(\nabla u)^*$ and $(\phi)^*$ are square integrable by assumption.
  Thus the divergence theorem is applicable and yields together with the product rule:
  \begin{align*}
    I_2
    &= 2 \Ret \int_\Omega \frac{\partial}{\partial x_i} \Big\{ h_k \frac{\partial \overline u_i}{\partial x_k} \phi \Big\} \d x \\[0.5em]
    &= 2 \Ret \int_\Omega \frac{\partial h_k}{\partial x_i} \cdot \frac{\partial \overline u_i}{\partial x_k} \, \phi \d x + 2 \Ret \int_\Omega h_k \frac{\partial^2 \overline u_i}{\partial x_j \partial x_k} \, \phi \d x + 2 \Ret \int_\Omega h_k \frac{\partial \overline u_i}{\partial x_k} \cdot \frac{\partial \phi}{\partial x_i} \d x\\
    &\eqqcolon I_8 + I_9 + I_6.
  \end{align*}
  One term that hasn't come up so far, the second summand of the right hand side in \eqref{eq:rellichIdentity}, will now be expanded:
  \begin{align*}
    \int_\Omega \div(h)\,  |\nabla u|^2 \d x
    &= \int_\Omega \div( h\,|\nabla u|^2) \d x - \int_\Omega h_k \frac{\partial}{\partial x_i} \Big\{ |\nabla u|^2 \Big\} \d x \\
    &\eqqcolon I_{10} - I_{11}.
  \end{align*}
  Expanding the Integral $I_{11}$ gives us the identity
  \begin{align*}
    I_{11}
    &= \int_\Omega h_i \frac{\partial}{\partial x_i} \bigg\{ \frac{\partial u_k}{\partial x_j} \cdot \frac{\partial \overline u_k}{\partial x_j} \bigg\} \d x
    = \int_\Omega h_i  \bigg\{ \frac{\partial^2 u_k}{\partial x_i \partial x_j} \cdot \frac{\partial \overline u_k}{\partial x_j} + \frac{\partial u_k}{\partial x_j} \cdot \frac{\partial^2 \overline u_k}{\partial x_i \partial x_j} \bigg\} \d x= I_4.
  \end{align*}
  If we now put everything together, the right hand side of \eqref{eq:rellichIdentity} reads
  \begin{align*}
    &(I_1 - I_2) + (I_{10} - I_{11}) - I_3 + I_8 - I_7 \\
    &\quad= (I_3 + I_4 + I_6 + I_7) - (I_8 + I_9 + I_6) + I_{10} - I_{11} - I_3 + I_8 - I_7 = I_{10}.
  \end{align*}
  Noting that by the divergence theorem, which is applicable with the same justification as for the integral $I_1$, we have
  \begin{align*}
    I_{10} = \int_{\partial\Omega} h_k n_k |\nabla u|^2 \d \sigma.
  \end{align*}
  Thus, the first identity is proven.

  In order to prove identity \eqref{eq:rellichIdentity2}, we show that the expression we get from considering (\eqref{eq:rellichIdentity} + \eqref{eq:rellichIdentity2}) holds, i.e. we show the identity
  \begin{align*}
    2 \int_{\partial\Omega} h_k n_k |\nabla u|^2 \d\sigma
    &= 2 \Ret \int_{\partial\Omega} h_k \frac{\partial \overline u_i}{\partial x_k} \bigg( \frac{\partial u}{\partial \nu} \bigg)_i \d \sigma \\[0.5em]
    &\quad + 2 \Ret \int_{\partial\Omega} h_k \frac{\partial\overline u_i}{\partial x_j} \bigg\{ n_k \frac{\partial u_i}{\partial x_j} - n_j \frac{\partial u_i}{\partial x_k} \bigg\} \d \sigma \\[0.5em]
    &\quad+ 2 \Ret \int_{\partial\Omega} h_k \, \overline \phi\,  \bigg\{ n_i \frac{\partial u_i}{\partial x_i} - n_k \frac{\partial u_i}{\partial x_i} \bigg\} \d\sigma.
  \end{align*}
  To this end, note that the left side of the identity equals $2 I_{10}$, whereas the right hand side can be written as 
  \begin{align*}
    (I_1 - I_2) + 2 (I_{10} - I_1) + (I_2 - 0),
  \end{align*}
  where we also used the fact that $\div u = \partial_i u_i = 0$.
\end{proof}

Consider the operators $\partial_{\tau_{jk}}$ which act on compactly supported continuously differentiable functions $\psi$ in the neighborhood of $\partial\Omega$ by
\begin{align}
  \label{eq:defnTangDerivative}
  \partial_{\tau_{jk}} \psi \coloneqq n_j \frac{\partial \psi}{\partial x_k} \bigg|_{\partial\Omega} - n_k \frac{\partial \psi}{\partial x_j} \bigg|_{\partial\Omega}, \quad j,k = 1,\dots,d.
\end{align}
These operators show up in identity \eqref{eq:rellichIdentity2} as
\begin{align*}
  \partial_{\tau_{kj}} u_i = \Big\{ n_k \frac{\partial u_i}{\partial x_j} - n_j \frac{\partial u_i}{\partial x_k} \Big\}
  \quad\text{and}\quad
  \partial_{\tau_{ik}} u_i =  \Big\{ n_i \frac{\partial u_i}{\partial x_k} - n_k \frac{\partial u_i}{\partial x_i} \Big\}
\end{align*}
and have been introduced by Mitrea and Wrigth \cite[p. 16]{mitreaWright}. 
These operators are called \emph{first-order tangential derivative operators} and relate to the tangential gradient, which has been introduced in 
\eqref{eq:tangentialGradient}, in the following way:
\begin{align}
  \label{eq:relTanGrad}
  \big(\nabla_{\mathrm{tan}} \psi\big)_j = \frac{\partial \psi}{\partial x_j} - n_k n_j \frac{\partial \psi}{\partial x_k} = n_k \, \partial_{\tau_{kj}} \psi.
\end{align}
The tangential derivative operators come with a helpful  ``integration by parts'' rule and can be used to define Sobolev spaces on the boundary $\partial\Omega$:
For $f \in \Ell^1_\loc(\Omega)$, we start by defining antilinear functionals on $\CC_0^\infty(\R^d)$ by setting
\begin{align*}
    \partial_{\tau_{kj}} f \colon \CC_0^\infty(\R^d) \ni \psi \mapsto \int_{\partial\Omega} f \, \overline{ \partial_{\tau_{jk}} \psi} \d\sigma.
\end{align*}
Now the weak tangential derivatives of $f$ are given by those functionals which are regular, i.e. elements of $\Ell^1_\loc(\partial\Omega)$.
In this case, the following integration by parts formula holds:
\begin{align*}
    \int_{\partial\Omega} f\, \overline{( \partial_{\tau_{jk}} \psi)} \d\sigma = \int_{\partial\Omega} (\partial_{\tau_{kj}} f)\, \overline{\psi} \d\sigma.
\end{align*}
Note that the minus sign that usually comes with the integration by parts is hidden in the new order of indices.
For $p \in (1,\infty)$, we define the corresponding Sobolev space via
\begin{align*}
  \WW^{1,\,p}(\partial\Omega) = \Big\{ f \in \Ell^p(\partial\Omega) \colon \partial_{\tau_{jk}} f \in \Ell^p(\partial\Omega), \; j,k = 1,\dots,d \Big\}.
\end{align*}

We extend our detour by the following basic lemma on a reverse triangle inequality for elements of the sector $\Sigma_\theta$.
A powerful generalization of this Lemma can be found in Tolksdorf \cite[Lem. 5.2.4]{tolksdorf}.

\begin{lem}
  \label{lem:lambdaIneq}
  Let $\theta \in (0,\pi/2)$.
  Then there exists $\alpha$ depending only on $\theta$ such that for all $\lambda \in \Sigma_\theta$ the following inequality holds:
  \begin{align*}
    \Re(\lambda) + \alpha \, \big|\Im(\lambda)\,\big| \geq |\lambda|.
  \end{align*}
\end{lem}

\begin{proof}
  For the moment being, suppose $|\lambda| = 1$ Then, we have $\Re(\lambda) = \cos(\varphi)$ and $\Im(\lambda) = \sin(\varphi)$ with $|\varphi| \in (0, \pi - \theta)$. 
  Set
  \begin{align*}
    \alpha = \frac{1 - \cos(\pi - \theta)}{\sin(\pi - \theta)} \geq \frac{1 - \cos(|\varphi|)}{\sin(|\varphi|)}.
  \end{align*}
  If $\varphi = |\varphi|$, this gives the inequality
  \begin{align*}
    \Re(\lambda) + \alpha \, |\Im(\lambda)\,| = \cos(\varphi) + \alpha \sin(\varphi) \geq 1.
  \end{align*}
  Conversely, if $\varphi = -|\varphi|$, then we have by the symmetry properties of $\sin$ and $\cos$ that
  \begin{align*}
    \Re(\lambda) + \alpha\, |\Im(\lambda)\,| = \cos(-\varphi) + \alpha \sin(-\varphi) \geq 1.
  \end{align*}
  For arbitrary $\lambda$ the claim follows by considering the normalized value $(\lambda / |\lambda|)$.
\end{proof}

The next lemma enables us to handle the solid integrals in \eqref{eq:rellichIdentity} and \eqref{eq:rellichIdentity2}.

\begin{lem}
  \label{lem:laxMilgramIneq}
  Under the same assumptions on $(u,\phi)$ and $\lambda$ as in Theorem \ref{thm:rellich}, we have
  \begin{align}
    \label{eq:laxMilgramIneq}
    \int_\Omega |\nabla u|^2 \d x + |\lambda| \int_\Omega |u|^2 \leq C \,\Big\| \frac{\partial u}{\partial \nu} \Big\|_\partial  \|u\|_\partial,
  \end{align}
  where $C$ depends only on $\theta$.
\end{lem}

\begin{proof}
  Inserting the solution $u$ into the the variational problem of the Stokes resolvent problem gives us
  \begin{align}
    \label{eq:variationalStokes}
    \int_\Omega -\Delta u \cdot \overline u \d x + \lambda \int_\Omega u \cdot \overline u \d x= - \int_\Omega \nabla \phi \cdot \overline u \d x.
  \end{align}
  Rewriting the first term of equation \eqref{eq:variationalStokes} leads to 
  \begin{align*}
    -\int_{\Omega} \frac{\partial^2 u_j}{\partial x_i \partial x_i}  \; \overline u_j \d x
    = -\int_{\Omega} \frac{\partial }{\partial x_i} \bigg\{ \overline u_j \frac{\partial u_j}{\partial x_i} \bigg\} \d x + \int_\Omega \frac{\partial u_j}{\partial x_i} \cdot \frac{\partial \overline u_j}{\partial x_i} \d x.
    \end{align*}
  Note that since $u$ is solenoidal, we have for the third term of equation \eqref{eq:variationalStokes}
  \begin{align*}
    - \int_{\Omega} \frac{\partial \phi}{\partial x_i} \; \overline u_i \d x = - \int_{\Omega} \frac{\partial}{\partial x_i} \Big\{ \phi\,  \overline u_i \Big\} \d x.
  \end{align*}
  Now we want to transform the first and third of the above solid integrals into boundary integrals through Proposition \ref{prop:approximationArgument}.
  By the assumptions formulated in Theorem \ref{thm:rellich}, $\phi$ and $\nabla u$ have a nontangential limit and for both nontangential maximal functions the inclusion $(\phi)^*, (\nabla u)^* \in \Ell^2(\partial\Omega)$ holds. 
  Furthermore, according to Remark \ref{rem:shenNontangential}, also $u$ has a nontangential limit and the nontangential maximal function satisfies $(u)^* \in \Ell^2(\partial\Omega)$. 
  Therefore, the function $|\phi\,  \overline u_i |$ may be dominated by $|(\phi)^* (u)^*| \in \Ell^2(\partial\Omega)$ and the function $|(\partial_j u_i) u_i|$ may be dominated by $|(\nabla u)^* (u)^*|$, respectively. Thus, the door to Proposition \ref{prop:approximationArgument} has been opened which allows to transform equation \eqref{eq:variationalStokes} into
  \begin{align*}
    \int_\Omega |\nabla u|^2 \d x - \int_{\partial\Omega} \frac{\partial u}{\partial n} \cdot \overline u \d \sigma + \lambda \int_\Omega |u|^2 \d x = - \int_{\partial\Omega}  \phi n \cdot \overline u \, \d \sigma.
  \end{align*}
  We can rearrange the terms of this identity and use the definition of conormal derivatives, see equation \eqref{eq:conormalDerivative}, to derive
  \begin{align}
    \label{eq:testedStokes}
    \int_\Omega |\nabla u|^2 \d x + \lambda \int_\Omega |u|^2 \d x = \int_{\partial\Omega} \frac{\partial u}{\partial \nu} \cdot \overline u \, \d \sigma.
  \end{align}
  If we now take the real and imaginary part of \eqref{eq:testedStokes} and sum them up with the prefactor $\alpha(\theta) > 0$ from Lemma \ref{lem:lambdaIneq}, we get
  \begin{align*}
    \int_\Omega |\nabla u|^2 \d x + \Big\{ \Re(\lambda) + \alpha\, \big|\Im(\lambda)\,\big|\, \Big\} \int_\Omega |u|^2 \d x
    \leq (1 + \alpha) \, \bigg| \int_{\partial\Omega} \frac{\partial u}{\partial \nu} \cdot \bar u \d \sigma \, \bigg|\,.
  \end{align*}
  Lemma \ref{lem:lambdaIneq} now gives
  \begin{align*}
    \int_\Omega |\nabla u|^2 \d x + |\lambda| \int_\Omega |u|^2 \d x \leq C\,  \bigg| \int_{\partial\Omega} \frac{\partial u}{\partial \nu} \cdot \bar u \d \sigma\, \bigg|\,,
  \end{align*}
  with $C = (1 + \alpha)$ from which we readily derive estimate \eqref{eq:laxMilgramIneq} after applying the Cauchy-Schwartz inequality.
\end{proof}

The next lemma combines Rellich identities \eqref{eq:rellichIdentity} and \eqref{eq:rellichIdentity2} with estimate \eqref{eq:laxMilgramIneq}.

\begin{lem}
  Under the same assumptions on $(u,\phi)$ and $\lambda$ as in Theorem \ref{thm:rellich}, we have
  \begin{align}
    \label{eq:gradEstimateRellich}
    \| \nabla u\|_\partial \leq C_\varepsilon\, \,\Big\| \frac{\partial u}{\partial \nu} \Big\|_\partial + \varepsilon \,\Big\{ \|\nabla u\|_\partial + \|\phi\|_\partial +  |\lambda|^{1/2} \| u\|_\partial \Big\}
  \end{align}
  and
  \begin{align}
    \label{eq:gradEstimateRellich2}
    \|\nabla u\|_\partial \leq C_\varepsilon\, \Big\{ \|\nabla_{\mathrm{tan}} u \|_\partial + |\lambda|^{1/2} \| u\|_\partial \Big\} + \varepsilon\, \Big\{ \|\nabla u\|_\partial + \|\phi\|_\partial \Big\}
  \end{align}
  for all $\varepsilon \in (0,1)$, where $C_\varepsilon$ depends only on $d$, $\theta$, $\tau$, $\varepsilon$ and the Lipschitz character of $\Omega$.
\end{lem}

\begin{proof}
  Let $h = (h_1, \dots, h_d) \in \CC_0^1(\R^d, \R^d)$ with $h_k n_k \geq c > 0$ on $\partial \Omega$ as given by Therorem \ref{thm:smoothApproximation} v). 
  The idea of the proof of the desired estimates \eqref{eq:gradEstimateRellich} and \eqref{eq:gradEstimateRellich2} is to first use the Rellich identities from Lemma \ref{lem:rellichIdentity} with this particular $h$ to estimate $\|\nabla u\|_\partial$ and then to bound the resulting right hand side by providing individual estimates.

  Before we start, note that we have $\Delta \phi = 0$ on the one hand and for the nontangential maximal function $(\phi)^* \in \Ell^2(\partial\Omega)$ on the other hand. According to Shen \cite[p. 410]{Shen2012}, a result from Dahlberg \cite{dahlberg77} gives the estimation
  \begin{align}
    \label{eq:dahlbergEstimate}
    \int_\Omega |\phi|^2 \d x \leq C \, \|(\phi)^* \|_\partial^2 \leq C\, \|\phi\|_\partial^2.
  \end{align}

  We will now prove the first estimate \eqref{eq:gradEstimateRellich}.
  In view of identity \eqref{eq:rellichIdentity}, we have
  \begin{align}
    \label{eq:normRellich}
    \begin{alignedat}{1}
    \|\nabla u\|_\partial^2
    &\leq C \, \bigg\{ \|\nabla u\|_\partial \, \Big\| \frac{\partial u}{\partial \nu} \Big\|_\partial + \int_\Omega |\nabla u|^2 \d x \\
    &\hspace{1.5cm}+ \int_\Omega |\nabla u| \, |\phi| \d x + |\lambda| \int_\Omega |\nabla u|\, |u| \d x \bigg\},
    \end{alignedat}
  \end{align}
  where the first term follows from the Cauchy-Schwartz inequality and $C$ only depends on $d$ and the Lipschitz character of $\Omega$.

  For now, we keep the first term of \eqref{eq:normRellich} as it is, the second term can be handled via Lemma \ref{lem:laxMilgramIneq}. 
  The goal for the remaining two integrals will be to bound each of them by a product of norms $\|\cdot\|_\partial$.
To this end, for the third integral we calculate
  \begin{align}
    \int_\Omega |\nabla u| |\phi| \d x
    \leq \Big( \int_\Omega |\nabla u|^2 \d x \Big)^{1/2} \Big( \int_\Omega |\phi|^2 \d x \Big)^{1/2}
    \leq C \, \Big\| \, \frac{\partial u}{\partial \nu}\, \Big\|_\partial^{1/2} \|u\|_\partial^{1/2} \, \|\phi\|_\partial\,, \label{eq:nablaPhi}
  \end{align}
  where the first step is due to the Cauchy-Schwartz inequality and the second step combines estimate \eqref{eq:laxMilgramIneq} with estimate \eqref{eq:dahlbergEstimate}.

  The last integral of \eqref{eq:normRellich} can be estimated as follows:
  \begin{align}
\label{eq:lambdaNablaU}
    |\lambda| \int_\Omega |\nabla u| \, |u| \d x 
    &\leq  \frac{|\lambda|^{3/2}}{2} \int_\Omega |u|^2 \d x + \frac{|\lambda|^{1/2}}{2 } \int_\Omega |\nabla u|^2 \d x \leq C \, \Big\| \, \frac{\partial u}{\partial \nu} \, \Big\|_\partial \, |\lambda|^{1/2}  \big\| u\,\big\|_\partial, 
  \end{align}
  where in the first step we used the weighted Young inequality and in the second step we applied estimate \eqref{eq:laxMilgramIneq}.
  Putting everything together, we calculate
  \begin{align*}
    \| \nabla u\|_\partial^2 
    &\leq C\, \bigg\{ \| \nabla u\|_\partial \, \Big\|\,\frac{\partial u}{\partial \nu} \, \Big\|_\partial 
    + \Big\|\,\frac{\partial u}{\partial \nu}\,\Big\|_\partial \|u\|^{}_\partial 
    + \Big\|\,\frac{\partial u}{\partial \nu}\,\Big\|_\partial^{1/2} \|u\|_\partial^{1/2} \|\phi\|^{}_\partial 
+ \Big\|\,\frac{\partial u}{\partial \nu}\,\Big\|_\partial \, |\lambda|^{1/2} \big\|  u \big\|^{}_\partial\bigg\}
  \end{align*}
  If we now use the assumption $|\lambda| \geq \tau$ which allows us to bound $\|u\|_\partial$ via
  \begin{align*}
      \|u\|^{}_\partial \leq \frac{|\lambda|^{1/2}}{\tau^{1/2}} \|u\|^{}_\partial = C \, |\lambda|^{1/2} \|u\|^{}_{\partial},
  \end{align*}
  the desired estimate \eqref{eq:gradEstimateRellich} now follows applying Young's weighted inequality with an $\varepsilon$ and the norm equivalence on finite dimensional vector spaces. 
  Note that for the product of three norms from inequality \eqref{eq:nablaPhi} we need to apply the Young inequality twice:
  \begin{align*}
      \Big\| \, \frac{\partial u}{\partial \nu} \, \Big\|_\partial^{1/2} \|u\|_\partial^{1/2} \|\phi\|^{}_\partial 
      &\leq \Big\{\, \frac{1}{4 \varepsilon}\, \Big\|\, \frac{\partial u}{\partial \nu}\, \Big\|^{}_\partial 
  + \varepsilon \|u\|^{}_\partial \Big\} \, \|\phi\|^{}_\partial  \\[0.5em]
  &\leq \frac{1}{32\, \varepsilon^3} \, \Big\|\, \frac{\partial u}{\partial \nu} \, \Big\|_\partial^2 + \frac{\varepsilon}{2} \|\phi\|_\partial^2 + \frac{1}{2} \|u\|_\partial^2 + \frac{\varepsilon^2}{2} \|\phi\|_\partial^2 \\[0.5em]
  &\leq C_\varepsilon\, \Big\{ \; \Big\|\, \frac{\partial u}{\partial\nu} \, \Big\|_\partial^2 + \|u\|_\partial^2 \,\Big\} + \varepsilon \|\phi\|_\partial^2,
  \end{align*}
  where for the last inequality we used the fact that $\varepsilon < 1$.

  For inequality \eqref{eq:gradEstimateRellich2}, we use the Rellich identity \eqref{eq:rellichIdentity2} and the relation \eqref{eq:relTanGrad} to obtain the estimate
  \begin{align}
    \label{eq:onTheWay}
    \begin{alignedat}{1}
    \| \nabla u\|_\partial^2 
    &\leq C \bigg\{ \|\nabla_{\mathrm{tan}} u\|^{}_\partial 
    \Big\{ \|\nabla u\|^{}_\partial + \|\phi\|^{}_\partial \Big\}   \\
    &\hspace{2cm} + \int_\Omega |\nabla u|^2 \d x + \int_\Omega |\nabla u| |\phi| \d x + |\lambda| \int_\Omega |\nabla u| |u|\d x \bigg\}.
  \end{alignedat}
  \end{align}
  As before we estimate the three terms on the right side of \eqref{eq:onTheWay} using \eqref{eq:laxMilgramIneq}, \eqref{eq:nablaPhi} and \eqref{eq:lambdaNablaU}, respectively, and obtain the estimate
  \begin{align*}
    \| \nabla u\|_\partial^2 
    &\leq C\, \bigg\{ \|\nabla_{\mathrm{tan}} u\|^{}_\partial 
    \Big\{ \|\nabla u\|^{}_\partial + \|\phi\|^{}_\partial \Big\}   \\
    &\hspace{2cm} + \Big\| \frac{\partial u}{\partial \nu} \Big\|^{}_\partial \| u\|^{}_\partial
+ \Big\| \frac{\partial u}{\partial \nu} \Big\|_\partial^{1/2} \|u\|_\partial^{1/2} \|\phi\|^{}_\partial + \Big\| \frac{\partial u}{\partial \nu} \Big\|^{}_\partial \, |\lambda|^{1/2} \big\| u  \big\|^{}_\partial \bigg\}.
  \end{align*}
  If we now use the Young inequality with an $\varepsilon$, we get
  \begin{align*}
    \|\nabla u\|_\partial^2 
    &\leq C_\varepsilon\, \Big\{ \, \|\nabla_{\mathrm{tan}} u\|_\partial^2 + |\lambda|^{1/2} \| u\|_\partial^2 \, \Big\}  + \varepsilon\, \bigg\{ \, \|\nabla u\|_\partial^2 + \|\phi\|_\partial^2 + \frac{1}{4}\,\Big\|\frac{\partial u}{\partial \nu} \Big\|_\partial^2 \, \bigg\}.
  \end{align*}
  The claim now follows if we use the definition of the conormal derivative and the norm equivalence on finite dimensional vector spaces.
\end{proof}

We prove one last lemma before we tackle the central theorem of this chapter.
The following lemma will not depend on the lemmata which were proven in the preceding part of this chapter, as the approach to derive the desired boundary estimates will be different:
We will not rely upon the rellich identities from Lemma \ref{lem:rellichIdentity} but directly part from a variational formulation of the Stokes resolvent problem on the boundary. 
Furthermore, we will work with the following Theorem about the regularity problem and the Neumann problem for the Laplacian on bounded Lipschitz domains
\begin{thm}
  \label{thm:jerisonKenig}
  Let $\Omega \subset \R^d$, $d \geq 2$, be a bounded Lipschitz domain. Then the following statements hold:
  \begin{enumerate}[a)]
    \item Given $f \in \HH^1(\partial\Omega)$, there exists a unique $\psi \in$ with $(\nabla \psi)^* \in \Ell^2(\partial\Omega)$ such that $\Delta u = 0$ in $\Omega$ and $\psi$ converges nontangentially to $f$ a.e.. Furthermore, the estimate 
      \begin{align*}
          \|\nabla\psi\|^{}_\partial \leq C\, \|f\|_{\HH^1(\partial\Omega)}
      \end{align*}
      holds with a constant only depending on the Lipschitz character of $\Omega$.
    \item Given $f \in \Ell^2(\partial\Omega)$ with $\int_{\partial\Omega} f \d\sigma = 0$, there exists a harmonic function $\psi$ on $\Omega$ with $\frac{\partial\psi}{\partial n} = f$ a.e.. Furthermore, the estimate
      \begin{align*}
          \| \psi\|_{\HH^1(\partial\Omega)} \leq C\, \|f\|^{}_\partial
      \end{align*}
      holds with C only depending on the Lipschitz character of $\Omega$.
  \end{enumerate}
\end{thm}

\begin{proof}
  According to Kenig, for the differential operator $\Delta$ the \emph{regularity problem}, $(R)_2$, and the \emph{Neumann problem}, $(N)_2$,  are solvable for data $f \in \Ell^2(\partial\Omega)$, see \cite[Thm. 2.1.10]{kenigBook}.
  Checking the definitions of $(R)_2$, see \cite[Defn. 1.7.10]{kenigBook}, and $(N)_2$, see \cite[Defn. 1.7.9]{kenigBook}, the claimed properties of follow from Theorems 1.8.2 and 1.8.3 in \cite[Chap. 1]{kenigBook}.
  We also refer to the works of Jerison and Kenig on the $\Ell^2$-regularity problem \cite{jerisonKenig2} and the Neumann problem \cite{jerisonKenig}.
\end{proof}

\begin{lem}
  Assume that $(u,\phi)$ satisfies the same conditions as in Theorem \ref{thm:rellich}.
  Then,
  \begin{align}
    \label{eq:phiDashintPhi}
    \bigg\|\phi - \bigg\{ \frac{1}{{r_0}^{d - 1}}\int_{\partial\Omega} \phi \d\sigma \bigg\} \bigg\|^{}_\partial 
    \leq C \Big\{ \|\nabla u\|^{}_\partial + |\lambda| \, \|u\cdot n\|_{\HH^{-1}(\partial\Omega)} \Big\}
  \end{align}
  and
  \begin{align}
    \label{eq:lambdaun}
    |\lambda|\, \| u\cdot n \|_{\HH^{-1}(\partial\Omega)} \leq C \Big\{ \|\phi\|^{}_\partial + \|\nabla u\|^{}_\partial \Big\},
  \end{align}
  where $C$ depends only on $d$ and the Lipschitz character of $\Omega$.
\end{lem}

\begin{proof}
  Our first goal will be to show that without loss of generality we may assume that $\Delta u = \nabla \phi + \lambda u$ on $\partial\Omega$.
  The central player will once again be Theorem \ref{thm:smoothApproximation}.
  To this end let $(\Omega_j)$ be a sequence of approximating $\CC^\infty$ domains as in Theorem \ref{thm:smoothApproximation} and suppose that
  \ref{eq:phiDashintPhi} holds for all $\Omega_j$, i.e. we have proven the inequality
  \begin{align}
    \label{eq:phiDashintPhij}
    \bigg\|\, \phi - \bigg\{ \frac{1}{{r_0}^{d - 1}}\int_{\partial\Omega} \phi \d\sigma \bigg\} \, \bigg\|_{\Ell^2(\partial\Omega_j; \C^d)}
     \leq C \Big\{ \|\nabla u\|_{\Ell^2(\partial\Omega_j; \C^{d\times d})} + |\lambda| \, \|u\cdot n\|_{\HH^{-1}(\partial\Omega_j)} \Big\}
  \end{align}
  and an inequality analogous to \eqref{eq:lambdaun}.
  Note that the mean value integral of $\phi$ should remain unchanged.
  It is now crucial that the constants in these inequalities only depend on the Lipschitz character of $\Omega$ and not on other geometric properties of the domain.
  Thus $C$ does not depend on $j$ and we may take the limit $j \to \infty$ using dominated convergence.

  For the rest of the proof we will thus assume that $(u,\phi)$ satisfies the Stokes resolvent problem in a domain $\Omega'$ for some $\overline \Omega \subseteq \Omega'$.
  In particular we have $\Delta u = \nabla \phi + \lambda u$  on $\partial\Omega$.
  Multiplying this identity  on $\partial\Omega$ with the outer normal vector $n$ and using the triangle inequality gives the following set of estimates:
  \begin{align}
    \label{eq:stokesEquationH1}
    \begin{alignedat}{1}
    \|\nabla \phi \cdot n\|_{\HH^{-1}(\partial\Omega)} 
    &\leq \|\Delta u \cdot n \|_{\HH^{-1}(\partial\Omega)} + |\lambda|\, \| u\cdot n\|_{\HH^{-1}(\partial\Omega)}, \\
    |\lambda|\, \| u\cdot n \|_{\HH^{-1}(\partial\Omega)} 
    &\leq \|\Delta u \cdot n \|_{\HH^{-1}(\partial\Omega)} + \|\nabla \phi \cdot n \|_{\HH^{-1}(\partial\Omega)}.
    \end{alignedat}
  \end{align}
  This looks almost like the desired pair of inequalities.
  We will now show that on the one hand the estimate
  \begin{align}
    \label{eq:deltaun}
    \|\Delta u \cdot n\|_{\HH^{-1}(\partial\Omega)}
    \leq C\, \|\nabla u\|^{}_\partial
  \end{align}
  and on the other hand the estimate
  \begin{align}
    \label{eq:nablaPhin}
    c \, \bigg\|\,\phi - \bigg\{ \frac{1}{{r_0}^{d - 1}}\int_{\partial\Omega} \phi \d\sigma \bigg\} \, \bigg\|^{}_\partial 
    %c \| \phi - \dashint_{\partial\Omega} \phi \d\sigma \|_\partial
    \leq \|\nabla \phi \cdot n\|_{\HH^{-1}(\partial\Omega)}
    \leq C\,  \|\phi\|^{}_\partial
  \end{align}
  holds.
  Using these two estimates applied to the respecive terms of \eqref{eq:stokesEquationH1}, we can directly verify \eqref{eq:phiDashintPhi} and \eqref{eq:lambdaun}.

  In order to prove \eqref{eq:deltaun}, we note the following identity of differential operators
  \begin{align*}
    \Delta u \cdot n = n_i \frac{\partial^2 u_i}{\partial x_j^2} = \Big\{ n_i \frac{\partial}{\partial x_j} - n_j \frac{\partial}{\partial x_i} \Big\} \frac{\partial u_i}{\partial x_j},
  \end{align*}
  where we used the fact that $\div u = 0$ in $\overline \Omega$.
  The expression between the brackets is the first-order tangential derivative $\partial_{\tau_{ij}}$, see \eqref{eq:defnTangDerivative}. 
  We can thus see that
  \begin{align*}
    \big| \langle \Delta u \cdot n, u \rangle \big| = \big| \langle \nabla u, \nabla_{\mathrm{tan}} u \rangle\big| \leq 2\, \|\nabla u\|_\partial^2\,,
  \end{align*}
  where we compared the tangential gradient with the usual gradient via the triangle inequality and applied the Cauchy-Schwartz estimate.  
  Identifying $\Delta u \cdot n$ with an element of $\HH^{-1}(\partial\Omega)$, the estimate \eqref{eq:deltaun} follows. 

  For the proof of estimate \eqref{eq:nablaPhin}, we will use $\Ell^2$-estimates for the Neumann and regularity problems for the Laplace equation in Lipschitz domains.
  Jerison and Kenig showed that for $g \in \Ell^2(\partial\Omega)$ with mean value zero the \emph{Neumann problem} for Laplace's equation on the Lipschitz domain $\Omega$ has a unique solution $\psi$ with  $(\nabla \psi)^* \in \Ell^2(\partial\Omega)$, $\frac{\partial \psi}{\partial n} = g$ on $\partial \Omega$ nontangentially and the solution fulfills the estimate $\|\psi\|_{\HH^1(\partial\Omega)} \leq C \|g\|^{}_\partial$, see Theorem \ref{thm:jerisonKenig}.
  By Proposition \ref{prop:approximationArgument}, we see that
  \begin{align*}
    \int_{\partial\Omega} \phi \, \frac{\partial\psi}{\partial x_i} n_i \d\sigma  
    = \int_\Omega \frac{\partial}{\partial x_i} \bigg\{ \phi \frac{\partial\psi}{\partial x_i} \bigg\} d x 
    = \int_\Omega \frac{\partial \phi}{\partial x_i} \cdot \frac{\partial \psi}{\partial x_i} \d x 
    = \int_{\partial\Omega} \psi \, \frac{\partial \phi}{\partial x_i} n_i \d \sigma.
  \end{align*}
  We can then use this identity and the estimate of $\psi$ against the data $g$ to derive
  \begin{align}
    \label{eq:dualityPhi}
    \bigg| \int_{\partial\Omega} \phi \, g \d \sigma\, \bigg|
    &\leq \Big\|\, \frac{\partial \phi}{\partial n}\, \Big\|_{\HH^{-1}(\partial\Omega)} \| \psi \|_{\HH^1(\partial\Omega)} 
    \leq C\, \Big\|\, \frac{\partial \phi}{\partial n} \,\Big\|_{\HH^{-1}(\partial\Omega)} \| g\|^{}_\partial.
  \end{align}
  
  Now, if we set $\overline g = \phi - \tilde \phi$, with $\tilde \phi \coloneqq |\partial \Omega|^{-1} \int_{\partial\Omega} \phi \d \sigma$, we arrive at the following estimate:
  \begin{align*}
      \| \phi - \tilde\phi \|_\partial^2
    = \int_{\partial\Omega} (\phi - \tilde \phi) \, \overline{(\phi - \tilde\phi)} \d\sigma 
    = \int_{\partial\Omega} \phi\, \overline{(\phi - \tilde\phi)} \d\sigma 
    \leq C\, \Big\|\, \frac{\partial\phi}{\partial n}\, \Big\|_{\HH^{-1}(\partial\Omega)} \|\phi - \tilde \phi\|^{}_\partial,
  \end{align*}
  where in the last step we used \eqref{eq:dualityPhi}.
  This together with Lemma \ref{lem:compareBoundaryWithBall} proves the left side of inequality \eqref{eq:nablaPhin}.

  For the right side of inequality \eqref{eq:nablaPhin}, we work in a similar way.
  We will use results for the \emph{regularity problem} of Laplace's equation by Jerison and Kenig, see Theorem \ref{thm:jerisonKenig}:
  Given $f \in \HH^1(\partial\Omega)$, there exists a harmonic function $\psi$ in $\Omega$ such that $(\nabla\psi)^* \in \Ell^2(\partial\Omega)$ and $\psi = f$ on $\partial\Omega$ nontangentially. Furthermore, the estimate $\|\nabla \psi\|^{}_\partial \leq C\, \|f\|_{\HH^1(\partial\Omega)}$ holds.
  As for \eqref{eq:dualityPhi}, we calculate
  \begin{align*}
    \bigg| \int_{\partial\Omega} \frac{\partial\phi}{\partial n} f \d\sigma\, \bigg|
    = \bigg| \int_{\partial\Omega} \phi\, \frac{\partial \psi}{\partial n} \d \sigma\, \bigg| 
    \leq \|\phi\|^{}_{\partial} \|\nabla \psi \|^{}_\partial 
    \leq C\, \|\phi\|^{}_\partial \|f\|_{\HH^1(\partial\Omega)},
  \end{align*}
  Interpreting elements from $\Ell^2(\partial\Omega)$ as functionals on $\HH^1(\partial\Omega)$, we obtain
  \begin{align*}
      \Big\|\,\frac{\partial\phi}{\partial n}\, \Big\|_{\HH^{-1}(\partial\Omega)} \leq C\, \|\phi\|^{}_\partial. 
  \end{align*}
  This finishes the proof of inequality \eqref{eq:nablaPhin}.
\end{proof}

\begin{rem}
  \label{rem:harmonicEstimate}
  A careful look at the proof of inequality \eqref{eq:nablaPhin} reveals that the estimate
  \begin{align*}
      c\, \| \phi \|^{}_\partial \leq \| \nabla \phi \cdot n \|_{\HH^{-1}(\partial\Omega)},
  \end{align*}
  holds for \emph{all} harmonic functions $\phi$ with vanishing mean on $\partial\Omega$ and not only those that correspond to the pressure term of the Stokes resolvent problem.
\end{rem}

After all this preparation we have accquainted enough tools and are now able to prove Theorem \ref{thm:rellich}.

\begin{proof}[Proof of Theorem \ref{thm:rellich}]
  For the proof of estimate \eqref{eq:rellich1}, we can assume without loss of generality that $\int_{\partial\Omega} \phi \d \sigma = 0$.

  We start by proving estimate \eqref{eq:rellich1}. 
  Using \eqref{eq:phiDashintPhi} to bound the second summand in \eqref{eq:rellich1} and then \eqref{eq:gradEstimateRellich2} for $\nabla u$, we get
  \begin{align*}
      \|\nabla u\|^{}_\partial + \|\phi\|^{}_\partial
      &\leq C\, \Big\{ \|\nabla u\|^{}_\partial + |\lambda| \|u \cdot n \|_{\HH^1(\partial\Omega)} \Big\} \\[0.5em]
      &\leq C_\varepsilon\, \Big\{ \big\|\nabla_{\mathrm{tan}} u \big\|^{}_\partial + |\lambda|^{1/2} \|u\|^{}_\partial + |\lambda| \| u\cdot n\|_{\HH^{-1}(\partial\Omega)} \Big\} 
      + C \,\varepsilon \,\Big\{ \|\nabla u\|^{}_\partial + \|\phi\|^{}_\partial \Big\}
  \end{align*}
  for all $\varepsilon \in (0,1)$.
  Chosing $\varepsilon$ such that $C\, \varepsilon < (1/2)$, we can rearrange the above inequality and obtain estimate \eqref{eq:rellich1}.

  Estimate \eqref{eq:rellich2} will need more effort to be proven.
  In order to obtain the desired estimate, we will divide the left hand side of inequality \eqref{eq:rellich2} into two groups as the following line suggests and then bound both groups separately.
  \begin{align*}
      \Big\{ \|\nabla u\|^{}_\partial + |\lambda|\, \|u \cdot n\|_{\HH^{-1}(\partial\Omega)} + \|\phi\|^{}_\partial \Big\} + |\lambda|^{1/2} \|u\|^{}_\partial \eqqcolon G_1 + G_2.
  \end{align*}
  We start with inequality \eqref{eq:lambdaun} and derive
  \begin{align*}
    G_1
    \leq C\, \Big\{ \|\nabla u\|^{}_\partial + \| \phi\|^{}_\partial \Big\}
    \leq C \, \Big\{\, \Big\|\,\frac{\partial u}{\partial \nu}\, \Big\|^{}_\partial + \|\nabla u\|^{}_\partial \Big\},
  \end{align*}
  where in the last step we used the definition of conormal derivatives.
  If we now apply \eqref{eq:gradEstimateRellich}, we get
  \begin{align*}
    G_1
    \leq C_\varepsilon\, \Big\| \, \frac{\partial u}{\partial \nu} \, \Big\|^{}_\partial + \varepsilon \, \Big\{ \|\nabla u\|^{}_\partial + \|\phi\|^{}_\partial + \| |\lambda|^{1/2} u \|^{}_\partial \Big\}
  \end{align*}
  for all $\varepsilon \in (0,1)$.
  Choosing $\varepsilon$ appropriately yields
  \begin{align}
    \label{eq:partOfRellich2}
    \|\nabla u\|^{}_\partial + \|\phi\|^{}_\partial + |\lambda| \, \|u \cdot n\|_{\HH^{-1}(\partial\Omega)}
    \leq C\, \Big\{ \, \Big\|\,\frac{\partial u}{\partial \nu} \, \Big\|^{}_\partial + |\lambda|^{1/2} \| u\|^{}_\partial \Big\},
  \end{align}
  and the first group has been successfully bounded.

  Now we need to estimate $G_2$.
  For this, we will work with the next identity which is a consequence of Green's theorem:
  \begin{align}
    \int_{\partial\Omega} h_k n_k |u|^2 \d \sigma
    %= \int_{\Omega} \frac{\partial}{\partial x_k} \big( h_k |u|^2 ) \d x
    &= \int_{\Omega} \frac{\partial h_k}{\partial x_k}\, |u|^2 \d x + \int_{\Omega} h_k \frac{\partial |u|^2}{\partial x_k}  \d x \nonumber\\[0.5em]
    \label{eq:hknkgreen}
    &= \int_\Omega \div(h)\, |u|^2 \d x + 2 \Re \int_\Omega h_k \frac{\partial \bar u_i}{\partial x_k} u_i \d x,
  \end{align}
  where $h \in \CC_0^1(\R^d, \R^d)$.
  This calculation is valid since Remark \ref{rem:shenNontangential} assures the existence of nontangential limits of $u$ and furthermore gives $(u)^* \in \Ell^2(\partial\Omega)$. Thus, $( h_k n_k |u|^2 )$ can be dominated by the integrable function $(\|h\|_\infty |(u)^*|^2)$ and the claim follows from Proposition \ref{prop:approximationArgument}.

  Next, in estimate \eqref{eq:hknkgreen}, we choose $h \in \CC_0^1(\R^d, \R^d)$ with $h_k n_k \geq c > 0$ on $\partial\Omega$, which is possible due to Theorem \ref{thm:smoothApproximation} and leads us to the estimate 
  \begin{align}
    \label{eq:estupartial}
    \|u\|_\partial^2 \leq C\, \Big\{ \int_\Omega |u|^2 \d x +  \int_\Omega |u| |\nabla u| \d x\, \Big\}.
  \end{align}
  Multiplying  \eqref{eq:estupartial} with $|\lambda|$ and using \eqref{eq:laxMilgramIneq}, we obtain 
  \begin{align*}
    |\lambda|\, \|u\|_\partial^2 
    &\leq C\, \Big\{\, |\lambda| \int_\Omega |u|^2 \d x + |\lambda|^{1/2}  \int_\Omega (|\lambda|^{1/2} |u|) |\nabla u| \d x\, \Big\} \nonumber\\[0.5em]
    %&\leq C \| \frac{\partial u}{\partial \nu} \|_\partial \|u\|_\partial + |\lambda|^{1/2} C \int_\Omega (|\lambda|^{1/2} |u|) |\nabla u| \d x \\
    &\leq C\, \Big\{ \, \Big\|\, \frac{\partial u}{\partial \nu} \, \Big\|^{}_\partial \|u\|^{}_\partial + |\lambda|^{1/2} \bigg( \int_\Omega |\lambda| |u|^2 \bigg)^{1/2} \bigg( \int_\Omega (|\nabla u|^2 \d x \bigg)^{1/2} \Big\} \\[0.5em]
    &\leq C\, \Big\|\,\frac{\partial u}{\partial \nu} \, \Big\|^{}_\partial \, |\lambda|^{1/2}  \|u\|^{}_\partial.
  \end{align*}
  Note that for the last estimate we also used the fact that $|\lambda| \geq \tau$ helps us to bound $\|u\|^{}_\partial$ by $C |\lambda|^{1/2} \|u\|^{}_\partial$.
  Rearranging terms in the last estimate, we now derive
  \begin{align}
    \label{eq:lambda12u}
    |\lambda|^{1/2} \|  u\|^{}_\partial \leq C\, \Big\|\, \frac{\partial u}{\partial \nu} \,\Big\|^{}_\partial.
  \end{align}
  Estimate \eqref{eq:rellich2} follows directly from \eqref{eq:partOfRellich2} in combination with \eqref{eq:lambda12u} and this concludes our proof.
\end{proof}

Shen proved that under reasonable assumptions a theorem similar to Theorem~\ref{thm:rellich} also holds for exterior domains, see \cite[Thm. 4.6]{Shen2012}.
%It is important to note that in the case $d = 2$ solutions $u$ that are given as a single layer potential do not fulfill the stated requirements on the decay.

\begin{thm}
  \label{thm:rellichExterior}
  Let $\lambda \in \Sigma_\theta$ and $|\lambda| \geq \tau$, where $\tau \in (0,1)$.
  Let $(u,\phi)$ be a solution of the Stokes resolvent problem in $\Omega_- = \R^d \setminus \overline\Omega$.
  Suppose additionally that $(\nabla u)^*$, $(\phi)^* \in \Ell^2(\partial\Omega)$ and that $\nabla u$, $\phi$ have nontangential limits almost everywhere on $\partial\Omega$.
  Furthermore, let for $|x| \to \infty$
  \begin{align*}
    |\phi(x)| + |\nabla u(x)| = O(|x|^{1 - d}) \quad\text{and}\quad 
    u(x) = \begin{cases} O(|x|^{2 - d}) &\quad\text{if } d \geq 3, \\ o(1) &\quad\text{if } d = 2. \end{cases}
  \end{align*}
  Then the estimates
  \begin{align}
    \label{eq:rellich1ext}
    \|\nabla u\|^{}_\partial + \|\phi\|^{}_\partial
    \leq C\, \Big\{ \|\nabla_{\mathrm{tan}} u\|^{}_\partial + |\lambda|^{1/2} \|u\|^{}_\partial + |\lambda|\, \|u \cdot n\|_{\HH^{-1}(\partial\Omega)} \Big\}
  \end{align}
  and
  \begin{align}
    \label{eq:rellich2ext}
    \|\nabla u\|^{}_\partial + |\lambda|^{1/2} \|u\|^{}_\partial + |\lambda| \, \|u \cdot n\|_{\HH^{-1}(\partial\Omega)} + \|\phi\|^{}_\partial
    \leq C \, \Big\|\,\frac{\partial u}{\partial \nu} \,\Big\|^{}_\partial
  \end{align}
  hold, where $C$ depends only on $d$, $\tau$, $\theta$ and the Lipschitz character of $\Omega$.
\end{thm}
