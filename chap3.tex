\chapter{Single and Double Layer Potentials}

In this chapter, we will deal with \emph{single} and \emph{double layer potentials}.
Both will serve as ``representation formulas'' for solutions to the Stokes resolvent problem.
We will study their properties as they will serve as the crucial ingredient to solving the Neumann and Dirichlet boundary problems associated to the Stokes resolvent problem.
In this chapter we will always assume that $\Omega$ is a bounded Lipschitz domain in $\R^d$ with $d \geq 2$ and $1 < p < \infty$.
We will also tacitly use the summation convention.

Let $\lambda \in \Sigma_\theta$. 
For $f \in \Ell^2(\partial\Omega; \C^d)$, the single layer potential $u = \slp_\lambda(f)$ is defined by
\begin{align}
  \label{eq:defSingleLayer}
  u_j(x) 
  = \int_{\partial\Omega} \Gamma_{jk}(x - y; \lambda) f_k(y) \d \sigma(y),
\end{align}
where $\Gamma_{jk}$ is the fundamental solution to the Stokes reolvent problem given by \eqref{eq:fundamentalMatrixStokes}.
For the pressure, respectively, we define the single layer potential $\phi = \slp_\Phi(f)$ by
\begin{align}
  \label{eq:defSingleLayerPressure}
  \phi(x) = \int_{\partial\Omega} \Phi_k(x - y) f_k(y) \d \sigma(y),
\end{align}
where $\Phi_k$ is given by \eqref{eq:fundamentalVectorPressure}.
As we have already shown, $(u,\phi)$ defines a solution to the Stokes resolvent problem \eqref{eq:stokesResolventProblem}.

We define two further integral operators
\begin{align}
  T_\lambda^*(f)(P) &= \sup_{t > 0} \big| \int_{\substack{y \in \partial\Omega \\ |y - P| > t}} \nabla_x \Gamma(P - y; \lambda) f(y) \d \sigma(y) \big| \label{eq:supTOperator}\\
  T_\lambda(f)(P) &= \pv \int_{\partial\Omega} \nabla_x \Gamma(P - y; \lambda) f(y) \d \sigma(y) \label{eq:pvTOperator}
\end{align}
for $P \in \partial\Omega$ which will be used to prove boundedness of maximal operators related to $u$.

\begin{lem}
  \label{lem:lpBoundednessT}
  Let $1 < p < \infty$ and $T_\lambda(f), T_\lambda^*(f)$ be defined by \eqref{eq:supTOperator} and \eqref{eq:pvTOperator}.
  Then $T_\lambda(f)(P)$ exists for almost everywhere $P \in \partial\Omega$ and
  \begin{align}
    \label{eq:lpBoundednessT}
    \|T_\lambda(f) \|_{\Ell^p(\partial\Omega)} 
    \leq \|T_\lambda^*(f) \|_{\Ell^p(\partial\Omega)}
    \leq C_p \, \|f\|_{\Ell^p(\partial\Omega)},
  \end{align}
  where $C_p$ depends only on $d$, $\theta$, $p$, and the Lipschitz character of $\Omega$.
\end{lem}

\begin{proof}
  If $\lambda = 0$, the Lemma is known \cite{fabesKenigVerchota} as a consequence of the seminal result of Coifman et al. \cite{coifmanEtAl}.
  One idea of the proof in the case $\lambda \in \Sigma_\theta$ will thus be to nourish from this result and to consider the difference $\Gamma(x - y; \lambda) - \Gamma(x - y; 0)$ as a well-disposed integral kernel.

  We start with the second inequality of \ref{eq:lpBoundednessT}.
  To this end, let $t > 0$ and additionally assume that $t^2 |\lambda| \geq \frac{1}{2}$. 
  Theorem \ref{thm:fundamentalMatrixEstimate} gives
  \begin{align*}
    \Big| \int_{|y - P| > t} \nabla \Gamma(P - y; \lambda) f(y) \d\sigma(y) \Big|
    &\leq C \int_{|P - y| > t} \frac{|f(y)|}{|\lambda| |P - y|^{d + 1}} \d \sigma(y) 
  \end{align*}
  Choose now $N \in \N$ such that $2^N t \leq \diam(\Omega) < 2^{N + 1} t$.
  We now exhaust the domain of integration by suitable annuli and calculate
  \begin{align*}
    &\sum_{k = 0}^N \int_{\BB(P, 2^{k + 1}t) \cap \partial\Omega} \frac{1}{|\lambda| 2^{k(d + 1)} t^{d + 1}} |f(y)| \d \sigma(y) \\
    &\quad\leq \frac{1}{|\lambda|t^2}\frac{1}{2^{1 -d}}\sum_{k = 0}^N \frac{1}{2^{2k}} \frac{1}{(2^{k+1} t)^{d - 1}} \int_{\BB(P, 2^{k + 1}t \cap \partial \Omega)}  |f(y)| \d\sigma(y) \\
    &\quad\leq C \sum_{k = 0}^N \frac{1}{2^{2k}} M_{\partial\Omega}(f)(P) \\
    &\quad\leq C  M_{\partial\Omega}(f)(P)
    %&\quad\leq C |\lambda|^{\frac{1}{2}} \frac{2^{N + 1}t - t}{1}  M_{\partial\Omega}(f)(P) \\
    %&\quad\leq C |\lambda|^{\frac{1}{2}} t
  \end{align*}
  where for the second inequality we used Lemma \ref{lem:compareBoundaryWithBall} to estimate
  \begin{align*}
    \frac{1}{(2^{k+1} t)^{d-1}} \leq C (\sigma(\BB(P, 2^{k + 1}d) \cap \partial\Omega))^{-1}.
  \end{align*}
  which gives the claimed estimate with a constant $C$ that depends on $d$, $\theta$ and the Lipschitz character of $\Omega$.
  Now let $t^2 |\lambda| < \frac{1}{2}$.
  We then split the integral as follows
  \begin{align*}
    &\Big| \int_{|y - P| > t} \nabla \Gamma(P - y; \lambda) f(y) \d\sigma(y) \Big| \\
    &\quad\leq \Big| \int_{|y - P| \geq (2|\lambda|)^{-1/2}} \nabla\Gamma(P - y; \lambda)f(y) \d\sigma\Big|
     + \Big| \int_{t < |y - P| < (2|\lambda|)^{-1/2}} \nabla\Gamma(P - y; \lambda)f(y) \d\sigma\Big|.
  \end{align*}
  The first summand can be estimated like in the step before, if we substitute $t$ by $(2|\lambda|)^{-1/2}$
  For the second term we use the principle of the nutrient zero and estimate
  \begin{align*}
     &\Big| \int_{t < |y - P| < (2|\lambda|)^{-1/2}} \nabla\Gamma(P - y; \lambda)f(y) \d\sigma\Big| \\
     &\qquad\leq \int_{t < |y - P| < (2|\lambda|)^{-1/2}} |\nabla\Gamma(P - y; \lambda) - \nabla\Gamma(P - y; 0) |f(y)| \d\sigma \\
     &\qquad\quad + \Big| \int_{t < |y - P| < (2|\lambda|)^{-1/2}} \nabla\Gamma(P - y; 0) f(y)  \d\sigma \Big|.
  \end{align*}
  We don't need to worry about the second summand here since the corresponding estimate is already coverd by the case of $\lambda = 0$ and therefore
  \begin{align*}
     &\Big| \int_{t < |y - P| < (2|\lambda|)^{-1/2}} \nabla\Gamma(P - y; 0) f(y)  \d\sigma \Big| \\
     &\quad\leq \Big| \int_{|y - P| > t} \nabla\Gamma(P - y; 0) f(y)  \d\sigma \Big| +  \Big| \int_{|y - P| > (2|\lambda|)^{-1/2}} \nabla\Gamma(P - y; 0) f(y)  \d\sigma \Big| \\
     &\quad\leq 2 T_0^*(f)(P).
   \end{align*}
   For the first summand we make use of Theorem \ref{thm:differenceFundamentalSolutionStokes} and more precisely of Corollary \ref{cor:differenceFundamentalSolutionStokes} which unifies all estimates.
   We then calculate
   \begin{align*}
     &\int_{t < |y - P| < (2|\lambda|)^{-1/2}} |\nabla\Gamma(P - y; \lambda) - \nabla\Gamma(P - y; 0) |f(y)| \d\sigma \\
     &\qquad\leq \int_{t < |y - P| < (2|\lambda|)^{-1/2}} |\lambda|^{\frac{1}{2}} |y - P|^{2 - d} |f(y)| \d\sigma,
     \intertext{and as before we choose adequate $N$ such that $2^{N + 1} t > (2|\lambda|)^{-1/2} \geq 2^N t$ which leads to}
     &\qquad\leq |\lambda|^{\frac{1}{2}} \sum_{k = 0}^N \int_{2^kt < |y - P| < 2^{k + 1} t} |y - P|^{2 - d} |f(y)| \d\sigma \\
     &\qquad\leq  |\lambda|^{\frac{1}{2}} t^{2 - d} \sum_{k = 0}^N 2^{k(2 - d)} \int_{\BB(P, 2^{k + 1}t)} |f(y)| \d\sigma\\
     &\qquad\leq 2^d  |\lambda|^{\frac{1}{2}} t \sum_{k = 0}^N 2^{k - 1}\; 2^{(k + 1)(1 - d)} t^{1 - d} \int_{\BB(P, 2^{k + 1}t)} |f(y)| \d\sigma \\
     &\qquad \leq C  |\lambda|^{\frac{1}{2}} t\frac{2^{N} - 1}{1} M_{\partial\Omega}(f)(P) \\
     &\qquad \leq C |\lambda|^{\frac{1}{2}} (2|\lambda|)^{-\frac{1}{2}} M_{\partial\Omega}(f)(P).
   \end{align*}
   Taking now the supremum over all $t > 0$ we see that
   \begin{align*}
     T_\lambda^*(f)(P)
     &\leq C (M_{\partial\Omega}(f)(P) + T_0^*(f)(P)),
   \end{align*}
   for all $P \in \partial\Omega$. Once again using the result for $\lambda = 0$ and the $L^p$-boundedness of the Hardy-Littlewood maximal operator we see that
   \begin{align*}
     \|T_\lambda^*(f) \|_{\Ell^p(\partial\Omega)} \leq C\, \|f\|_{\Ell^p(\partial\Omega)}
   \end{align*}

   To conclude the first inequality in \eqref{eq:lpBoundednessT}, we want to use a standard result from harmonic analysis \cite[2.1.14]{grafakos}.
   First we will show that the integral operator
   \begin{align*}
     T_\lambda(f)(P) = \lim_{t \to 0} \int_{\substack{y \in \partial\Omega \\ |y - P| > t}} \nabla_x \Gamma(P - y; \lambda) f(y) \d\sigma(y)
   \end{align*}
   exists for almost every $P \in \partial\Omega$ and all $f \in \CC(\partial\Omega; \C^d)$.
   In a first step, we can split this operator formally in
   \begin{align*}
     T_\lambda(f)(P) = T_0(f)(P) + \lim_{t \to 0} \int_{\substack{y \in \partial\Omega \\ |y - P| > t}} \nabla_x \{ \Gamma(P - y; \lambda) - \Gamma(P - y; 0)\}  f(y) \d\sigma(y)
   \end{align*}
   The right expression is well defined for $f \in \CC_0^\infty$, once we prove integrability of 
   \begin{align*}
     |\nabla \{\Gamma(P - y; \lambda) - \Gamma(P - y; 0)\} |
   \end{align*}
   on $\partial\Omega$.
   To this end we first note that it suffices to consider the integral
   \begin{align*}
     \int_{|P - y| \leq \varepsilon} | \nabla\{ \Gamma(P - y; \lambda) - \Gamma(P - y; 0) \} | \d\sigma(y), 
   \end{align*}
   for $\varepsilon \leq \min(2|\lambda|^{-1/2}, r_0/4)$ as the integrand is smooth away from $0$ and the domain of integration is bounded.
   Now Corollary \ref{cor:differenceFundamentalSolutionStokes} and Tolksdorf 4.3.2 give that this can be estimated by
   \begin{align*}
     \int_{|P - y| \leq \varepsilon} |\lambda|^{1/2} |P - y|^{2 - d} \d\sigma(y) \leq C |\lambda|^{1/2} \varepsilon \leq C.
   \end{align*}
   Based on the preceding calculation we conclude that for all $f \in \CC(\partial\Omega,\C^d)$ the operator $T_\lambda(f)(P)$ exists whenever $T_0(f)(P)$ exists.
   $T_0(f)(P)$ exists for almost everywhere $P \in \partial\Omega$ because of Fabes, Kenig and Verchota \cite{fabesKenigVerchota}.
   As furthermore $T_\lambda^*(f)(P)$ is bounded on $\Ell^p(\partial\Omega)$  we may now apply Theorem 2.1.14 from Grafakos \cite{grafakos} to conclude that $T_\lambda(f)(P)$ exists now for all $f \in \Ell^p(\partial\Omega; \C^d)$ and almost everywhere $P \in \partial\Omega$.
   The desired $\Ell^p$ estimate on $T_\lambda(f)$ now follows from the observation that 
   \begin{align*}
     | T_\lambda(f)(P) | \leq T_\lambda^*(f)(P)
   \end{align*}
   for almost everywhere $P \in \partial\Omega$.
\end{proof}

For a function $u$ in $\Omega$, we define the nontangential maximal function $(u)^*$ by
\begin{align}
  \label{eq:defnNontangMaxFunction}
  (u)^*(P) = \sup\{ |u(x)| \colon x \in \Omega \text{ and } |x - P| < C \operatorname{dist}(x, \partial\Omega)\}
\end{align}
for $P \in \partial\Omega$, where $C > 2$ is a fixed and sufficiently large constant depending on $d$ and the Lipschitz character of $\Omega$.
Note that in Shen cones we have that for $P, y \in \partial\Omega$ and $x \in \ShenCone(P)$
\begin{align}
  \label{eq:shenConeEstimate}
  |P - y| 
  &\leq |P - x| + |x - y| 
  \leq C \operatorname{dist}(x \partial\Omega) + |x - y|  \nonumber\\
  &\leq (C + 1) |x - y|
\end{align}
where $C$ is the constant from \eqref{eq:defnNontangMaxFunction}

We can now prove the boundedness of certain nontangential maximal operators.
\begin{lem}
  \label{lem:nontangentialMaximalFunctions}
  Let $1 < p < \infty$ and $(u,\phi)$ be given by \eqref{eq:defSingleLayer} and \eqref{eq:defSingleLayerPressure}.
  Let furthermore $d \geq 3$.
  Then 
  \begin{align}
    \| (\nabla u)^* \|_{\Ell^p(\partial\Omega)} + \| (\phi)^* \|_{\Ell^p(\partial\Omega)} + |\lambda|^{\frac{1}{2}} \|(u)^*\|_{\Ell^p(\partial\Omega)} \leq C_p \|f\|_{\Ell^p(\partial\Omega)},
  \end{align}
  where $C_p$ depends only on $d$, $\theta$, $p$ and the Lipschitz character of $\Omega$.
\end{lem}

\begin{proof}
  A proof of the estimate $\|(\phi)^*\|_{\Ell^p(\partial\Omega)} \leq C_p \|f\|_{\Ell^p(\partial\Omega)}$ can be found in Verchota \cite{verchota}.
  The proof for $\|(\nabla u)^*\|_{\Ell^p(\partial\Omega)}$ works in the same way. 
  We will provide a proof for the sake of completeness.
  To immitate the proof of Verchota, we will work with the corresponding type of cones.
  Therefore the results for $\nabla u$ and $\phi$ will at first only be established for the type of maximal operators defined by Verchota.
  The transferability to Shen's maximal operators is given by Tolksdorf \cite{tolksdorfDiss} as the solution $(u,\phi)$ has a representation as a single layer potential.

  Let $x \in \verCone(P)$ and set $t = |x - P|$.
  Then,
  \begin{align*}
    |(\nabla u)(x) |
    &= \big| \int_{\partial\Omega} \nabla \Gamma_{jk} (x - y; \lambda) f_k \d \sigma(y)\big| \\
    &\leq \big| \int_{|y - P| > t} \nabla\Gamma_{jk}(x - y; \lambda) f_k \d \sigma(y) \big| + \big| \int_{|y - P| \leq t} \nabla \Gamma_{jk}(x - y; \lambda) f_k \d \sigma(y)\big| \\
    &= I_1 + I_2.
  \end{align*}
  We will now estimate $I_1$ and $I_2$ seperately.
  Note that in Verchota cones we have that for all $Q \in \partial\Omega$ we have $|x - Q| > C |x - P|$, where $C$ is a constant only depending on $d$ and the Lipschitz character of $\Omega$.
  By Theorem \ref{thm:fundamentalMatrixEstimate} we know that
  \begin{align*}
    I_2 
    &\leq C \int_{|y - P| \leq t} \frac{1}{|x - y|^{d - 1}} |f(y)| \d \sigma(y) \\
    &\leq \frac{C}{t^{n - 1}} \int_{|y - P| \leq t} |f(y)| \d \sigma(y)
    \leq C M_{\partial\Omega} (f)(P).
  \end{align*}
%  For $I_1$, we calculate
%  \begin{align*}
%    &\Big| \int_{| y - P | > t} \nabla \Gamma_{j k} (x - y; \lambda) f_k(y) - \nabla \Gamma_{jk}(P - y; \lambda) f_k(y) + \nabla \Gamma_{jk}(P - y; \lambda) f_k(y) \d \sigma(y) \Big| \\
%    &\quad\leq \Big| \int_{|y - P| > t} \nabla\Gamma_{jk} (x - y; \lambda) f_k(y) - \nabla\Gamma_{jk}(P - y; \lambda) f_k(y) \d \sigma(y) \Big| \\
%    &\qquad+ \Big| \int_{|y - P| > t} \nabla\Gamma_{jk}(P - y; \lambda) f_k(y) \d\sigma(y) \Big|.
%  \end{align*}
   For $I_1$, we calculate
   \begin{align*}
     &\Big| \int_{| y - P | > t} \nabla \Gamma_{j k} (x - y; \lambda) f_k(y) - \nabla \Gamma_{jk}(x - y; 0) f_k(y) + \nabla \Gamma_{jk}(x - y; 0) f_k(y) \d \sigma(y) \Big| \\
     &\quad\leq \Big| \int_{|y - P| > t} \nabla\Gamma_{jk}(x - y; \lambda) f_k(y) - \nabla\Gamma_{jk}(P - y; \lambda) f_k(y) \d \sigma(y) \Big| \\
     &\qquad+ \Big| \int_{|y - P| > t} \nabla\Gamma_{jk}(P - y; \lambda) f_k(y) \d\sigma(y) \Big|.
   \end{align*}
  The second summand can directly be estimated by $T_\lambda^*(f)(P)$.
  For the second one we apply the mean value theorem and derive using once again Theorem \ref{thm:fundamentalMatrixEstimate}
  \begin{align*}
    &\int_{|y - P| > t} \big| \nabla\Gamma_{jk}(x - y; \lambda) - \nabla\Gamma_{jk}(P - y; \lambda) \big| |f(y)| \d\sigma(y) \\ 
    &\qquad\leq \int_{|y - P| > t} |\nabla^2 \Gamma_{jk}(s - y; \lambda)| |x - P| |f(y) | \d\sigma(y) \\
    &\qquad\leq C t \int_{|y - P| > t} \frac{1}{|s - y|^{d}} |f(y)| \d\sigma(y) \\
    &\qquad\leq C t \int_{|y - P| > t} \frac{1}{|y - P|^{d}} |f(y)| \d\sigma(y) \\
    &\qquad\leq C \int_{\partial\Omega} \frac{t}{(t + |y - P|)^d} |f(y)| \d\sigma(y).
  \end{align*}
  where $s$ is an element on the line connecting $x$ and $P$ and we used the property of Verchota-cones that $|s - y| \geq C |y - P|$.
  Note that Verchota cones are convex.
  By exhausting the domain of integration using annuli, we can estimate this integral by $M_{\partial\Omega}f(P)$:
  Choose $N$ such that $2^N t \leq \diam(\Omega) < 2^{N + 1}t$.
  Then
  \begin{align*}
    &\int_{|y - P| > t} \frac{1}{|y - P|^{d - 1}} |f(y)| \d\sigma(y) \\
    &\quad= \sum_{k = 0}^N \int_{2^{k + 1}t > |y - P| \geq 2^k t} \frac{1}{|y - P|^{d - 1}} |f(y)| \d\sigma(y) \\
    &\quad\leq \sum_{k = 0}^N \frac{1}{2^{k(d - 1)} t^{d - 1}} \int_{|y - P| < 2^{k + 1} t} |f(y)| \d\sigma(y) \\
    &\quad \leq  C \sum_{k = 0}^N 2^{k - d + 1)} M_{\partial\Omega}(f)(P)
    &\quad\leq
  \end{align*}
  Taking the supremum over all $x \in \Omega$ the claim follows.

  We will now work on the proof of the estimate for $(u)^*$.
  In order to derive $\Ell^p$ estimates on this maximal operator we will work directly with the Definition of the single layer potential \eqref{eq:defSingleLayer}.
  For $P \in \partial\Omega$, estimate \eqref{eq:fundamentalMatrixEstimate} together with the estimate for Shen cones \eqref{eq:shenConeEstimate} gives that for all $x \in \ShenCone(P)$
  \begin{align*}
    |u^*(x)| 
    &\leq  C \int_{\partial\Omega} \frac{1}{|x - y|^{d - 2}} |f(y)| \d\sigma(y) 
    \leq  C \int_{\partial\Omega} \frac{1}{|P - y|^{d - 2}} |f(y)| \d\sigma(y),
  \end{align*}
  where $C$ only depends on $d$, $\theta$ and the Lipschitz character of $\Omega$.
  Passing to the maximal operator yields the inequality
  \begin{align*}
    u^*(P) \leq  C \int_{\partial\Omega} \frac{1}{|P - y|^{d - 2}} |f(y)| \d\sigma(y),
  \end{align*}
  We are now left with the task to estimate the integral 
  \begin{align*}
    \int_{\partial\Omega} \frac{1}{|P - y|^{d - 2}} \d\sigma(y)
  \end{align*}
  uniformly for all $P \in \partial\Omega$, as the rest can be handled using the Young inequality.
  Let $r_0$ be the Radius from the definition of Lipschitz cylinders.
  Then
  \begin{align*}
    \int_{\partial\Omega} \frac{1}{|P - y|^{d - 2}} \d\sigma(y)
    &\leq \int_{\partial\Omega \cap \BB(P; r_0/4)} \frac{1}{|P - y|^{d - 2}} \d\sigma(y) + \int_{\partial\Omega \setminus \BB(P; r_0/4)} \frac{1}{|P - y|^{d - 2}} \d\sigma(y). \\
    & \leq C r_0/4 +  \sigma(\partial\Omega) r_0^{2 - d} 4^{d - 2}.
  \end{align*}
  where $C$ only depends on $d$ and the Lipschitz character of $\Omega$.
\end{proof}

The next Lemma deals with \emph{trace formulas} for $\nabla u$ and $\phi$. We can now finally talk about boundary values as the existence of nontangential limits guarantees that there exists something on $\partial\Omega$ that is related to the function inside $\Omega$.

\begin{lem}
  \label{lem:traceFormulas}
  Let $(u,\phi)$ be given by \eqref{eq:defSingleLayer} and \eqref{eq:defSingleLayerPressure} with $f \in \Ell^p(\partial\Omega; \C^d)$ and $1 < p < \infty$.
  Then
  \begin{align}
    \big( \frac{\partial u_i}{\partial x_j} \big)_{\pm}(x) 
    &= \pm \frac{1}{2} \{ n_j(x) f_i(x) - n_i(x) n_j(x) n_k(x) f_k(x) \} \nonumber\\
    &\quad+ \pv \int_{\partial\Omega} \frac{\partial}{\partial x_j} \{ \Gamma_{ik} (x - y; \lambda) \} f_k(y) \d\sigma(y), \label{eq:traceFormula} \\
    \phi_\pm(x) &= \mp \frac{1}{2} n_k(x) f_k(x) + \pv\int_{\partial\Omega} \Phi_k(x - y) f_k(y) \d \sigma(y) \nonumber
  \end{align}
  for almost everywhere $x \in \partial\Omega$.
  The subscripts $+$ and $-$ indicate nontangential limits taken inside $\Omega$ and outside $\overline\Omega$, respectively.
\end{lem}

\begin{proof}
  The correctness of the trace formulas \eqref{eq:traceFormula} is known for the case $\lambda = 0$ since Fabes, Kenig and Verchota \cite{fabesKenigVerchota}.
  This fact will now be reused for $\lambda \in \Sigma_\theta$.
  We insert a $0$ to the nontangential limit as
  \begin{align*}
    (\nabla u_j)_\pm(x) = 
    (\nabla v_j)_\pm(x) + (\nabla u_j - \nabla v_j)_\pm(x),
  \end{align*}
  where $v_j(x) = \int_{\partial\Omega} \Gamma_{jk}(x - y; 0) f_k(y) \d\sigma(y)$.
  Because of \cite{fabesKenigVerchota} we know that the first nontangential limit exists and is given by \eqref{eq:traceFormula} with $\lambda = 0$.
  It therefore remains to show that
  \begin{align*}
    (\nabla u_j - \nabla v_j)_\pm(x) = \int_{\partial\Omega} \nabla \{ \Gamma_{jk}(x - y; \lambda) - \Gamma_{jk}(x - y; 0) \} f_k(y) \d\sigma(y)
  \end{align*}
  for all $x \in \partial \Omega$.
  To this end let $(x_l)_{l \in \N}$ a sequence in $\ShenCone(x)$ with $\lim_{l \to \infty} x_l = x$.
  Furthermore let us note that for almost everywhere $x \in \partial\Omega$ we have that 
  \begin{align*}
    \int_{\partial\Omega} \frac{1}{|x - y|^{d - 2}} |f(y)| \d\sigma(y) < \infty.
  \end{align*}
  This is a consequence of the fact that
  \begin{align*}
    \sup_{x \in \partial\Omega} \Big| \int_{\partial\Omega} \frac{1}{|x - y|^{d - 2}} \d\sigma(y) \Big| < \infty
  \end{align*}
  and an application of Young's inequality which can be found in Tolksdorf \cite{tolksdorfDiss}:
  Let $x \in \partial\Omega$.
  Then
  \begin{align*}
    &\int_{\partial\Omega} \frac{1}{|x - y|^{d - 2}} \d\sigma(y) \\
    &\quad\leq \int_{\partial\Omega \cap \BB(x,r_0/4)} \frac{1}{|x - y|^{d - 2}} \d\sigma(y) + \int_{\partial\Omega \setminus \BB(x, r_0/4)} \frac{1}{|x - y|^{d - 2}} \d\sigma(y) \\
    &\quad\leq C r_0 + r^{2 - d} 4^{d - 2} \sigma(\partial\Omega)
  \end{align*}
  by Lemma \ref{lem:compareBoundaryWithBall}.
  Now Young's inequality gives us the desired result.
  In the next step we will show that
  \begin{align*}
    \frac{1}{|x -y|^{d - 2}} |f(y)|
  \end{align*}
  gives a suitable function for dominated convergence.
  Set $\varepsilon = (4 |\lambda|^2)^{-1}$ and without loss of generality assume that $\supp f \subseteq \BB(x,\varepsilon)$.
  Furthermore assume that $|x_l - x| < \varepsilon$ for all $l \in \N$.
  Then $|x_l - y| \leq (2|\lambda|^2)^{-1}$ and Corollary \ref{cor:differenceFundamentalSolutionStokes} give
  \begin{align*}
    & (\nabla u_j - \nabla v_j)(x_l) \int_{\partial\Omega} \nabla \{ \Gamma_{jk}(x_l - y; \lambda) - \Gamma_{jk}(x_l - y; 0)\} f_k(y) \d\sigma(y) \\
    &\quad\leq \int_{\partial\Omega} \frac{1}{\sqrt{|\lambda|}|x_l - y|^{d - 2}} |f(y)| \d\sigma(y) \\
    &\quad\leq \frac{C}{\sqrt{|\lambda|}} \int_{\partial\Omega} \frac{1}{|x - y|^{d - 2}} |f(y)| \d\sigma(y) < \infty.
  \end{align*}
  Now dominated convergence gives the claim for $x_l \to x$.
  Note that it does not affect the proof if the sequence $x_l$ lays inside $\Omega$ or outside $\overline\Omega$.
\end{proof}

The previous Lemma enables us to talk about boundary values of partial derivatives. 
The next theorem will now give a similar result but for conormal derivatives which are defined by
\begin{align*}
  \frac{\partial u}{\partial \nu} = \frac{\partial u}{\partial n} - \phi n.
\end{align*}
We will also be working with tangential derivatives which are defined via
\begin{align*}
  DEFINE
\end{align*}

\begin{thm}
  \label{thm:jumpConditions}
  Let $\lambda \in \Sigma_\theta$ and $\Omega$ be a bounded Lipschitz domain in $\R^d$, $d \geq 3$. 
  Let $(u,\phi)$ be given by \eqref{eq:defSingleLayer} and \eqref{eq:defSingleLayerPressure} with $f \in \Ell^p(\partial\Omega; \C^d)$ and $1 < p < \infty$.
  Then $\nabla_{\mathrm{tan}} u_+ = \nabla_{\mathrm{tan}} u_-$ and
  \begin{align}
    \label{eq:nontangentialConormalDerivative}
    \Big( \frac{\partial u}{\partial \nu} \Big)_\pm = \Big( \pm \frac{1}{2} I + \K_\lambda \Big) f
  \end{align}
  on $\partial\Omega$, with $\K_\lambda$ a bounded operator on $\Ell^p(\partial\Omega; \C^d)$ with
  \begin{align*}
    \| \K_\lambda f \|_{\Ell^p(\partial\Omega)} \leq C_p \|f\|_{\Ell^p(\partial\Omega)},
  \end{align*}
  where $C_p$ depends only on $d$, $\theta$, $p$ and the Lipschitz character of $\Omega$.
\end{thm}

\begin{proof}
  For the the $j$th component of the tangential derivative of $u_i$, $1\leq i,j \leq d$, we calculate using the results from Lemma \ref{lem:traceFormulas}
  \begin{align*}
    ((\nabla_{\mathrm{tan}} u_i)_+)_j
    &= (\frac{\partial u_i}{\partial x_j})_+ - \langle (\nabla u_i)_+, n \rangle n_j \\
    &= (\frac{\partial u_i}{\partial x_j} )_+ - (\frac{\partial u_i}{\partial x_k} )_+ n_k n_j \\
    &= \frac{1}{2} \{ n_j f_i - n_i n_j n_k f_k\} - \frac{1}{2} \{ n_k f_i - n_i n_k n_l f_l \} n_k n_j  \\
    &\quad+ \pv \int_{\partial\Omega} \frac{\partial}{\partial x_j} \{ \Gamma_{ik} (x - y; \lambda) \} f_k(y) \d\sigma(y) \\
    &\quad+ \pv \int_{\partial\Omega} \frac{\partial}{\partial x_k} \{ \Gamma_{il} (x - y; \lambda) \} f_l(y) \d\sigma(y) n_k n_j.
  \end{align*}
  As the first two summands add up to zero, the entire expression does not depend on the direction of the nontangential limit. 
  This gives
  \begin{align*}
    (\nabla_{\mathrm{tan}} u)_+ = (\nabla_{\mathrm{tan}} u)_-
  \end{align*}
  We calculate for the $j$th component of the nontangential limit of the conormal derivative of $u$ at $x \in \partial\Omega$ using the results from Lemma \ref{lem:traceFormulas}
  \begin{align*}
    &\big(\frac{\partial u_j}{\partial x_i}\big)_+(x) n_i - \phi_+(x) n_j\\
    &\quad= \frac{1}{2} \{ n_i f_j(x) - n_j n_i n_k f_k(x) \} n_i + \pv \int_{\partial\Omega} \frac{\partial}{\partial{x_i}} \big\{ \Gamma_{jk}(x - y; \lambda) \big\} f_k(y) \d\sigma(y) n_i \\
    &\qquad+ \frac{1}{2} n_k f_k(x) n_j - \pv \int_{\partial\Omega} \Phi_k(x - y) f_k(y) \d\sigma(y) n_j \\
    &\quad= \frac{1}{2} f_j(x) + (\K_\lambda f)_j(x),
  \end{align*}
  where $n$ denotes the normal vector at $x$ and
  \begin{align}
    \label{eq:defnKlambda}
    (\K_\lambda f) (x)
    &= \pv \int_{\partial\Omega} \nabla_x \Gamma(x - y; \lambda)  f(y) \d\sigma(y) n - \pv \int_{\partial\Omega} \Phi_k(x - y) f_k(y) \d\sigma(y) n.
  \end{align}
  We note that $\K_\lambda$ essentially consists of two boundary layer potentials. 
  The $\Ell^p$-boundedness of the first one was proven in Lemma \ref{lem:lpBoundednessT}.
  The $\Ell^p$-boundedness of the second boundary layer potential follows in an analogous way using the fact that the operators
  \begin{align*}
    A^*(f)(P) = \sup_{t > 0} \Big| \int_{\substack{y \in \partial\Omega \\ |y - P| > t}} \frac{P - y}{|P - y|^n} f(y) \d\sigma(y)\Big| , \quad P \in \partial\Omega
  \end{align*}
  are bounded by Lemma 1.2 of Verchota \cite{verchotaDiss}.
\end{proof}

Similar to $\K_\lambda$ for $\lambda = 0$ we have
\begin{align}
  \label{eq:defnK0}
    (\K_0 f)(x)= \pv \int_{\partial\Omega} \nabla_x \Gamma(x - y; 0)  f(y) \d\sigma(y) n - \pv \int_{\partial\Omega} \Phi_k(x - y) f_k(y) \d\sigma(y) n,
\end{align}
as was shown by Fabes, Kenig and Verchota \cite[(0.12)]{fabesKenigVerchota}.

We now note a fact that will be crucial for solving the $\Ell^2$ Dirichlet problem in Chapter 5 and will fortify the hopes of translating results for $\lambda = 0$ to $\lambda \in \Sigma_\theta$.
\begin{lem}
  \label{lem:compactness}
  Let $\lambda \in \Sigma_\theta$ and $d \geq 3$.
  Then the operator $\K_\lambda - \K_0$ on $\Ell^2(\partial\Omega; \C^d)$ is compact.
\end{lem}

\begin{proof}
  The idea of this proof is similar to the one in Tolksdorf \cite[Lemma 4.3.5]{tolksdorfDiss}.
  Let $f \in \Ell^2(\partial\Omega; \C^d)$.
  Let's denote $\K \coloneqq \K_\lambda - \K_0$. 
  We will now try to approximate $\K$ by compact operators in the operator norm.
  To this end we define for all $\varepsilon > 0$
  \begin{align*}
    (\K^{(\varepsilon)}f)(x) \coloneqq \int_{\partial\Omega \setminus \BB(x,\varepsilon)} \nabla \{ \Gamma(x - y; \lambda) - \Gamma(x - y; 0) \} f(y) \d \sigma(y), \quad x \in \partial\Omega.
  \end{align*}
  We can now estimate by Young's inequality \ref{lim:young} 
  \begin{align*}
    \| (\K^{(\varepsilon)} f)(x) \|_{\Ell^2(\partial\Omega)}
    \leq \sup_{p \in \partial\Omega} \| \nabla \{ \Gamma(p - \cdot; \lambda) - \Gamma(p - \cdot; 0) \} 1_{\BB(p,\varepsilon)} \|_{\Ell^1(\partial\Omega)} \|f\|_{\Ell^2(\partial\Omega)}.
  \end{align*}
  Our goal is to show that
  \begin{align*}
    \sup_{p \in \partial\Omega} \| \nabla \{ \Gamma(p - \cdot; \lambda) - \Gamma(p - \cdot; 0) \} 1_{\BB(p,\varepsilon)} \|_{\Ell^1(\partial\Omega)} \to 0 \quad \text{as} \quad \varepsilon \to 0.
  \end{align*}
  To this end, let $\varepsilon$ be small enough such that we can apply the estimates from Corollary \ref{cor:differenceFundamentalSolutionStokes} to calculate for some $p \in \partial\Omega$
  \begin{align*}
     &\| \nabla \{ \Gamma(p - \cdot; \lambda) - \Gamma(p - \cdot; 0) \} 1_{\BB(p,\varepsilon)} \|_{\Ell^1(\partial\Omega)}  \\
     &\qquad\qquad\leq C \int_{\partial\Omega \cap \BB(p, \varepsilon)} \sqrt{|\lambda|} |p - y|^{2 - d} \d\sigma(y) 
     \leq C \sqrt{|\lambda|} \varepsilon
  \end{align*}
  where for the last step we applied Lemma \ref{lem:comparability}.
  For $\varepsilon \to 0$ this gives us $\K^{(\varepsilon)} \to \K$ in the operator norm.
  The last step is to verify the compactness of $\K^(\varepsilon)$.
  We note that the integral kernel of $\K^{(\varepsilon)}$ is bounded which gives us that in particular the kernel is an element of $\Ell^2(\partial\Omega \times \partial\Omega; \C^{d \times d}$.
  The compactness of $\K^{(\varepsilon)}$ now follows from Weidmann \cite[Thm. 6.11]{weidmann}.
\end{proof}

Our next step is to introduce the \emph{double layer potential} $u(x) = \dlp_\lambda(f)(x)$ for the Stokes resolvent problem, where
\begin{align}
  \label{eq:defDoubleLayer}
  u_j(x) = \int_{\partial\Omega} \Big\{ \frac{\partial}{\partial y_i} \{ \Gamma_{jk}(y - x; \lambda) \} n_i(y) - \Phi_j(y - x) n_k(y) \Big\} f_k(y) \d\sigma(y).
\end{align}
The corresponding pressure $\phi(x) = \dlp_{\phi}(f)(x)$ is defined via
\begin{align}
  \label{eq:defDoubleLayerPressure}
  \phi(x)
  &= \frac{\partial^2}{\partial x_i \partial x_k} \int_{\partial\Omega} G(y - x; 0) n_i(y) f_k(y) \d\sigma(y) + \lambda \int_{\partial\Omega} G(y - x; 0) n_k(y) f_k(y) \d\sigma(y).
\end{align}
Using \ref{eq:fundamentalVectorPressure} and \ref{eq:solutionStokesSystem} one can show that $(u,\phi)$ defines again a solution to the Stokes resolvent problem in $\R^d \setminus \partial\Omega$.

The next theorem will give us a suitable operator which maps a given function $f \in \Ell^p(\partial\Omega; \C^d)$ to boundary values of $u = \dlp_\lambda(f)$.

\begin{thm}
  \label{thm:nontangentialLimitDoubleLayer}
Let $\lambda \in \Sigma_\theta$ and $\Omega$ be a bounded Lipschitz domain in $\R^d$, $d \geq 3$.
Let $u$ be given by \eqref{eq:defDoubleLayer} for $f \in \Ell^p(\partial\Omega; \C^d)$, $1 < p < \infty$.
Then
\begin{align}
  \label{eq:lpBoundednessUNontangentialMax}
  \|(u)^*\|_{\Ell^p(\partial\Omega)} \leq C_p \|f\|_{\Ell^p(\partial\Omega)}
\end{align}
where $C_p$ depends only on $d$, $p$, $\theta$ and the Lipschitz character of $\Omega$.
Furthermore 
\begin{align}
  \label{eq:nontangentialLimitDoubleLayer}
  u_\pm = \Big(\mp \frac{1}{2} I + \K_{\bar\lambda}^* \Big) f,
\end{align}
where $K_{\bar\lambda}^*$ is the adjoint of the operator $K_{\bar\lambda}$ in \eqref{eq:nontangentialConormalDerivative}
\end{thm}

\begin{proof}
  The estimate for $(u)^*$ is a direct consequence of Lemma \ref{lem:nontangentialMaximalFunctions}, in particular of the estimates on $(\nabla u)^*$ and $(\phi)^*$.

  For the proof of \eqref{eq:nontangentialLimitDoubleLayer}, we begin by determining the adjoint of the operator $\K_{\bar\lambda}$.
  To this end we fill first work with truncated operators $\K_{\lambda}^{(\varepsilon)}$ which are defined as
  \begin{align*}
    (\K_\lambda^{(\varepsilon)} f) (x)
    &= \int_{\partial\Omega} 1_{\EE(x,\varepsilon)} \nabla_x \Gamma(x - y; \lambda)  f(y) \d\sigma(y) n - \int_{\partial\Omega} 1_{\EE(x,\varepsilon)} \Phi_k(x - y) f_k(y) \d\sigma(y) n,
  \end{align*}
  for $x \in \partial\Omega$ and $\EE(x,\varepsilon) \coloneqq \R^d \setminus \BB(x,\varepsilon)$.
  Now for $f \in \Ell^p(\partial\Omega; \C^d)$ and $g \in \Ell^q(\partial\Omega; \C^d)$ with $1/p + 1/q = 1$ we calculate
  \begin{align*}
    \langle \K_{\bar\lambda}^{(\varepsilon)}f, g \rangle
    &= \int_{\partial\Omega} (\K_{\bar\lambda}^{(\varepsilon)} f_j)(x) \overline{g_j(x)} \d\sigma(x) \\
    &= \int_{\partial\Omega} \int_{\partial\Omega} \frac{\partial}{\partial x_i} \{ \Gamma_{jk} (x - y; \overline\lambda) \} f_k(y) 1_{\EE(x,\varepsilon}(y) \d\sigma(y) n_i(x) \overline{g_j(x)} \d\sigma(x) \\
    &\quad+ \int_{\partial\Omega} \int_{\partial\Omega} \Phi_k(x -y)f_k(y) 1_{\EE(x,\varepsilon)}(y) \d\sigma(y) n_j(x) \overline{g_j(x)} \d\sigma(x).
  \end{align*}
  Note that $1_{\EE(x,\varepsilon)}(y) = 1_{\EE(y,\varepsilon)}(x)$.
  Now an application of Fubini and factoring out $f_k(y)$ gives that the lengthy expression is equal to
  \begin{align*}
    \int_{\partial\Omega} f_k(y) \int_{\partial\Omega} &\Big\{ \frac{\partial}{\partial x_i} \{ \Gamma_{jk}(x - y; \bar\lambda)\} n_i(x)
    - \Phi_k(x - y)  n_j(x) \Big\}1_{\EE(y, \varepsilon)}(x) \overline{g_j(x)} \d\sigma(x) \d\sigma(y).
  \end{align*}
  Therefore we see that the adjoint of the truncated operator $\K_{\bar\lambda}^{(\varepsilon)}$ is given by
  \begin{align*}
    \big( ( K_{\bar\lambda}^{(\varepsilon)})^* g\big)_k(y)
    = \int_{\partial\Omega} \Big\{ \frac{\partial}{\partial x_i} \{ \Gamma_{jk}(x - y; \lambda)\} n_i(x) - \Phi_k(x - y) n_j(x) \Big\} 1_{\EE(y,\varepsilon)}(x) g_j(x) \d\sigma(x), 
  \end{align*}
  for $y \in \partial\Omega$ since $\overline{\Gamma_{jk}(x - y; \lambda)} = \Gamma_{jk}(x - y; \bar\lambda)$.

  In the next step we will go from truncated operators to principal value operators. 
  For this to work we will look for suitable majorants.
  If $x \in \partial\Omega$ we estimate
  \begin{align*}
    |(\K_{\bar\lambda}^{(\varepsilon)}f)_j (x)|
    &= \Big| \int_{|x - y| > \varepsilon} \frac{\partial}{\partial x_i} \{ \Gamma_{jk}(x - y; \lambda) \} f_k(y) \d\sigma(y) n_i(x)  \\
    &\qquad - \int_{|x - y| > \varepsilon} \Phi_k(x - y) f_k(y) n_j(x) \d\sigma(y) \Big| \\
    &\leq T_\lambda^*(f)(x) + A^*(f)(x).
  \end{align*}
  Now dominated convergence gives
  \begin{align*}
    \lim_{\varepsilon \to 0} \langle K_{\bar\lambda}^{(\varepsilon)} f, g \rangle = \langle K_{\bar\lambda} f, g\rangle.
  \end{align*}
  A similar argument gives
  \begin{align*}
    \lim_{\varepsilon \to 0} \langle f, {K_{\bar\lambda}^{(\varepsilon)}}^{(*)} g \rangle = \langle f,K_{\bar\lambda}^{(*)}  g\rangle,
  \end{align*}
  where
  \begin{align*}
    \big( ( K_{\bar\lambda}^* g\big)_k(y)
    = \pv\int_{\partial\Omega} \Big\{ \frac{\partial}{\partial x_i} \{ \Gamma_{kj}(x - y; \lambda)\} n_i(x) - \Phi_k(x - y) n_j(x) \Big\} g_j(x) \d\sigma(x). 
  \end{align*}
  Note that we have used the symmetry of $(\Gamma_{\alpha\beta})$.

  The last part now consists of proving that the equality \eqref{eq:nontangentialLimitDoubleLayer} holds.
  To simplify the calculations and make Lemma \ref{lem:traceFormulas} more accessible note that on the one hand
  \begin{align*}
    \int_{\partial\Omega} \frac{\partial}{\partial y_i} \{ \Gamma_{jk}(y - x; \lambda) \} n_i(y) f_k(y) \d\sigma(y)
    &= -\int_{\partial\Omega} \frac{\partial}{\partial x_i} \{ \Gamma_{jk}(x - y; \lambda) \} n_i(y) f_k(y) \d\sigma(y)\\
    &= -\frac{\partial}{\partial x_i} \slp(n_i f)_j(x)
  \end{align*}
  and on the other hand
  \begin{align*}
    -\int_{\partial\Omega} \Phi_j(y - x) n_k(y) f_k(y) \d\sigma(y)
    = \int_{\partial\Omega} \Phi_l(x - y) \delta_{lj} n_k(y) f_k(y) \d\sigma(y)
    = \slp_\Phi(\tilde f^j)(x), 
  \end{align*}
  where $\tilde f^j_l = \delta_{lj} n_k f_k$.
  For $x \in \partial\Omega$ we can now calculate
  \begin{align*}
    &\Big( \int_{\partial\Omega} \frac{\partial}{\partial y_i} \{ \Gamma_{jk}(y - \cdot\, ; \lambda) \} n_i(y) f_k(y) \d\sigma(y) \Big)_\pm(x)\\
    &\qquad= - \big( \frac{\partial}{\partial x_i} \slp_\lambda(n_i f)_j\big)_\pm(x) \\
    &\qquad= \mp \frac{1}{2} \{ n_i(x) n_i(x) f_j(x) - n_j(x) n_i(x) n_k(x) n_i(x) f_k(x)  \} \\
    &\qquad\quad - \pv\int_{\partial\Omega} \frac{\partial}{\partial x_i} \{\Gamma_{jk}(x - y; \lambda) \} n_i(y) f_k(y) \d\sigma(y) \\
    &\qquad= \mp \frac{1}{2} \{ f_j(x) - n_j(x) n_k(x) f_k(x) \} \\
    &\qquad\quad + \pv\int_{\partial\Omega} \frac{\partial}{\partial y_i} \{\Gamma_{jk}(x - y; \lambda) \} n_i(y) f_k(y) \d\sigma(y),
  \end{align*}
  where we used trace formula \eqref{eq:traceFormula}.
  A similar procedure for the second integral part of the double layer potential gives
  \begin{align*}
    &-\Big(\int_{\partial\Omega} \Phi_j(y - \cdot\,) n_k(y) f_k(y) \d\sigma(y)\Big)_\pm(x) \\
    &\qquad= \big(\slp_\Phi(\tilde f^j)\big)_\pm(x)  \\
    &\qquad= \mp \frac{1}{2} n_k(x) \tilde f^j_k(x) - \pv \int_{\partial\Omega} \Phi_k(x - y) \tilde f^j_k(x) \d\sigma(y) \\
    &\qquad= \mp \frac{1}{2} n_j(x) n_k(x) f_k(x) - \pv \int_{\partial\Omega} \Phi_j(x - y) n_k(x) f_k(x) \d\sigma(y)
  \end{align*}
  Putting everything together we get
  \begin{align*}
    (u_j)_\pm(x) = \mp\frac{1}{2} f_j(x) + (K_{\bar\lambda}^* f)_j(x)
  \end{align*}
  which proves the claim.
\end{proof}
