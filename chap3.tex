\chapter{Single and Double Layer Potentials}
\label{chap:3}

In this chapter, we will deal with \emph{single} and \emph{double layer potentials}.
Both will serve as ``representation formulas'' for solutions to the Stokes resolvent problem.
We will study their properties as they will serve as the crucial ingredient to solving the $\Ell^2$ Dirichlet problem associated to the Stokes resolvent problem on bounded Lipschitz domains $\Omega \subseteq \R^d$:
For $\lambda \in \C \setminus (-\infty,0)$ and
\begin{align*}
  g \in \Ell^2_n(\partial\Omega) \coloneqq \bigg\{g \in \Ell^2(\partial\Omega; \C^d) \colon \int_{\partial\Omega} g \cdot n \d \sigma = 0\bigg\}
\end{align*}
we are looking for smooth functions $u$ and $\phi$ that satisfy
{
  \label{eq:dirProblem}
\begin{align*}
  (\mathrm{Dir}_\lambda)
  \left\{
  \begin{alignedat}{1}
    -\Delta u + \nabla \phi + \lambda u =&\; 0 \quad\text{in }\Omega\,, \\
    \div u =&\; 0 \quad\text{in } \Omega\,, \\ 
    u =&\; g  \quad\text{nontangentially on } \partial\Omega\,, \\
    (u)^* &\in \Ell^2(\partial\Omega).
  \end{alignedat}
    \right.
\end{align*}
}
In this chapter we will thus always assume that $\Omega$ is a bounded Lipschitz domain in $\R^d$ with $d \geq 2$ and $1 < p < \infty$.
We will also tacitly use the summation convention.

We note that due to the new two dimensional estimates on fundamental solutions in Chapter~\ref{chap:2}, namely the continuation of  Theorem~\ref{thm:fundamentalMatrixEstimate}and Theorem~\ref{thm:differenceFundamentalSolutionStokes} for the case $d=2$, we could extend all results from Chapter 3 of Shen's seminal paper \cite{Shen2012} that are relevant to the analysis of the $\Ell^2$ Dirichlet problem in a straightforward way.

Let $\lambda \in \Sigma_\theta$, $\theta \in (0,\pi/2)$. 
Furthermore, let $f \in \Ell^p(\partial\Omega; \C^d)$. 
The single layer potential $u = \slp_\lambda(f)$ is defined by
\begin{align}
  \label{eq:defSingleLayer}
  (\slp_\lambda(f))_j(x) 
  \coloneqq \int_{\partial\Omega} \Gamma_{jk}(x - y; \lambda) f_k(y) \d \sigma(y),
\end{align}
where $\Gamma_{jk}$ is the fundamental solution to the Stokes resolvent problem given by \eqref{eq:fundamentalMatrixStokes}.
For the pressure, respectively, we define the single layer potential $\phi = \slp_\Phi(f)$ by
\begin{align}
  \label{eq:defSingleLayerPressure}
  \slp_{\Phi}(f)(x) \coloneqq \int_{\partial\Omega} \Phi_k(x - y) f_k(y) \d \sigma(y),
\end{align}
where $\Phi_k$ is given by \eqref{eq:fundamentalVectorPressure}.
The pair $(u,\phi)$ defines a solution to the Stokes resolvent problem \eqref{eq:stokesResolventProblem} in $\R^d \setminus \partial\Omega$ by the properties of the fundamental solution together with the dominated convergence theorem..

We define two further integral operators that map to functions living on $\partial\Omega$:
\begin{align}
  T_\lambda^*(f)(q) &= \sup_{t > 0} \bigg| \int_{\substack{y \in \partial\Omega \\ |y - q| > t}} \nabla_x \Gamma(q - y; \lambda) f(y) \d \sigma(y) \, \bigg| \label{eq:supTOperator}\\
  T_\lambda(f)(q) &= \pv \int_{\partial\Omega} \nabla_x \Gamma(q - y; \lambda) f(y) \d \sigma(y) \label{eq:pvTOperator}
\end{align}
for $q \in \partial\Omega$ which will be used to prove boundedness of maximal operators related to $u$ and its gradient.

The following lemma will be a good companion for the forthcoming calculation of estimates.
\begin{lem}
  \label{lem:compareBoundaryWithBall}
  Let $\Omega \subset \R^d$, $d \geq 2$, be a bounded Lipschitz domain with corresponding numbers $r_0$ and $M$.
  Then there exist $c, C > 0$ depending only on $d$ and $M$ such that
  \begin{align*}
    c r^{d - 1} 
    \leq \sigma(\BB(q,r) \cap \partial\Omega) \leq C r^{d -1}
  \end{align*}
  for all $r > 0$ and $q \in \partial\Omega$.
  Furthermore, there exist constants $\tilde c, \tilde C > 0$, depending only on $d$ and the Lipschitz character of $\Omega$, such that
  \begin{align*}
    \tilde c r_0^{d - 1} 
    \leq \sigma(\partial\Omega) 
    \leq C r_0^{d - 1}.
\end{align*}
\end{lem}

Another cornerstone in the theory of the single and double layer potentials is the following lemma, see Tolksdorf \cite[Lem.\@~4.3.2]{tolksdorf}, as it will allow us to bring into play the estimates from Section~\ref{sec:2.2}.

\begin{lem}
  \label{lem:central}
  Let $\Omega \subset \R^d$, $d \geq 2$, be a bounded Lipschitz domain with corresponding numbers $r_0$ and $M$.
  Let $x \in \R^d$, $0 < \varepsilon \leq (r_0/4)$, and $l \in \N_0$ with $l < d - 1$.
  Then there exists a constant $C > 0$ depending only on $d$, $l$ and $M$ such that
  \begin{align*}
    \int_{\partial\Omega \cap \BB(x,\, \varepsilon)} \frac{1}{|x - y|^l} \d\sigma(y) \leq C \varepsilon^{d - l - 1}.
  \end{align*}
\end{lem}

We are now in the position to prove our first lemma on the way to establish the single layer potential as a benevolent operator for tackling boundary value problems on bounded Lipschitz domains. 
The lemma deals with mapping properties of the aforementioned integral operators $T_\lambda$ and $T_\lambda^*$.
The main idea will be to deduce pointwise estimates that bound the operator $T_\lambda^*$ by the \emph{Hardy-Littlewood maximal operator} $M_{\partial\Omega}$ which is defined for functions $f \in \Ell^1_{\mathrm{loc}}(\partial\Omega)$ via
\begin{align*}
  M_{\partial\Omega}(f)(q) \coloneqq \sup_{\varepsilon > 0} \; \frac{1}{\sigma(\partial\Omega \cap \BB(q, \varepsilon))} \int_{\partial\Omega \cap \BB(q,\,\varepsilon)} |f(y)| \d \sigma(y), \quad q \in \partial\Omega.
\end{align*}

\begin{lem}
  \label{lem:lpBoundednessT}
  Let $1 < p < \infty$ and $T_\lambda(f), T_\lambda^*(f)$ be defined by \eqref{eq:supTOperator} and \eqref{eq:pvTOperator}.
  Then $T_\lambda(f)(P)$ exists for almost every $P \in \partial\Omega$ and
  \begin{align}
    \label{eq:lpBoundednessT}
    \|T_\lambda(f) \|_{\Ell^p(\partial\Omega; \C^{d \times d})}
    \leq \|T_\lambda^*(f) \|_{\Ell^p(\partial\Omega)}
    \leq C_p \, \|f\|_{\Ell^p(\partial\Omega; \C^d)},
  \end{align}
  where $C_p$ depends only on $d$, $\theta$, $p$, and the Lipschitz character of $\Omega$.
\end{lem}

\begin{proof}
  %If $\lambda = 0$, the lemma is known due to Fabes, Kenig and Verchota \cite{fabesKenigVerchota} as a consequence of the seminal result of Coifman, McIntosh and Meyer \cite{coifman}.
  %If $\lambda = 0$, the lemma is known due to Mitra and Wright \cite{mitreaWright} as a consequence of the seminal result of Coifman, McIntosh and Meyer \cite{coifman}.
  If $\lambda = 0$, the lemma is a consequence of the seminal result of Coifman, McIntosh and Meyer \cite[Thm.\@9]{coifman}.
  One idea of the proof in the case $\lambda \in \Sigma_\theta$ will thus be to nourish from this result and to consider the difference $\Gamma(x - y; \lambda) - \Gamma(x - y; 0)$.

  We start with the second inequality of \eqref{eq:lpBoundednessT}.
  To this end, let $t > 0$ and additionally assume that $t^2 |\lambda| > ({1}/{2})$. 
  In this case, Theorem~\ref{thm:fundamentalMatrixEstimate} gives us the estimate
  \begin{align*}
    \bigg| \int_{|y - q| > t} \nabla_x \Gamma(q - y; \lambda) f(y) \d\sigma(y) \bigg|
    &\leq C \int_{|q - y| > t} \frac{|f(y)|}{|\lambda| |q - y|^{d + 1}} \d \sigma(y) ,
  \end{align*}
  where $C$ depends on $d$ and $\theta$.
  Choose now $N \in \N$ such that $2^N t \leq \diam(\Omega) < 2^{N + 1} t$.
  We now exhaust the domain of integration by suitable annuli and use the inner radii to simplify the integrand and the outer radii to amplify the domain of integration:
  \begin{align}
    &\sum_{k = 0}^N \int_{2^k t < |q - y| < 2^{k + 1}t}  \frac{1}{|\lambda| |q - y|^{d + 1}} |f(y)| \d \sigma(y) \nonumber\\
    &\qquad\qquad\leq \sum_{k = 0}^N \int_{2^k t < |q - y| < 2^{k + 1}t}  \frac{1}{|\lambda| 2^{k(d + 1)}t^{d + 1}} |f(y)| \d \sigma(y) \nonumber\\
    %&\sum_{k = 0}^N \int_{\BB(P, 2^{k + 1}t) \cap \partial\Omega} \frac{1}{|\lambda| 2^{k(d + 1)} t^{d + 1}} |f(y)| \d \sigma(y) \\
    &\qquad\qquad\leq \frac{1}{|\lambda|t^2}\; \frac{1}{2^{1 -d}}\sum_{k = 0}^N \frac{1}{2^{2k}}\, \frac{1}{(2^{k+1} t)^{d - 1}} \int_{\BB(q,\, 2^{k + 1}t) \cap \partial \Omega}  |f(y)| \d\sigma(y)\label{eq:gtt}.
    %&\qquad\qquad\leq C \sum_{k = 0}^N \frac{1}{2^{2k}} M_{\partial\Omega}(f)(P) \\
    %&\qquad\qquad\leq C  M_{\partial\Omega}(f)(P)
    %&\quad\leq C |\lambda|^{\frac{1}{2}} \frac{2^{N + 1}t - t}{1}  M_{\partial\Omega}(f)(P) \\
    %&\quad\leq C |\lambda|^{\frac{1}{2}} t
  \end{align}
  Note that due to Lemma~\ref{lem:compareBoundaryWithBall} we have that
  %where for the third inequality we used Lemma~\ref{lem:compareBoundaryWithBall} to estimate
  %\begin{align*}
    %\frac{1}{(2^{k+1} t)^{d-1}} \leq C (\sigma(\BB(P, 2^{k + 1}d) \cap \partial\Omega))^{-1},
  %^\end{align*}
  \begin{align}
    \label{eq:applLem31}
    \frac{1}{(2^{k+1} t)^{d - 1}} \int_{\BB(q,\, 2^{k + 1}t) \cap \partial \Omega}  |f(y)| \d\sigma(y)
    \leq C M_{\partial\Omega}(f)(q), \quad k=0,\dots,N, 
  \end{align}
  with a constant $C > 0$ that depends on $d$ and the Lipschitz character of $\Omega$.
  Now, we glue together \eqref{eq:gtt} and \eqref{eq:applLem31}, take $N \to \infty$ noting the geometric series and get the estimate
  \begin{align}
    \label{eq:finalgtt}
    \bigg| \int_{|y - q| > t} \nabla_x \Gamma(q - y; \lambda) f(y) \d\sigma(y) \,\bigg|
    \leq C M_{\partial\Omega}(f)(q).
  \end{align}

  Now let $t^2 |\lambda| \leq ({1}/{2})$.
  We then split the integral as follows:
  \begin{align*}
    \bigg| \int_{|q - y| > t} \nabla_x \Gamma(q - y; \lambda) f(y) \d\sigma(y) \, \bigg| 
     &\leq \bigg| \int_{|q -y | \geq (2|\lambda|)^{-1/2}} \nabla_x\Gamma(q - y; \lambda)f(y) \d\sigma(y)\, \bigg| \\
     &\quad + \bigg| \int_{t < |q - y| < (2|\lambda|)^{-1/2}} \nabla_x\Gamma(q - y; \lambda)f(y) \d\sigma(y)\, \bigg|.
  \end{align*}
  For the first summand, note that estimate \eqref{eq:finalgtt} holds for all $t > 0$ and thus in particular for  $t = (2|\lambda|)^{-1/2}$.
  For the second term, we add a special zero and use the triangle inequality to estimate
  \begin{align*}
    &\bigg| \int_{t < |q - y| < (2|\lambda|)^{-1/2}} \nabla_x\Gamma(q - y; \lambda)f(y) \d\sigma(y) \,\bigg| \\
     &\qquad\qquad\leq \int_{t < |q - y| < (2|\lambda|)^{-1/2}} \big|\nabla_x\Gamma(q - y; \lambda) - \nabla_x\Gamma(q - y; 0) \big| \, |f(y)| \d\sigma(y) \\
     &\qquad\qquad\quad + \bigg| \int_{t < |q - y| < (2|\lambda|)^{-1/2}} \nabla_x\Gamma(q - y; 0) f(y)  \d\sigma(y)\, \bigg|.
  \end{align*}
  We don't need to worry about the second summand here since the corresponding estimate is already covered by the $\lambda = 0$ case:
  \begin{align}
    \label{eq:lambda0case}
    \begin{alignedat}{1}
     &\bigg| \int_{t < |q - y| < (2|\lambda|)^{-1/2}} \nabla_x\Gamma(q - y; 0) \, f(y)  \d\sigma\, \bigg| \\
      &\qquad\qquad \leq \bigg| \int_{|q - y| > t} \nabla_x\Gamma(q - y; 0) \, f(y)  \d\sigma \,\bigg| 
     \leq T_0^*(f)(q).
    \end{alignedat}
   \end{align}
   For the first summand we make use of Theorem~\ref{thm:differenceFundamentalSolutionStokes} and more precisely of Corollary~\ref{cor:differenceFundamentalSolutionStokes} which unifies all estimates: We start by estimating
   \begin{align}
     \label{eq:estDiff}
     \begin{alignedat}{1}
       &\int_{t < |q - y| < (2|\lambda|)^{-1/2}} \big|\nabla_x\Gamma(q - y; \lambda) - \nabla_x\Gamma(q - y; 0) \big| \, |f(y)| \d\sigma(y) \\
       &\hspace{4cm}\leq C \int_{t < |q - y| < (2|\lambda|)^{-1/2}} \sqrt{|\lambda|} \, |q - y|^{2 - d}\, |f(y)| \d\sigma(y),
     \end{alignedat}
   \end{align}
   where $C$ depends on $d$ and $\theta$. 
   Now we choose $N \in \N$ such that 
   \begin{align}
     \label{eq:choiceOfN}
     2^{N + 1} t > (2|\lambda|)^{-1/2} \geq 2^N t
   \end{align}
   holds. 
   Once again we integrate over annuli, use the inner radii to loose the term $|q - y|^{2 - d}$ and use the outer radii to expand the domain of integration to balls with this radius:
   \begin{align*}&\int_{t < |q - y| < (2|\lambda|)^{-1/2}} \frac{1}{|q - y|^{d - 2}} \, |f(y)| \d\sigma(y) \\
     &\qquad\qquad\leq \sum_{k = 0}^N \int_{2^kt \leq |q - y| < 2^{k + 1} t} \frac{1}{|q - y|^{d - 2}}\, |f(y)| \d\sigma(y) \\
     &\qquad\qquad\leq  \sum_{k = 0}^N \frac{1}{(2^{k}t)^{d - 2}}  \int_{\BB(q,\, 2^{k + 1}t) \cap \partial\Omega} |f(y)| \d\sigma(y) \\
     &\qquad\qquad =  2^{d-1} \sum_{k = 0}^N 2^k t \;  \frac{1}{(2^{k + 1}t)^{d - 1}}  \int_{\BB(q,\, 2^{k + 1}t) \cap \partial\Omega} |f(y)| \d\sigma(y).
   \end{align*}
   %\begin{align*}
%     &\qquad\qquad\leq  |\lambda|^{\frac{1}{2}} t^{2 - d} \sum_{k = 0}^N 2^{k(2 - d)} \int_{\BB(P, 2^{k + 1}t) \cap \partial\Omega} |f(y)| \d\sigma\\
%     &\qquad\qquad\leq 2^d  |\lambda|^{\frac{1}{2}} t \sum_{k = 0}^N 2^{k - 1}\; 2^{(k + 1)(1 - d)} t^{1 - d} \int_{\BB(P, 2^{k + 1}t) \cap \partial\Omega} |f(y)| \d\sigma.
%   \end{align*}
   As before we use Lemma~\ref{lem:compareBoundaryWithBall} to bring the Hardy-Littlewood maximal operator into the game like for inequality \eqref{eq:applLem31}.
   This time we cannot take $N \to \infty$ as the resulting geometric series wouldn't converge. 
   But $N$ was chosen wisely, see \eqref{eq:choiceOfN}, and thus
   \begin{align*}
     \sum_{k = 0}^N 2^k t \leq 2^{N + 1} t \leq 2^{1/2} |\lambda|^{-1/2}
   \end{align*}
   which yields the inequality
   \begin{align*}\int_{t < |q - y| < (2|\lambda|)^{-1/2}} \frac{1}{|q - y|^{d - 2}} \, |f(y)| \d\sigma
     \leq C |\lambda|^{-1/2} M_{\partial\Omega}(f)(q),
   \end{align*}
   where $C > 0$ depends on $d$ and the Lipschitz character of $\Omega$.
   Taking into account the foregoing calculations together with estimate \eqref{eq:estDiff} and \eqref{eq:lambda0case} we derive
   \begin{align}
     \label{eq:finalltt}
     \bigg| \int_{t < |q - y| < (2|\lambda|)^{-1/2}} \nabla_x\Gamma(q - y; \lambda) f(y)  \d\sigma\, \bigg| 
     \leq C\, \Big\{ T_0^*(f)(q) + M_{\partial\Omega}(f)(q) \Big\},
   \end{align}
   with $C > 0$ depending only on $d$, $\theta$ and the Lipschitz character of $\Omega$.

   It is now the time to take the supremum over all $t > 0$, and, considering estimates \eqref{eq:finalgtt} and \eqref{eq:finalltt}, we finally see that
   \begin{align*}
     T_\lambda^*(f)(q)
     &\leq C\, \Big\{  T_0^*(f)(q)+ M_{\partial\Omega}(f)(q) \Big\},
   \end{align*}
   for all $q \in \partial\Omega$. Once again using the result for $\lambda = 0$ and the $\Ell^p$ boundedness of the Hardy-Littlewood maximal operator, we conclude the first part of the claimed inequality
   \begin{align*}
     \|T_\lambda^*(f) \|_{\Ell^p(\partial\Omega)} \leq C\, \|f\|_{\Ell^p(\partial\Omega; \C^d)}.
   \end{align*}

   To conclude the left inequality in \eqref{eq:lpBoundednessT}, we want to use a standard result from argument harmonic analysis.
   To this end, we define the operators
   \begin{align*}
     T_\lambda^{(t)}(f)(q) \coloneqq \int_{\substack{y \in \partial\Omega \\ |y - q| > t}} \nabla_x \Gamma(q - y; \lambda) f(y) \d\sigma(y), \quad t > 0.
   \end{align*}
   Suppose we can show that
   \begin{align}
     \label{eq:pointwiseLimit}
     T_\lambda(f)(q) = \lim_{t \to 0} T_\lambda^{(t)}(f)(q) 
   \end{align}
   exists for almost every $q \in \partial\Omega$ and all $f \in \CC(\partial\Omega; \C^d)$. Now, note that $\CC(\partial\Omega; \C^d)$ is dense in $\Ell^p(\partial\Omega; \C^d)$ and that $T_\lambda^*(f)$ is bounded on $\Ell^p(\partial\Omega; \C^d)$ as we showed earlier.
   Then, Grafakos \cite[Thm.\@~2.1.14]{grafakos2014classical} gives that $T_\lambda$ is bounded from $\Ell^p(\partial\Omega; \C^{d})$ to $\Ell^p(\partial\Omega; \C^{d \times d})$.

   In order to prove the existence of the pointwise limit \eqref{eq:pointwiseLimit},  we split the operator $T_\lambda$ as follows:
   \begin{align*}
     T_\lambda(f)(q) = T_0(f)(q) + \lim_{t \to 0} \int_{\substack{y \in \partial\Omega \\ |y - q| > t}} \nabla_x \big\{ \Gamma(q - y; \lambda) - \Gamma(q - y; 0)\big\}  f(y) \d\sigma(y).
   \end{align*}
   The right summand is well defined for $f \in \CC(\partial\Omega; \C^d)$, once we prove integrability of the integral kernel $\big|\nabla_x \big\{\Gamma(q - y; \lambda) - \Gamma(q - y; 0)\big\} \big|$ on $\partial\Omega$.
   To this end, we first note that it suffices to consider the integral
   \begin{align*}
     \int_{|q - y| \leq \varepsilon} \big| \nabla_x\big\{ \Gamma(q - y; \lambda) - \Gamma(q - y; 0) \big\} \big| \d\sigma(y), 
   \end{align*}
   for $\varepsilon \leq \min(2|\lambda|^{-1/2}, r_0/4)$ as the integrand is bounded on $\partial\Omega \setminus \BB(q,\,\varepsilon)$  and the domain of integration $\partial\Omega$ is bounded.
   Now, Corollary~\ref{cor:differenceFundamentalSolutionStokes} and Lemma~\ref{lem:central} give that the integrand can be estimated by
   \begin{align*}
     C \int_{|q - y| \leq \varepsilon} \sqrt{|\lambda|} |q - y|^{2 - d} \d\sigma(y) 
     \leq C\, \sqrt{|\lambda|} \varepsilon \leq C,
   \end{align*}
   where $C$ is a constant depending on $d$, $\theta$ and the Lipschitz character of $\Omega$.
   Based on the preceding calculation we conclude that for all $f \in \CC(\partial\Omega; \C^d)$ the limit $T_\lambda(f)(q)$ exists whenever $T_0(f)(q)$ exists.
   %$T_0(f)(q)$ exists for almost every $q \in \partial\Omega$ because of Fabes, Kenig and Verchota \cite{fabesKenigVerchota}.
   $T_0(f)(q)$ exists for almost every $q \in \partial\Omega$ because of Mitrea and Wright \cite{mitreaWright}.
   As furthermore, $T_\lambda^*(f)(q)$ is bounded on $\Ell^p(\partial\Omega)$  we may now apply Theorem 2.1.14 from Grafakos \cite{grafakos2014classical} to conclude that $T_\lambda(f)(q)$ exists now for all $f \in \Ell^p(\partial\Omega; \C^d)$ and almost every $q \in \partial\Omega$.
   The desired $\Ell^p$ estimate for $T_\lambda(f)$ now follows from the observation that $| T_\lambda(f)(q) | \leq T_\lambda^*(f)(q)$ for almost every $q \in \partial\Omega$.
\end{proof}

%For a function $u$ in $\Omega$, we define the nontangential maximal function $(u)^*$ by
%\begin{align}
%  \label{eq:defnNontangMaxFunction}
%  (u)^*(P) = \sup\{ |u(x)| \colon x \in \Omega \text{ and } |x - P| < C \operatorname{dist}(x, \partial\Omega)\}
%\end{align}
%for $P \in \partial\Omega$, where $C > 2$ is a fixed and sufficiently large constant depending on $d$ and the Lipschitz character of $\Omega$.
%Note that in Shen cones we have that for $P, y \in \partial\Omega$ and $x \in \ShenCone(P)$
%\begin{align}
%  \label{eq:shenConeEstimate}
%  |P - y| 
%  &\leq |P - x| + |x - y| 
%  \leq C \operatorname{dist}(x \partial\Omega) + |x - y|  \nonumber\\
%  &\leq (C + 1) |x - y|
%\end{align}
%where $C$ is the constant from \eqref{eq:defnNontangMaxFunction}

For further use, we state a very useful lemma which can be considered a \emph{Young}-type inequality for $\Ell^p$ spaces on boundaries of Lipschitz domains.
A proof can be found in Tolksdorf \cite[Prop.\@~1.1.4]{tolksdorf}.

\begin{lem}
  \label{lem:young}
  Let $\Omega \subset \R^d$, $\mu$ a $\sigma$-finite measure on $\Omega$, and $1 \leq p < \infty$.
  Let $g \colon \R^d \setminus \{ 0\} \to \C$ be a function such that the function $\Omega \times \Omega \ni (x, y) \mapsto g(x-y)$ is measurable with respect to the product measure $\mu \times \mu$ and such that
  \begin{align*}
    A + B \coloneqq \sup_{x \in \Omega} \| g(x - \cdot) \|_{\Ell^1(\Omega, \mu)} + \sup_{y \in \Omega} \|g(\cdot - y) \|_{\Ell^1(\Omega, \mu)} < \infty.
  \end{align*}
  If $f \in \Ell^p(\Omega, \mu)$, then $x \mapsto \int_\Omega g(x - y) f(y) \d\mu(y) \in \Ell^p(\Omega, \mu)$ and
  \begin{align*}
    \bigg\| \int_\Omega g(\cdot - y) f(y) \d\mu(y)\, \bigg\|_{\Ell^p(\Omega, \mu)} \leq A^{1 - 1/p} B^{1/p} \|f\|_{\Ell^p(\Omega, \mu)}.
  \end{align*}
\end{lem}

For us, Lemma~\ref{lem:young} will be applied often to integral kernels $g$ that result from an application of the theorems in Chapter~\ref{chap:2}.
The following lemma shows that these integral kernels fulfill the requirements from Lemma $\ref{lem:young}$.

\begin{lem}
  \label{lem:youngApp}
  Let $\Omega \subset \R^d$ be a bounded Lipschitz domain. Then
  \begin{align*}
    \sup_{q \in \partial\Omega} \int_{\partial\Omega} \frac{1}{|q - y|^{d - 2}} \leq C r_0,
  \end{align*}
  where $C$ is a constant depending only on $d$ and the Lipschitz character of $\Omega$.
\end{lem}

\begin{proof}
  Let $r_0$ be the radius from the definition of Lipschitz cylinders and $q \in \partial\Omega$.
  Splitting the domain of integration and applying Lemma~\ref{lem:central}, we get
  \begin{align*}
    &\int_{\partial\Omega} \frac{1}{|q - y|^{d - 2}} \d\sigma(y) \\[0.5em]
    &\qquad \leq \int_{\partial\Omega \cap \BB(q,\, r_0/4)} \frac{1}{|q - y|^{d - 2}} \d \sigma(y) 
    + \int_{\partial\Omega \setminus \BB(q,\, r_0/4)} \frac{1}{|q - y|^{d - 2}} \d\sigma(y) \\[0.5em]
    &\qquad\leq C r_0 + r_0^{2 - d} 4^{d - 2} \sigma(\partial\Omega)
    \leq C\, ( r_0 + r_0^{2 - d} r_0^{d - 1}),
  \end{align*}
  where $C > 0$ depends only on $d$ and the Lipschitz character of $\Omega$.
  This proves the claim.
\end{proof}

We can now prove the boundedness of certain nontangential maximal operators.
\begin{lem}
  \label{lem:nontangentialMaximalFunctions}
  Let $1 < p < \infty$ and $(u,\phi)$ be given by \eqref{eq:defSingleLayer} and \eqref{eq:defSingleLayerPressure}.
  Then 
  \begin{align}
    \| (\nabla u)^* \|_{\Ell^p(\partial\Omega)}  +  \| (\phi)^* \|_{\Ell^p(\partial\Omega)} \leq C_p \|f\|_{\Ell^p(\partial\Omega; \C^d)},
  \end{align}
  where $C_p > 0$ depends only on $d$, $\theta$, $p$ and the Lipschitz character of $\Omega$.
  Let furthermore $d \geq 3$. Then
  \begin{align}
    \label{eq:estimateUstar}
     \|(u)^*\|_{\Ell^p(\partial\Omega)} \leq C_p \|f\|_{\Ell^p(\partial\Omega; \C^d)},
  \end{align}
  where $C_p > 0$ depends only on $d$, $\theta$, $p$ and the Lipschitz character of $\Omega$.
\end{lem}

\begin{proof}
  A proof of the estimate $\|(\phi)^*\|_{\Ell^p(\partial\Omega)} \leq C_p \|f\|_{\Ell^p(\partial\Omega; \C^d)}$ can be found in Verchota's dissertation \cite[Lem.\@~1.3]{verchota}.
  The proof for $\|(\nabla u)^*\|_{\Ell^p(\partial\Omega)}$ works in the same way. 
  We will provide a proof for the sake of completeness.
  To imitate the proof of Verchota, we will work with the corresponding type of cones.
  Therefore, the results for $\nabla u$ and $\phi$ will at first only be established for the type of maximal operators defined by Verchota.
  The transferability to Shen's maximal operators is given by Tolksdorf \cite[p.\@~90ff.]{tolksdorf} as the solution $(u,\phi)$ has a representation as a single layer potential.

  Let $q \in \partial\Omega$,  $x \in \verCone(q)$, and set $t = |x - q|$.
  Then,
  \begin{align*}
    &|(\nabla u_j)(x) | \\
    &\quad= \bigg| \int_{\partial\Omega} \nabla_x \Gamma_{jk} (x - y; \lambda) f_k(y) \d \sigma(y)\, \bigg| \\
    &\quad\leq \bigg| \int_{|y - q| > t} \nabla_x\Gamma_{jk}(x - y; \lambda) f_k(y) \d \sigma(y) \, \bigg| + \bigg| \int_{|y - q| \leq t} \nabla_x \Gamma_{jk}(x - y; \lambda) f_k(y) \d \sigma(y)\, \bigg| \\
    &\quad\eqqcolon I_1 + I_2.
  \end{align*}
  We will now estimate $I_1$ and $I_2$ separately.
  Note that in Verchota cones $\verCone(q)$ it holds that for all $s \in \partial\Omega$ we have $|x - s| \geq C\, |x - q|$, where $C > 0$ is a constant only depending on $d$ and the Lipschitz character of $\Omega$, see inequality~\eqref{eq:verCone1}.
  By Theorem~\ref{thm:fundamentalMatrixEstimate} we know that
  \begin{align*}
    I_2 
    &\leq C \int_{|y - q| \leq t} \frac{1}{|x - y|^{d - 1}} |f(y)| \d \sigma(y) \\
    &\leq \frac{C}{t^{d - 1}} \int_{|y - q| \leq t} |f(y)| \d \sigma(y)
    \leq C\, M_{\partial\Omega} (f)(q),
  \end{align*}
  where we used also Lemma~\ref{lem:compareBoundaryWithBall} to bring the Hardy-Littlewood maximal operator into play.
  Here, $C > 0$ depends on $d$, $\theta$ and the Lipschitz character of $\Omega$.
%  For $I_1$, we calculate
%  \begin{align*}
%    &\Big| \int_{| y - P | > t} \nabla \Gamma_{j k} (x - y; \lambda) f_k(y) - \nabla \Gamma_{jk}(P - y; \lambda) f_k(y) + \nabla \Gamma_{jk}(P - y; \lambda) f_k(y) \d \sigma(y) \Big| \\
%    &\quad\leq \Big| \int_{|y - P| > t} \nabla\Gamma_{jk} (x - y; \lambda) f_k(y) - \nabla\Gamma_{jk}(P - y; \lambda) f_k(y) \d \sigma(y) \Big| \\
%    &\qquad+ \Big| \int_{|y - P| > t} \nabla\Gamma_{jk}(P - y; \lambda) f_k(y) \d\sigma(y) \Big|.
%  \end{align*}
   For $I_1$, we calculate
   \begin{align*}
     &\bigg| \int_{| x - q | > t} \nabla_x \Gamma_{j k} (x - y; \lambda) f_k(y) - \nabla_x \Gamma_{jk}(q - y; \lambda) f_k(y) + \nabla_x \Gamma_{jk}(q - y; \lambda) f_k(y) \d \sigma(y) \, \bigg| \\
     &\qquad\leq \bigg| \int_{|y - q| > t} \nabla_x \Big\{ \Gamma_{jk}(x - y; \lambda) -  \Gamma_{jk}(q - y; \lambda)\Big\} f_k(y) \d \sigma(y) \, \bigg| \\
     &\qquad\qquad+ \bigg| \int_{|y - q| > t} \nabla_x\Gamma_{jk}(q - y; \lambda) f_k(y) \d\sigma(y) \, \bigg|.
   \end{align*}
  The second summand can directly be estimated by $T_\lambda^*(f)(q)$.
  For the first one we apply the mean value theorem and use Theorem~\ref{thm:fundamentalMatrixEstimate} to derive the following estimation:
  \begin{align*}
    &\int_{|y - q| > t} \big| \nabla_x\Gamma_{jk}(x - y; \lambda) - \nabla_x\Gamma_{jk}(q - y; \lambda) \big| \, |f(y)| \d\sigma(y) \\ 
    &\qquad\qquad\leq \int_{|y - q| > t} \big|\nabla^2 \Gamma_{jk}(s - y; \lambda)\big| |x - q|\, |f(y) | \d\sigma(y) \\
    &\qquad\qquad\leq C \int_{|y - q| > t} \frac{t}{|s - y|^{d}} |f(y)| \d\sigma(y) \\
    &\qquad\qquad\leq C \int_{|y - q| > t} \frac{t}{|y - q|^{d}} |f(y)| \d\sigma(y) \\
    &\qquad\qquad\leq C \int_{\partial\Omega} \frac{t}{(t + |y - q|)^d} |f(y)| \d\sigma(y),
  \end{align*}
  where $s$ is an element on the line connecting $x$ and $q$ and we used the property of Verchota cones that $|s - y| \geq C\, |y - q|$, see inequality \eqref{eq:verCone2}.
  Note that Verchota cones are convex.
  As in Verchota \cite[Lem.\@~1.3]{verchota}, the integral may now be bounded by the Hardy-Littlewood maximal operator due to an application of a suitable result from Grafakos \cite[Thm.\@~2.1.10]{grafakos2014classical} as the kernel ${t}(t + |y - q|)^{-d}$ is uniformly integrable and radially decreasing.% on $\partial\Omega$.
  Summing up we have shown that
  \begin{align*}
    |(\nabla u)(x)| \leq C\, \Big\{ M_{\partial\Omega} f(P) + T_\lambda^*(f)(P) \Big\},
  \end{align*}
  where $C > 0$ only depends on $d$, $\theta$ and the Lipschitz character of $\Omega$.
  We thus may take the supremum over all $x \in \verCone(q)$ and conclude the desired estimate by the well known mapping properties of the Hardy-Littlewood maximal operator and the respective results from Lemma~\ref{lem:lpBoundednessT}.
%  By exhausting the domain of integration using annuli, we can estimate this integral by $M_{\partial\Omega}f(P)$:
%  Choose $N$ such that $2^N t \leq \diam(\Omega) < 2^{N + 1}t$.
%  Then
%  \begin{align*}
%    &\int_{|y - P| > t} \frac{1}{|y - P|^{d - 1}} |f(y)| \d\sigma(y) \\
%    &\quad= \sum_{k = 0}^N \int_{2^{k + 1}t > |y - P| \geq 2^k t} \frac{1}{|y - P|^{d - 1}} |f(y)| \d\sigma(y) \\
%    &\quad\leq \sum_{k = 0}^N \frac{1}{2^{k(d - 1)} t^{d - 1}} \int_{|y - P| < 2^{k + 1} t} |f(y)| \d\sigma(y) \\
%    &\quad \leq  C \sum_{k = 0}^N 2^{k - d + 1)} M_{\partial\Omega}(f)(P)
%    &\quad\leq
%  \end{align*}
%  Taking the supremum over all $x \in \Omega$ the claim follows.

  We will now work on the proof of the estimate for $(u)^*$ for $d \geq 3$.
  In order to derive $\Ell^p$ bounds on this maximal operator, we will work directly with the Definition of the single layer potential \eqref{eq:defSingleLayer}.
  For $q \in \partial\Omega$, estimate \eqref{eq:fundamentalMatrixEstimate} together with the estimate for Shen cones \eqref{eq:shenConeEstimate} gives that for all $x \in \ShenCone(q)$
  \begin{align*}
    |u(x)| 
    &\leq C \int_{\partial\Omega} \frac{1}{|x - y|^{d - 2}} |f(y)| \d\sigma(y) 
    \leq  C \int_{\partial\Omega} \frac{1}{|q - y|^{d - 2}} |f(y)| \d\sigma(y),
  \end{align*}
  where $C > 0$ only depends on $d$, $\theta$ and the Lipschitz character of $\Omega$.
  Passing to the maximal operator yields the inequality
  \begin{align*}
    (u)^*(q) \leq  C \int_{\partial\Omega} \frac{1}{|q - y|^{d - 2}} |f(y)| \d\sigma(y).
  \end{align*}
  Estimating the kernel via Lemma~\ref{lem:youngApp} and applying the Young inequality for convolutions from Lemma~\ref{lem:young} the claim follows.
\end{proof}

\begin{rem}
  We note that in addition to the consideration of $d = 2$, Lemma~\ref{lem:nontangentialMaximalFunctions} differs in the form of estimate \eqref{eq:estimateUstar} from the original statement in Shen's work \cite[Lem.\@~3.2]{Shen2012}.
  There, the author derives an estimate of the form $|\lambda|^{1/2}\|(u)^*\|_{\Ell^p(\partial\Omega)} \leq C_p \|f\|_{\Ell^p(\partial\Omega; \C^d)}$ which is based on Shen's version of Lemma~\ref{lem:estimateHelmholtzDerivatives}, namely \cite[Lem.\@~2.1]{Shen2012}.
  As we could not follow the proof in \cite{Shen2012}, we provided a similar estimate and since the estimate won't be needed in the course of this thesis, we will not pursue the verification of Shen's estimate further.
  Another approach to the integrability of $(u)^*$ for the $\Ell^2$ case will be given in Chapter~\ref{chap:4}.
\end{rem}

The next lemma deals with \emph{trace formulas} for $\nabla u$ and $\phi$. 
We will then finally be able to talk about boundary values since the existence of nontangential limits guarantees that there exists something on $\partial\Omega$ that is related to the function inside $\Omega$ or inside $\R^d \setminus \overline\Omega$, respectively.

\begin{lem}
  \label{lem:traceFormulas}
  Let $(u,\phi)$ be given by \eqref{eq:defSingleLayer} and \eqref{eq:defSingleLayerPressure} with $f \in \Ell^p(\partial\Omega; \C^d)$ and $1 < p < \infty$.
  Then
  \begin{align}
    \Big( \frac{\partial u_i}{\partial x_j} \Big)_{\pm}(x) 
    &= \pm \frac{1}{2} \big\{ n_j(x) f_i(x) - n_i(x) n_j(x) n_k(x) f_k(x) \big\} \nonumber\\
    &\quad+ \pv \int_{\partial\Omega} \frac{\partial}{\partial x_j} \Big\{ \Gamma_{ik} (x - y; \lambda) \Big\} f_k(y) \d\sigma(y), \label{eq:traceFormula} \\
    \phi_\pm(x) &= \mp \frac{1}{2} n_k(x) f_k(x) + \pv\int_{\partial\Omega} \Phi_k(x - y) f_k(y) \d \sigma(y) \nonumber
  \end{align}
  for almost every $x \in \partial\Omega$.
  The subscripts $+$ and $-$ indicate nontangential limits taken inside $\Omega$ and outside $\overline\Omega$, respectively.
\end{lem}

\begin{proof}
  The correctness of the trace formulas \eqref{eq:traceFormula} is known for the case $\lambda = 0$ due to Mitrea and Wright \cite[Prop.\@~4.4]{mitreaWright}.
  This fact will now be reused for $\lambda \in \Sigma_\theta$.
  We insert a zero to the nontangential limit such that
  \begin{align*}
    (\nabla u_j)_\pm(x) = 
    (\nabla v_j)_\pm(x) + (\nabla u_j - \nabla v_j)_\pm(x),
  \end{align*}
  where $v_j(x) = \int_{\partial\Omega} \Gamma_{jk}(x - y; 0) f_k(y) \d\sigma(y)$.
  Because of \cite{mitreaWright} we know that the first nontangential limit exists and is given by \eqref{eq:traceFormula} with $\lambda = 0$.
  It therefore remains to show the identity
  \begin{align*}
    (\nabla u_j - \nabla v_j)_\pm(x) = \int_{\partial\Omega} \nabla_x \Big\{ \Gamma_{jk}(x - y; \lambda) - \Gamma_{jk}(x - y; 0) \Big\} f_k(y) \d\sigma(y)
  \end{align*}
  for all $x \in \partial \Omega$.
  To this end, let $(x_l)_{l \in \N}$ be a sequence in $\ShenCone(x)$ with $\lim_{l \to \infty} x_l = x$.
  Furthermore, let us note that for almost every $x \in \partial\Omega$ we have that 
  \begin{align*}
    \int_{\partial\Omega} \frac{1}{|x - y|^{d - 2}} |f(y)| \d\sigma(y) < \infty.
  \end{align*}
  This is a consequence of Lemma~\ref{lem:youngApp} and Young's inequality from Lemma~\ref{lem:young}.
%  \begin{align*}
%    \sup_{x \in \partial\Omega} \Big| \int_{\partial\Omega} \frac{1}{|x - y|^{d - 2}} \d\sigma(y) \Big| < \infty
%  \end{align*}
%  and an application of Lemma~\ref{lem:young} regarding Young's inequality for convolutions:
%  Let $x \in \partial\Omega$.
%  Then
%  \begin{align*}
%    &\int_{\partial\Omega} \frac{1}{|x - y|^{d - 2}} \d\sigma(y) \\
%    &\quad\leq \int_{\partial\Omega \cap \BB(x,r_0/4)} \frac{1}{|x - y|^{d - 2}} \d\sigma(y) + \int_{\partial\Omega \setminus \BB(x, r_0/4)} \frac{1}{|x - y|^{d - 2}} \d\sigma(y) \\
%    &\quad\leq C r_0 + r^{2 - d} 4^{d - 2} \sigma(\partial\Omega)
%  \end{align*}
%  by Lemma~\ref{lem:compareBoundaryWithBall}.
%  Now Young's inequality gives us the desired result.
  Now, we will show that for the members of the sequence $x_l$ the function
  \begin{align*}
    \frac{1}{|x -y|^{d - 2}} |f(y)|
  \end{align*}
  serves as a suitable function for dominated convergence.
  Set $\varepsilon = (4 |\lambda|^2)^{-1}$ and without loss of generality assume that $\supp f \subseteq \BB(x,\varepsilon)$.
  Furthermore assume that $|x_l - x| < \varepsilon$ for all $l \in \N$.
  Then $|x_l - y| \leq (2|\lambda|^2)^{-1}$ and Corollary~\ref{cor:differenceFundamentalSolutionStokes} give
  \begin{align*}
    %(\nabla u_j - \nabla v_j)(x_l) 
    &\bigg| \int_{\partial\Omega} \nabla_x \Big\{ \Gamma_{jk}(x_l - y; \lambda) - \Gamma_{jk}(x_l - y; 0)\Big\} f_k(y) \d\sigma(y)\, \bigg| \\
    &\qquad\qquad\leq C \int_{\partial\Omega} \sqrt{|\lambda|} \frac{1}{|x_l - y|^{d - 2}}\, |f(y)| \d\sigma(y) \\
    &\qquad\qquad\leq C \,\sqrt{|\lambda|} \int_{\partial\Omega} \frac{1}{|x - y|^{d - 2}}\, |f(y)| \d\sigma(y) < \infty,
  \end{align*}
  where for the last step we applied inequality~\eqref{eq:shenConeEstimate}
  Now dominated convergence gives the claim for $x_l \to x$.
  Note that it does not affect the proof if the sequence $x_l$ lays inside $\Omega$ or outside $\overline\Omega$ and thus the same proof holds for a sequence $(x_l)_{l \in \N}$ in $\ShenCone^{\mathrm{ext}}(x)$.
\end{proof}

The previous lemma enables us to talk about boundary values of partial derivatives. 
The next theorem will now give a similar result but for \emph{conormal derivatives}, which are defined for solutions $(u,\phi)$ to the Stokes (resolvent) system via 
\begin{align}
  \label{eq:conormalDerivative}
  \frac{\partial u}{\partial \nu} \coloneqq \frac{\partial u}{\partial n} - \phi n,
\end{align}
see Mitrea and Wright \cite[Eq.\@~(1.2)]{mitreaWright}, where $n$ denotes the outer unit normal vector.
We will also be working with the tangential gradient which is defined via
\begin{align}
  \label{eq:tangentialGradient}
  \nabla_{\mathrm{tan}} u_j \coloneqq \nabla u_j - \big(\nabla u_j \cdot  n\big) n,
\end{align}
see Mitrea and Wright \cite[p.\@~17]{mitreaWright}.

\begin{thm}
  \label{thm:jumpConditions}
  Let $\lambda \in \Sigma_\theta$ and $\Omega$ be a bounded Lipschitz domain in $\R^d$, $d \geq 2$. 
  Let $(u,\phi)$ be given by \eqref{eq:defSingleLayer} and \eqref{eq:defSingleLayerPressure} with $f \in \Ell^p(\partial\Omega; \C^d)$ and $1 < p < \infty$.
  Then $\nabla_{\mathrm{tan}} u_+ = \nabla_{\mathrm{tan}} u_-$ and
  \begin{align}
    \label{eq:nontangentialConormalDerivative}
    \Big( \frac{\partial u}{\partial \nu} \Big)_\pm = \Big( \pm \frac{1}{2} I + \K_\lambda \Big) f
  \end{align}
  on $\partial\Omega$, with $\K_\lambda$ a bounded operator on $\Ell^p(\partial\Omega; \C^d)$ satisfying
  \begin{align*}
    \| \K_\lambda f \|_{\Ell^p(\partial\Omega; \C^d)} \leq C_p\, \|f\|_{\Ell^p(\partial\Omega, \C^d)},
  \end{align*}
  where $C_p > 0$ depends only on $d$, $\theta$, $p$ and the Lipschitz character of $\Omega$.
\end{thm}

\begin{proof}
  For the $j$th component of the tangential derivative of $u_i$, $1\leq i,j \leq d$, we calculate using the results from Lemma~\ref{lem:traceFormulas}
  \begin{align*}
    ((\nabla_{\mathrm{tan}} u_i)_+)_j
    &= \Big(\frac{\partial u_i}{\partial x_j}\Big)_+ - \langle (\nabla u_i)_+, n \rangle n_j \\
    &= \Big(\frac{\partial u_i}{\partial x_j} \Big)_+ - \Big(\frac{\partial u_i}{\partial x_k} \Big)\,_+ n_k n_j \\
    &= \frac{1}{2} \big\{ n_j f_i - n_i n_j n_k f_k\big\} - \frac{1}{2} \big\{ n_k f_i - n_i n_k n_l f_l \big\} n_k n_j  \\
    &\quad+ \pv \int_{\partial\Omega} \frac{\partial}{\partial x_j} \Big\{ \Gamma_{ik} (\,\cdot - y; \lambda) \Big\} f_k(y) \d\sigma(y) \\
    &\quad+ \pv \int_{\partial\Omega} \frac{\partial}{\partial x_k} \Big\{ \Gamma_{il} (\,\cdot - y; \lambda) \Big\} f_l(y) \d\sigma(y) \, n_k n_j,
  \end{align*}
  for almost every $x \in \partial\Omega$.
  As the first two summands add up to zero, the entire expression does not depend on the direction of the nontangential limit. 
  This gives
  \begin{align*}
    (\nabla_{\mathrm{tan}} u)_+ = (\nabla_{\mathrm{tan}} u)_-,
  \end{align*}
  for almost every $x \in \partial\Omega$.
  We calculate for the $j$th component of the nontangential limit of the conormal derivative of $u$ on $\partial\Omega$ using the results from Lemma~\ref{lem:traceFormulas}
  \begin{align*}
    &\Big(\frac{\partial u_j}{\partial x_i}\Big)_+ n_i - \phi_+ n_j\\
    &\quad= \frac{1}{2} \big\{ n_i f_j - n_j n_i n_k f_k \big\} n_i + \pv \int_{\partial\Omega} \frac{\partial}{\partial{x_i}} \Big\{ \Gamma_{jk}(\,\cdot - y; \lambda) \Big\} f_k(y) \d\sigma(y) n_i \\
    &\qquad+ \frac{1}{2} n_k f_k n_j - \pv \int_{\partial\Omega} \Phi_k(\,\cdot - y) f_k(y) \d\sigma(y) n_j \\
    &\quad= \frac{1}{2} f_j + (\K_\lambda f)_j
  \end{align*}
  almost every and where $\K_\lambda$ is a singular integral operator defined via
  \begin{align}
    \label{eq:defnKlambda}
    \begin{alignedat}{1}
    (\K_\lambda f)_j (x)
      &\coloneqq \pv \int_{\partial\Omega} \frac{\partial}{\partial x_i} \Big\{\Gamma_{jk}(x - y; \lambda) \Big\}  f_k(y) \d\sigma(y) \, n_i(x) \\
      &\qquad - \pv \int_{\partial\Omega} \Phi_k(x - y) f_k(y) \d\sigma(y)\, n_j(x).
    \end{alignedat}
  \end{align}
  We note that $\K_\lambda$ essentially consists of two boundary layer potentials. 
  The $\Ell^p$ boundedness of the first one was proven in Lemma~\ref{lem:lpBoundednessT}.
  The $\Ell^p$ boundedness of the second boundary layer potential follows in an analogous way using the fact that the operators
  \begin{align*}
    A^*(f)(q) \coloneqq \sup_{t > 0} \bigg| \int_{\substack{y \in \partial\Omega \\ |y - q| > t}} \frac{q - y}{|q - y|^d} f(y) \d\sigma(y)\, \bigg|\, , \quad q \in \partial\Omega,
  \end{align*}
  are bounded by the corresponding result from Verchota \cite[Lem.\@~1.2]{verchota}.
\end{proof}

Similar to $\K_\lambda$, for $\lambda = 0$, we have
\begin{align}
  \begin{alignedat}{1}
  \label{eq:defnK0}
    (\K_0 f)_j(x)&= \pv \int_{\partial\Omega} \frac{\partial}{\partial x_i} \Big\{ \Gamma_{jk}(x - y; 0) \Big\}  f_k(y) \d\sigma(y) \,n_i(x) \\
    &\qquad - \pv \int_{\partial\Omega} \Phi_k(x - y) f_k(y) \d\sigma(y)\, n_j(x),
  \end{alignedat}
\end{align}
as was shown by Mitrea and Wright \cite[Prop.\@~4.4]{mitreaWright}.
If one compares \eqref{eq:defnKlambda} with \eqref{eq:defnK0}, then the only difference lies within the boundary integral involving the fundamental solutions $\Gamma_{jk}(\,\cdot\,; 0)$ instead of $\Gamma_{jk}(\,\cdot\,; \lambda)$.

The next result will be crucial for solving the $\Ell^2$ Dirichlet problem in Chapter~\ref{chap:5} and will fortify the hopes of translating results for $\lambda = 0$ to $\lambda \in \Sigma_\theta$.
\begin{lem}
  \label{lem:compactness}
  Let $\lambda \in \Sigma_\theta$ and $d \geq 2$ and let $\K_\lambda$ and $\K_0$ be defined by \eqref{eq:defnKlambda} and \eqref{eq:defnK0}, respectively.
  Then the operator $\K_\lambda - \K_0$ on $\Ell^2(\partial\Omega; \C^d)$ is compact.
\end{lem}

\begin{proof}
  The idea of this proof is similar to the one in Tolksdorf \cite[Lem.\@~4.3.5]{tolksdorf}.
  Let $f \in \Ell^2(\partial\Omega; \C^d)$ and let us denote $\K \coloneqq \K_\lambda - \K_0$. 
  We will now try to approximate $\K$ by compact operators in the operator norm.
  To this end, we define for all $\varepsilon > 0$
  \begin{align*}
    &(\K^{(\varepsilon)}f)(x) 
    \coloneqq \int_{\partial\Omega \setminus \BB(x,\,\varepsilon)} \nabla_x \Big\{ \Gamma(x - y; \lambda) - \Gamma(x - y; 0) \Big\} f(y) \d \sigma(y) \, n, \quad x \in \partial\Omega.
  \end{align*}
  We can now estimate by Young's inequality, see Lemma~\ref{lem:young}.
  \begin{align*}
    &\Big\| \,\K f - \K^{(\varepsilon)} f\, \Big\|_{\Ell^2(\partial\Omega; \C^d)} \\
    &\qquad\leq \sup_{p \in \partial\Omega} \Big\| \nabla_x \Big\{ \Gamma(p - \cdot; \lambda) - \Gamma(p - \cdot; 0) \Big\} 1_{\BB(p,\,\varepsilon)} \Big\|_{\Ell^1(\partial\Omega; \C^{d \times d})} \|f\|_{\Ell^2(\partial\Omega; \C^d)}.
  \end{align*}
  Our goal is to show that
  \begin{align*}
    \sup_{p \in \partial\Omega} \Big\| \nabla_x \Big\{ \Gamma(p - \cdot; \lambda) - \Gamma(p - \cdot; 0) \Big\} 1_{\BB(p,\,\varepsilon)} \Big\|_{\Ell^1(\partial\Omega; \C^{d \times d})} \to 0 \quad \text{as} \quad \varepsilon \to 0.
  \end{align*}
  To this end, let $\varepsilon$ be small enough such that we can apply the estimates from Corollary~\ref{cor:differenceFundamentalSolutionStokes} to calculate for some $p \in \partial\Omega$
  \begin{align*}
    &\Big\| \nabla_x \Big\{ \Gamma(p - \cdot; \lambda) - \Gamma(p - \cdot; 0) \Big\} 1_{\BB(p,\,\varepsilon)} \Big\|_{\Ell^1(\partial\Omega; \C^{d \times d})}  \\
     &\qquad\qquad\leq C \int_{\partial\Omega \cap \BB(p,\, \varepsilon)} \sqrt{|\lambda|} |p - y|^{2 - d} \d\sigma(y) 
     \leq C\, \sqrt{|\lambda|} \varepsilon
  \end{align*}
  where for the last step we applied Lemma~\ref{lem:compareBoundaryWithBall}.
  For $\varepsilon \to 0$ this gives us $\K^{(\varepsilon)} \to \K$ in the operator norm.

  The last step is to verify the compactness of $\K^{(\varepsilon)}$.
  We note that the integral kernel of $\K^{(\varepsilon)}$ is bounded which gives us that in particular the kernel is an element of the space $\Ell^2(\partial\Omega \times \partial\Omega; \C^{d \times d})$.
  The compactness of $\K^{(\varepsilon)}$ now follows from Weidmann \cite[Thm.\@~6.11]{weidmann}.

  As a consequence, $\K$ is compact since the limit of compact operators with respect to the operator norm gives again a compact operator.
\end{proof}

Our next step is to introduce the \emph{double layer potential} $u(x) = \dlp_\lambda(f)(x)$ for the Stokes resolvent problem via
\begin{align}
  \label{eq:defDoubleLayer}
  (\dlp_\lambda(f))_j(x) &\coloneqq \int_{\partial\Omega} \Big\{ \frac{\partial}{\partial y_i} \{ \Gamma_{jk}(y - x; \lambda) \} n_i(y) - \Phi_j(y - x) n_k(y) \Big\} f_k(y) \d\sigma(y).
\intertext{The corresponding pressure $\phi(x) = \dlp_{\Phi}(f)(x)$ is defined as}
  \dlp_{\Phi}(f)(x)
  &\coloneqq \frac{\partial^2}{\partial x_i \partial x_k} \int_{\partial\Omega} G(y - x; 0) \, n_i(y) f_k(y) \d\sigma(y) \nonumber\\
  \label{eq:defDoubleLayerPressure}
  &\hspace{3cm}+ \lambda \int_{\partial\Omega} G(y - x; 0) \, n_k(y) f_k(y) \d\sigma(y).
\end{align}
Using \eqref{eq:fundamentalVectorPressure} and \eqref{eq:solutionStokesSystem} one can show that $(u,\phi)$ defines again a solution to the Stokes resolvent problem in $\R^d \setminus \partial\Omega$.

The following theorem will give us a suitable operator which maps a given function $f \in \Ell^p(\partial\Omega; \C^d)$ to boundary values of $u = \dlp_\lambda(f)$ in the form of nontangential limits.
It will then be the task of the following chapters to prove the invertibility of this operator.

\begin{thm}
  \label{thm:nontangentialLimitDoubleLayer}
Let $\lambda \in \Sigma_\theta$ and $\Omega$ be a bounded Lipschitz domain in $\R^d$, $d \geq 2$.
Let $u$ be given by \eqref{eq:defDoubleLayer} for $f \in \Ell^p(\partial\Omega; \C^d)$, $1 < p < \infty$.
Then,
\begin{align}
  \label{eq:lpBoundednessUNontangentialMax}
  \|(u)^*\|_{\Ell^p(\partial\Omega)} \leq C_p \|f\|_{\Ell^p(\partial\Omega; \C^d)},
\end{align}
where $C_p > 0$ depends only on $d$, $p$, $\theta$ and the Lipschitz character of $\Omega$.
Furthermore, 
\begin{align}
  \label{eq:nontangentialLimitDoubleLayer}
  u_\pm = \Big(\mp \frac{1}{2} I + \K_{\bar\lambda}^* \Big) f,
\end{align}
where $\K_{\bar\lambda}^*$ is the adjoint of the operator $\K_{\bar\lambda}$ in \eqref{eq:nontangentialConormalDerivative}.
\end{thm}

\begin{proof}
  The estimate for $(u)^*$ is a direct consequence of Lemma~\ref{lem:nontangentialMaximalFunctions} and the estimates on the nontangential maximal functions for the single layer potentials $(\nabla \slp_\lambda(f))^*$ and $(\slp_\Phi(f))^*$:
  We have on the one hand
  \begin{align}
    \label{eq:rep1}
    \begin{alignedat}{1}
    &\int_{\partial\Omega} \frac{\partial}{\partial y_i} \Big\{ \Gamma_{jk}(y - x; \lambda) \Big\} \, n_i(y) f_k(y) \d\sigma(y) \\
    &\qquad\qquad= -\int_{\partial\Omega} \frac{\partial}{\partial x_i} \Big\{ \Gamma_{jk}(x - y; \lambda) \Big\}\, n_i(y) f_k(y) \d\sigma(y)
    = -\frac{\partial}{\partial x_i} \slp_\lambda(n_i f)_j(x)
    \end{alignedat}
  \end{align}
  and on the other hand
  \begin{align}
    \label{eq:rep2}
    \begin{alignedat}{1}
    &-\int_{\partial\Omega} \Phi_j(y - x) n_k(y) f_k(y) \d\sigma(y) \\
    &\qquad\qquad= \int_{\partial\Omega} \Phi_l(x - y)\, \delta_{lj} n_k(y) f_k(y) \d\sigma(y)
    = \slp_\Phi(\tilde f^j)(x), 
    \end{alignedat}
  \end{align}
  where $\tilde f^j_l = \delta_{lj} n_k f_k$.
  This shows that
  \begin{align*}
    u_j(x) = (\dlp_\lambda(f))_j(x) = -\frac{\partial}{\partial x_i} \slp_\lambda(n_i f)_j(x) + \slp_\Phi(\tilde f^j)(x).
  \end{align*}
  Therefore, we have for $x \in \ShenCone(q)$, $q \in \partial\Omega$,
  \begin{align*}
    |u(x)| \leq C\, \Big\{ |\nabla_x \slp_\lambda(n_i f)(x)| + |\slp_\Phi(\tilde f^j) | \Big\}, 
  \end{align*}
  with $C > 0$ depending only on $d$.
  Hence, by Lemma~\ref{lem:nontangentialMaximalFunctions}, we derive the followign chain of estimates:
  \begin{align*}
    \| (u)^* \|_{\Ell^p(\partial\Omega)} 
    \leq C\, \Big\{ \sum_{i = 1}^d \| n_i f \|_{\Ell^p(\partial\Omega; \C^d)} + \sum_{j = 1}^d \| \slp_\Phi(\tilde f^j) \|_{\Ell^p(\partial\Omega)}\Big\} 
    \leq C\, \|f\|_{\Ell^p(\partial\Omega; \C^d)},
  \end{align*}
  where $C > 0$ depends on $d$, $p$, $\theta$ and the Lipschitz character of $\Omega$.

  For the proof of \eqref{eq:nontangentialLimitDoubleLayer}, we begin by determining the adjoint of the operator $\K_{\bar\lambda}$.
  To this end, we will first work with truncated operators $\K_{\lambda}^{(\varepsilon)} \colon \Ell^2(\partial\Omega; \C^d) \to \Ell^2(\partial\Omega; \C^d)$ which are defined via
  \begin{align*}
    (\K_\lambda^{(\varepsilon)} f)_j (x)
    &\coloneqq \int_{\partial\Omega} 1_{\EE(x, \,\varepsilon)}(y) \frac{\partial}{\partial x_i} \Gamma_{jk}(x - y; \lambda)  f_k(y) \d\sigma(y) \, n_i(x) \\
    &\qquad - \int_{\partial\Omega} 1_{\EE(x, \,\varepsilon)}(y)\, \Phi_k(x - y) f_k(y) \d\sigma(y) \, n_j(x),
  \end{align*}
  for $x \in \partial\Omega$ and $\EE(x,\varepsilon) \coloneqq \R^d \setminus \BB(x, \varepsilon)$.
  Now, for $f \in \Ell^p(\partial\Omega, \C^d)$ and $g \in \Ell^q(\partial\Omega, \C^d)$ with $1/p + 1/q = 1$ we calculate
  \begin{align*}
    \Big\langle \K_{\bar\lambda}^{(\varepsilon)}f, g \Big\rangle
    &= \int_{\partial\Omega} (\K_{\bar\lambda}^{(\varepsilon)} f)_j(x) \, \overline{g_j(x)} \d\sigma(x) \\
    &= \int_{\partial\Omega} \int_{\partial\Omega} \frac{\partial}{\partial x_i} \Big\{ \Gamma_{jk} (x - y; \overline\lambda) \Big\} f_k(y) 1_{\EE(x, \,\varepsilon)}(y) \d\sigma(y) \, n_i(x) \, \overline{g_j(x)} \d\sigma(x) \\
    &\quad+ \int_{\partial\Omega} \int_{\partial\Omega} \Phi_k(x -y)f_k(y) 1_{\EE(x,\, \varepsilon)}(y) \d\sigma(y) \, n_j(x) \, \overline{g_j(x)} \d\sigma(x).
  \end{align*}
  Note that $1_{\EE(x,\, \varepsilon)}(y) = 1_{\EE(y,\, \varepsilon)}(x)$ for all $x, y \in \partial\Omega$.
  Now, an application of Fubini's theorem and factoring out $f_k(y)$ gives that 
  \begin{align*}
    \Big\langle \K_{\bar\lambda}^{(\varepsilon)}f, g \Big\rangle 
    =
    \int_{\partial\Omega} f_k(y) \int_{\partial\Omega} \Big\{ & \frac{\partial}{\partial x_i} \big\{  \Gamma_{jk}(x - y; \overline\lambda)\big\}\, n_i(x) \\
    &\qquad \quad - \Phi_k(x - y)  n_j(x) \Big\}\, 1_{\EE(y,\, \varepsilon)}(x)\, \overline{g_j(x)} \d\sigma(x) \d\sigma(y).
  \end{align*}
  Therefore, we see that the adjoint of the truncated operator $\K_{\bar\lambda}^{(\varepsilon)}$ is given by
  \begin{align*}
    \big( ( \K_{\bar\lambda}^{(\varepsilon)})^* g\big)_k(y)
    = \int_{\partial\Omega} \Big\{ &\frac{\partial}{\partial x_i} \{ \Gamma_{jk}(x - y; \lambda)\} n_i(x) \\
    &\qquad\quad - \Phi_k(x - y) n_j(x)\Big\}\, 1_{\EE(y,\,\varepsilon)}(x) \, g_j(x) \d\sigma(x), 
  \end{align*}
  for $y \in \partial\Omega$ since $\overline{\Gamma_{jk}(x - y; \lambda)} = \Gamma_{jk}(x - y; \overline\lambda)$.

  In the next step, we will go from truncated operators to principal value operators through the dominated convergence theorem. 
  For this to work we will look for suitable majorants.
  For $x \in \partial\Omega$, we estimate
  \begin{align}
    \big|(\K_{\bar\lambda}^{(\varepsilon)}f)_j (x)\big|
    &= \bigg| \int_{|x - y| > \varepsilon} \frac{\partial}{\partial x_i} \Big\{ \Gamma_{jk}(x - y; \lambda) \Big\} f_k(y) \d\sigma(y) \, n_i(x)  \nonumber\\
    &\qquad - \int_{|x - y| > \varepsilon} \Phi_k(x - y) f_k(y)\, n_j(x) \d\sigma(y)\, \bigg| \nonumber\\
    \label{eq:pIntegrableMajorant}
    &\leq T_{\bar\lambda}^*(f)(x) + A^*(fn_j)(x).
  \end{align}
  We know from Lemma~\ref{lem:lpBoundednessT} and the respective result for $A^*$ that the right-hand side of inequality \eqref{eq:pIntegrableMajorant} is $p$-integrable and hence we get from dominated convergence
  \begin{align*}
    \lim_{\varepsilon \to 0} \Big\langle \K_{\bar\lambda}^{(\varepsilon)} f, g \Big\rangle = \Big\langle \K_{\bar\lambda} f, g \Big\rangle\,.
  \end{align*}
  With a similar argument we get
  \begin{align*}
    \big| \big((\K_{\bar \lambda}^{(\varepsilon)})^* g \big)_k(y) \big|
    &=
    \bigg| \int_{|x - y| > \varepsilon} \frac{\partial}{\partial x_i} \Big\{ \Gamma_{jk}(x - y; \overline\lambda) \Big\} \, n_i(x) g_j(x) \d \sigma(x)\\
    &\qquad - \int_{|x - y| > \varepsilon} \Phi_k(x - y) n_j(x) g_j(x)   \d\sigma(x) \bigg| \\
    &\leq \sum_{i = 1}^d T_\lambda^*(n_i g)(y)  + \sum_{i = 1}^d A^*(\tilde g^j)(y),
  \end{align*}
  where $\tilde g^j_l = \delta_{lj} n_k f_k$ and therefore the dominated convergence theorem yields
  \begin{align*}
    \lim_{\varepsilon \to 0} \Big\langle f, {\Big(\K_{\bar\lambda}^{(\varepsilon)}}\Big)^* g \Big\rangle = \Big\langle f,\K_{\bar\lambda}^{(*)}  g \Big\rangle\,,
  \end{align*}
  where the limit operator $\K_{\bar\lambda}^{(*)}$ is defined via
  \begin{align*}
    \big( ( \K_{\bar\lambda}^{(*)} g\big)_k(y)
    \coloneqq \pv\int_{\partial\Omega} \Big\{ \frac{\partial}{\partial x_i} \big\{ \Gamma_{kj}(x - y; \lambda)\big\} n_i(x) - \Phi_k(x - y) n_j(x) \Big\} g_j(x) \d\sigma(x). 
  \end{align*}
  Of course by the uniqueness of the adjoint operator, the identity $\langle \K_{\bar\lambda} f, g \rangle = \langle f, \K_{\bar\lambda}^{(*)} g \rangle$ shows that $\K_{\bar\lambda}^* = \K_{\bar\lambda}^{(*)}$.
  Note that we have used the symmetry of the matrix $(\Gamma_{jk})_{1 \leq j,k \leq d}$.

  In the last part of this proof we will show that the equality \eqref{eq:nontangentialLimitDoubleLayer} holds.
  Note that by \eqref{eq:rep1} and \eqref{eq:rep2} we have made Lemma~\ref{lem:traceFormulas} accessible.
%  \begin{align*}
%    \int_{\partial\Omega} \frac{\partial}{\partial y_i} \{ \Gamma_{jk}(y - x; \lambda) \} n_i(y) f_k(y) \d\sigma(y)
%    &= -\int_{\partial\Omega} \frac{\partial}{\partial x_i} \{ \Gamma_{jk}(x - y; \lambda) \} n_i(y) f_k(y) \d\sigma(y)\\
%    &= -\frac{\partial}{\partial x_i} \slp(n_i f)_j(x)
%  \end{align*}
%  and on the other hand
%  \begin{align*}
%    -\int_{\partial\Omega} \Phi_j(y - x) n_k(y) f_k(y) \d\sigma(y)
%    = \int_{\partial\Omega} \Phi_l(x - y) \, \delta_{lj} n_k(y) f_k(y) \d\sigma(y)
%    = \slp_\Phi(\tilde f^j)(x), 
%  \end{align*}
%  where $\tilde f^j_l = \delta_{lj} n_k f_k$.
  For $x \in \partial\Omega$ we can now calculate
  \begin{align*}
    &\Big( \int_{\partial\Omega} \frac{\partial}{\partial y_i} \Big\{ \Gamma_{jk}(y - \cdot\, ; \lambda) \Big\} n_i(y) f_k(y) \d\sigma(y) \Big)_\pm(x)\\
    &\qquad\qquad= - \big( \frac{\partial}{\partial x_i} \slp_\lambda(n_i f)_j\big)_\pm(x) \\
    &\qquad\qquad= \mp \frac{1}{2} \big\{ n_i(x) n_i(x) f_j(x) - n_j(x) n_i(x) n_k(x) n_i(x) f_k(x)  \big\} \\
    &\qquad\qquad\qquad - \pv\int_{\partial\Omega} \frac{\partial}{\partial x_i} \Big\{\Gamma_{jk}(x - y; \lambda) \Big\} \, n_i(y) f_k(y) \d\sigma(y) \\
    &\qquad\qquad= \mp \frac{1}{2} \big\{ f_j(x) - n_j(x) n_k(x) f_k(x) \big\} \\
    &\qquad\qquad\qquad + \pv\int_{\partial\Omega} \frac{\partial}{\partial y_i} \Big\{\Gamma_{jk}(y - x; \lambda) \Big\}\,  n_i(y) f_k(y) \d\sigma(y),
  \end{align*}
  where we used trace formula \eqref{eq:traceFormula}.
  A similar procedure for the second integral part of the double layer potential gives
  \begin{align*}
    &-\Big(\int_{\partial\Omega} \Phi_j(y - \cdot\,) \, n_k(y) f_k(y) \d\sigma(y)\Big)_\pm(x) \\
    &\qquad\qquad= \big(\slp_\Phi(\tilde f^j)\big)_\pm(x)  \\
    &\qquad\qquad= \mp \frac{1}{2} n_k(x) \tilde f^j_k(x) - \pv \int_{\partial\Omega} \Phi_k(x - y) \tilde f^j_k(x) \d\sigma(y) \\
    &\qquad\qquad= \mp \frac{1}{2} n_j(x) n_k(x) f_k(x) - \pv \int_{\partial\Omega} \Phi_j(x - y)\, n_k(x) f_k(x) \d\sigma(y).
  \end{align*}
  Putting everything together we get
  \begin{align*}
    (u_j)_\pm(x) = \mp\frac{1}{2} f_j(x) + (\K_{\bar\lambda}^* f)_j(x)
  \end{align*}
  and the proof is finished.
\end{proof}
