% Vorlage fuer Abschlussarbeiten am Fachbereich Mathematik der TU Darmstadt.
% Geeignet fuer Bachelorarbeiten und Masterarbeiten.

\documentclass[12pt,a4paper,twoside]{report}
% Benutzt man die Option draft, so kann man die Umbrueche ueberpruefen.

% Hier werden alle benoetigten Pakete und Einstellungen geladen. Auch
% hier sind Sie frei diese direkt in der Praeambel zu laden.

% Folgende Pakete sind Minimalvoraussetzung.

% Anpassung von LaTeX an die deutsche Sprache. Zum Beispiel wird
% \chapter{} direkt als "`Kapitel"' uebersetzt usw. Ausserdem
% wird die Silbentrennung nach deutscher Rechtschreibung vorgenommen.
% Sonderzeichen, z.B. Umlaute, koennen direkt eingegeben werden.

%\usepackage[ngerman]{babel}
\usepackage[english]{babel}

% Kodierung fuer unixoide Systeme und Windowssysteme.

\usepackage[latin1]{inputenc}

% Schriftart Times New Roman.

%\usepackage{times}
%\usepackage{lmodern}
\usepackage{mathptmx}

% Erweitert den Zeichenvorrat, so dass z.B. auch Umlaute im PDF-Dokument
% gefunden werden.

\usepackage[T1]{fontenc}

% Zur Einbindung von Bildern.

\usepackage{graphicx}

% Erweiterte enumerate-Umgebung.

\usepackage{enumerate}

% Verschiedene Pakete, die nuetzlich sind um Mathematik in LaTeX zu setzen.

\usepackage{mathtools}
\usepackage{amsmath, amssymb, amsthm, dsfont}
\usepackage{hyperref}

%%%%%%%%%%%%%%%%
% Seitenlayout %
%%%%%%%%%%%%%%%%

% DIV# gibt den Divisor f�r die Layoutberechnung an.
% Vergr��ern des Divisors vergr��ert den Textbereich.
% BCOR#cm gibt die Breite des Bundstegs an.
\usepackage[DIV14,BCOR2cm]{typearea}

% Abstand obere Blattkante zur Kopfzeile ist 2.54cm - 15mm
%\setlength{\topmargin}{-15mm}

% Keine Einrueckung nach einem Absatz.

%\parindent 0pt

% Abstand zwischen zwei Abs\"atzen.

%\parskip 12pt

% Zeilenabstand.
\usepackage{bibgerm}

\linespread{1.25}

% Inhaltsverzeichnis erstellen.

\usepackage{makeidx}
\makeindex

%\usepackage[inline]{showlabels}


% Hier werden Makros und Umgebungen eingebunden. Diese werden separat
% in der Datei befehle.tex definiert. Sie sind frei diese Befehle auch
% direkt in der Praeambel zu definieren.
% Makros

\newcommand{\N}{\mathbb{N}} % natuerliche Zahlen
\newcommand{\Z}{\mathbb{Z}} % ganze Zahlen
\newcommand{\Q}{\mathbb{Q}} % rationale Zahlen
\newcommand{\R}{\mathbb{R}} % reelle Zahlen
\newcommand{\C}{\mathbb{C}}
\DeclareMathOperator{\Ret}{Re} % Guter Realteil

%%Mengen
\newcommand{\BB}{\mathrm{B}} 
\newcommand{\Dom}{\mathrm{D}} 
\newcommand{\EE}{\mathrm{E}} 
\newcommand{\verCone}{\Gamma_\mathrm{V}} 
\newcommand{\ShenCone}{\Gamma} 

%%Funktionenr�ume
\newcommand{\CC}{\mathrm{C}} 
\newcommand{\Ell}{\mathrm{L}} 
\newcommand{\Li}{\mathfrak{L}}
\newcommand{\WW}{\mathrm{W}}
\newcommand{\HH}{\mathrm{H}}

%%Mathematische Operatoren

\newcommand{\aform}{\mathfrak{a}}
\newcommand{\loc}{\mathrm{loc}}
\newcommand{\PP}{\mathbb{P}}
\newcommand{\diam}{\operatorname{diam}}
\newcommand{\K}{\mathcal{K}}
\newcommand{\slp}{\mathcal{S}}
\newcommand{\MM}{\mathcal{M}}
\newcommand{\supp}{\operatorname{supp}}
\newcommand{\dlp}{\mathcal{D}}
\DeclareMathOperator{\pv}{p. v.}
\renewcommand{\Im}{\operatorname{Im}}
\renewcommand{\Re}{\operatorname{Re}}

%Differentialoperatoren
\let\divsymb=\div % rename builtin command \div to \divsymb
\renewcommand{\div}[1]{\mathrm{div\,} #1} % for divergence
\renewcommand{\d}[1]{\ensuremath\, {\operatorname{d}\!{#1}}}
\newcommand{\DD}{\ensuremath\,{\operatorname{D}}}

\newcommand{\ii}{\mathrm{i}}
\newcommand{\e}{\mathrm{e}}

% Umgebungen f�r Definitionen, S�tze, usw.
\theoremstyle{plain}
\newtheorem{thm}{Theorem}[chapter]
\newtheorem{lem}[thm]{Lemma}
\newtheorem{cor}[thm]{Corollary}
\newtheorem{prop}[thm]{Proposition}


\theoremstyle{definition}
\newtheorem{defn}[thm]{Definition}
\newtheorem{Beispiel}{Beispiel}[chapter]
\newtheorem*{RBsp}{Regelbeispiel}

\theoremstyle{remark}
\newtheorem{rem}[thm]{Remark}

\def\Xint#1{\mathchoice
  {\XXint\displaystyle\textstyle{#1}}%
  {\XXint\textstyle\scriptstyle{#1}}%
  {\XXint\scriptstyle\scriptscriptstyle{#1}}%
  {\XXint\scriptscriptstyle\scriptscriptstyle{#1}}%
\!\int}
\def\XXint#1#2#3{{\setbox0=\hbox{$#1{#2#3}{\int}$ }
\vcenter{\hbox{$#2#3$ }}\kern-.6\wd0}}
\def\ddashint{\Xint=}
\def\dashint{\Xint-}


\begin{document}
% Auf der Titelseite und im Inhaltsverzeichnis sollen keine
% Seitenzahlen dargestellt werden.
\pagestyle{empty}

% Hier wird die Titelseite eingebunden.
\begin{titlepage}
  \begin{center}
    \includegraphics[width=0.5\linewidth]{TU_Darmstadt_Logo.pdf}
    \vfill
    
    \large{Fachbereich Mathematik}
    \vfill
    
    \large{Masterthesis}
    \vfill

    \huge{On Resolvent Estimates in $\Ell^p$ for the Stokes Operator in Lipschitz Domains}
    \vfill
    
		\large
    Fabian Gabel

    \large 15.04.2018
    \vfill
\begin{tabular}{rl}
    Supervisor:& Dr. Patrick Tolksdorf
    \\
    Zweiter Gutachter:& Name des Zweitgutachters
\end{tabular}
  \end{center}
\end{titlepage}


% Inhaltsverzeichnis erstellen.
\cleardoublepage
\tableofcontents

% Ab sofort werden Seitenzahlen in der Kopfzeile angezeigt.
\pagestyle{headings}

% Einbinden der Einleitung.
\phantomsection
\chapter*{Introduction}
\addcontentsline{toc}{chapter}{Introduction}

In the solution theory for nonlinear partial differential equations, an integral part of the solution process is often to develop a semigroup theory for the linearization of the equation.
In the case of the famous \emph{Navier-Stokes equations} which for a given domain $\Omega \subseteq \R^d$, $d \geq 2$, describe the behavior of a Newtonian fluid over time, the linearization is given by the \emph{Stokes equation}s
\begin{align*}
  \partial_t u - \Delta u + \nabla \pi &= 0 \quad\text{in } \Omega, t > 0,\\
  \div(u) &= 0 \quad\text{in } \Omega, t > 0, \\
  u(0) &= a \quad\text{in } \Omega, \\
  u &= 0 \quad\text{on } \partial\Omega, t > 0,
\end{align*}
where $u \colon \R^+ \times \Omega \to \R^d$ stands for the velocity field and $\pi \colon \R^+ \times \Omega \to \R$ represents the pressure of the fluid.
The so-called \emph{Stokes semigroup} $(\e^{-tA})_{t \geq 0}$ describes the evolution of the velocity $u$ and the \emph{Stokes operator} $A$ corresponds to the term ``$-\Delta u + \nabla \pi$'' in the Stokes equations. 

Having a semigroup makes it possible to look for \emph{mild solutions} to the Navier-Stokes equations using a variation of constants formula to construct an iteration method.
This approach was introduced by Fujita and Kato \cite{fujitaKato,katoFujita} and builds mainly on resolvent estimates for the Stokes operator $A$ and the analyticity property of the Stokes semigroup.
Fujita and Kato applied their methods to find solutions in $\Ell^2$ spaces, given that the initial values $a$ were to be found in domains of fractional powers of the Stokes operator.
Fractional power domains are in general difficult to describe which is why Fujita and Kato suggested that an $\Ell^p$ theory for the Navier-Stokes equations could overcome this problem, see \cite[p.\@~313]{fujitaKato}.
%The interest in an $\Ell^p$ theory of the Stokes equations became eminent as they were looking to find spaces with pertinence conditions that are easier to verify.
Resolvent estimates in $\Ell^p$ for the Stokes operator and the analyticity of the Stokes semigroup are crucial ingredients of this classical functional analytic approach to the solution theory of the Navier-Stokes equations.

On smooth domains, the first milestone in the direction of an $\Ell^p$ theory was laid by Giga \cite{giga} and Solonnikov \cite{solonnikov} who proved that on bounded smooth domains the Stokes operator generates a bounded analytic semigroup in $\Ell^p(\Omega)$ for $1 < p < \infty$.
This result was then used by Giga and Miyakawa \cite{gigaMiyakawa} to prove the existence of mild solutions given $\Ell^3$ initial data.
Kato \cite{katoExtend} then extended this theory to the whole space and later Giga \cite{gigaAdjust} adjusted the approach for bounded smooth domains.

A natural assumption in real world applications is that the boundary of the domain $\Omega$ is not smooth but merely Lipschitz continuous. 
On Lipschitz domains, the analysis of partial differential equations is more complicated as classical localization techniques fail due to the lack of smoothness of the boundary.
%In order to extend the advances of the $\Ell^p$ theory on smooth domains to Lipschitz domains, the first existence and unique results were given by Mitrea and Monniaux.
In order to overcome these problems, new techniques had to be developed which gave rise to the work of Fabes, Kenig and Verchota~\cite{fabesKenigVerchota} who constructed solutions to the $\Ell^2$ Dirichlet problem of the Stokes equations by the method of layer potentials.

As on smooth domains, the further expansion of the $\Ell^p$ theory by means of the classical iteration method was reliant on the analyticity of the Stokes semigroup.
Due to the boundedness properties of the Helmholtz projection, Taylor conjectured in his paper \cite{taylor} from 2000 that the analyticity of the Stokes semigroup would only hold in a neighborhood of $p = 2$, namely for $p \in  (3/2 - \varepsilon, 3 + \varepsilon)$ in the three dimensional case.
His conjecture was supported by the fact that, not much later, Deuring showed in \cite{deuring} the existence of a three dimensional Lipschitz domain such that the needed resolvent estimates would fail for sufficiently large $p$.

Twelve years passed until a positive result could be given to Taylor's conjecture:  Shen showed in his seminal paper \cite{Shen2012} that in bounded Lipschitz domains $\Omega \subseteq \R^d$, $d \geq 3$, for all $p \in (2d/( d + 1) - \varepsilon , 2d/(d - 1) + \varepsilon)$ the Stokes operator generates a bounded analytic semigroup on $\Ell^p_\sigma(\Omega)$.
Shen's result on the analyticity of the semigroup thus made the Fujita-Kato approach available for the study of the Navier-Stokes equations on Lipschitz domains which, together with other methods, was used by Tolksdorf to prove the existence of mild solutions in $\Ell^3$ to the Navier-Stokes equations \cite{tolksdorf}.

In the present thesis, we will study Shen's approach to the analyticity problem of the Stokes semigroup in $\Ell^p$ for bounded Lipschitz domains in $\R^d$.
While Shen's result was only formulated for $d \geq 3$, we will show that his approach can be extended to the two dimensional case by proving suitable estimates on the fundamental solutions of the Stokes resolvent problem.

Except for the first chapter, in which we will gather the fundamentals and formulate the problem under consideration, the rest of this thesis will closely follow the structure of Shen's work \cite{Shen2012}. 

In Chapter \ref{chap:2}, we will derive the central estimates of the fundamental solutions to the Stokes resolvent problem on $\R^d$ by taking advantage of their explicit representation formula.

Chapter \ref{chap:3} will introduce the method of boundary layer potentials for the solution of the Stokes resolvent problem on bounded Lipschitz domains.
We will discover the central relation between boundary values that are attained nontangentially and the singular integral operators that provide a representation formula for solutions to the Stokes resolvent problem. 

In Chapter \ref{chap:4}, we take a more general look on solutions to the Stokes resolvent problem and derive Rellich-type estimates for these solutions.

Based on the results of Chapter 3 and 4, in Chapter \ref{chap:5}, we will study the $\Ell^2$ Dirichlet problem of the Stokes resolvent system. In this chapter, we will see that for given boundary values in $\Ell^2(\partial\Omega; \C^d)$, there exists a unique solution to the $\Ell^2$ Dirichlet problem which is given via a double layer potential.

Finally, in Chapter \ref{chap:6}, we will tackle the problem of analyticity of the Stokes operator. In this chapter, we will derive the necessary resolvent estimates on $\Ell^p$ by making use of Shen's extrapolation theorem which can be considered a refined version of the Calder\'on-Zygmund Lemma.

This last paragraph is reserved to mention some of the persons that were involved in the course of this thesis.
I thank Prof. Dr. Robert Haller-Dintelmann for taking the responsibility of supervising this thesis.
Furthermore, I want to express my gratitude to Dr. Patrick Tolksdorf. 
This thesis would not have been possible without his generous support in innumerable occasions.
I also want to thank Sebastian Bechtel for proofreading parts of this thesis. 
Last but not least, I want to thank my family for supporting me during my studies.



% Einbinden der einzelnen Kapitel (zur �bersicht eine Datei je Kapitel)

\chapter{Fundamentals}

\section{Lipschitz-Domains}

\subsection{Basic Definitions and Properties}

\subsection{Tangential and Nontangential Operators}


\section{The Stokes Operator}



\chapter{Estimating Fundamental Solutions}
\label{chap:2}

The purpose of this section is to study fundamental solutions of the Stokes resolvent problem and to deduce related estimates which will be crucial for  the next chapters.
Before working on the Stokes resolvent problem, we will take a look at the atoms of the fundamental solution of this problem: the Hankel functions. 

As a basis for the subsequent sections and chapters, let us fix recurring quantities regarding sectors in the complex plane $\C$.

Let $\theta \in (0, \pi/2)$ and $\lambda \in \Sigma_\theta$ as in Section \ref{sec:stokesOperator}.  
The polar form of $\lambda$ is given as $\lambda = r \e^{\ii \tau}$ with $0 < r < \infty$ and $-\pi + \theta < \tau < \pi - \theta$.
Now, set 
\begin{align*}
  k \coloneqq \sqrt{r} \, \e^{\ii(\pi + \tau)/2}.
\end{align*}
Then, we have
\begin{align*}
  k^2 = -\lambda\quad\text{ and }\quad \frac{\theta}{2} < \arg(k) < \pi - \frac{\theta}{2}
\end{align*}
as it holds
\begin{align*}
  \arg(k) = \frac{\pi + \tau}{2} &> \frac{\pi}{2} + \frac{-\pi + \theta}{2} = \frac{\theta}{2} 
  \intertext{on the one hand and}
  \arg(k) = \frac{\pi + \tau}{2} &< \frac{\pi}{2} + \frac{\pi - \theta}{2} = \pi  - \frac{\theta}{2}
\end{align*}
on the other hand.
The preceding calculation gives rise to the following estimate:
\begin{align}
  \label{eq:imaginaryPartEstimate}
  \Im(k) > \sqrt{|\lambda|}  \sin(\theta/2) > 0.
\end{align}
Indeed, we have 
\begin{align*}
  \Im( k) = \sqrt{r} \sin\left( \frac{\pi + \tau}{2} \right) = \sqrt{|\lambda|} \sin\left( \frac{\pi + \tau}{2} \right)\quad\text{and}\quad \frac{\theta}{2} < \frac{\pi + \tau}{2} < \pi - \frac{\theta}{2}
\end{align*}
which gives for $\tau$ with $\frac{\pi + \tau}{2} \leq \frac{\pi}{2}$ that $\sin(\frac{\pi + \tau}{2}) \geq \sin({\theta}/{2} )$ and for $\tau$ with $\frac{\pi + \tau}{2} > \frac{\pi}{2}$ that $\sin(\frac{\pi + \tau}{2}) > \sin(\pi - {\theta}/{2} ) = \sin({\theta}/{2} )$.

\section{Hankel Functions and the Helmholtz Equation}
\label{sec:hankel}

Before diving into fundamental solutions of the Stokes resolvent problem, we will first consider a fundamental solution for the (scalar) Helmholtz equation in $\R^d$
\begin{align*}
-\Delta u + \lambda u = 0.
\end{align*}
One fundamental solution with pole at the origin is given by
\begin{align}
  \label{eq:definitionFundamentalHelmholtz}
  G(x; \lambda) = \frac{\ii}{4 ( 2\pi )^{\frac{d}{2} - 1}} \cdot \frac{1}{|x|^{d - 2}} \cdot (k |x|)^{\frac{d}{2} - 1} H_{\frac{d}{2} - 1}^{(1)} (k|x|),
\end{align}
see McLean \cite[Eq.\@~9.14)]{mclean}, where $H_{\nu}^{(1)}(z)$ is the Hankel function of the first kind which according to Lebedev \cite[Sec.\@~5.11]{lebedev} can be also be written as
\begin{align}
  \label{eq:integralRepresentationHankel}
  H_\nu^{(1)}(z) = \frac{2^{\nu + 1} \, \e^{\ii(z - \nu \pi)}\, z^\nu}{\ii\, \sqrt{\pi}\, \Gamma(\nu + \frac{1}{2})} \int_0^\infty \e^{2 z \ii s} s^{\nu - \frac{1}{2}} (1 + s)^{\nu - \frac{1}{2}} \d s.
\end{align}
This formula holds for $\nu > -\frac{1}{2}$ and $0 < \arg(z) < \pi$.
We will usually set 
\begin{align*}
  \nu = \nu_d = \frac{d}{2} - 1 \quad\text{and}\quad z = k|x|. 
\end{align*}
Note that by \eqref{eq:imaginaryPartEstimate} we will always have $\Im(z) > 0$. 
Since $\nu_d < \nu_{d + 1}$ for all $d \geq 2$ and $\nu_2 = 0$, formula~\eqref{eq:integralRepresentationHankel} will hold for all dimensions $d \geq 2$ and all $x \in \R^d$.

In the case $d = 2$, formula~\eqref{eq:definitionFundamentalHelmholtz} simplifies to 
\begin{align}
  \label{eq:2dDefinitionFundamentalHelmholtz}
  G(x;\lambda) = \frac{\ii}{4} H_{0}^{(1)}(k|x|).
\end{align}
In the case $d = 3$, one has an even easier formula, namely
\begin{align}
  \label{eq:3dDefinitionFundamentalHelmholtz}
  G(x; \lambda) = \frac{\ii}{4\, (2\pi)^{1/2}} \cdot \frac{1}{|x|} \cdot (k |x|)^{1/2} H_{1/2}^{(1)}(k|x|) =  \frac{\e^{\ii k |x|}}{4\, \pi\, |x|},
\end{align}
which is due to this easy formula for $H_{1/2}^{(1)}(z)$:
\begin{align}
  \label{eq:Hankel3d}
  H_{1/2}^{(1)}(z) = -\ii\, \bigg(\frac{2}{\pi z} \bigg)^{1/2} \e^{\ii z},
\end{align}
see Lebedev \cite[Eq.\@~(5.8.4)]{lebedev} or McLean \cite[Eq.\@~(9.15)]{mclean}.

Our first estimate is concerned with estimates on the fundamental solution $G(\,\cdot\,;\lambda)$ and its derivatives.
The main concern of this lemma is with the asymptotic behavior of $G(\,\cdot\,; \lambda)$ for large values of $|x|$.

\begin{lem}
  \label{lem:estimateHelmholtzDerivatives}
  Let $\lambda \in \Sigma_\theta$.
  Then
  \begin{align}
    \label{eq:estimateHelmholtzDerivatives}
    |\nabla_x^l G(x; \lambda)| \leq \frac{ C_l \, \e^{-c \sqrt{|\lambda|} |x|}}{|x|^{d - 2 + l}}
  \end{align}
  for any integer $l \geq 0$ if $d \geq 3$ and for $l \geq 1$ if $d = 2$.
  Here, $c > 0$ depends only on $\theta$ and $C_l$ depends only on $d$, $l$ and $\theta$.

  Let $d = 2$. Then $|G(x; \lambda)| = o(1)$ as $|x| \to \infty$.
\end{lem}

\begin{proof}
  We start with the case $l = 0$ and $d \geq 3$.
  Let $\Im(z) > 0$ and $\nu = \nu_d = (d/2) - 1$.
  In particular, we have $\nu - \frac{1}{2} \geq 0$.
  Then, \eqref{eq:integralRepresentationHankel} gives
  \begin{align}
    \label{eq:firstEstimate}
    |H_\nu^{(1)}(z)|
    &\leq C_d \, \e^{-\Im(z)} |z|^\nu \int_0^\infty \e^{-2s \Im(z)} s^{\nu - \frac{1}{2}} (1 + s)^{\nu - \frac{1}{2}} \d s,
  \end{align}
  where $C_d > 0$ depends only on $d$.
  We apply the substitution rule with $t = s + (1/2)$ and calculate
  \begin{align*}
    \e^{\frac{-\Im(z)}{2}} \int_0^\infty \e^{-2s \Im(z)} s^{\nu - \frac{1}{2}} ( 1 + s )^{\nu - \frac{1}{2}} \d s
    &\leq \int_0^\infty \e^{- (s + \frac{1}{2}) \Im(z)} s^{\nu - \frac{1}{2}} ( 1 + s )^{\nu - \frac{1}{2}} \d s \\
    &= \int_{\frac{1}{2}}^\infty \e^{-t \Im(z) } \Big(t^2 - \frac{1}{4} \Big) ^{\nu - \frac{1}{2}} \d t \\
    &\leq \int_0^\infty e^{-t \Im(z)} t^{2\nu - 1} \d t \\
    &= \int_0^\infty e^{-u}  u^{2 \nu - 1} \Im(z)^{1 - 2\nu} \Im(z)^{-1} \d u \\
    &= \Im(z)^{-2 \nu} \int_0^\infty \e^{-u} u^{2 \nu - 1} \d u \\
    &= C_\nu \Im(z)^{-2 \nu},
  \end{align*}
  where we also used the substitution rule with $u = t \Im(z)$.
  Now, we multiply \eqref{eq:firstEstimate} by $|z|^{v}$ and reuse the previous estimate to arrive at
  \begin{align*}
    |z|^\nu\,  \big|H_\nu^{(1)}(z)\big| \leq C_d \, C_\nu\, |z|^{2 \nu} \, \Im(z)^{-2\nu} \e^{-\frac{\Im(z)}{2}},
  \end{align*}
  which for $z = k|x|$ gives
  \begin{align}
    \label{eq:zHEstimate}
    |kx|^\nu \, \big|H_\nu^{(1)}(k |x|)\big| \leq C \sin(\theta/2)^{-2\nu} \, \e^{-\frac{1}{2} \sin(\theta/2) \sqrt{|\lambda|} |x|},
  \end{align}
  where $C > 0$ depends only on $d$ and we used \eqref{eq:imaginaryPartEstimate} to estimate
  \begin{align*}
    (|kx|)^{2\nu} \cdot \Im(k|x|)^{-2\nu} 
    = |\lambda|^\nu \cdot \Im(k)^{-2\nu} 
    \leq \sin(\theta/2)^{-2\nu}.
  \end{align*}
  Note that per constructionem $k$ has positive imaginary part.
  Using \eqref{eq:definitionFundamentalHelmholtz}, we estimate for $d \geq 3$
  \begin{align*}
    |G(x; \lambda)| 
    \leq C\,  |x|^{2 - d} \e^{-c \sqrt{|\lambda|} |x|}
  \end{align*}
  and it is clear that the generic constant $C>0$ depends on $d$ and $\theta$ while $c > 0$ depends only on $\theta$.
  This gives the estimate for $l = 0$ and $d \geq 3$.

  Using the relation for the derivatives of Hankel functions which one finds in the book of Lebedev~\cite[Eq.\@~(5.6.3)]{lebedev},
  \begin{align*}
    \frac{\d{}}{\d z} \Big\{ z^{-\nu} H_\nu^{(1)}(z) \Big\} = -z^{-\nu} H_{\nu + 1}^{(1)} (z),
  \end{align*}
  we inductively establish the estimate~\eqref{eq:estimateHelmholtzDerivatives} for $l \geq 1$ and $d \geq 2$:
  For $1 \leq j \leq d$, we calculate using the product and chain rule 
  \begin{align}
    \label{eq:derivativeOfG}
    \nabla_x^{} G(x; \lambda)
    &= C  \, \Big\{\, |x|^{1 - d} \cdot (k|x|)^{\frac{d}{2} - 1} H_{\frac{d}{2} - 1}^{(1)}(k|x|) 
       - |x|^{2 - d} \cdot (k|x|)^{\frac{d}{2} - 1} H_{\frac{d}{2}}^{(1)}(k|x|) \cdot k \, \Big\} \nonumber\\[0.5em]
    &= C \, |x|^{1 - d}\Big\{\,(k|x|)^{\frac{d}{2} - 1} H_{\frac{d}{2} - 1}^{(1)}(k|x|) -  (k|x|)^{\frac{d}{2}} H_{\frac{d}{2}}^{(1)}(k|x|) \, \Big\},
  \end{align}
  where $C > 0$ is a generic constant that depends on $d$. 
  Note that the first summand in \eqref{eq:derivativeOfG} does not arise in the case $d = 2$ as is easily seen from equation~\eqref{eq:2dDefinitionFundamentalHelmholtz}.
  The terms in the bracket can now be estimated individually by \eqref{eq:zHEstimate}.
  The extension of this proof to orders of differentiation $l \geq 2$ is straightforward using the Leibniz product rule for higher derivatives.

  Now, for the last part of the proof, let us verify the claim regarding the asymptotic behavior of $|G(x; \lambda)|$ if $d = 2$.
  Based on the integral representation~\eqref{eq:integralRepresentationHankel}, Lebedev derived an asymptotic expansion for the Hankel function \cite[Sec.\@~5.11, Eq.\@~(5.11.3)]{lebedev}. 
  For $\nu = (d/2) - 1 = 0$ and $z = k|x|$ this expansion reads
  \begin{align*}
    H_0^{(1)}(z) = \left(\frac{2}{\pi z}\right)^{1/2} \e^{\ii (z - (1/4) \pi)} \Big( 1 + O(|z|^{-1})\Big).
  \end{align*}
  As $\Im(z) > 0$, we see that $\big|H_0^{(1)}(k |x|)\big| = O(|x|^{-1/2})$.
  Due to the simple structure of $G(x; \lambda)$ for $d = 2$, as shown in equation~\eqref{eq:2dDefinitionFundamentalHelmholtz}, the claim follows easily.
\end{proof}

In the derivation of the next estimates, we will use the following useful interior estimate for solutions to Poisson's equation which we state her for further use.
\begin{lem}
  \label{lem:interiorEstimatePoisson}
  Let $r > 0$ and $x \in \R^d$, $d \geq 2$. 
  If $w \in \CC^{k}(\BB(x,r)) \cap \CC^0(\overline{\BB(x,r)})$ is a solution to $\Delta w = f$ in $\BB(x,r)$ for $f \in \CC^{k - 1}(\BB(x,r))$, then,
  \begin{align}
    \label{eq:interiorEstimatePoisson}
    \big|\nabla^l w(x)\big| \leq C r^{-l} \sup_{\BB(x,\,r)} |w| + C \max_{0 \leq j \leq l - 1} \sup_{\BB(x,\,r)} r^{j - l + 2} \, \big|\nabla^j f\big|, \quad l \leq k,
  \end{align}
  where $C > 0$ only depends on $d$ and $l$.
\end{lem}

\begin{proof}
  If $l = 1$, then estimate~\eqref{eq:interiorEstimatePoisson} is a consequence of the \emph{comparison principle} and a proof of this fact can be found in the book of Gilbarg and Trudinger \cite[Sec.\@~3.4,~~ Eq.\@~(3.16)]{gilbarg}.
  We will now use this estimate to inductively deduce the estimates for higher derivatives:
  Note that by translating from $x$ to $0$ and rescaling like
  \begin{align*}
    u_r(x) \coloneqq u(rx) \quad\text{and}\quad f_r(x) \coloneqq r^2 f(rx)
  \end{align*}
  we may assume that $\Delta w = f$ in $\BB(0,1)$ and that it suffices to prove
  \begin{align}
    \label{eq:interiorEstimatePoissonSimple}
    \big|\nabla^l w(0) \big| \leq C \sup_{\BB(0,\,1)} |w| + C \max_{0 \leq j \leq l - 1} \sup_{\BB(0,\,1)} \big|\nabla^j f\big|
  \end{align}
  for $l > 1$.
  By the Schwarz theorem we have that if $w$ solves Poisson's equation with right-hand side $f$ and $w$ and $f$ are sufficiently regular, then $\nabla^l w$ solves Poisson's equation with right-hand side $\nabla^l f$.
  We thus estimate inductively
  \begin{align*}
    \big|\nabla^l w(0)\big|
    &\leq C_l \sup_{\BB(0,\, (1/2)^{l - 1})} \big|\nabla^{l - 1} w\big| + C_l \sup_{\BB(0,\, (1/2)^{l-1})} \big|\nabla^{l - 1} f\big| \\[0.5em]
    &\leq C_l \sup_{\BB(0,\, (1/2)^{l-2})} \big|\nabla^{l - 2} w\big| + C_l\, \bigg(\sup_{\BB(0,\,1)} \big|\nabla^{l - 2} f\big| + \sup_{\BB(0,\,1)} \big|\nabla^{l - 1} f\big| \,\bigg) \\[0.5em]
    %&\leq \dots \\
    &\leq C_l  \sup_{\BB(0,\,1)} |w| + C_l \sum_{j = 0}^{l - 1} \sup_{B(0,\,1)} |\nabla^j f|
  \end{align*}
  which readily yields the desired estimate.
\end{proof}

We will need the following asymptotic expansions for the function $z^\nu H_\nu^{(1)}(z)$ in $\C \setminus (-\infty, 0]$.
The derivation of these asymptotic expansions is based on asymptotic expansions of the \emph{Bessel functions of the first} and \emph{the second kind} and can be found in Tolksdorf \cite[Sec.\@~4.2]{tolksdorf}:
\begin{alignat}{2}
  z^{\nu}H_\nu^{(1)}(z) &= \frac{2^\nu \Gamma(\nu)}{\ii \pi} + \frac{\ii}{\pi} z^2 \log(z) + \omega z^2 + O(|z|^4 \big|\log(z)\,\big|) \quad&&\text{if } d = 4, \label{eq:asymptoticd4}\\[0.5em]
  z^{\nu} H_{\nu}^{(1)}(z) &= \frac{2^\nu \Gamma(\nu)}{\ii\pi} + \frac{2^\nu \Gamma(\nu - 1)}{4 \pi \ii} z^2 + \omega z^3 + O(|z|^4) &&\text{if } d = 5, \label{eq:asymptoticd5}\\[0.5em]
  z^\nu H_\nu^{(1)}(z) &= \frac{2^\nu \Gamma(\nu)}{\ii \pi} + \frac{2^\nu \Gamma(\nu - 1)}{4\pi \ii}z^2 + O(|z|^4 \big|\log(z)\,\big|) &&\text{if } d = 6,\label{eq:asymptoticd6} \\[0.5em]
  z^\nu H_\nu^{(1)}(z) &= \frac{2^\nu \Gamma(\nu)}{\ii \pi} + \frac{2^\nu \Gamma(\nu - 1)}{4 \pi \ii} z^2 + O(|z|^4) &&\text{if } d \geq 7,\label{eq:asymptoticd7}
\end{alignat}
where $\omega \in \C$ is a constant that needs to be chosen depending on $d$, see \cite[Eq.\@~(4.19)]{tolksdorf} and \cite[Eq.\@~(4.20)]{tolksdorf} for $d = 4$ and $d = 5$, respectively.


The next lemma will be concerned with estimating the difference $G(x;\lambda) - G(x; 0)$ (and derivatives of this difference) of the fundamental solution to the scalar Helmholtz equation and the fundamental solution to $-\Delta u = 0$ in $\R^d$ which is given by
\begin{align}
  \label{eq:laplace}
  G(x;0) \coloneqq \begin{cases} -\frac{1}{2\pi} \log(|x|)\,, &\quad\text{for } d = 2\,, \\
                                 c_d \, \frac{1}{|x|^{d - 2}}\,, &\quad\text{for } d > 2\,,  \end{cases}
\end{align}
where inverse of the coefficient $c_d$ is given as as a multiple of the surface measure of the $(d-1)$-dimensional sphere $\mathbb{S}^{d - 1}$:
\begin{align}
  \label{eq:wd}
  c_d = \frac{1}{(d - 2) \, \omega_d}, \quad\text{with}\quad \omega_d = \frac{2\,\pi^{\frac{d}{2}}}{\Gamma(\frac{d}{2})} = |\mathbb{S}^{d - 1}|.
\end{align}
%Note that $c_2 = (2 \pi)^{-1}$.
By rearranging terms and using the functional equation of the Gamma function 
\begin{align}
  \label{eq:functionalGamma}
  \begin{alignedat}{1}
    \Gamma(z + 1) &= z\, \Gamma(z), \quad z \in \C, \Re(z) > 0 \\
    \Gamma(1) &= 1,
  \end{alignedat}
\end{align}
we get
\begin{align*}
  (d - 2)\, \omega_d 
  = 2  \, \bigg( \frac{d}{2} - 1\bigg)\, \frac{2\, \pi^{\frac{d}{2}}}{\Gamma(\frac{d}{2})}
  =  2 \, \bigg( \frac{d}{2} - 1\bigg)\, \frac{2\, \pi^{\frac{d}{2}}}{\Gamma(\frac{d}{2} - 1)\,(\frac{d}{2} - 1)}
  = \frac{4\, \pi^{\frac{d}{2}}}{\Gamma(\frac{d}{2} - 1)},
\end{align*}
and thus, we will also sometimes use the equivalent definition
\begin{align}
  \label{eq:defncd}
  c_d \coloneqq \frac{\Gamma(\frac{d}{2} - 1)}{4 \pi^{\frac{d}{2}}}.
\end{align}
Furthermore, the leading coefficient of the asymptotic expansions of the Hankel functions \eqref{eq:asymptoticd4}-\eqref{eq:asymptoticd7} for $d \geq 4$ will be denoted as
\begin{align}
  \label{eq:Defnad}
  a_d \coloneqq \frac{2^{\frac{d}{2} - 1}\, \Gamma(\frac{d}{2} - 1)}{\ii \pi}.
\end{align}
Note that the series expansion of the Hankel function in the case $d = 3$, see \eqref{eq:Hankel3d}, gives that the above definition of $a_d$ also extends to $d = 3$ since we have
\begin{align*}
  -\ii\, \bigg( \frac{2}{\pi} \bigg)^{1/2} = \frac{2\, \Gamma(\frac{1}{2})}{\ii\pi}.
\end{align*}
The coefficients $a_d$ and $c_d$ are related in the following way:
\begin{align*}
  c_d 
  %= \frac{\Gamma(\frac{d}{2} - 1) 2^\nu}{\ii \pi} 
  %=\frac{4 (2 \pi)^{\frac{d}{2} - 1}}{\ii} \frac{\Gamma(\frac{d}{2} - 1)}{4 \pi^{\frac{d}{2}}} 
  =\frac{\ii} {4\, (2 \pi)^{\frac{d}{2} - 1}}\; a_d .
\end{align*}
This allows us to write for $d \geq 3$
\begin{align}
  \label{eq:HelmholtzLaplaceDifference}
  G(x;\lambda) - G(x; 0) = \frac{\ii}{4\,(2\pi)^{\frac{d}{2} - 1}} \cdot \frac{1}{|x|^{d - 2}} \Big\{ (k|x|)^{\frac{d}{2} - 1} H_{\frac{d}{2} - 1}^{(1)}(k|x|) - a_d \Big\}.
\end{align}

The following lemma will help us to estimate the expression~\eqref{eq:HelmholtzLaplaceDifference} together with its derivatives and their two dimensional counterparts.

\begin{lem}
  \label{lem:HelmholtzLaplaceDifference}
  Let $\lambda \in \Sigma_\theta$.
  Then
  \begin{align}
    \label{eq:HelmholtzLaplaceDifferenceEstimate}
    \Big|\,\nabla_x^l \Big\{ G(x; \lambda) - G(x; 0) \Big\}\,\Big| \leq C\, |\lambda| |x|^{4 - d - l}
  \end{align}
  if $d \geq 5$ and $l \geq 0$, where $C  > 0$ depends only on $d$, $l$ and $\theta$.
  If $d = 3$ or $4$, estimate \eqref{eq:HelmholtzLaplaceDifferenceEstimate} holds for $l \geq 1$ and if $d = 2$, the estimate holds for $l \geq 3$.
\end{lem}

\begin{proof}
  \begin{enumerate}[(a)]
    \item In this part, we will show that the desired estimate~\eqref{eq:HelmholtzLaplaceDifferenceEstimate} holds if we assume that $|\lambda| |x|^2 > ({1}/{2})$.
    In this case, Lemma \ref{lem:estimateHelmholtzDerivatives} gives
    \begin{align*}
      \Big|\nabla_x^l \Big\{ G(x; \lambda) - G(x; 0) \Big\} \Big|
      \leq C\, \Bigg\{ \frac{\e^{-c \sqrt{|\lambda|} |x|}}{|x|^{d - 2 + l}} + \frac{1}{|x|^{d - 2 + l}} \Bigg\} \leq C\, \frac{|\lambda|}{|x|^{d - 4 + l}},
    \end{align*}
    where $C > 0$ depends only on $d$, $l$ and $\theta$.
      Therefore, for the remaining proof we will suppose $|\lambda||x|^2 \leq ({1}/{2})$.
  \item In this step, we show that we can restrict ourselves to proving~\eqref{eq:HelmholtzLaplaceDifferenceEstimate} in three cases: (1) $d \geq 5$ and $l = 0$; (2) $d = 3$ or $4$ and $l = 1$; (3) $d = 2$ and $l = 3$.
    
    Suppose \eqref{eq:HelmholtzLaplaceDifferenceEstimate} holds in case (1) and let $l \geq 1$.
    If we set $ w(x) = G(x;\lambda) - G(x;0) $, we have $\Delta_x w = \lambda G(x; \lambda)$ in $\R^d \setminus \{0\}$.
    For $f = \lambda G(x; \lambda)$, estimate~\eqref{eq:interiorEstimatePoisson} now gives
    \begin{align*}
      \big|\nabla^l w(x) \big|
      &\leq C r^{-l} \sup_{\BB(x,r)} |w| + C \max_{0 \leq j \leq l - 1} \sup_{\BB(x,r)} r^{j - l + 2} \big|\nabla^j f \big| \\[0.5em]
      &\leq C r^{-l} \sup_{y \in \BB(x,r)} |\lambda| |y|^{4 - d} + C \sum_{j = 0}^{l - 1} \sup_{y \in \BB(x,r)} r^{j - l + 2} |\lambda| |y|^{2 - d - j} \\[0.5em]
      &= C r^{-l} |\lambda| \, \bigg|x - r \frac{x}{|x|} \bigg|^{4 - d} \! + C \sum_{j = 0}^{l - 1} r^{j - l + 2} |\lambda|\, \bigg|x - r \frac{x}{|x|} \bigg|^{2 - d - j}\! ,
      \intertext{for all $0 < r < |x|$, where we used \eqref{eq:HelmholtzLaplaceDifferenceEstimate} with $l = 0$ for the first summand and \eqref{eq:estimateHelmholtzDerivatives} to estimate the second summand.
    We choose $r = \frac{|x|}{2}$ and receive}
      \big|\nabla^l w(x)\big| 
      &\leq C\, |\lambda |x|^{-l} |x|^{4 - d} + C \sum_{j = 0}^{l - 1} |x|^{j - l + 2} |\lambda| |x|^{2 - d - j} \\
      &\leq C\, |\lambda| |x|^{4 - d - l}.
    \end{align*}

    The proof for case (2) is completely analogous if one sets 
      \begin{align*}
        w(x) = \nabla_x \Big\{ G(x; \lambda) - G(x; 0) \Big\}\quad\text{and}\quad f(x) = \lambda \, \nabla_x G(x; \lambda).
      \end{align*}
    Also case (3) is proven in a similar fashion. 
%      We will give the proof for the sake of completeness.
%
%    For $w$ and $f$ as in case (2) by \eqref{eq:HelmholtzLaplaceDifferenceEstimate} we get
%    \begin{align*}
%      |\nabla^l w(x) |
%      &\leq C r^{-l} \sup_{\BB(x,r)} |w| + C \max_{0 \leq j \leq l - 1} \sup_{\BB(x,r)} r^{j - l + 2} |\nabla^j f| \\
%      &\leq C r^{-l} \sup_{y \in \BB(x,r)} |\lambda| |y|(|\log |\lambda| |y|^2 | + 1 ) + C \sum_{j = 0}^{l - 1} \sup_{y \in \BB(x,r)} r^{j - l + 2} |\lambda| |y|^{- j - 1} \\
%      &\leq S_1 + S_2,
%      \intertext{wheras}
%      S_1 
%      &\leq C r^{-l} |\lambda| |x + r \frac{x}{|x|} | ( |\log |\lambda| |x - r \frac{x}{|x|}|^2 | + |\log( |\lambda||x + r \frac{x}{|x|}|^2 )|+ 1) \\
%      &\leq C  |\lambda| |x|^{1-l}(|\log( |\lambda| |x|^2)| + 1)
%      \intertext{
%        if we choose $r = \frac{|x|}{2}$. For $S_2$ we calculate as before, using estimate \eqref{eq:estimateHelmholtzDerivatives}
%  }
%      S_2
%      &\leq C \sum_{j = 0}^{l - 1} C |x|^{j - l + 2} |\lambda| |x|^{-j - 1} \\
%      &\leq C |\lambda| |x|^{1 - l}.
%    \end{align*}

  \item In this step we prove \eqref{eq:HelmholtzLaplaceDifferenceEstimate} for $d \geq 5$ and $l = 0$. 
    First, note that for the functions
    \begin{align*}
      &g(x) \coloneqq (k|x|)^{\frac{d}{2} - 1} H_{\frac{d}{2} - 1}^{(1)}(k|x|), \quad g(0) = a_d, \\
      &h(z) \coloneqq z^{\frac{d}{2} - 1} H_{\frac{d}{2} - 1}^{(1)}(z), \quad h(0) = a_d,
    \end{align*}
    the mean value theorem yields the estimate
    \begin{align*}
      |g(x) - g(0)| \leq |x| \sup_{y \in \BB(0,|x|)} |\nabla g(y)| \leq |x| |k| \sup_{y \in \BB(0,|x|)}  \Big| \, \frac{\d{} }{\d z} h (k|y|)\,\Big|.
    \end{align*}
    Using representation~\eqref{eq:HelmholtzLaplaceDifference}, we estimate
    \begin{align}
      |G(x; \lambda) - G(x; 0) |
      &\leq C |x|^{2 - d} \cdot |k| |x| \max_{\substack{|z| \leq |k| |x| \\ \Im(z) > 0}} \Big|\, \frac{\d{}}{\d z} \Big\{ z^{\frac{d}{2} - 1} H_{\frac{d}{2} - 1}(z) \Big\} \,\Big| \nonumber\\[0.5em]
      &= C |x|^{2 - d} \cdot |k| |x| \max_{\substack{|z| \leq |k||x| \\ \Im(z) > 0}} \Big|\,z^{\frac{d}{2} - 1} H_{\frac{d}{2} - 2}^{(1)}(z) \,\Big| \, , \label{eq:HelmholtzLaplaceDifferenceGt5}
    \end{align}
    where for the last equality we used another useful relation that can be found in the book of Lebedev \cite[Eq.\@~(5.6.3)]{lebedev},
    \begin{align}
      \frac{\d{}}{\d z} \Big\{ z^\nu H_\nu^{(1)}(z) \Big\} = z^\nu H_{\nu - 1}^{(1)}(z).
    \end{align}
    Since the asymptotic expansions yield that $|z^{\nu} H_\nu^{(1)}(z)| \leq C_\nu$ for $\nu > 0$ and $|z| \leq 1$ with $\Im(z) > 0$, it follows from \eqref{eq:HelmholtzLaplaceDifferenceGt5} that
    \begin{align*}
      |G(x;\lambda) - G(x; 0)| \leq C |x|^{2 - d} \cdot |k| |x| \cdot |k| |x| \max_{\substack{|z| \leq |k||x| \\ \Im(z) > 0}} \Big|\,z^{\frac{d}{2} - 2} H_{\frac{d}{2} - 2}^{(1)}(z) \, \Big|
      \leq C \, |\lambda| |x|^{4 - d}.
    \end{align*}
    
  \item Now we consider the case $d = 4$ and $l = 1$.
    The asymptotic expansion~\eqref{eq:asymptoticd4} gives that
    \begin{align}
      \label{eq:mwt4d}
      \bigg|\, \frac{\d{}}{\d z} \bigg\{ \frac{z H_1^{(1)}(z) - a_4}{z^2} \bigg\} \, \bigg| \leq C\, |z|^{-1}
    \end{align}
    for all $|z| \leq \frac{1}{2}$ with $\Im(z) > 0$.
    Since by identity~\eqref{eq:HelmholtzLaplaceDifference} we have that
    \begin{align*}
      \frac{G(x; \lambda) - G(x; 0)}{\lambda} = - \frac{C \, \big(z H_1^{(1)}(z) - a_4 \big)}{z^2},
    \end{align*}
    where $z = k |x|$. With \eqref{eq:mwt4d} we conclude that
    \begin{align*}
      \bigg|\,\frac{\nabla_x \big\{ G(x; \lambda) - G(x; 0) \big\}}{\lambda} \,\bigg|
      \leq C\, |k| \; \bigg|\, \frac{\d{}}{\d z} \bigg\{ \frac{z H_1^{(1)}(z) - a_4}{z^2} \bigg\} \bigg|_{z = k|x|} \bigg|
      \leq C\, |k| |k|^{-1} |x|^{-1},
    \end{align*}
    which after rearrangement of the involved terms gives the claim.

  \item Next, we consider thee case $d = 3$ and $l = 1$. 
    We start with the compact definition of $G(x; \lambda)$, see \eqref{eq:3dDefinitionFundamentalHelmholtz}, which makes this dimension stand out.  
    From equation~\eqref{eq:defncd} and a well known fact of the Gamma function, $\Gamma(1/2) = \sqrt{\pi}$, we then derive the following identity:
    \begin{align*}
      G(x;\lambda) - G(x; 0) = \frac{\e^{\ii k |x|}}{4\pi |x|} - \frac{c_3}{|x|} = \frac{\e^{\ii k|x|} - 1}{4 \pi |x|}.
    \end{align*}
    Now we calculate
    \begin{align*}
      \frac{\partial}{\partial x_j} \bigg\{ \frac{\e^{\ii k|x|} - 1}{|x|} \bigg\} 
      &= \frac{\partial}{\partial x_j} \bigg\{ \frac{\e^{\ii k |x|} - 1 - \ii k |x|}{|x|} \bigg\} 
      = \frac{\partial}{\partial x_j} \bigg\{ \sum_{n = 2}^\infty \frac{(\ii k |x|)^n}{n!} \cdot \frac{1}{|x|} \bigg\} \\
      &= \sum_{n = 2}^\infty \frac{(\ii k)^n}{n!} (n - 1) \cdot \frac{x_j}{|x|} |x|^{n - 2} 
    \end{align*}
    which in turn implies 
    \begin{align*}
      \bigg| \,\frac{\partial}{\partial x_j} \bigg\{ \frac{\e^{\ii k |x|} - 1}{|x|} \bigg\} \,\bigg|
      &\leq |\lambda| \sum_{n = 2}^\infty \frac{n - 1}{n!} |k|^{n - 2} |x|^{n - 2} 
      \leq C \, |\lambda|
    \end{align*}
    since $|\lambda||x| \leq (1/2)$.

  \item For the last case $d = 2$ and $l = 3$, we will directly calculate the estimate using the asymptotic expansion of $H_0^{(1)}(z)$ with $z = k|x|$. The calculations are omitted from this chapter. Instead, they can be found in the appendix of this thesis, see \hyperref[sec:A1]{A.1}. \qedhere
%    \begin{align*}
%      H_0^{(1)}(z) 
%      &= J_0(z) + \ii Y_0(z) \\
%      &= \sum_{l = 0}^\infty \frac{(-1)^l}{(l!)^2 4^l} z^{2l} \big( 1 - \frac{2\ii \log(2)}{\pi} \big) 
%      - \frac{2 \ii}{\pi} \sum_{l = 0}^\infty \frac{(-1)^l}{(l!)^2 4^l} \psi(l + 1) \cdot z^{2l} \\
%      &\quad + \frac{2\ii}{\pi} \sum_{l = 0}^\infty \frac{(-1)^l}{(l!)^2 4^l} z^{2l} \log(z)
%    \end{align*}
%    The first complex derivative of $H_0^{(1)}(z)$ reads
%    \begin{align*}
%      \frac{\d{}}{\d z} H_0^{(1)}(z) = 
%    \end{align*}
%
  \end{enumerate} 
\end{proof}

\begin{rem}
  \label{rem:HelmholtzLaplaceDifference}
  In the situation of Lemma \ref{lem:HelmholtzLaplaceDifference}, one can show for $|\lambda| |x|^2 \leq (1/2)$ by considering the asymptotic expansions that 
  \begin{align*}
    |G(x; \lambda) - G(x; 0) | \leq \begin{cases}
      C \sqrt{|\lambda|} \quad&\text{if } d = 3, \\
      C\, |\lambda| \,\Big\{\, \big|\log(|\lambda| |x|^2) \,\big| + 1 \,\Big\} \quad&\text{if } d = 4.
    \end{cases}
  \end{align*}
  Also using the asymptotic expansions, it can be shown that if $d = 2$, then
  \begin{align*}
    \big|\nabla_x^l \{ G(x; \lambda) - G(x; 0) \} \big| \leq C\, |\lambda| |x|^{2 - l} \Big\{\, \big|\log(|\lambda| |x|^2 ) \,\big| + 1\, \Big\},
  \end{align*}
  for $l \in \{1, 2\}$.  
\end{rem}

\newpage
\section{The Stokes Resolvent Problem}
\label{sec:2.2}

We will now analyze fundamental solutions to the \emph{Stokes resolvent problem}
\begin{align}
  \label{eq:stokesResolventProblem}
  \begin{alignedat}{1}
  -\Delta u + \nabla \phi + \lambda u &= 0 \\
  \div u &= 0
  \end{alignedat}
\end{align}
in $\R^d$ with $\lambda \in \Sigma_\theta$ with the goal to deduce helpful estimates for the following chapters.
  The fundamental solutions to the (scalar) Helmholtz equation and the Laplace equation will form the main ingredients for the following matrix of fundamental solutions to the Stokes resolvent problem with pole at the origin:
  \begin{align}
    \label{eq:fundamentalMatrixStokes}
    \Gamma_{\alpha\beta}(x;\lambda) = G(x; \lambda) \delta_{\alpha\beta} - \frac{1}{\lambda}\, \frac{\partial^2}{\partial x_\alpha \partial x_\beta} \Big\{ G(x; \lambda) - G(x; 0) \Big\}, \quad \alpha,\beta = 1,\dots,d.
  \end{align}
  As the matrix of fundamental solutions $\Gamma(x; \lambda) = (\Gamma_{\alpha\beta}(x; \lambda))_{d \times d}$ carries two arguments it cannot be confused with the Gamma function.
  Having formula~\eqref{eq:fundamentalMatrixStokes} at sight, the following observations are obvious:
  \begin{align*}
    \Gamma_{\alpha\beta}(x; \lambda) = \Gamma_{\beta\alpha}(x; \lambda), \quad 
    \overline{\Gamma_{\alpha\beta}(x; \lambda)} = \Gamma_{\alpha\beta}(x; \bar\lambda)
    \quad\text{and}\quad
    \Gamma_{\alpha\beta}(x; \lambda) = \Gamma_{\alpha\beta}(-x; \lambda).
  \end{align*}
  For the pressure, we define the vector of fundamental solutions
  \begin{align}
    \label{eq:fundamentalVectorPressure}
      \Phi_\beta(x) = -\frac{\partial}{\partial x_\beta} \Big\{ G(x; 0) \Big\} = \frac{x_\beta}{\omega_d |x|^d}, \quad \beta= 1,\dots,d.
  \end{align}
  We note that $\Phi_\beta(x) = -\Phi_\beta(-x)$.

  Using the fact that $\Delta_x G(x; \lambda) = \lambda G(x; \lambda)$ in $\R^d \setminus \{0\}$, one can see that on $\R^d \setminus \{0\}$ and for all $1 \leq \beta \leq d$
  \begin{align}
    \label{eq:solutionStokesSystem}
    \begin{alignedat}{1}
      (-\Delta_x + \lambda)\, \Gamma_{\alpha\beta}(x;\lambda) + \frac{\partial}{\partial x_\alpha} \Big\{ \Phi_\beta(x) \Big\} &= 0\,, \\
      \frac{\partial}{\partial x_\alpha} \Big\{ \Gamma_{\alpha\beta} (x; \lambda) \Big\} &= 0\,, \quad\text{for } 1 \leq \alpha \leq d.
    \end{alignedat}
  \end{align}
  Note that in the last equation the summation convention was used.

  We now keep up to the spirit of this exhausting chapter by proving further estimates, this time for the fundamental solutions to the Stokes resolvent problem \eqref{eq:stokesResolventProblem}.

\begin{thm}
  \label{thm:fundamentalMatrixEstimate}
  Let $\lambda \in \Sigma_\theta$.
  Then, for any $d \geq 3$ and $l \geq 0$,
  \begin{align}
    \label{eq:fundamentalMatrixEstimate}
    \big| \nabla_x^l \Gamma(x; \lambda) \big| &\leq \frac{C}{(1 + |\lambda||x|^2) |x|^{d - 2 + l}} 
  \end{align}
    where $C > 0$ depends only on $d$, $l$ and $\theta$. For $d = 2$ and $l \geq 1$, the same estimate holds.
\end{thm}

  \begin{proof}
    Let $|\lambda| |x|^2 > (1/2)$. 
    Then, there exist constants $C_a$, $C_b$, $C_c > 0$ such that
    \begin{align*}
      \e^{-c \sqrt{|\lambda|} |x|} ( 1 + |\lambda| |x|^2) &\leq C_a, \\[0.5em]
      1 &\leq \frac{C_b |\lambda| |x|^2}{1 + |\lambda| |x|^2}, \\[0.5em]
      \e^{-c \sqrt{|\lambda|} |x|} &\leq \frac{ C_c |\lambda| |x|^2}{1 + |\lambda| |x|^2},
    \end{align*}
    where $c > 0$ is the constant from Lemma \ref{lem:estimateHelmholtzDerivatives}.
    Using these estimates and Lemma \ref{lem:estimateHelmholtzDerivatives} gives
    \begin{align*}
      \big|\nabla_x^l \Gamma(x; \lambda) \big|
      &\leq \big|\nabla_x^l G(x; \lambda)\big| + \frac{1}{|\lambda|}\cdot \big|\nabla_x^{l + 2} G(x; \lambda) \big| + \frac{1}{|\lambda|} \cdot \big|\nabla_x^{l + 2} G(x; 0)\big| \\[0.5em]
      &\leq\frac{C_l\, \e^{-c \sqrt{|\lambda|} |x|}}{|x|^{d - 2 + l}} + \frac{1}{|\lambda|} \cdot \frac{C_{l + 2}\, \e^{-c \sqrt{|\lambda|} |x|}}{|x|^2 |x|^{d - 2 + l}} + \frac{1}{|\lambda|} \cdot \frac{C}{|x|^2 |x|^{d - 2 + l}} \\[0.5em]
    %&\leq \frac{C}{(1 + |\lambda||x|^2)} \frac{1}{|x|^{d - 2 + l}} + \frac{C_{l + 2} |\lambda||x|^2}{(1 + |\lambda||x|^2)} \frac{1}{|\lambda| |x|^2} \frac{1}{|x|^{d - 2 + l}} + \frac{C}{1 + |\lambda||x|^2} \frac{1}{|x|^{d - 2 + l}}
    &\leq \frac{C}{1 + |\lambda| |x|^2} \cdot \frac{1}{|x|^{d - 2 + l}}.
    \end{align*}
    Now let $|\lambda| |x|^2 \leq ({1}/{2})$.
    Then, by Lemma \ref{lem:estimateHelmholtzDerivatives} and Lemma \ref{lem:HelmholtzLaplaceDifference} we get
    \begin{align*}
      \big|\nabla_x^l \Gamma(x; \lambda) \big|
      &\leq \big|\nabla_x^l G(x; \lambda) \big| 
      + \frac{1}{|\lambda|} \cdot \Big|\,\nabla_x^{l + 2} \Big\{ G(x; \lambda) - G(x; 0) \Big\}\,\Big| \\
      &\leq \frac{C}{|x|^{d - 2 + l}} + \frac{1}{|\lambda|} \cdot C\, |\lambda| |x|^{4 - d - (l + 2)} \\
      &\leq \frac{C}{|x|^{d - 2 + l}}\cdot  \frac{(1 + |\lambda| |x|^2)}{(1 + |\lambda| |x|^2)} \\
      &\leq \frac{C}{(1 + |\lambda| |x|^2) |x|^{d - 2 + l}}
    \end{align*}
    which gives the claim.
    %If $d = 2$ the steps are analogous considering the different structure of the estimate \eqref{eq:HelmholtzLaplaceDifferenceEstimate2d}.
  \end{proof}

%  \begin{rem}
%    It is important to note that using the information from Remark \ref{rem:HelmholtzLaplaceDifference} regarding the case $d = 2$, the above proof would also work using only minor modifications starting from $l = 1$.
%    As a consequence one would derive estimates of the form
%    \begin{align*}
%      \big|\nabla_x^l \Gamma(x; \lambda)| \leq \frac{C}{(1 + |\lambda| |x|^2) |x|^l} (1 + \big|\log(|\lambda||x|^2)\, \big|), \quad l \geq 1 ,
%    \end{align*}
%    which offer a different behavior as estimates of the form \ref{eq:fundamentalMatrixEstimate}.
%    It will be important in later chapters to have estimates on $\nabla^l_x \Gamma(x; \lambda)$ that do not depend on $\lambda$.
%    Affortunately this type of estimates will only be needed for orders of differentiation $l \geq 1$.
%  \end{rem}

 %Regarding the structure of the above proof, the derived estimate for the case $d = 2$ seems natural at first sight. 
 %One important difference compared to $d \geq 3$ is that it is not possible to derive an estimate of the form $|\nabla_x^l \Gamma(x; \lambda)| \leq C |x|^{-l}$.

  If $\lambda = 0$, the Stokes resolvent problem becomes just the Stokes problem in $\R^d$
\begin{align}
  \label{eq:stokesProblem}
  \begin{alignedat}{1}
  -\Delta u + \nabla \phi &= 0\,, \\
  \div u &= 0\,.
  \end{alignedat}
\end{align}
  Whereas the fundamental solution for the pressure is maintained, the matrix of fundamental solutions to the Stokes problem in $\R^d$ with pole at the origin is given by $\Gamma(x; 0) = (\Gamma_{\alpha\beta}(x; 0))_{d \times d}$, where
  \begin{align}
    \label{eq:fundamentalSolutionStokes}
    \Gamma_{\alpha\beta}(x; 0) &\coloneqq \frac{1}{2 \omega_d} \bigg\{ \,\frac{\delta_{\alpha\beta}}{(d - 2)\, |x|^{d - 2}} + \frac{x_{\alpha} x_\beta}{|x|^d} \,\bigg\}
    \intertext{if $d \geq 3$ and}
    \label{eq:fundamentalSolutionStokes2d}
    \Gamma_{\alpha\beta}(x; 0) &\coloneqq \frac{1}{2 \omega_2} \bigg\{ \,- \delta_{\alpha\beta} \log(|x|) + \frac{x_\alpha x_\beta}{|x|^2} \,\bigg\}  \end{align}
  for $d = 2$.
  Note that the given fundamental solution for the case $d = 2$ differs from the one given by Mitrea and Wright \cite[Sec.\@~4.2]{mitreaWright} by having summands with alternating signs.
  Considering the structure of the fundamental solution for $d \geq 3$, our choice seems more natural with regard to the structure of the fundamental solutions to the Laplace equation~\eqref{eq:laplace}.
  The alternating sign is necessary for $\Gamma_{\alpha\beta}$ to be divergence free.
  The ordering of the signs is also crucial as we will see in later calculations.

  One important technique in the following chapter will be to reduce problems formulated for $\Gamma(x; \lambda)$ to problems formulated in $\Gamma(x; 0)$ perturbed by the difference $\Gamma(x; \lambda) - \Gamma(x; 0)$.
  Under this aspect it seems reasonable to study estimates of the difference of fundamental solutions.
  To this end, it is helpful to rewrite parts of the fundamental solution.
  Using the fact that for $d \geq 5$ or $d = 3$, we have
  \begin{align*}
    \frac{\partial^2}{\partial x_\alpha \partial x_\beta} \bigg\{ \frac{1}{|x|^{d - 4}} \bigg\}
    = -(d - 4)\, \frac{\partial}{\partial x_\alpha} \bigg\{ \frac{x_\beta}{|x|^{d - 2}} \bigg\}
    = -(d - 4)\, \frac{\delta_{\alpha\beta}}{|x|^{d - 2}} + \frac{(d - 4)\,(d - 2)\, x_\alpha x_\beta}{|x|^d}.
  \end{align*}
  This allows us to write
  \begin{align*}
    \frac{x_\alpha x_\beta}{|x|^d} = \frac{\delta_{\alpha \beta}}{(d - 2)\,|x|^{d - 2}} + \frac{1}{(d - 4)\,(d - 2)}\, \frac{\partial^2}{\partial x_\alpha \partial x_\beta} \bigg\{ \frac{1}{|x|^{d - 4}} \bigg\},
  \end{align*}
  which, considering definition~\eqref{eq:fundamentalSolutionStokes}, gives 
  \begin{align}
    \label{eq:fundamentalSolutionStokes35}
    \Gamma_{\alpha\beta}(x; 0) &= G(x; 0)\delta_{\alpha\beta} + \frac{1}{2\omega_d\,(d - 4)\,(d - 2)} \,\frac{\partial^2}{\partial x_\alpha \partial x_\beta} \bigg\{ \frac{1}{|x|^{d - 4}} \bigg\}.
  \intertext{A similar trick works for $d = 4$:
  Since $\omega_4 = 2 \pi^2$, we have}
    \Gamma_{\alpha \beta}(x; 0)
    &= \frac{1}{2 \omega_4} \,\frac{1}{|x|^2} \delta_{\alpha\beta} - \frac{1}{8\pi^2}\, \bigg(\, \frac{\delta_{\alpha\beta}}{|x|^2} - \frac{2 x_\alpha x_\beta}{|x|^4} \, \bigg) \nonumber\\[0.5em]
    &= G(x; 0) \delta_{\alpha\beta} - \frac{1}{8\pi^2} \, \frac{\partial^2}{\partial x_\alpha \partial x_\beta}\Big\{ \log(|x|) \Big\} \nonumber \\[0.5em]
    \label{eq:fundamentalSolutionStokes4}
    &= G(x; 0) \delta_{\alpha\beta} - \frac{1}{4 \omega_4} \, \frac{\partial^2}{\partial x_\alpha \partial x_\beta} \Big\{ \log(|x|) \Big\}.
  \intertext{In the case $d = 2$, we use
    $$
    \frac{1}{8\pi}\, \frac{\partial^2}{\partial x_\alpha \partial x_\beta} \Big\{ |x|^2 \log(|x|) \Big\}
  = \frac{\delta_{\alpha\beta}}{4\pi}\, \log(|x|) 
    + \frac{1}{4\pi}\, \frac{x_\alpha x_\beta}{|x|^2} 
    + \frac{\delta_{\alpha\beta}}{8\pi}
$$
to find the identity}
    \label{eq:fundamentalSolutionStokes2}
    \Gamma_{\alpha \beta}(x; 0) &= G(x; 0)\delta_{\alpha\beta} - \frac{\delta_{\alpha\beta}}{8\pi} - \frac{\partial^2}{\partial x_\alpha \partial x_\beta} \Big\{ |x|^2 \log(|x|)\Big\} .
  \end{align}
  This ends the preparatory step and brings us to the next theorem.
  
%\begin{thm}
%  \label{thm:differenceFundamentalSolutionStokes}
%  Let $\lambda \in \Sigma_\theta$.
%  Suppose that $|\lambda| |x|^2 \leq ({1}/{2})$.
%  Then
%  \begin{align}
%    \big|\nabla_x \{ \Gamma(x; \lambda) - \Gamma(x; 0) \} \big|
%    \leq 
%    \begin{cases}
%      C |\lambda| |x|^{3 - d} &\quad\text{if } d \geq 7 \text{ or } d = 5, \\
%      C |\lambda| |x|^{3 - d} \, \big| \log(|\lambda| |x|^2)\, \big| &\quad\text{if } d = 4 \text{ or } 6, \\
%      C \sqrt{|\lambda|} |x|^{-1} &\quad\text{if } d = 3,\\
%      %C |\lambda| |x| ( |\log(|\lambda| |x|^2)| + 1 ) &\quad\text{if } d = 2,
%      C |\lambda| |x| \, \big|\log(|\lambda| |x|^2)\,\big|  &\quad\text{if } d = 2,
%    \end{cases}
%  \end{align}
%  where $C$ depends only on $d$ and $\theta$.
%\end{thm}
\begin{thm}
  \label{thm:differenceFundamentalSolutionStokes}
  Let $\lambda \in \Sigma_\theta$.
  Suppose that $|\lambda| |x|^2 \leq ({1}/{2})$.
  Then,
  \begin{align}
    \!\Big|\,\nabla_x \Big\{ \Gamma(x; \lambda) - \Gamma(x; 0) \Big\}\, \Big|
    \leq 
    \begin{cases}
      C\,|\lambda| |x|^{3 - d} &\quad\text{if } d = 3\text{, } 5 \text{ or } d \geq 7\, , \\
      C\,|\lambda| |x|^{3 - d} \, \big| \log(|\lambda| |x|^2)\, \big| &\quad\text{if } d = 2\text{, } 4 \text{ or } 6\,, \\
    \end{cases}
  \end{align}
  where $C > 0$ depends only on $d$ and $\theta$.
\end{thm}

\begin{proof}
  We will split the proof in several parts.
  According to the preparatory step, for $d \geq 2$ and all $\alpha, \beta = 1,\dots,d$, the difference $\partial_\gamma\big\{\Gamma_{\alpha\beta}(x; \lambda) - \Gamma_{\alpha\beta}(x; 0)\big\}, \gamma = 1,\dots,d$, is always of the form
  \begin{align*}
    &\frac{\partial}{\partial x_\gamma}\Big\{\Gamma_{\alpha\beta}(x; \lambda) - \Gamma_{\alpha\beta}(x; 0)\Big\} \\
    &    \qquad\quad= \frac{\partial}{\partial x_\gamma}\Big\{ G(x; \lambda) - G(x; 0) \Big\}\delta_{\alpha\beta} \
        - \frac{1}{\lambda}\, \frac{\partial^3}{\partial x_\gamma \partial x_\alpha \partial x_\beta} \Big\{ G(x; \lambda) - G(x; 0) + \big[ \dots \big] \Big\}.
  \end{align*}
  But the first term on the right-hand side of the above expression is already under control thanks to Lemma~\ref{lem:HelmholtzLaplaceDifference} and Remark~\ref{rem:HelmholtzLaplaceDifference}.
  It thus suffices to estimate the second term on the right-hand side.

  We start by considering the cases $d = 3$ and $d \geq 5$.
      Taking into account identity~\eqref{eq:fundamentalSolutionStokes35}, we have for all $\alpha, \beta, \gamma = 1,\dots,d$:
      \begin{align*}
        %&\Gamma_{\alpha\beta}(x; \lambda) - \Gamma_{\alpha\beta}(x; 0) \\
        %&\frac{1}{\lambda} \frac{\partial^3}{\partial x_\gamma \partial x_\alpha \partial x_\beta} \bigg\{ G(x; \lambda) - G(x; 0) + \big[ \dots \big] \bigg\}\\
        & G(x; \lambda) - G(x; 0) + \big[ \dots \big]
        %= \frac{\partial}{\partial x_\gamma} \big\{ G(x; \lambda) - G(x; 0) \big\}\delta_{\alpha\beta}  \\
        %&\qquad\qquad\qquad = \frac{1}{\lambda} \frac{\partial^3}{\partial x_\gamma \partial x_\alpha \partial x_\beta} \bigg\{ G(x; \lambda) - G(x; 0) + \frac{\lambda}{2 \omega_d (d - 4) (d - 2) |x|^{d - 4}} \bigg\}.
        = \frac{1}{\lambda}\, \bigg\{ G(x; \lambda) - G(x; 0) + \frac{\lambda}{2\, \omega_d \, (d - 4)\, (d - 2)\, |x|^{d - 4}} \bigg\}.
      \end{align*}
  %As the first term can already be estimated via Lemma \ref{lem:HelmholtzLaplaceDifference} in a satisfactory way, we will only be concerned about the second one.
  If $d = 3$, a direct calculation will then yield the desired result:
  We start by noting that $\omega_3 = 4 \pi$ gives
  \begin{align*}
    G(x; \lambda) - G(x; 0) - \frac{\lambda}{2\omega_3 {|x|}^{-1}}
    &= \frac{\e^{\ii k |x|}}{4\pi |x|} - \frac{1}{4 \pi |x|} - \frac{(\ii k)^2}{2 \omega_3 |x|^{-1}} \\
    &= \frac{1}{4 \pi |x|} \, \Big( \e^{\ii k |x|} - 1 - \frac{ (\ii k)^2 |x|^2}{2} \Big)\\
    &= \frac{1}{4 \pi |x|} \, \Big( \ii k |x| + \sum_{n = 3}^\infty \frac{(\ii k |x| )^n}{n!} \Big) \\
    &= \frac{1}{4\pi} \, \Big( \ii k + \sum_{n = 3}^\infty \frac{ (\ii k )^n |x|^{n - 1}}{n!} \Big) \eqqcolon I.
  \end{align*}
  Taking the first derivative of this expression we get
  \begin{align*}
    \frac{\partial}{\partial x_\beta} \big\{ \, I \, \big\} = 
    \frac{x_\beta}{4 \pi} \sum_{n = 3}^\infty \frac{(ik)^n (n - 1) }{n!} |x|^{n - 3}
  \end{align*}
  and differentiating with respect to $x_\alpha$ yields
  \begin{align*}
    \frac{\partial^2}{\partial x_\alpha \partial x_\beta}  \big\{ \, I \, \big\}
    = \frac{\delta_{\alpha\beta}}{4 \pi} \sum_{n = 3}^\infty \frac{(ik)^n (n - 1) }{n!} |x|^{n - 3}
    + \frac{x_\beta x_\alpha}{4\pi} \sum_{n = 4}^\infty \frac{(\ii k)^n (n - 1) (n - 3)}{n!} |x|^{n - 5}.
  \end{align*}
  As we are interested in estimating the \emph{gradient} of the difference of $\Gamma(x; \lambda)$ and $\Gamma(x; 0)$, we have to consider one additional derivative. This leaves us with
\begin{align*}
  \frac{\partial^3}{\partial x_\gamma \partial x_\alpha \partial x_\beta} \big\{ \, I\, \big\}
    &= \frac{\delta_{\alpha\beta} x_\gamma + \delta_{\beta\gamma} x_\alpha + \delta_{\alpha\gamma} x_\beta}{4\pi} \sum_{n = 4}^\infty \frac{(\ii k)^n (n - 1) (n - 3)}{n!}  |x|^{n - 5} \\
    &\qquad + \frac{x_\beta x_\alpha x_\gamma}{4\pi} \sum_{\substack{n = 4}}^\infty \frac{(\ii k)^n (n - 1) (n - 3) (n - 5)}{n!} |x|^{n - 7}.
\end{align*}
The desired estimate follows now via
\begin{align*}
  \bigg|\, \frac{1}{\lambda}\, \frac{\partial^3}{\partial x_\gamma \partial x_\alpha \partial x_\beta} \big\{ \, I\, \big\}  \, \bigg|
  &\leq \frac{1}{|k|^2\pi}  \sum_{n = 4}^\infty \frac{|k|^n (n - 1) (n - 3)(1 + |n - 5|)}{n!} |x|^{n - 4} \\
  %&\leq \frac{1}{|k|^2 |x|\pi} |k|^3 \sum_{n = 4}^\infty \frac{(n - 1)(n - 3)(1 + (n - 5))}{n!} |k|^{n - 3} |x|^{n - 3}\\
  &\leq \frac{1}{|k|^2 \pi} |k|^4 \sum_{n = 4}^\infty \frac{(n - 1)(n - 3)(1 + |n - 5|)}{n!} |k|^{n - 4} |x|^{n - 4}\\
  &\leq C \, |k|^2.
\end{align*}
This gives the claim for $d = 3$.
If $d \geq 5$, equation~\eqref{eq:HelmholtzLaplaceDifference} gives
\begin{align}
  \label{eq:secondTerm}
  \begin{alignedat}{1}
    &G(x; \lambda)- G(x; 0) + \frac{\lambda}{2 \omega_d (d - 4) (d - 2) |x|^{d - 4}} \\[0.5em]
  &\hspace{4cm}= \frac{\ii}{4 (2\pi)^{\frac{d}{2} - 1}} \, \frac{1}{|x|^{d - 2}} \Big\{ z^{\frac{d}{2} - 1} H_{\frac{d}{2} - 1}^{(1)}(z) - a_d - b_d z^2\Big\},
  \end{alignedat}
\end{align}
where $z = k|x|$, $a_d$ was calculated in~\eqref{eq:Defnad} and $b_d$ is given by
\begin{align*}
 b_d 
 &= -\frac{2\ii\, (2\pi)^{\frac{d}{2} - 1}}{\omega_d (d - 4) (d - 2)}.
  \end{align*}
    Using relation~\eqref{eq:wd} and the functional equation of the Gamma function~\eqref{eq:functionalGamma} twice, we see that
    \begin{align*}
      b_d &= -\frac{2\ii\, (2\pi)^{\frac{d}{2} - 1}\, \Gamma(\frac{d}{2})}{2\pi^{\frac{d}{2}} (d - 2)(d - 4)}
 = \frac{2^{\frac{d}{2} - 1} }{\pi \ii (d - 4)}\, \frac{\Gamma(\frac{d}{2})}{(d - 2)} 
      = \frac{2^{\frac{d}{2} - 1} }{2\pi \ii } \, \frac{\Gamma(\frac{d}{2} - 1)}{(d - 4)} \\[0.5em]
      &= \frac{2^{\frac{d}{2} - 1}}{4 \pi \ii } \, \frac{\Gamma(\frac{d}{2} - 1)}{(\frac{d}{2} - 1 - 1)}
      = \frac{2^{\frac{d}{2} - 1} \, \Gamma(\frac{d}{2} - 1 - 1)}{4 \pi \ii } = \frac{2^{\nu_d}\, \Gamma(\nu_d - 1)}{4 \pi \ii}.
\end{align*}
This shows that for $d \geq 5$, $b_d$ is the second coefficient of the asymptotic expansions~\eqref{eq:asymptoticd5}-\eqref{eq:asymptoticd7}, respectively. 
Now we split the proof for $d \geq 5$ into (1) $d \geq 7$, (2) $d = 6$ and (3) $d = 5$.
If $d \geq 7$, we use the asymptotic expansion~\eqref{eq:asymptoticd7} to estimate the part of~\eqref{eq:secondTerm} which involves the Hankel function as
\begin{align}
  \label{eq:estimateDerivativesd7}
  \bigg|\, \frac{\mathrm{d}^l}{\d z^l} \Big\{ z^{\frac{d}{2} - 1} H_{\frac{d}{2} - 1}^{(1)}(z) - a_d - b_d z^2 \Big\} \,\bigg| 
  \leq C |z|^{4 - l}
\end{align}
for $0 \leq l \leq 3$ and $z \in \C \setminus (-\infty, 0]$.
For better readability, we define the function
\begin{align*}
  g(z) \coloneqq z^{\frac{d}{2} - 1} H_{\frac{d}{2} - 1}^{(1)}(z) - a_d - b_d z^2
\end{align*}
and consider the function $f(x) \coloneqq g(k |x|)$ on $\R^d \setminus \{0\}$.
The derivatives of $f$ read
\begin{align*}
  \frac{\partial}{\partial x_\beta} f(x)
  &= \Big(\frac{\d{}}{\d z} g\Big)(k|x|) \cdot k\, \frac{x_\beta}{|x|}\,, \\[0.5em]
  %
  \frac{\partial^2}{\partial x_\alpha \partial x_\beta} f(x)
  &= \Big(\frac{\mathrm{d}^2}{\d z^2} g\Big)(k|x|) \cdot k^2\,  \frac{x_\alpha x_\beta}{|x|^2} 
  + \Big(\frac{\d{}}{\d z} g\Big)(k |x|) \cdot k\, \bigg( \frac{\delta_{\alpha\beta}}{|x|} - \frac{x_\beta x_\alpha}{|x|^3} \bigg)\,, \\[0.5em]
  %
  \frac{\partial^3}{\partial x_\gamma \partial x_\alpha \partial x_\beta} f(x)
  &= \Big(\frac{\mathrm{d}^3}{\d z^3}g\Big)(k |x|) \cdot k^3 \,\frac{x_\alpha x_\beta x_\gamma}{|x|^3} \\[0.5em]
  &\quad + \Big(\frac{\mathrm{d}^2}{\d z^2}g\Big)(k|x|) 
  \cdot k^2\, \bigg( \frac{\delta_{\beta\gamma} x_\alpha + \delta_{\alpha\gamma} x_\beta + \delta_{\alpha\beta} x_\gamma }{|x|^2} 
  - \frac{x_\alpha x_\beta x_\gamma}{|x|^4} \bigg) \\[0.5em]
  &\quad + \Big(\frac{\d{}}{\d z}g\Big)(k|x|) \cdot k\, \bigg(-\frac{ \delta_{\alpha\beta}x_\gamma + \delta_{\beta\gamma}x_\alpha  + \delta_{\alpha\gamma}x_\beta }{|x|^3} + \frac{3 x_\alpha x_\beta x_\gamma}{|x|^5} \bigg)\,.
\end{align*}
If we now look for estimates on the absolute value of the derivatives, we see that by \eqref{eq:estimateDerivativesd7}
\begin{align*}
% |\nabla_x f(x) | 
% &\leq C |k|^{4 - 1 + 1} |x|^{4 - 1} = C |k|^{4} |x|^{3}\\
% |\nabla_x^2 f(x) | 
% &\leq C \{ |k|^{4 - 2 + 2} |x|^{4 - 2} + |k|^{4 - 1 + 1} |x|^{4 -1 -1} \}  = C |k|^4 |x|^2 \\
% |\nabla^3_x f(x) | 
% &\leq C \{ |k|^{4 - 3 + 3} |x|^{4 - 3} + |k|^{4 - 2 + 2} |x|^{4 - 2 -1} + |k|^{4 - 1 + 1} |x|^{4 - 1 - 2}
  \big|\nabla^l f(x)\big| \leq C \, |k|^4 |x|^{4 - l}, \quad 1 \leq l \leq 3,
\end{align*}
where $C > 0$ only depends on $l$.
We finally uncover the desired estimate via
\begin{align*}
  &\bigg|\, 
  \frac{1}{\lambda} \, \nabla_x^3
  \Bigg\{ \, \frac{\ii}{4\, (2\pi)^{\frac{d}{2} - 1}}\, \frac{1}{|x|^{d - 2}} 
  \Big\{ \,z^{\frac{d}{2} - 1} H_{\frac{d}{2} - 1}^{(1)}(z) - a_d - b_d z^2 \,
  \Big\} 
  \Bigg\} \,
  \bigg| \\[0.5em]
  &\qquad\leq C\, 
  \frac{1}{|k|^2} \sum_{l = 0}^3 \bigg| \, \nabla^{3 - l} \bigg\{\, \frac{1}{|x|^{d - 2}}\, \bigg\} \, \bigg|
  \; \big| \nabla^l f(x) \big| \\[0.5em]
  &\qquad\leq C \sum_{l = 0}^3 |x|^{-d + 2 - 3 + l} |k|^2 |x|^{4 - l} = C\, |\lambda| |x|^{3 - d},
\end{align*}
where $C > 0$ is a constant only depending on $d$.

If $d = 6$, the asymptotic expansion~\eqref{eq:asymptoticd6} gives us in analogy to \eqref{eq:estimateDerivativesd7} the estimate
\begin{align}
  \label{eq:estimateDerivativesd6}
  \bigg| \, \frac{\mathrm{d}^l}{\d z^l} \bigg\{\, z^{\frac{d}{2} - 1} H_{\frac{d}{2} - 1}^{(1)}(z) - a_d - b_d z^2 \, \bigg\}\,  \bigg| \leq C\, |z|^{4 - l} \, \big| \log(z)\, \big|\, ,
\end{align}
for $0 \leq l \leq 3$ as the absolute values of $z$ are bounded by assumption.% and $z \in \C \setminus (-\infty, 0]$.
Using as before the expressions for the derivatives of $f$, we estimate their absolute values as
\begin{align*}
  \big|\nabla^l f(x) \big| \leq C\, |k|^4 |x|^{4 - l}\, \big|\log(|\lambda| |x|^2)\, \big|\, , 
\end{align*}
which, by a calculation analogous to the case $d \geq 7$, yields the claim.
%\begin{align*}
%  \big| \nabla_x \{ \Gamma(x; \lambda) - \Gamma(x; 0) \}\big| \leq C |\lambda| |x|^{3 - d} \, \big| \log(|\lambda| |x|^2)\, \big|\,.
%\end{align*}

For $d = 5$, we differentiate \eqref{eq:secondTerm} twice and use relation~\eqref{eq:fundamentalSolutionStokes35} for the fundamental solution of the Stokes problem to  write
\begin{align*}
  &\frac{\partial^2}{\partial x_\alpha \partial x_\beta} \bigg\{ G(x; \lambda) - G(x; 0) + \frac{\lambda}{6\,\omega_5\, |x|^{d - 4}} \bigg\} \\
  &\hspace{3cm}
  = \frac{\partial^2}{\partial x_\alpha x_\beta} \Bigg\{ \, \frac{\ii}{4\, (2\pi)^{\frac{3}{2}}} \cdot \frac{1}{|x|^3} \bigg\{ \, z^{\frac{3}{2}} H_{\frac{3}{2}}^{(1)}(z) - a_5 - b_5 z^2 - w z^3\, \bigg\}\,  \Bigg\},
\end{align*}
where $w\in\C$ can be an arbitrary constant if we set $z = k|x|$.
Now, for the appropriate choice of $w \in \C$ the asymptotic expansion~\eqref{eq:asymptoticd5} gives the same estimate as \eqref{eq:estimateDerivativesd7} which, like for $d \geq 7$, proves the claim for $d = 5$.

In the case $d = 4$, we use the respective relation for the fundamental solution~\eqref{eq:fundamentalSolutionStokes4} in order to simplify the difference
\begin{align*}
  %&\Gamma_{\alpha\beta}(x; \lambda) - \Gamma_{\alpha\beta}(x; 0) \\
  %&\quad= \big\{ G(x; \lambda) - G(x; 0) \big\}\delta_{\alpha\beta}  
  & \frac{\partial^2}{\partial x_\alpha \partial x_\beta} \bigg\{ G(x; \lambda) - G(x; 0) - \frac{\lambda \log(|x|)}{4\, \omega_4} \bigg\} \\
  &\qquad\qquad\qquad= \frac{\partial^2}{\partial x_\alpha \partial x_\beta} \bigg\{ \,\frac{\ii}{8\, \pi\, |x|^2} \Big\{ \,z H_1^{(1)}(z) - a_4 - w z^2 - b_4 z^2 \log(z)\,\Big\} \, \bigg\},
\end{align*}
where $z = k |x|$, $b_4 = ({\ii}/{\pi})$ and $w \in \C$ is an arbitrary constant. Using the asymptotic expansion~\eqref{eq:asymptoticd4} and the appropriate constant $w \in \C$, we get the estimate
\begin{align*}
  \bigg| \, \frac{\mathrm{d}^l}{\d z^l} \Big\{ \,z H_1^{(1)}(z) - a_4 - w z^2 - b_4 z^2 \log(z)\, \Big\} \, \bigg| \leq C |z|^{4 - l}\, \big|\log(z)\,\big|\,.
\end{align*}
The estimate has the same right-hand side as \eqref{eq:estimateDerivativesd6} and the proof can be carried out just as in the previous cases.

For $d = 2$, the claimed estimate follows from a direct calculation which is postponed until appendix \hyperref[sec:A2]{A.2}.
\end{proof}

We can now use the assumption $|\lambda||x|^2 \leq ({1}/{2})$  to unify the structure of the estimates from Theorem \ref{thm:differenceFundamentalSolutionStokes}.
\begin{cor}
  \label{cor:differenceFundamentalSolutionStokes}
  Let $\lambda \in \Sigma_\theta$. 
  Suppose that $|\lambda||x|^2 \leq ({1}/{2})$.
  Then, for all $d \geq 2$
  \begin{align*}
    \Big|\,\nabla_x\Big\{ \Gamma(x; \lambda) - \Gamma(x; 0)\Big\}\, \Big| \leq C \sqrt{|\lambda|} |x|^{2 - d},
  \end{align*}
  where $C$ depends only on $d$ and $\theta$.
\end{cor}

\begin{proof}
  We just extend the estimates given in Theorem \ref{thm:differenceFundamentalSolutionStokes}.
  Let $d \geq 7$ or $d = 5$. Since $\sqrt{|\lambda|} \leq C{|x|}^{-1}$, we have
  \begin{align*}
    C |\lambda| |x|^{3 - d} \leq C \sqrt{|\lambda|} |x|^{2 - d}.
  \end{align*}
  For $d = 2,4,6$, we have
  \begin{align*}
    |\lambda| |x|^{3 - d} \, \big| \log(|\lambda||x|^2)\,\big|
    = C \sqrt{|\lambda|} |x|^{2 - d} \cdot \sqrt{|\lambda|} |x|\, \big| \log(|\lambda| |x|^2)\, \big|
    \leq C \sqrt{|\lambda|} |x|^{2 - d},
  \end{align*}
  since $\sqrt{|\lambda|} |x| \,\big|\log(|\lambda| |x|^2)\,\big|$ is bounded for $|\lambda| |x|^2 \leq ({1}/{2})$.
  %For $d = 2$, the same argument applies to the expression $\sqrt{|\lambda|} |x| (|\log(|\lambda| |x|^2)| + 1)$.
\end{proof}


\chapter{Single and Double Layer Potentials}

In this chapter, we will deal with \emph{single} and \emph{double layer potentials}.
Both will serve as ``representation formulas'' for solutions to the Stokes resolvent problem.
We will study their properties as they will serve as the crucial ingredient to solving the Neumann and Dirichlet boundary problems associated to the Stokes resolvent problem.
In this chapter we will always assume that $\Omega$ is a bounded Lipschitz domain in $\R^d$ with $d \geq 2$ and $1 < p < \infty$.
We will also tacitly use the summation convention.

Let $\lambda \in \Sigma_\theta$. 
For $f \in \Ell^2(\partial\Omega; \C^d)$, the single layer potential $u = \slp_\lambda(f)$ is defined by
\begin{align}
  \label{eq:defSingleLayer}
  u_j(x) 
  = \int_{\partial\Omega} \Gamma_{jk}(x - y; \lambda) f_k(y) \d \sigma(y),
\end{align}
where $\Gamma_{jk}$ is the fundamental solution to the Stokes reolvent problem given by \eqref{eq:fundamentalMatrixStokes}.
For the pressure, respectively, we define the single layer potential $\phi = \slp_\Phi(f)$ by
\begin{align}
  \label{eq:defSingleLayerPressure}
  \phi(x) = \int_{\partial\Omega} \Phi_k(x - y) f_k(y) \d \sigma(y),
\end{align}
where $\Phi_k$ is given by \eqref{eq:fundamentalVectorPressure}.
As we have already shown, $(u,\phi)$ defines a solution to the Stokes resolvent problem \eqref{eq:stokesResolventProblem}.

We define two further integral operators
\begin{align}
  T_\lambda^*(f)(P) &= \sup_{t > 0} \big| \int_{\substack{y \in \partial\Omega \\ |y - P| > t}} \nabla_x \Gamma(P - y; \lambda) f(y) \d \sigma(y) \big| \label{eq:supTOperator}\\
  T_\lambda(f)(P) &= \pv \int_{\partial\Omega} \nabla_x \Gamma(P - y; \lambda) f(y) \d \sigma(y) \label{eq:pvTOperator}
\end{align}
for $P \in \partial\Omega$ which will be used to prove boundedness of maximal operators related to $u$.

\begin{lem}
  \label{lem:lpBoundednessT}
  Let $1 < p < \infty$ and $T_\lambda(f), T_\lambda^*(f)$ be defined by \eqref{eq:supTOperator} and \eqref{eq:pvTOperator}.
  Then $T_\lambda(f)(P)$ exists for almost everywhere $P \in \partial\Omega$ and
  \begin{align}
    \label{eq:lpBoundednessT}
    \|T_\lambda(f) \|_{\Ell^p(\partial\Omega)} 
    \leq \|T_\lambda^*(f) \|_{\Ell^p(\partial\Omega)}
    \leq C_p \, \|f\|_{\Ell^p(\partial\Omega)},
  \end{align}
  where $C_p$ depends only on $d$, $\theta$, $p$, and the Lipschitz character of $\Omega$.
\end{lem}

\begin{proof}
  If $\lambda = 0$, the Lemma is known \cite{fabesKenigVerchota} as a consequence of the seminal result of Coifman et al. \cite{coifmanEtAl}.
  One idea of the proof in the case $\lambda \in \Sigma_\theta$ will thus be to nourish from this result and to consider the difference $\Gamma(x - y; \lambda) - \Gamma(x - y; 0)$ as a well-disposed integral kernel.

  We start with the second inequality of \ref{eq:lpBoundednessT}.
  To this end, let $t > 0$ and additionally assume that $t^2 |\lambda| \geq \frac{1}{2}$. 
  Theorem \ref{thm:fundamentalMatrixEstimate} gives
  \begin{align*}
    \Big| \int_{|y - P| > t} \nabla \Gamma(P - y; \lambda) f(y) \d\sigma(y) \Big|
    &\leq C \int_{|P - y| > t} \frac{|f(y)|}{|\lambda| |P - y|^{d + 1}} \d \sigma(y) 
  \end{align*}
  Choose now $N \in \N$ such that $2^N t \leq \diam(\Omega) < 2^{N + 1} t$.
  We now exhaust the domain of integration by suitable annuli and calculate
  \begin{align*}
    &\sum_{k = 0}^N \int_{\BB(P, 2^{k + 1}t) \cap \partial\Omega} \frac{1}{|\lambda| 2^{k(d + 1)} t^{d + 1}} |f(y)| \d \sigma(y) \\
    &\quad\leq \frac{1}{|\lambda|t^2}\frac{1}{2^{1 -d}}\sum_{k = 0}^N \frac{1}{2^{2k}} \frac{1}{(2^{k+1} t)^{d - 1}} \int_{\BB(P, 2^{k + 1}t \cap \partial \Omega)}  |f(y)| \d\sigma(y) \\
    &\quad\leq C \sum_{k = 0}^N \frac{1}{2^{2k}} M_{\partial\Omega}(f)(P) \\
    &\quad\leq C  M_{\partial\Omega}(f)(P)
    %&\quad\leq C |\lambda|^{\frac{1}{2}} \frac{2^{N + 1}t - t}{1}  M_{\partial\Omega}(f)(P) \\
    %&\quad\leq C |\lambda|^{\frac{1}{2}} t
  \end{align*}
  where for the second inequality we used Lemma \ref{lem:compareBoundaryWithBall} to estimate
  \begin{align*}
    \frac{1}{(2^{k+1} t)^{d-1}} \leq C (\sigma(\BB(P, 2^{k + 1}d) \cap \partial\Omega))^{-1}.
  \end{align*}
  which gives the claimed estimate with a constant $C$ that depends on $d$, $\theta$ and the Lipschitz character of $\Omega$.
  Now let $t^2 |\lambda| < \frac{1}{2}$.
  We then split the integral as follows
  \begin{align*}
    &\Big| \int_{|y - P| > t} \nabla \Gamma(P - y; \lambda) f(y) \d\sigma(y) \Big| \\
    &\quad\leq \Big| \int_{|y - P| \geq (2|\lambda|)^{-1/2}} \nabla\Gamma(P - y; \lambda)f(y) \d\sigma\Big|
     + \Big| \int_{t < |y - P| < (2|\lambda|)^{-1/2}} \nabla\Gamma(P - y; \lambda)f(y) \d\sigma\Big|.
  \end{align*}
  The first summand can be estimated like in the step before, if we substitute $t$ by $(2|\lambda|)^{-1/2}$
  For the second term we use the principle of the nutrient zero and estimate
  \begin{align*}
     &\Big| \int_{t < |y - P| < (2|\lambda|)^{-1/2}} \nabla\Gamma(P - y; \lambda)f(y) \d\sigma\Big| \\
     &\qquad\leq \int_{t < |y - P| < (2|\lambda|)^{-1/2}} |\nabla\Gamma(P - y; \lambda) - \nabla\Gamma(P - y; 0) |f(y)| \d\sigma \\
     &\qquad\quad + \Big| \int_{t < |y - P| < (2|\lambda|)^{-1/2}} \nabla\Gamma(P - y; 0) f(y)  \d\sigma \Big|.
  \end{align*}
  We don't need to worry about the second summand here since the corresponding estimate is already coverd by the case of $\lambda = 0$ and therefore
  \begin{align*}
     &\Big| \int_{t < |y - P| < (2|\lambda|)^{-1/2}} \nabla\Gamma(P - y; 0) f(y)  \d\sigma \Big| \\
     &\quad\leq \Big| \int_{|y - P| > t} \nabla\Gamma(P - y; 0) f(y)  \d\sigma \Big| +  \Big| \int_{|y - P| > (2|\lambda|)^{-1/2}} \nabla\Gamma(P - y; 0) f(y)  \d\sigma \Big| \\
     &\quad\leq 2 T_0^*(f)(P).
   \end{align*}
   For the first summand we make use of Theorem \ref{thm:differenceFundamentalSolutionStokes} and more precisely of Corollary \ref{cor:differenceFundamentalSolutionStokes} which unifies all estimates.
   We then calculate
   \begin{align*}
     &\int_{t < |y - P| < (2|\lambda|)^{-1/2}} |\nabla\Gamma(P - y; \lambda) - \nabla\Gamma(P - y; 0) |f(y)| \d\sigma \\
     &\qquad\leq \int_{t < |y - P| < (2|\lambda|)^{-1/2}} |\lambda|^{\frac{1}{2}} |y - P|^{2 - d} |f(y)| \d\sigma,
     \intertext{and as before we choose adequate $N$ such that $2^{N + 1} t > (2|\lambda|)^{-1/2} \geq 2^N t$ which leads to}
     &\qquad\leq |\lambda|^{\frac{1}{2}} \sum_{k = 0}^N \int_{2^kt < |y - P| < 2^{k + 1} t} |y - P|^{2 - d} |f(y)| \d\sigma \\
     &\qquad\leq  |\lambda|^{\frac{1}{2}} t^{2 - d} \sum_{k = 0}^N 2^{k(2 - d)} \int_{\BB(P, 2^{k + 1}t)} |f(y)| \d\sigma\\
     &\qquad\leq 2^d  |\lambda|^{\frac{1}{2}} t \sum_{k = 0}^N 2^{k - 1}\; 2^{(k + 1)(1 - d)} t^{1 - d} \int_{\BB(P, 2^{k + 1}t)} |f(y)| \d\sigma \\
     &\qquad \leq C  |\lambda|^{\frac{1}{2}} t\frac{2^{N} - 1}{1} M_{\partial\Omega}(f)(P) \\
     &\qquad \leq C |\lambda|^{\frac{1}{2}} (2|\lambda|)^{-\frac{1}{2}} M_{\partial\Omega}(f)(P).
   \end{align*}
   Taking now the supremum over all $t > 0$ we see that
   \begin{align*}
     T_\lambda^*(f)(P)
     &\leq C (M_{\partial\Omega}(f)(P) + T_0^*(f)(P)),
   \end{align*}
   for all $P \in \partial\Omega$. Once again using the result for $\lambda = 0$ and the $L^p$-boundedness of the Hardy-Littlewood maximal operator we see that
   \begin{align*}
     \|T_\lambda^*(f) \|_{\Ell^p(\partial\Omega)} \leq C\, \|f\|_{\Ell^p(\partial\Omega)}
   \end{align*}

   To conclude the first inequality in \eqref{eq:lpBoundednessT}, we want to use a standard result from harmonic analysis \cite[2.1.14]{grafakos}.
   First we will show that the integral operator
   \begin{align*}
     T_\lambda(f)(P) = \lim_{t \to 0} \int_{\substack{y \in \partial\Omega \\ |y - P| > t}} \nabla_x \Gamma(P - y; \lambda) f(y) \d\sigma(y)
   \end{align*}
   exists for almost every $P \in \partial\Omega$ and all $f \in \CC(\partial\Omega; \C^d)$.
   In a first step, we can split this operator formally in
   \begin{align*}
     T_\lambda(f)(P) = T_0(f)(P) + \lim_{t \to 0} \int_{\substack{y \in \partial\Omega \\ |y - P| > t}} \nabla_x \{ \Gamma(P - y; \lambda) - \Gamma(P - y; 0)\}  f(y) \d\sigma(y)
   \end{align*}
   The right expression is well defined for $f \in \CC_0^\infty$, once we prove integrability of 
   \begin{align*}
     |\nabla \{\Gamma(P - y; \lambda) - \Gamma(P - y; 0)\} |
   \end{align*}
   on $\partial\Omega$.
   To this end we first note that it suffices to consider the integral
   \begin{align*}
     \int_{|P - y| \leq \varepsilon} | \nabla\{ \Gamma(P - y; \lambda) - \Gamma(P - y; 0) \} | \d\sigma(y), 
   \end{align*}
   for $\varepsilon \leq \min(2|\lambda|^{-1/2}, r_0/4)$ as the integrand is smooth away from $0$ and the domain of integration is bounded.
   Now Corollary \ref{cor:differenceFundamentalSolutionStokes} and Tolksdorf 4.3.2 give that this can be estimated by
   \begin{align*}
     \int_{|P - y| \leq \varepsilon} |\lambda|^{1/2} |P - y|^{2 - d} \d\sigma(y) \leq C |\lambda|^{1/2} \varepsilon \leq C.
   \end{align*}
   Based on the preceding calculation we conclude that for all $f \in \CC(\partial\Omega,\C^d)$ the operator $T_\lambda(f)(P)$ exists whenever $T_0(f)(P)$ exists.
   $T_0(f)(P)$ exists for almost everywhere $P \in \partial\Omega$ because of Fabes, Kenig and Verchota \cite{fabesKenigVerchota}.
   As furthermore $T_\lambda^*(f)(P)$ is bounded on $\Ell^p(\partial\Omega)$  we may now apply Theorem 2.1.14 from Grafakos \cite{grafakos} to conclude that $T_\lambda(f)(P)$ exists now for all $f \in \Ell^p(\partial\Omega; \C^d)$ and almost everywhere $P \in \partial\Omega$.
   The desired $\Ell^p$ estimate on $T_\lambda(f)$ now follows from the observation that 
   \begin{align*}
     | T_\lambda(f)(P) | \leq T_\lambda^*(f)(P)
   \end{align*}
   for almost everywhere $P \in \partial\Omega$.
\end{proof}

For a function $u$ in $\Omega$, we define the nontangential maximal function $(u)^*$ by
\begin{align}
  \label{eq:defnNontangMaxFunction}
  (u)^*(P) = \sup\{ |u(x)| \colon x \in \Omega \text{ and } |x - P| < C \operatorname{dist}(x, \partial\Omega)\}
\end{align}
for $P \in \partial\Omega$, where $C > 2$ is a fixed and sufficiently large constant depending on $d$ and the Lipschitz character of $\Omega$.
Note that in Shen cones we have that for $P, y \in \partial\Omega$ and $x \in \ShenCone(P)$
\begin{align}
  \label{eq:shenConeEstimate}
  |P - y| 
  &\leq |P - x| + |x - y| 
  \leq C \operatorname{dist}(x \partial\Omega) + |x - y|  \nonumber\\
  &\leq (C + 1) |x - y|
\end{align}
where $C$ is the constant from \eqref{eq:defnNontangMaxFunction}

We can now prove the boundedness of certain nontangential maximal operators.
\begin{lem}
  \label{lem:nontangentialMaximalFunctions}
  Let $1 < p < \infty$ and $(u,\phi)$ be given by \eqref{eq:defSingleLayer} and \eqref{eq:defSingleLayerPressure}.
  Let furthermore $d \geq 3$.
  Then 
  \begin{align}
    \| (\nabla u)^* \|_{\Ell^p(\partial\Omega)} + \| (\phi)^* \|_{\Ell^p(\partial\Omega)} + |\lambda|^{\frac{1}{2}} \|(u)^*\|_{\Ell^p(\partial\Omega)} \leq C_p \|f\|_{\Ell^p(\partial\Omega)},
  \end{align}
  where $C_p$ depends only on $d$, $\theta$, $p$ and the Lipschitz character of $\Omega$.
\end{lem}

\begin{proof}
  A proof of the estimate $\|(\phi)^*\|_{\Ell^p(\partial\Omega)} \leq C_p \|f\|_{\Ell^p(\partial\Omega)}$ can be found in Verchota \cite{verchota}.
  The proof for $\|(\nabla u)^*\|_{\Ell^p(\partial\Omega)}$ works in the same way. 
  We will provide a proof for the sake of completeness.
  To immitate the proof of Verchota, we will work with the corresponding type of cones.
  Therefore the results for $\nabla u$ and $\phi$ will at first only be established for the type of maximal operators defined by Verchota.
  The transferability to Shen's maximal operators is given by Tolksdorf \cite{tolksdorfDiss} as the solution $(u,\phi)$ has a representation as a single layer potential.

  Let $x \in \verCone(P)$ and set $t = |x - P|$.
  Then,
  \begin{align*}
    |(\nabla u)(x) |
    &= \big| \int_{\partial\Omega} \nabla \Gamma_{jk} (x - y; \lambda) f_k \d \sigma(y)\big| \\
    &\leq \big| \int_{|y - P| > t} \nabla\Gamma_{jk}(x - y; \lambda) f_k \d \sigma(y) \big| + \big| \int_{|y - P| \leq t} \nabla \Gamma_{jk}(x - y; \lambda) f_k \d \sigma(y)\big| \\
    &= I_1 + I_2.
  \end{align*}
  We will now estimate $I_1$ and $I_2$ seperately.
  Note that in Verchota cones we have that for all $Q \in \partial\Omega$ we have $|x - Q| > C |x - P|$, where $C$ is a constant only depending on $d$ and the Lipschitz character of $\Omega$.
  By Theorem \ref{thm:fundamentalMatrixEstimate} we know that
  \begin{align*}
    I_2 
    &\leq C \int_{|y - P| \leq t} \frac{1}{|x - y|^{d - 1}} |f(y)| \d \sigma(y) \\
    &\leq \frac{C}{t^{n - 1}} \int_{|y - P| \leq t} |f(y)| \d \sigma(y)
    \leq C M_{\partial\Omega} (f)(P).
  \end{align*}
  For $I_1$, we calculate
  \begin{align*}
    &\Big| \int_{| y - P | > t} \nabla \Gamma_{j k} (x - y; \lambda) f_k(y) - \nabla \Gamma_{jk}(P - y; \lambda) f_k(y) + \nabla \Gamma_{jk}(P - y; \lambda) f_k(y) \d \sigma(y) \Big| \\
    &\quad\leq \Big| \int_{|y - P| > t} \nabla\Gamma_{jk} (x - y; \lambda) f_k(y) - \nabla\Gamma_{jk}(P - y; \lambda) f_k(y) \d \sigma(y) \Big| \\
    &\qquad+ \Big| \int_{|y - P| > t} \nabla\Gamma_{jk}(P - y; \lambda) f_k(y) \d\sigma(y) \Big|.
  \end{align*}
  The second summand can directly be estimated by $T_\lambda^*(f)(P)$.
  For the second one we apply the mean value theorem and derive using once again Theorem \ref{thm:fundamentalMatrixEstimate}
  \begin{align*}
    &\int_{|y - P| > t} \big| \nabla\Gamma_{jk}(x - y; \lambda) - \nabla\Gamma_{jk}(P - y; \lambda) \big| |f(y)| \d\sigma(y) \\ 
    &\qquad\leq \int_{|y - P| > t} |\nabla^2 \Gamma_{jk}(s - y; \lambda)| |x - P| |f(y) | \d\sigma(y) \\
    &\qquad\leq C t \int_{|y - P| > t} \frac{1}{|s - y|^{d}} |f(y)| \d\sigma(y) \\
    &\qquad\leq C t \int_{|y - P| > t} \frac{1}{|y - P|^{d}} |f(y)| \d\sigma(y) ,
  \end{align*}
  where $s$ is an element on the line connecting $x$ and $P$ and we used the property of Verchota-cones that $|s - y| \geq C |y - P|$.
  Note that Verchota cones are convex.
  By exhausting the domain of integration using annuli, we can estimate this integral by $M_{\partial\Omega}f(P)$:
  Choose $N$ such that $2^N t \leq \diam(\Omega) < 2^{N + 1}t$.
  Then
  \begin{align*}
    &\int_{|y - P| > t} \frac{1}{|y - P|^{d - 1}} |f(y)| \d\sigma(y) \\
    &\quad= \sum_{k = 0}^N \int_{2^{k + 1}t > |y - P| \geq 2^k t} \frac{1}{|y - P|^{d - 1}} |f(y)| \d\sigma(y) \\
    &\quad\leq \sum_{k = 0}^N \frac{1}{2^{k(d - 1)} t^{d - 1}} \int_{|y - P| < 2^{k + 1} t} |f(y)| \d\sigma(y) \\
    &\quad \leq  C \sum_{k = 0}^N 2^{k - d + 1)} M_{\partial\Omega}(f)(P)
    &\quad\leq
  \end{align*}
  Taking the supremum over all $x \in \Omega$ the claim follows.

  We will now work on the proof of the estimate for $(u)^*$.
  In order to derive $\Ell^p$ estimates on this maximal operator we will work directly with the Definition of the single layer potential \eqref{eq:defSingleLayer}.
  For $P \in \partial\Omega$, estimate \eqref{eq:fundamentalMatrixEstimate} together with the estimate for Shen cones \eqref{eq:shenConeEstimate} gives that for all $x \in \ShenCone(P)$
  \begin{align*}
    |u^*(x)| 
    &\leq  C \int_{\partial\Omega} \frac{1}{|x - y|^{d - 2}} |f(y)| \d\sigma(y) 
    \leq  C \int_{\partial\Omega} \frac{1}{|P - y|^{d - 2}} |f(y)| \d\sigma(y),
  \end{align*}
  where $C$ only depends on $d$, $\theta$ and the Lipschitz character of $\Omega$.
  Passing to the maximal operator yields the inequality
  \begin{align*}
    u^*(P) \leq  C \int_{\partial\Omega} \frac{1}{|P - y|^{d - 2}} |f(y)| \d\sigma(y),
  \end{align*}
  We are now left with the task to estimate the integral 
  \begin{align*}
    \int_{\partial\Omega} \frac{1}{|P - y|^{d - 2}} \d\sigma(y)
  \end{align*}
  uniformly for all $P \in \partial\Omega$, as the rest can be handled using the Young inequality.
  Let $r_0$ be the Radius from the definition of Lipschitz cylinders.
  Then
  \begin{align*}
    \int_{\partial\Omega} \frac{1}{|P - y|^{d - 2}} \d\sigma(y)
    &\leq \int_{\partial\Omega \cap \BB(P; r_0/4)} \frac{1}{|P - y|^{d - 2}} \d\sigma(y) + \int_{\partial\Omega \setminus \BB(P; r_0/4)} \frac{1}{|P - y|^{d - 2}} \d\sigma(y). \\
    & \leq C r_0/4 +  \sigma(\partial\Omega) r_0^{2 - d} 4^{d - 2}.
  \end{align*}
  where $C$ only depends on $d$ and the Lipschitz character of $\Omega$.
\end{proof}

The next Lemma deals with \emph{trace formulas} for $\nabla u$ and $\phi$. We can now finally talk about boundary values as the existence of nontangential limits guarantees that there exists something on $\partial\Omega$ that is related to the function inside $\Omega$.

\begin{lem}
  \label{lem:traceFormulas}
  Let $(u,\phi)$ be given by \eqref{eq:defSingleLayer} and \eqref{eq:defSingleLayerPressure} with $f \in \Ell^p(\partial\Omega; \C^d)$ and $1 < p < \infty$.
  Then
  \begin{align}
    \big( \frac{\partial u_i}{\partial x_j} \big)_{\pm}(x) 
    &= \pm \frac{1}{2} \{ n_j(x) f_i(x) - n_i(x) n_j(x) n_k(x) f_k(x) \} \nonumber\\
    &\quad+ \pv \int_{\partial\Omega} \frac{\partial}{\partial x_j} \{ \Gamma_{ik} (x - y; \lambda) \} f_k(y) \d\sigma(y), \label{eq:traceFormula} \\
    \phi_\pm(x) &= \mp \frac{1}{2} n_k(x) f_k(x) + \pv\int_{\partial\Omega} \Phi_k(x - y) f_k(y) \d \sigma(y) \nonumber
  \end{align}
  for almost everywhere $x \in \partial\Omega$.
  The subscripts $+$ and $-$ indicate nontangential limits taken inside $\Omega$ and outside $\overline\Omega$, respectively.
\end{lem}

\begin{proof}
  The correctness of the trace formulas \eqref{eq:traceFormula} is known for the case $\lambda = 0$ since Fabes, Kenig and Verchota \cite{fabesKenigVerchota}.
  This fact will now be reused for $\lambda \in \Sigma_\theta$.
  We insert a $0$ to the nontangential limit as
  \begin{align*}
    (\nabla u_j)_\pm(x) = 
    (\nabla v_j)_\pm(x) + (\nabla u_j - \nabla v_j)_\pm(x),
  \end{align*}
  where $v_j(x) = \int_{\partial\Omega} \Gamma_{jk}(x - y; 0) f_k(y) \d\sigma(y)$.
  Because of \cite{fabesKenigVerchota} we know that the first nontangential limit exists and is given by \eqref{eq:traceFormula} with $\lambda = 0$.
  It therefore remains to show that
  \begin{align*}
    (\nabla u_j - \nabla v_j)_\pm(x) = \int_{\partial\Omega} \nabla \{ \Gamma_{jk}(x - y; \lambda) - \Gamma_{jk}(x - y; 0) \} f_k(y) \d\sigma(y)
  \end{align*}
  for all $x \in \partial \Omega$.
  To this end let $(x_l)_{l \in \N}$ a sequence in $\ShenCone(x)$ with $\lim_{l \to \infty} x_l = x$.
  Furthermore let us note that for almost everywhere $x \in \partial\Omega$ we have that 
  \begin{align*}
    \int_{\partial\Omega} \frac{1}{|x - y|^{d - 2}} |f(y)| \d\sigma(y) < \infty.
  \end{align*}
  This is a consequence of the fact that
  \begin{align*}
    \sup_{x \in \partial\Omega} \Big| \int_{\partial\Omega} \frac{1}{|x - y|^{d - 2}} \d\sigma(y) \Big| < \infty
  \end{align*}
  and an application of Young's inequality which can be found in Tolksdorf \cite{tolksdorfDiss}:
  Let $x \in \partial\Omega$.
  Then
  \begin{align*}
    &\int_{\partial\Omega} \frac{1}{|x - y|^{d - 2}} \d\sigma(y) \\
    &\quad\leq \int_{\partial\Omega \cap \BB(x,r_0/4)} \frac{1}{|x - y|^{d - 2}} \d\sigma(y) + \int_{\partial\Omega \setminus \BB(x, r_0/4)} \frac{1}{|x - y|^{d - 2}} \d\sigma(y) \\
    &\quad\leq C r_0 + r^{2 - d} 4^{d - 2} \sigma(\partial\Omega)
  \end{align*}
  by Lemma \ref{lem:compareBoundaryWithBall}.
  Now Young's inequality gives us the desired result.
  In the next step we will show that
  \begin{align*}
    \frac{1}{|x -y|^{d - 2}} |f(y)|
  \end{align*}
  gives a suitable function for dominated convergence.
  Set $\varepsilon = (4 |\lambda|^2)^{-1}$ and without loss of generality assume that $\supp f \subseteq \BB(x,\varepsilon)$.
  Furthermore assume that $|x_l - x| < \varepsilon$ for all $l \in \N$.
  Then $|x_l - y| \leq (2|\lambda|^2)^{-1}$ and Corollary \ref{cor:differenceFundamentalSolutionStokes} give
  \begin{align*}
    & (\nabla u_j - \nabla v_j)(x_l) \int_{\partial\Omega} \nabla \{ \Gamma_{jk}(x_l - y; \lambda) - \Gamma_{jk}(x_l - y; 0)\} f_k(y) \d\sigma(y) \\
    &\quad\leq \int_{\partial\Omega} \frac{1}{\sqrt{|\lambda|}|x_l - y|^{d - 2}} |f(y)| \d\sigma(y) \\
    &\quad\leq \frac{C}{\sqrt{|\lambda|}} \int_{\partial\Omega} \frac{1}{|x - y|^{d - 2}} |f(y)| \d\sigma(y) < \infty.
  \end{align*}
  Now dominated convergence gives the claim for $x_l \to x$.
  Note that it does not affect the proof if the sequence $x_l$ lays inside $\Omega$ or outside $\overline\Omega$.
\end{proof}

The previous Lemma enables us to talk about boundary values of partial derivatives. 
The next theorem will now give a similar result but for conormal derivatives which are defined by
\begin{align*}
  \frac{\partial u}{\partial \nu} = \frac{\partial u}{\partial n} - \phi n.
\end{align*}
We will also be working with tangential derivatives which are defined via
\begin{align*}
  DEFINE
\end{align*}

\begin{thm}
  \label{thm:jumpConditions}
  Let $\lambda \in \Sigma_\theta$ and $\Omega$ be a bounded Lipschitz domain in $\R^d$, $d \geq 3$. 
  Let $(u,\phi)$ be given by \eqref{eq:defSingleLayer} and \eqref{eq:defSingleLayerPressure} with $f \in \Ell^p(\partial\Omega; \C^d)$ and $1 < p < \infty$.
  Then $\nabla_{\mathrm{tan}} u_+ = \nabla_{\mathrm{tan}} u_-$ and
  \begin{align}
    \label{eq:nontangentialConormalDerivative}
    \Big( \frac{\partial u}{\partial \nu} \Big)_\pm = \Big( \pm \frac{1}{2} I + \K_\lambda \Big) f
  \end{align}
  on $\partial\Omega$, with $\K_\lambda$ a bounded operator on $\Ell^p(\partial\Omega; \C^d)$ with
  \begin{align*}
    \| \K_\lambda f \|_{\Ell^p(\partial\Omega)} \leq C_p \|f\|_{\Ell^p(\partial\Omega)},
  \end{align*}
  where $C_p$ depends only on $d$, $\theta$, $p$ and the Lipschitz character of $\Omega$.
\end{thm}

\begin{proof}
  For the the $j$th component of the tangential derivative of $u_i$, $1\leq i,j \leq d$, we calculate using the results from Lemma \ref{lem:traceFormulas}
  \begin{align*}
    ((\nabla_{\mathrm{tan}} u_i)_+)_j
    &= (\frac{\partial u_i}{\partial x_j})_+ - \langle (\nabla u_i)_+, n \rangle n_j \\
    &= (\frac{\partial u_i}{\partial x_j} )_+ - (\frac{\partial u_i}{\partial x_k} )_+ n_k n_j \\
    &= \frac{1}{2} \{ n_j f_i - n_i n_j n_k f_k\} - \frac{1}{2} \{ n_k f_i - n_i n_k n_l f_l \} n_k n_j  \\
    &\quad+ \pv \int_{\partial\Omega} \frac{\partial}{\partial x_j} \{ \Gamma_{ik} (x - y; \lambda) \} f_k(y) \d\sigma(y) \\
    &\quad+ \pv \int_{\partial\Omega} \frac{\partial}{\partial x_k} \{ \Gamma_{il} (x - y; \lambda) \} f_l(y) \d\sigma(y) n_k n_j.
  \end{align*}
  As the first two summands add up to zero, the entire expression does not depend on the direction of the nontangential limit. 
  This gives
  \begin{align*}
    (\nabla_{\mathrm{tan}} u)_+ = (\nabla_{\mathrm{tan}} u)_-
  \end{align*}
  We calculate for the $j$th component of the nontangential limit of the conormal derivative of $u$ at $x \in \partial\Omega$ using the results from Lemma \ref{lem:traceFormulas}
  \begin{align*}
    &\big(\frac{\partial u_j}{\partial x_i}\big)_+(x) n_i - \phi_+(x) n_j\\
    &\quad= \frac{1}{2} \{ n_i f_j(x) - n_j n_i n_k f_k(x) \} n_i + \pv \int_{\partial\Omega} \frac{\partial}{\partial{x_i}} \big\{ \Gamma_{jk}(x - y; \lambda) \big\} f_k(y) \d\sigma(y) n_i \\
    &\qquad+ \frac{1}{2} n_k f_k(x) n_j - \pv \int_{\partial\Omega} \Phi_k(x - y) f_k(y) \d\sigma(y) n_j \\
    &\quad= \frac{1}{2} f_j(x) + (\K_\lambda f)_j(x),
  \end{align*}
  where $n$ denotes the normal vector at $x$ and
  \begin{align}
    \label{eq:defnKlambda}
    (\K_\lambda f) (x)
    &= \pv \int_{\partial\Omega} \nabla_x \Gamma(x - y; \lambda)  f(y) \d\sigma(y) n - \pv \int_{\partial\Omega} \Phi_k(x - y) f_k(y) \d\sigma(y) n.
  \end{align}
  We note that $\K_\lambda$ essentially consists of two boundary layer potentials. 
  The $\Ell^p$-boundedness of the first one was proven in Lemma \ref{lem:lpBoundednessT}.
  The $\Ell^p$-boundedness of the second boundary layer potential follows in an analogous way using the fact that the operators
  \begin{align*}
    A^*(f)(P) = \sup_{t > 0} \Big| \int_{\substack{y \in \partial\Omega \\ |y - P| > t}} \frac{P - y}{|P - y|^n} f(y) \d\sigma(y)\Big| , \quad P \in \partial\Omega
  \end{align*}
  are bounded by Lemma 1.2 of Verchota \cite{verchotaDiss}.
\end{proof}

Similar to $\K_\lambda$ for $\lambda = 0$ we have
\begin{align}
  \label{eq:defnK0}
    (\K_0 f)(x)= \pv \int_{\partial\Omega} \nabla_x \Gamma(x - y; 0)  f(y) \d\sigma(y) n - \pv \int_{\partial\Omega} \Phi_k(x - y) f_k(y) \d\sigma(y) n,
\end{align}
as was shown by Fabes, Kenig and Verchota \cite[(0.12)]{fabesKenigVerchota}.

We now note a fact that will be crucial for solving the $\Ell^2$ Dirichlet problem in Chapter 5 and will fortify the hopes of translating results for $\lambda = 0$ to $\lambda \in \Sigma_\theta$.
\begin{lem}
  \label{lem:compactness}
  Let $\lambda \in \Sigma_\theta$ and $d \geq 3$.
  Then the operator $\K_\lambda - \K_0$ on $\Ell^2(\partial\Omega; \C^d)$ is compact.
\end{lem}

\begin{proof}
  The idea of this proof is similar to the one in Tolksdorf \cite[Lemma 4.3.5]{tolksdorfDiss}.
  Let $f \in \Ell^2(\partial\Omega; \C^d)$.
  Let's denote $\K \coloneqq \K_\lambda - \K_0$. 
  We will now try to approximate $\K$ by compact operators in the operator norm.
  To this end we define for all $\varepsilon > 0$
  \begin{align*}
    (\K^{(\varepsilon)}f)(x) \coloneqq \int_{\partial\Omega \setminus \BB(x,\varepsilon)} \nabla \{ \Gamma(x - y; \lambda) - \Gamma(x - y; 0) \} f(y) \d \sigma(y), \quad x \in \partial\Omega.
  \end{align*}
  We can now estimate by Young's inequality \ref{lim:young} 
  \begin{align*}
    \| (\K^{(\varepsilon)} f)(x) \|_{\Ell^2(\partial\Omega)}
    \leq \sup_{p \in \partial\Omega} \| \nabla \{ \Gamma(p - \cdot; \lambda) - \Gamma(p - \cdot; 0) \} 1_{\BB(p,\varepsilon)} \|_{\Ell^1(\partial\Omega)} \|f\|_{\Ell^2(\partial\Omega)}.
  \end{align*}
  Our goal is to show that
  \begin{align*}
    \sup_{p \in \partial\Omega} \| \nabla \{ \Gamma(p - \cdot; \lambda) - \Gamma(p - \cdot; 0) \} 1_{\BB(p,\varepsilon)} \|_{\Ell^1(\partial\Omega)} \to 0 \quad \text{as} \quad \varepsilon \to 0.
  \end{align*}
  To this end, let $\varepsilon$ be small enough such that we can apply the estimates from Corollary \ref{cor:differenceFundamentalSolutionStokes} to calculate for some $p \in \partial\Omega$
  \begin{align*}
     &\| \nabla \{ \Gamma(p - \cdot; \lambda) - \Gamma(p - \cdot; 0) \} 1_{\BB(p,\varepsilon)} \|_{\Ell^1(\partial\Omega)}  \\
     &\qquad\qquad\leq C \int_{\partial\Omega \cap \BB(p, \varepsilon)} \sqrt{|\lambda|} |p - y|^{2 - d} \d\sigma(y) 
     \leq C \sqrt{|\lambda|} \varepsilon
  \end{align*}
  where for the last step we applied Lemma \ref{lem:comparability}.
  For $\varepsilon \to 0$ this gives us $\K^{(\varepsilon)} \to \K$ in the operator norm.
  The last step is to verify the compactness of $\K^(\varepsilon)$.
  We note that the integral kernel of $\K^{(\varepsilon)}$ is bounded which gives us that in particular the kernel is an element of $\Ell^2(\partial\Omega \times \partial\Omega; \C^{d \times d}$.
  The compactness of $\K^{(\varepsilon)}$ now follows from Weidmann \cite[Thm. 6.11]{weidmann}.
\end{proof}

Our next step is to introduce the \emph{double layer potential} $u(x) = \dlp_\lambda(f)(x)$ for the Stokes resolvent problem, where
\begin{align}
  \label{eq:defDoubleLayer}
  u_j(x) = \int_{\partial\Omega} \Big\{ \frac{\partial}{\partial y_i} \{ \Gamma_{jk}(y - x; \lambda) \} n_i(y) - \Phi_j(y - x) n_k(y) \Big\} f_k(y) \d\sigma(y).
\end{align}
The corresponding pressure $\phi(x) = \dlp_{\phi}(f)(x)$ is defined via
\begin{align}
  \label{eq:defDoubleLayerPressure}
  \phi(x)
  &= \frac{\partial^2}{\partial x_i \partial x_k} \int_{\partial\Omega} G(y - x; 0) n_i(y) f_k(y) \d\sigma(y) + \lambda \int_{\partial\Omega} G(y - x; 0) n_k(y) f_k(y) \d\sigma(y).
\end{align}
Using \ref{eq:fundamentalVectorPressure} and \ref{eq:solutionStokesSystem} one can show that $(u,\phi)$ defines again a solution to the Stokes resolvent problem in $\R^d \setminus \partial\Omega$.

The next theorem will give us a suitable operator which maps a given function $f \in \Ell^p(\partial\Omega; \C^d)$ to boundary values of $u = \dlp_\lambda(f)$.

\begin{thm}
  \label{thm:nontangentialLimitDoubleLayer}
Let $\lambda \in \Sigma_\theta$ and $\Omega$ be a bounded Lipschitz domain in $\R^d$, $d \geq 3$.
Let $u$ be given by \eqref{eq:defDoubleLayer} for $f \in \Ell^p(\partial\Omega; \C^d)$, $1 < p < \infty$.
Then
\begin{align}
  \label{eq:lpBoundednessUNontangentialMax}
  \|(u)^*\|_{\Ell^p(\partial\Omega)} \leq C_p \|f\|_{\Ell^p(\partial\Omega)}
\end{align}
where $C_p$ depends only on $d$, $p$, $\theta$ and the Lipschitz character of $\Omega$.
Furthermore 
\begin{align}
  \label{eq:nontangentialLimitDoubleLayer}
  u_\pm = \Big(\mp \frac{1}{2} I + \K_{\bar\lambda}^* \Big) f,
\end{align}
where $K_{\bar\lambda}^*$ is the adjoint of the operator $K_{\bar\lambda}$ in \eqref{eq:nontangentialConormalDerivative}
\end{thm}

\begin{proof}
  The estimate for $(u)^*$ is a direct consequence of Lemma \ref{lem:nontangentialMaximalFunctions}, in particular of the estimates on $(\nabla u)^*$ and $(\phi)^*$.

  For the proof of \eqref{eq:nontangentialLimitDoubleLayer}, we begin by determining the adjoint of the operator $\K_{\bar\lambda}$.
  To this end we fill first work with truncated operators $\K_{\lambda}^{(\varepsilon)}$ which are defined as
  \begin{align*}
    (\K_\lambda^{(\varepsilon)} f) (x)
    &= \int_{\partial\Omega} 1_{\EE(x,\varepsilon)} \nabla_x \Gamma(x - y; \lambda)  f(y) \d\sigma(y) n - \int_{\partial\Omega} 1_{\EE(x,\varepsilon)} \Phi_k(x - y) f_k(y) \d\sigma(y) n,
  \end{align*}
  for $x \in \partial\Omega$ and $\EE(x,\varepsilon) \coloneqq \R^d \setminus \BB(x,\varepsilon)$.
  Now for $f \in \Ell^p(\partial\Omega; \C^d)$ and $g \in \Ell^q(\partial\Omega; \C^d)$ with $1/p + 1/q = 1$ we calculate
  \begin{align*}
    \langle \K_{\bar\lambda}^{(\varepsilon)}f, g \rangle
    &= \int_{\partial\Omega} (\K_{\bar\lambda}^{(\varepsilon)} f_j)(x) \overline{g_j(x)} \d\sigma(x) \\
    &= \int_{\partial\Omega} \int_{\partial\Omega} \frac{\partial}{\partial x_i} \{ \Gamma_{jk} (x - y; \overline\lambda) \} f_k(y) 1_{\EE(x,\varepsilon}(y) \d\sigma(y) n_i(x) \overline{g_j(x)} \d\sigma(x) \\
    &\quad+ \int_{\partial\Omega} \int_{\partial\Omega} \Phi_k(x -y)f_k(y) 1_{\EE(x,\varepsilon)}(y) \d\sigma(y) n_j(x) \overline{g_j(x)} \d\sigma(x).
  \end{align*}
  Note that $1_{\EE(x,\varepsilon)}(y) = 1_{\EE(y,\varepsilon)}(x)$.
  Now an application of Fubini and factoring out $f_k(y)$ gives that the lengthy expression is equal to
  \begin{align*}
    \int_{\partial\Omega} f_k(y) \int_{\partial\Omega} &\Big\{ \frac{\partial}{\partial x_i} \{ \Gamma_{jk}(x - y; \bar\lambda)\} n_i(x)
    - \Phi_k(x - y)  n_j(x) \Big\}1_{\EE(y, \varepsilon)}(x) \overline{g_j(x)} \d\sigma(x) \d\sigma(y).
  \end{align*}
  Therefore we see that the adjoint of the truncated operator $\K_{\bar\lambda}^{(\varepsilon)}$ is given by
  \begin{align*}
    \big( ( K_{\bar\lambda}^{(\varepsilon)})^* g\big)_k(y)
    = \int_{\partial\Omega} \Big\{ \frac{\partial}{\partial x_i} \{ \Gamma_{jk}(x - y; \lambda)\} n_i(x) - \Phi_k(x - y) n_j(x) \Big\} 1_{\EE(y,\varepsilon)}(x) g_j(x) \d\sigma(x), 
  \end{align*}
  for $y \in \partial\Omega$ since $\overline{\Gamma_{jk}(x - y; \lambda)} = \Gamma_{jk}(x - y; \bar\lambda)$.

  In the next step we will go from truncated operators to principal value operators. 
  For this to work we will look for suitable majorants.
  If $x \in \partial\Omega$ we estimate
  \begin{align*}
    |(\K_{\bar\lambda}^{(\varepsilon)}f)_j (x)|
    &= \Big| \int_{|x - y| > \varepsilon} \frac{\partial}{\partial x_i} \{ \Gamma_{jk}(x - y; \lambda) \} f_k(y) \d\sigma(y) n_i(x)  \\
    &\qquad - \int_{|x - y| > \varepsilon} \Phi_k(x - y) f_k(y) n_j(x) \d\sigma(y) \Big| \\
    &\leq T_\lambda^*(f)(x) + A^*(f)(x).
  \end{align*}
  Now dominated convergence gives
  \begin{align*}
    \lim_{\varepsilon \to 0} \langle K_{\bar\lambda}^{(\varepsilon)} f, g \rangle = \langle K_{\bar\lambda} f, g\rangle.
  \end{align*}
  A similar argument gives
  \begin{align*}
    \lim_{\varepsilon \to 0} \langle f, {K_{\bar\lambda}^{(\varepsilon)}}^{(*)} g \rangle = \langle f,K_{\bar\lambda}^{(*)}  g\rangle,
  \end{align*}
  where
  \begin{align*}
    \big( ( K_{\bar\lambda}^* g\big)_k(y)
    = \pv\int_{\partial\Omega} \Big\{ \frac{\partial}{\partial x_i} \{ \Gamma_{kj}(x - y; \lambda)\} n_i(x) - \Phi_k(x - y) n_j(x) \Big\} g_j(x) \d\sigma(x). 
  \end{align*}
  Note that we have used the symmetry of $(\Gamma_{\alpha\beta})$.

  The last part now consists of proving that the equality \eqref{eq:nontangentialLimitDoubleLayer} holds.
  To simplify the calculations and make Lemma \ref{lem:traceFormulas} more accessible note that on the one hand
  \begin{align*}
    \int_{\partial\Omega} \frac{\partial}{\partial y_i} \{ \Gamma_{jk}(y - x; \lambda) \} n_i(y) f_k(y) \d\sigma(y)
    &= -\int_{\partial\Omega} \frac{\partial}{\partial x_i} \{ \Gamma_{jk}(x - y; \lambda) \} n_i(y) f_k(y) \d\sigma(y)\\
    &= -\frac{\partial}{\partial x_i} \slp(n_i f)_j(x)
  \end{align*}
  and on the other hand
  \begin{align*}
    -\int_{\partial\Omega} \Phi_j(y - x) n_k(y) f_k(y) \d\sigma(y)
    = \int_{\partial\Omega} \Phi_l(x - y) \delta_{lj} n_k(y) f_k(y) \d\sigma(y)
    = \slp_\Phi(\tilde f^j)(x), 
  \end{align*}
  where $\tilde f^j_l = \delta_{lj} n_k f_k$.
  For $x \in \partial\Omega$ we can now calculate
  \begin{align*}
    &\Big( \int_{\partial\Omega} \frac{\partial}{\partial y_i} \{ \Gamma_{jk}(y - \cdot\, ; \lambda) \} n_i(y) f_k(y) \d\sigma(y) \Big)_\pm(x)\\
    &\qquad= - \big( \frac{\partial}{\partial x_i} \slp_\lambda(n_i f)_j\big)_\pm(x) \\
    &\qquad= \mp \frac{1}{2} \{ n_i(x) n_i(x) f_j(x) - n_j(x) n_i(x) n_k(x) n_i(x) f_k(x)  \} \\
    &\qquad\quad - \pv\int_{\partial\Omega} \frac{\partial}{\partial x_i} \{\Gamma_{jk}(x - y; \lambda) \} n_i(y) f_k(y) \d\sigma(y) \\
    &\qquad= \mp \frac{1}{2} \{ f_j(x) - n_j(x) n_k(x) f_k(x) \} \\
    &\qquad\quad + \pv\int_{\partial\Omega} \frac{\partial}{\partial y_i} \{\Gamma_{jk}(x - y; \lambda) \} n_i(y) f_k(y) \d\sigma(y),
  \end{align*}
  where we used trace formula \eqref{eq:traceFormula}.
  A similar procedure for the second integral part of the double layer potential gives
  \begin{align*}
    &-\Big(\int_{\partial\Omega} \Phi_j(y - \cdot\,) n_k(y) f_k(y) \d\sigma(y)\Big)_\pm(x) \\
    &\qquad= \big(\slp_\Phi(\tilde f^j)\big)_\pm(x)  \\
    &\qquad= \mp \frac{1}{2} n_k(x) \tilde f^j_k(x) - \pv \int_{\partial\Omega} \Phi_k(x - y) \tilde f^j_k(x) \d\sigma(y) \\
    &\qquad= \mp \frac{1}{2} n_j(x) n_k(x) f_k(x) - \pv \int_{\partial\Omega} \Phi_j(x - y) n_k(x) f_k(x) \d\sigma(y)
  \end{align*}
  Putting everything together we get
  \begin{align*}
    (u_j)_\pm(x) = \mp\frac{1}{2} f_j(x) + (K_{\bar\lambda}^* f)_j(x)
  \end{align*}
  which proves the claim.
\end{proof}

\chapter{Rellich Estimates}
\label{chap:4}

In this section we will establish Rellich type estimates for the Stokes resolvent problem which will be used to prove the invertibility of the operators $\pm(1/2)I + \K_{\lambda}$ and their adjoints from Theorems  \ref{thm:jumpConditions} and \ref{thm:nontangentialLimitDoubleLayer}.
We will for this entire section always assume that $\Omega$ is a bounded Lipschitz domain in $\R^d$, $d \geq 2$, with connected boundary.
Furthermore we will use the shorthand notation
\begin{align*}
  \| \cdot \|_{\partial} \coloneqq \| \cdot \|_{\Ell^2(\partial\Omega; \C^k)}, \quad k \in \N,
\end{align*}
and we will tacitly use the summation convention whenever it is applicable.

The following Theorem is the central result of this chapter:

\begin{thm}
  \label{thm:rellich}
  Let $\lambda \in \Sigma_\theta$ and $|\lambda| \geq \tau$, where 
  $\tau \in (0,1)$.
  Let $(u,\phi)$ be a smooth solution to the Stokes resolvent problem in $\Omega$ and suppose that $(\nabla u)^* \in \Ell^2(\partial\Omega)$ and $(\phi)^* \in \Ell^2(\partial\Omega)$.
  Furthermore, assume that $\nabla u$, $\phi$ have nontangential limits almost everywhere on $\partial\Omega$.
  Then
  \begin{align}
    \label{eq:rellich1}
    \begin{alignedat}{1}
      &\|\nabla u\|_\partial + \bigg\|\phi - \bigg\{ \frac{1}{{r_0}^{d - 1}}\int_{\partial\Omega} \phi \d\sigma \bigg\} \bigg\|_\partial \\[1em]
      &\hspace{2cm} \leq C\, \bigg\{ \|\nabla_{\mathrm{tan}} u\|_\partial + |\lambda|^{1/2} \|u\|_\partial + |\lambda| \, \|u \cdot n\|_{\HH^{-1}(\partial\Omega)} \bigg\}
    \end{alignedat}
  \end{align}
  and
  \begin{align}
    \label{eq:rellich2}
    \|\nabla u\|_\partial + |\lambda|^{1/2} \|u\|_\partial + |\lambda| \, \|u \cdot n\|_{\HH^{-1}(\partial\Omega)} + \|\phi\|_\partial
    \leq C \, \Big\|\frac{\partial u}{\partial \nu} \Big\|_\partial,
  \end{align}
  where $\frac{\partial}{\partial \nu}$ denotes the conormal derivative, and $C$ depends only on $d$, $\tau$, $\theta$ and the Lipschitz character of $\Omega$.
\end{thm}

\begin{rem}
  \label{rem:shenNontangential}
  The assumptions on $u$ in Theorem \ref{thm:rellich} are sufficient for $u$ to have a nontangential limit and a square integrable maximal function $(u)^*$. 
  Indeed for $d = 2$ we have $(u)^* \in \Ell^\infty(\partial\Omega)$, for $d = 3$ we have $(u)^* \in \Ell^p(\partial\Omega), p \in (1,\infty)$, and for $d \geq 3$ we have $(u)^* \in \Ell^p(\partial\Omega)$, $p \in \big(1, 2 (d - 1) / (d - 3) \big)$.
  A proof of these facts can be found in Shen's notes \cite[Prop. 7.1.3]{Shen2017}.
\end{rem}

We will now prepare the proof of Theorem \ref{thm:rellich} by proving several helpful lemmata.
The first lemma deals with so called \emph{Rellich identities} for solutions to the Stokes resolvent system.

\begin{lem}
  \label{lem:rellichIdentity}
  Under the same conditions on $(u,\phi)$ as in Theorem \ref{thm:rellich}, we have
  \begin{align}
    \int_{\partial\Omega} h_k n_k |\nabla u|^2 \d\sigma 
    &= 2 \Ret \int_{\partial\Omega} h_k \frac{\partial \overline u_i}{\partial x_k} \Big( \frac{\partial u}{\partial \nu} \Big)_i \d\sigma + \int_\Omega \div(h) |\nabla u|^2 \d x \nonumber\\[0.5em]
    &\quad - 2 \Ret \int_\Omega \frac{\partial h_k}{\partial x_j} \cdot \frac{\partial u_i}{\partial x_k} \cdot \frac{\partial \overline u_i}{\partial x_j} \d x + 2 \Ret \int_\Omega \frac{\partial h_k}{\partial x_i} \cdot \frac{\partial u_i}{\partial x_k} \, \overline \phi \d x \nonumber\\[0.5em]
    &\quad - 2 \Ret \int_\Omega h_k \frac{\partial u_i}{\partial x_k} \cdot \overline {\lambda u_i} \d x \label{eq:rellichIdentity}
  \end{align}
  and
  \begin{align}
    \int_{\partial\Omega} h_k n_k |\nabla u|^2 \d\sigma
    &= 2\Ret \int_{\partial\Omega} h_k \frac{\partial \overline u_i}{\partial x_j} \bigg\{ n_k \frac{\partial u_i}{\partial x_j} - n_j \frac{\partial u_i}{\partial x_k} \bigg\} \d\sigma \nonumber\\[0.5em]
    &\quad + 2 \Ret \int_{\partial\Omega} h_k \, \overline \phi \, \bigg\{ n_i \frac{\partial u_i}{\partial x_k} - n_k \frac{\partial u_i}{\partial x_i} \bigg\} \d\sigma - \int_\Omega \div(h) |\nabla u|^2 \d x \nonumber\\[0.5em]
    &\quad + 2 \Ret \int_\Omega \frac{\partial h_k}{\partial x_j} \cdot \frac{\partial u_i}{\partial x_k} \cdot \frac{\partial \overline u_i}{\partial x_j} \d x - 2 \Ret \int_\Omega \frac{\partial h_k}{\partial x_i} \cdot \frac{\partial u_i}{\partial x_k} \, \overline \phi \d x \nonumber\\[0.5em]
    &\quad + 2 \Ret \int_\Omega h_k \frac{\partial u_i}{\partial x_k} \cdot \overline \lambda \overline u_i \d x, \label{eq:rellichIdentity2}
  \end{align}
  where $h = (h_1, \dots,h_d) \in \CC_0^1(\R^d, \R^d)$.
\end{lem}

\begin{proof}
  The proof of the stated identities reduces to several applications of the divergence theorem once we establish its applicability.
  To this end, we want to make Proposition \ref{prop:approximationArgument} available. We note that the assumptions given in Theorem \ref{thm:rellich} are sufficient for this purpose and we will verify them, once they are used.

  Let's expand the first summand in \eqref{eq:rellichIdentity} using the definition of conormal derivatives
  \begin{align*}
    2\Ret \int_{\partial\Omega} h_k \frac{\partial \overline u_i}{\partial x_k} \bigg( \frac{\partial u}{\partial \nu} \bigg)_i \d\sigma
    &= 2 \Ret \int_{\partial\Omega} h_k \frac{\partial \overline u_i}{\partial x_k} \cdot \frac{\partial u_i}{\partial x_j}  n_j \d\sigma  - 2\Ret \int_{\partial\Omega} h_k \frac{\partial \overline u_i}{\partial x_k} \, \phi n_i \d x \\
    &\eqqcolon I_1 - I_2.
  \end{align*}
  The divergence theorem is applicable for $I_1$ as $h$ is bounded and defined everywhere and the integrand has nontangential limits that can be dominated by $|(\nabla u)^*|^2 \in \Ell^2(\partial\Omega)$.
  Therefore, we find using the divergence theorem and the product rule:
  \begin{align*}
    I_1
    &= 2\Ret \int_\Omega \frac{\partial}{\partial x_j} \bigg\{ h_k \frac{\partial \overline u_i}{\partial x_k} \cdot \frac{\partial u_i}{\partial x_j} \bigg\} \d x \\
    &= 2\Ret \int_\Omega \frac{\partial h_k}{\partial x_j} \cdot \frac{\partial \overline u_i}{\partial x_k} \cdot \frac{\partial u_i}{\partial x_j} \d x 
    + 2\Ret \int_{\Omega} h_k \frac{\partial^2 \overline u_i}{\partial x_j \partial x_k} \cdot \frac{\partial u_i}{\partial x_j} \d x  \\
    &\hphantom{= 2\Ret \int_\Omega \frac{\partial h_k}{\partial x_j} \cdot \frac{\partial \overline u_i}{\partial x_k} \cdot \frac{\partial u_i}{\partial x_j} \d x } \;
    + 2 \Ret \int_\Omega h_k \frac{\partial \overline u_i}{\partial x_k} \cdot \frac{\partial^2 u_i}{\partial x_j^2} \d x \\
    &\eqqcolon I_3 + I_4 + I_5.
  \end{align*}
  For $I_5$ we use the fact that $u$ solves the Stokes resolvent problem which gives
  \begin{align*}
     I_5 
     &= 2 \Ret \int_\Omega h_k \frac{\partial \overline u_i}{\partial x_k}\cdot  \frac{\partial \phi}{\partial x_i} \d x + 2\Ret \int_\Omega h_k \frac{\partial\overline u_i}{\partial x_k} \lambda u_i \d x \\
     &\eqqcolon I_6 + I_7 .
  \end{align*}
  Now we want to apply the divergence theorem, i.e. Proposition \ref{prop:approximationArgument} to integral $I_2$.
  This is possible since $h$ is defined everywhere and bounded, $(\partial_k u_i) \cdot \phi$ has a nontangential limit and can be bounded by $\big( |(\nabla u)^*| |(\phi)^*| \big)$ which is integrable due to Hölder's inequality as $(\nabla u)^*$ and $(\phi)^*$ are square integrable by assumption.
  Thus the divergence theorem is applicable and yields together with the product rule:
  \begin{align*}
    I_2
    &= 2 \Ret \int_\Omega \frac{\partial}{\partial x_i} \Big\{ h_k \frac{\partial \overline u_i}{\partial x_k} \phi \Big\} \d x \\
    &= 2 \Ret \int_\Omega \frac{\partial h_k}{\partial x_i} \cdot \frac{\partial \overline u_i}{\partial x_k} \, \phi \d x + 2 \Ret \int_\Omega h_k \frac{\partial^2 \overline u_i}{\partial x_j \partial x_k} \, \phi \d x + 2 \Ret \int_\Omega h_k \frac{\partial \overline u_i}{\partial x_k} \cdot \frac{\partial \phi}{\partial x_i} \d x\\
    &\eqqcolon I_8 + I_9 + I_6.
  \end{align*}
  One term that hasn't come up so far, the second summand of the right hand side in \eqref{eq:rellichIdentity}, will now be expanded:
  \begin{align*}
    \int_\Omega \div(h)\,  |\nabla u|^2 \d x
    &= \int_\Omega \div( h|\nabla u|^2) \d x - \int_\Omega h_k \frac{\partial}{\partial x_i} \Big\{ |\nabla u|^2 \Big\} \d x \\
    &\eqqcolon I_{10} - I_{11}.
  \end{align*}
  Expanding the Integral $I_{11}$ gives us the identity
  \begin{align*}
    I_{11}
    &= \int_\Omega h_i \frac{\partial}{\partial x_i} \bigg\{ \frac{\partial u_k}{\partial x_j} \cdot \frac{\partial \overline u_k}{\partial x_j} \bigg\} \d x
    = \int_\Omega h_i  \bigg\{ \frac{\partial^2 u_k}{\partial x_i \partial x_j} \cdot \frac{\partial \overline u_k}{\partial x_j} + \frac{\partial u_k}{\partial x_j} \cdot \frac{\partial^2 \overline u_k}{\partial x_i \partial x_j} \bigg\} \d x= I_4.
  \end{align*}
  If we now put everything together, the right hand side of \eqref{eq:rellichIdentity} reads
  \begin{align*}
    &(I_1 - I_2) + (I_{10} - I_{11}) - I_3 + I_8 - I_7 \\
    &\quad= (I_3 + I_4 + I_6 + I_7) - (I_8 + I_9 + I_6) + I_{10} - I_{11} - I_3 + I_8 - I_7 = I_{10}.
  \end{align*}
  Noting that by the divergence theorem, which is applicable with the same justification as for the integral $I_1$, we have
  \begin{align*}
    I_{10} = \int_{\partial\Omega} h_k n_k |\nabla u|^2 \d \sigma.
  \end{align*}
  Thus, the first identity is proven.

  In order to prove identity \eqref{eq:rellichIdentity2}, we show that the expression we get from considering (\eqref{eq:rellichIdentity} + \eqref{eq:rellichIdentity2}) holds, i.e. we show the identity
  \begin{align*}
    2 \int_{\partial\Omega} h_k n_k |\nabla u|^2 \d\sigma
    &= 2 \Ret \int_{\partial\Omega} h_k \frac{\partial \overline u_i}{\partial x_k} \bigg( \frac{\partial u}{\partial \nu} \bigg)_i \d \sigma \\
    &\quad + 2 \Ret \int_{\partial\Omega} h_k \frac{\partial\overline u_i}{\partial x_j} \bigg\{ n_k \frac{\partial u_i}{\partial x_j} - n_j \frac{\partial u_i}{\partial x_k} \bigg\} \d \sigma \\
    &\quad+ 2 \Ret \int_{\partial\Omega} h_k \, \overline \phi\,  \bigg\{ n_i \frac{\partial u_i}{\partial x_i} - n_k \frac{\partial u_i}{\partial x_i} \bigg\} \d\sigma.
  \end{align*}
  To this end, note that the left side of the identity equals $2 I_{10}$, whereas the right hand side can be written as 
  \begin{align*}
    (I_1 - I_2) + 2 (I_{10} - I_1) + (I_2 - 0),
  \end{align*}
  where we also used the fact that $\div u = \partial_i u_i = 0$.
\end{proof}

Consider the operators $\partial_{\tau_{jk}}$ which act on compactly supported continuously differentiable functions $\psi$ in the neighborhood of $\partial\Omega$ by
\begin{align}
  \label{eq:defnTangDerivative}
  \partial_{\tau_{jk}} \psi \coloneqq n_j \frac{\partial \psi}{\partial x_k} \bigg|_{\partial\Omega} - n_k \frac{\partial \psi}{\partial x_j} \bigg|_{\partial\Omega}, \quad j,k = 1,\dots,d.
\end{align}
They have been introduced by Mitrea and Wrigth \cite[p. 16]{mitreaWright} and come with a helpful  ``integration by parts'' rule that can be used to define Sobolev spaces on the boundary $\partial\Omega$. However for our purposes it will suffice to formulate this rule for the specific case

These operators show up in identity \eqref{eq:rellichIdentity2} as
\begin{align*}
  \partial_{\tau_{kj}} u_i = \Big\{ n_k \frac{\partial u_i}{\partial x_j} - n_j \frac{\partial u_i}{\partial x_k} \Big\}
  \quad\text{and}\quad
  \partial_{\tau_{ik}} u_i =  \Big\{ n_i \frac{\partial u_i}{\partial x_k} - n_k \frac{\partial u_i}{\partial x_i} \Big\}
\end{align*}

We make a quick detour that gives us the following basic lemma on elements of the sector $\Sigma_\theta$.
A powerful generalization of this Lemma can be found in Tolksdorf \cite[Lem. 5.2.4]{tolksdorf}.

\begin{lem}
  \label{lem:lambdaIneq}
  Let $\theta \in (0,\pi/2)$.
  Then there exists $\alpha$ depending only on $\theta$ such that for all $\lambda \in \Sigma_\theta$ the following inequality holds:
  \begin{align*}
    \Re(\lambda) + \alpha \, \big|\Im(\lambda)\,\big| \geq |\lambda|.
  \end{align*}
\end{lem}

\begin{proof}
  For the moment being, suppose $|\lambda| = 1$ Then, we have $\Re(\lambda) = \cos(\varphi)$ and $\Im(\lambda) = \sin(\varphi)$ with $|\varphi| \in (0, \pi - \theta)$. 
  Set
  \begin{align*}
    \alpha = \frac{1 - \cos(\pi - \theta)}{\sin(\pi - \theta)} \geq \frac{1 - \cos(|\varphi|)}{\sin(|\varphi|)}.
  \end{align*}
  If $\varphi = |\varphi|$, this gives the inequality
  \begin{align*}
    \Re(\lambda) + \alpha \, |\Im(\lambda)\,| = \cos(\varphi) + \alpha \sin(\varphi) \geq 1.
  \end{align*}
  Conversely, if $\varphi = -|\varphi|$, then we have by the symmetry properties of $\sin$ and $\cos$ that
  \begin{align*}
    \Re(\lambda) + \alpha\, |\Im(\lambda)\,| = \cos(-\varphi) + \alpha \sin(-\varphi) \geq 1.
  \end{align*}
  For arbitrary $\lambda$ the claim follows by considering the normalized value $(\lambda / |\lambda|)$.
\end{proof}

The next lemma enables us to handle the solid integrals in \eqref{eq:rellichIdentity} and \eqref{eq:rellichIdentity2}.

\begin{lem}
  \label{lem:laxMilgramIneq}
  Under the same assumptions on $(u,\phi)$ and $\lambda$ as in Theorem \ref{thm:rellich}, we have
  \begin{align}
    \label{eq:laxMilgramIneq}
    \int_\Omega |\nabla u|^2 \d x + |\lambda| \int_\Omega |u|^2 \leq C \,\Big\| \frac{\partial u}{\partial \nu} \Big\|_\partial  \|u\|_\partial,
  \end{align}
  where $C$ depends only on $\theta$.
\end{lem}

\begin{proof}
  Inserting the solution $u$ into the the variational problem of the Stokes resolvent problem gives us
  \begin{align}
    \label{eq:variationalStokes}
    \int_\Omega -\Delta u \cdot \overline u \d x + \lambda \int_\Omega u \cdot \overline u \d x= - \int_\Omega \nabla \phi \cdot \overline u \d x.
  \end{align}
  Rewriting the first term of equation \eqref{eq:variationalStokes} leads to 
  \begin{align*}
    -\int_{\Omega}  \frac{\partial^2 u_j}{\partial x_i \partial x_i}  \, \overline u_j \d x
    = -\int_{\Omega}  \frac{\partial }{\partial x_i} \bigg\{ \overline u_j \frac{\partial u_j}{\partial x_i} \bigg\} \d x + \int_\Omega \frac{\partial u_j}{\partial x_i} \cdot \frac{\partial \overline u_j}{\partial x_i} \d x.
    \end{align*}
  Note that since $u$ is solenoidal, we have for the third term of equation \eqref{eq:variationalStokes}
  \begin{align*}
    - \int_{\Omega} \frac{\partial \phi}{\partial x_i} \, \overline u_i \d x = - \int_{\Omega} \frac{\partial}{\partial x_i} \Big\{ \phi\,  \overline u_i \Big\} \d x.
  \end{align*}
  Now we want to transform the first and third of the above solid integrals into boundary integrals through Proposition \ref{prop:approximationArgument}.
  By the assumptions formulated in Theorem \ref{thm:rellich}, $\phi$ and $\nabla u$ have a nontangential limit and for both nontangential maximal functions the inclusion $(\phi)^*, (\nabla u)^* \in \Ell^2(\partial\Omega)$ holds. 
  Furthermore, according to Remark \ref{rem:shenNontangential}, also $u$ has a nontangential limit and the nontangential maximal function satisfies $(u)^* \in \Ell^2(\partial\Omega)$. 
  Therefore, the function $|\phi\,  \overline u_i |$ may be dominated by $|(\phi)^* (u)^*| \in \Ell^2(\partial\Omega)$ and the function $|(\partial_j u_i) u_i|$ may be dominated by $|(\nabla u)^* (u)^*|$, respectively. Thus, the door to Proposition \ref{prop:approximationArgument} has been opened which allows to transform equation \eqref{eq:variationalStokes} into
  \begin{align*}
    \int_\Omega |\nabla u|^2 \d x - \int_{\partial\Omega} \frac{\partial u}{\partial n} \cdot \overline u \d \sigma + \lambda \int_\Omega |u|^2 \d x = - \int_{\partial\Omega}  \phi n \cdot \overline u \, \d \sigma.
  \end{align*}
  We can rearrange the terms of this identity and use the definition of conormal derivatives, see equation \eqref{eq:conormalDerivative}, to derive
  \begin{align}
    \label{eq:testedStokes}
    \int_\Omega |\nabla u|^2 \d x + \lambda \int_\Omega |u|^2 \d x = \int_{\partial\Omega} \frac{\partial u}{\partial \nu} \cdot \overline u \, \d \sigma.
  \end{align}
  If we now take the real and imaginary part of \eqref{eq:testedStokes} and sum them up with the prefactor $\alpha(\theta) > 0$ from Lemma \ref{lem:lambdaIneq}, we get
  \begin{align*}
    \int_\Omega |\nabla u|^2 \d x + \Big\{ \Re(\lambda) + \alpha\, \big|\Im(\lambda)\,\big|\, \Big\} \int_\Omega |u|^2 \d x
    \leq (1 + \alpha) \, \bigg| \int_{\partial\Omega} \frac{\partial u}{\partial \nu} \cdot \bar u \d \sigma \bigg|\,.
  \end{align*}
  Lemma \ref{lem:lambdaIneq} now gives
  \begin{align*}
    \int_\Omega |\nabla u|^2 \d x + |\lambda| \int_\Omega |u|^2 \d x \leq C\,  \bigg| \int_{\partial\Omega} \frac{\partial u}{\partial \nu} \cdot \bar u \d \sigma \bigg|\,,
  \end{align*}
  with $C = (1 + \alpha)$ from which we readily derive estimate \eqref{eq:laxMilgramIneq} after applying the Cauchy-Schwartz inequality.
\end{proof}

The next lemma combines Rellich identities \eqref{eq:rellichIdentity} and \eqref{eq:rellichIdentity2} with estimate \eqref{eq:laxMilgramIneq}.

\begin{lem}
  Under the same assumptions on $(u,\phi)$ and $\lambda$ as in Theorem \ref{thm:rellich}, we have
  \begin{align}
    \label{eq:gradEstimateRellich}
    \| \nabla u\|_\partial \leq C_\varepsilon \,\Big\| \frac{\partial u}{\partial \nu} \Big\|_\partial + \varepsilon \Big\{ \|\nabla u\|_\partial + \|\phi\|_\partial + \| \, |\lambda|^{1/2} u\|_\partial \Big\}
  \end{align}
  and
  \begin{align}
    \label{eq:gradEstimateRellich2}
    \|\nabla u\|_\partial \leq C_\varepsilon \Big\{ \|\nabla_{\mathrm{tan}} u \|_\partial + \|\, |\lambda|^{1/2} u\|_\partial \Big\} + \varepsilon \{ \|\nabla u\|_\partial + \|\phi\|_\partial \}
  \end{align}
  for all $\varepsilon \in (0,1)$, where $C_\varepsilon$ depends only on $d$, $\theta$, $\tau$, $\varepsilon$ and the Lipschitz character of $\Omega$.
\end{lem}

\begin{proof}
  Let $h = (h_1, \dots, h_d) \in \CC_0^1(\R^d, \R^d)$ with $h_k n_k \geq c > 0$ on $\partial \Omega$ as given by Therorem \ref{thm:smoothApproximation} v). 
  The idea of the proof of the desired estimates \eqref{eq:gradEstimateRellich} and \eqref{eq:gradEstimateRellich2} is to first use the Rellich identities from Lemma \ref{lem:rellichIdentity} with this particular $h$ to estimate $\|\nabla u\|_\partial$ and then to bound the resulting right hand side by providing individual estimates.

  Before we start, note that we have $\Delta \phi = 0$ on the one hand and for the nontangential maximal function $(\phi)^* \in \Ell^2(\partial\Omega)$ on the other hand. According to Shen \cite[p. 410]{Shen2012}, a result from Dahlberg \cite{dahlberg77} gives the estimation
  \begin{align}
    \label{eq:dahlbergEstimate}
    \int_\Omega |\phi|^2 \d x \leq C \|(\phi)^* \|_\partial^2 \leq C \|\phi\|_\partial^2.
  \end{align}

  We will now prove the first estimate \eqref{eq:gradEstimateRellich}.
  In view of identity \eqref{eq:rellichIdentity}, we have
  \begin{align}
    \label{eq:normRellich}
    \begin{alignedat}{1}
    \|\nabla u\|_\partial^2
    &\leq C \, \bigg\{ \|\nabla u\|_\partial \, \Big\| \frac{\partial u}{\partial \nu} \Big\|_\partial + \int_\Omega |\nabla u|^2 \d x \\
    &\hspace{1.5cm}+ \int_\Omega |\nabla u| \, |\phi| \d x + |\lambda| \int_\Omega |\nabla u|\, |u| \d x \bigg\},
    \end{alignedat}
  \end{align}
  where the first term follows from the Cauchy-Schwartz inequality and $C$ only depends on $d$ and the Lipschitz character of $\Omega$.

  For now, we keep the first term of \eqref{eq:normRellich} as it is, the second term can be handled via Lemma \ref{lem:laxMilgramIneq}. 
  The goal for the remaining two integrals will be to bound each of them by a product of norms $\|\cdot\|_\partial$.
To this end, for the third integral we calculate
  \begin{align}
    \int_\Omega |\nabla u| |\phi| \d x
    \leq \Big( \int_\Omega |\nabla u|^2 \d x \Big)^{1/2} \Big( \int_\Omega |\phi|^2 \d x \Big)^{1/2}
    \leq C \, \Big\| \frac{\partial u}{\partial \nu} \Big\|_\partial^{1/2} \|u\|_\partial^{1/2} \, \|\phi\|_\partial\,, \label{eq:nablaPhi}
  \end{align}
  where the first step is due to the Cauchy-Schwartz inequality and the second step combines estimate \eqref{eq:laxMilgramIneq} with estimate \eqref{eq:dahlbergEstimate}.

  The last integral of \eqref{eq:normRellich} can be estimated as follows:
  \begin{align}
    |\lambda| \int_\Omega |\nabla u| \, |u| \d x 
    &\leq  \frac{|\lambda|^{3/2}}{2} \int_\Omega |u|^2 \d x + \frac{|\lambda|^{1/2}}{2 } \int_\Omega |\nabla u|^2 \d x \nonumber\\\label{eq:lambdaNablaU}
    &\leq C \, \Big\| \frac{\partial u}{\partial \nu} \Big\|_\partial \big\|\, |\lambda|^{1/2} u\,\big\|_\partial, 
  \end{align}
  where in the first step we used the weighted Young inequality and in the second step we applied estimate \eqref{eq:laxMilgramIneq}.
  Putting everything together, we calculate
  \begin{align*}
    \| \nabla u\|_\partial^2 
    &\leq C \bigg\{ \| \nabla u\|_\partial \, \Big\|\frac{\partial u}{\partial \nu} \Big\|_\partial 
    + \Big\| \frac{\partial u}{\partial \nu} \Big\|_\partial \|u\|_\partial 
    + \Big\| \frac{\partial u}{\partial \nu} \Big\|_\partial^{1/2} \|u\|_\partial^{1/2} \|\phi\|_\partial 
    + \Big\| \frac{\partial u}{\partial \nu} \Big\|_\partial \big\| \, |\lambda|^{1/2} u\, \big\|_\partial\bigg\}
  \end{align*}
  If we now use the assumption $|\lambda| \geq \tau$ which allows us to bound $\|u\|_\partial$ via
  \begin{align*}
    \|u\|_\partial \leq \frac{|\lambda|^{1/2}}{\tau^{1/2}} \|u\|_\partial = C |\lambda|^{1/2} \|u\|_{\partial},
  \end{align*}
  the desired estimate \eqref{eq:gradEstimateRellich} now follows applying Young's weighted inequality with an $\varepsilon$ and the norm equivalence on finite dimensional vector spaces. 
  Note that for the product of three norms from inequality \eqref{eq:nablaPhi} we need to apply the Young inequality twice:
  \begin{align*}
    \Big\| \frac{\partial u}{\partial \nu} \Big\|_\partial^{1/2} \|u\|_\partial^{1/2} \|\phi\|_\partial 
    &\leq \Big\{\, \frac{1}{4 \varepsilon} \Big\| \frac{\partial u}{\partial \nu} \Big\|_\partial 
  + \varepsilon \|u\|_\partial \Big\} \, \|\phi\|_\partial  \\[0.5em]
  &\leq \frac{1}{32\, \varepsilon^3} \Big\| \frac{\partial u}{\partial \nu} \Big\|_\partial^2 + \frac{\varepsilon}{2} \|\phi\|_\partial^2 + \frac{1}{2} \|u\|_\partial^2 + \frac{\varepsilon^2}{2} \|\phi\|_\partial^2 \\[0.5em]
  &\leq C_\varepsilon \Big\{ \; \Big\|\frac{\partial u}{\partial\nu} \Big\|_\partial^2 + \|u\|_\partial^2 \Big\} + \varepsilon \|\phi\|_\partial^2,
  \end{align*}
  where for the last inequality we used the fact that $\varepsilon < 1$.

  For inequality \eqref{eq:gradEstimateRellich2}, we use the Rellich identity \eqref{eq:rellichIdentity2} to obtain the estimate
  \begin{align}
    \label{eq:onTheWay}
    \begin{alignedat}{1}
    \| \nabla u\|_\partial^2 
    &\leq C \bigg\{ \|\nabla_{\mathrm{tan}} u\|_\partial 
    \Big\{ \|\nabla u\|_\partial + \|\phi\|_\partial \Big\}   \\
    &\hspace{2cm} + \int_\Omega |\nabla u|^2 \d x + \int_\Omega |\nabla u| |\phi| \d x + |\lambda| \int_\Omega |\nabla u| |u|\d x \bigg\}.
  \end{alignedat}
  \end{align}
  As before we estimate the three terms on the right side of \eqref{eq:onTheWay} using \eqref{eq:laxMilgramIneq}, \eqref{eq:nablaPhi} and \eqref{eq:lambdaNablaU}, respectively and obtain the estimate
  \begin{align*}
    \| \nabla u\|_\partial^2 
    &\leq C \bigg\{ \|\nabla_{\mathrm{tan}} u\|_\partial 
    \Big\{ \|\nabla u\|_\partial + \|\phi\|_\partial \Big\}   \\
    &\hspace{2cm} + \Big\| \frac{\partial u}{\partial \nu} \Big\|_\partial \| u\|_\partial
    + \Big\| \frac{\partial u}{\partial \nu} \Big\|_\partial^{1/2} \|u\|_\partial^{1/2} \|\phi\|_\partial + \Big\| \frac{\partial u}{\partial \nu} \Big\|_\partial \big\|\, |\lambda|^{1/2} u \, \big\|_\partial \bigg\}.
  \end{align*}
  If we now use the Young inequality with an $\varepsilon$, we get
  \begin{align*}
    \|\nabla u\|_\partial^2 
    &\leq C_\varepsilon \big\{ \, \|\nabla_{\mathrm{tan}} u\|_\partial^2 + \| |\lambda|^{1/2} u\|_\partial^2 \, \big\}  + \varepsilon \bigg\{ \, \|\nabla u\|_\partial^2 + \|\phi\|_\partial^2 + \frac{1}{4}\Big\|\frac{\partial u}{\partial \nu} \Big\|_\partial^2 \, \bigg\}.
  \end{align*}
  The claim now follows if we use the definition of the conormal derivative and the norm equivalence on finite dimensional vector spaces.
\end{proof}

We prove one last lemma befor we tackle the central theorem of this chapter.

\begin{lem}
  Assume that $(u,\phi)$ satisfies the same conditions as in Theorem \ref{thm:rellich}.
  Then
  \begin{align}
    \label{eq:phiDashintPhi}
    \| \phi - \dashint_{\partial\Omega} \phi \|_\partial \leq C \{ \|\nabla u\|_\partial + |\lambda| \|u\cdot n\|_{\HH^{-1}(\partial\Omega}
  \end{align}
  and
  \begin{align}
    \label{eq:lambdaun}
    |\lambda| \| u\cdot n \|_{\HH^{-1}(\partial\Omega)} \leq C \{ \|\phi\|_\partial + \|\nabla u\|_\partial,
  \end{align}
  where $C$ depends only on $d$ and the Lipschitz character of $\Omega$.
\end{lem}

\begin{proof}
  By Theorem \ref{thm:smoothApproximation} we may assume that $\Delta u = \nabla \phi + \lambda u$ on $\partial\Omega$.
  Multiplying the Stokes resolvent equation on $\partial\Omega$ with $n$ and using the triangle inequality gives
  \begin{align}
    \|\nabla \phi \cdot n\|_{\HH^{-1}(\partial\Omega)} 
    &\leq \|\Delta u \cdot n \|_{\HH^{-1}(\partial\Omega)} + |\lambda| \| u\cdot n\|_{\HH^{-1}(\partial\Omega)}, \nonumber\\
    \label{eq:stokesEquationH1}
    |\lambda| \| u\cdot n \|_{\HH^{-1}(\partial\Omega)} 
    &\leq \|\Delta u \cdot n \|_{\HH^{-1}(\partial\Omega)} + \|\nabla \phi \cdot n \|_{\HH^{-1}(\partial\Omega)}.
  \end{align}
  We will now show that
  \begin{align}
    \label{eq:deltaun}
    \|\Delta u \cdot n\|_{\HH^{-1}(\partial\Omega)}
    \leq C \|\nabla u\|_\partial
  \end{align}
  and 
  \begin{align}
    \label{eq:nablaPhin}
    c \| \phi - \dashint_{\partial\Omega} \phi \d\sigma \|_\partial
    \leq \|\nabla \phi \cdot n\|_{\HH^{-1}(\partial\Omega)}
    \leq C \|\phi\|_\partial
  \end{align}
  Using these two estimates applied to \eqref{eq:stokesEquationH1}, we can directly derive \eqref{eq:phiDashintPhi} and \eqref{eq:lambdaun}.

  In order to prove \eqref{eq:deltaun}, note that
  \begin{align*}
    \Delta u \cdot n = n_i \frac{\partial^2 u_i}{\partial x_j^2} = \Big( n_i \frac{\partial}{\partial x_j} - n_j \frac{\partial}{\partial x_i} \Big) \frac{\partial u_i}{\partial x_j}
  \end{align*}
  since $\div u = 0$ in $\overline \Omega$.
  As the expression in between the brackets is a tangential derivative we derive estimate \eqref{eq:deltaun} from
  \begin{align*}
    | \langle \Delta u \cdot n, u \rangle | = | \langle \nabla u, \nabla_{\mathrm{tan}} u \rangle| \leq \|\nabla u\|_\partial^2
  \end{align*}
  since this implies 
  \begin{align*}
    \|\nabla u \cdot  n\|_{\HH^{-1}(\partial\Omega)} \leq \|\nabla u\|_\partial.
  \end{align*}

  Now for the proof of estimate \eqref{eq:nablaPhin} we will use $\Ell^2$-estimates for the Neumann and regularity problems for the Laplace equation in Lipschitz domains.
  For $g \in \Ell^2(\partial\Omega)$ with mean value zero, by Jerison and Kenig \cite{jerisonKenig} the Neumann problem for Laplace's equation on the Lipschitz domain $\Omega$ has a solution $\psi$ with  $(\nabla \psi)^* \in \Ell^2(\partial\Omega)$ and $\frac{\partial \psi}{\partial n} = g$ on $\partial \Omega$.
  Green's identity we have that since $\phi$ and $\psi$ are harmonic
  \begin{align}
    \big| \int_{\partial\Omega} \phi g \d \sigma \big|
    &=  \big| \int_{\partial\Omega} \phi \frac{\partial \psi}{\partial n} \d \sigma \big|
    = \big| \int_{\partial\Omega} \frac{\partial \phi}{\partial n} \psi \d \sigma \big|\nonumber \\
    \label{eq:dualityPhi}
    &\leq \| \frac{\partial \phi}{\partial n} \|_{\HH^{-1}(\partial\Omega)} \| \psi \|_{\HH^1(\partial\Omega)} \leq C \| \frac{\partial \phi}{\partial n} \|_{\HH^{-1}(\partial\Omega)} \| g\|_\partial ,
  \end{align}
  where in the last step we used the estimate $\|\psi\|_{\HH^1(\partial\Omega)} \leq C \|g\|_\partial$ for the $\Ell^2$ Neumann problem which can be found in Jerison and Kenig \cite{jerisonKenig}.
  Now if we set $\bar g = \phi - \tilde \phi$, with $\tilde \phi = \dashint_{\partial\Omega} \phi \d \sigma$ and use that $\int_{\partial\Omega} (\phi - \tilde \phi) \overline{(\phi - \tilde\phi)} \d\sigma = \int_{\partial\Omega} \phi \overline{(\phi - \tilde\phi)} \d\sigma$, we get from \eqref{eq:dualityPhi}
  \begin{align*}
    \| \phi - \tilde\phi \|_\partial^2
    \leq C \|\frac{\partial\phi}{\partial n} \|_{\HH^{-1}(\partial\Omega)} \|\phi - \tilde \phi\|_\partial
  \end{align*}
  or, after rearranging and expanding
  \begin{align*}
    \| \phi - \dashint_{\partial\Omega} \phi \d\sigma \|_\partial \leq C \|\frac{\partial\phi}{\partial n} \|_{\HH^{-1}(\partial\Omega)}
  \end{align*}
  We work in a similar way with results from the regularity problem of Laplace's equation by Jerison and Kenig \cite{jerisonKenig2}.
  Given $f \in \HH^1(\partial\Omega)$, there exists a harmonic function $\psi$ in $\Omega$ such that $(\nabla\psi)^* \in \Ell^2(\partial\Omega)$ and $\psi = f$ on $\partial\Omega$.
  As for \eqref{eq:dualityPhi}, we calculate
  \begin{align*}
    \big| \int_{\partial\Omega} \frac{\partial\phi} f \d\sigma \big|
    &= \big| \int_{\partial\Omega} \frac{\partial\phi} \psi \d\sigma \big|
    = \big| \int_{\partial\Omega} \phi \frac{\partial \psi}{\partial n} \d \sigma \big| \\
    &\leq \|\phi\|_{\partial} \|\nabla \psi \|_\partial 
    \leq C \|\phi\|_\partial \|f\|_{\HH^1(\partial\Omega)},
  \end{align*}
  where in the last step we used the estimate $\|\nabla \psi\|_\partial \leq C \|f\|_{\HH^1(\partial\Omega)}$ for the $\Ell^2$ regularity problem.
  By duality this gives that
  \begin{align*}
    \|\frac{\partial\phi}{\partial n} \|_{\HH^{-1}(\partial\Omega)} \leq C \|\phi\|_\partial. 
  \end{align*}
\end{proof}

\begin{rem}
  \label{rem:harmonicEstimate}
  A careful look at the proof of inequality \eqref{eq:nablaPhin} reveals that the estimate
  \begin{align*}
    c \| \phi \|_\partial \leq \| \nabla \phi \cdot n \|_{\HH^{-1}(\partial\Omega)},
  \end{align*}
  holds for all harmonic functions $\phi$ with vanishing mean on $\partial\Omega$.
\end{rem}

After all this preparation we are now able to prove Theorem \ref{thm:rellich}.

\begin{proof}[Proof of Theorem \ref{thm:rellich}]
  For the proof of estimate \eqref{eq:rellich1}, without loss of generality we can assume that $\int_{\partial\Omega} \phi \d \sigma = 0$.

  Using \eqref{eq:phiDashintPhi} for the second summand  in \eqref{eq:rellich1} and and then \eqref{eq:gradEstimateRellich2} for the terms involving $\nabla u$ we get
  \begin{align*}
    \|\nabla u\|_\partial + \|\phi\|_\partial
    &\leq C \{ \|\nabla u\|_\partial + |\lambda| \|u \cdot n \|_{\HH^1(\partial\Omega)} \} \\
    &\leq C_\varepsilon \Big\{ \|\nabla_{\mathrm{tan}} u \|_\partial + |\lambda|^{1/2} \|u\|_\partial + |\lambda| \| u\cdot n\|_{\HH^{-1}(\partial\Omega)} \Big\} \\
    &\quad + C \varepsilon \{ \|\nabla u\|_\partial + \|\phi\|_\partial \}
  \end{align*}
  for all $\varepsilon \in (0,1)$.
  Chosing $\varepsilon$ such that $C \varepsilon < (1/2)$ we can rearrange the above inequality and obtain estimate \eqref{eq:rellich1}.

  Estimate \eqref{eq:rellich2} will need more effort to be proven.
  We start with inequality \eqref{eq:lambdaun} and derive
  \begin{align*}
    \|\nabla u\|_\partial + \|\phi\|_\partial + |\lambda| \|u \cdot n\|_{\HH^{-1}(\partial\Omega)}
    \leq C \{ \|\nabla u\|_\partial + \| \phi\|_\partial \}
    \leq C \Big\{ \|\frac{\partial u}{\partial \nu} \|_\partial + \|\nabla u\|_\partial \Big\},
  \end{align*}
  where in the last step we used the definition of conormal derivatives.
  If we now apply \eqref{eq:gradEstimateRellich} we get
  \begin{align*}
    \|\nabla u\|_\partial + \|\phi\|_\partial + |\lambda| \|u \cdot n\|_{\HH^{-1}(\partial\Omega)}
    \leq C_\varepsilon \| \frac{\partial u}{\partial \nu} \|_\partial + \varepsilon \big\{ \|\nabla u\|_\partial + \|\phi\|_\partial + \| |\lambda|^{1/2} u \|_\partial \big\}
  \end{align*}
  for all $\varepsilon \in (0,1)$.
  Choosing $\varepsilon$ appropriately yields
  \begin{align}
    \label{eq:partOfRellich2}
    \|\nabla u\|_\partial + \|\phi\|_\partial + |\lambda| \|u \cdot n\|_{\HH^{-1}(\partial\Omega)}
    \leq C \|\frac{\partial u}{\partial \nu} \|_\partial + C |\lambda|^{1/2} \| u\|_\partial.
  \end{align}
  Now we need to estimate $|\lambda|^{1/2} \|u\|_\partial$.
  Green's identity yields
  \begin{align}
    \int_{\partial\Omega} h_k n_k |u|^2 \d \sigma
    = \int_{\Omega} \frac{\partial}{\partial x_k} \big( h_k |u|^2 ) \d x
    &= \int_{\Omega} \frac{\partial h_k}{\partial x_k} |u|^2 \d x + \int_{\Omega} h_k \frac{\partial |u|^2}{\partial x_k}  \d x \nonumber\\
    \label{eq:hknkgreen}
    &= \int_\Omega \div(h) |u|^2 \d x + 2 \Re \int_\Omega h_k \frac{\partial \bar u_i}{\partial x_k} u_i \d x.
  \end{align}
  We choose $h \in \CC_0^1(\R^d, \R^d)$ with $h_k n_k \geq c > 0$ on $\partial\Omega$. 
  The existence of such a function $h$ follows from Theorem \ref{thm:smoothApproximation}.
  Using this, we can continue the estimate \eqref{eq:hknkgreen} as
  \begin{align}
    \label{eq:estupartial}
    \|u\|_\partial^2 \leq C \int_\Omega |u|^2 \d x + C \int_\Omega |u| |\nabla u| \d x.
  \end{align}
  The next estimate uses \eqref{eq:estupartial} and \eqref{eq:laxMilgramIneq} which gives
  \begin{align*}
    |\lambda| \|u\|_\partial^2 
    &\leq |\lambda| C \int_\Omega |u|^2 \d x + |\lambda| C \int_\Omega |u| |\nabla u| \d x \nonumber\\
    &\leq C \| \frac{\partial u}{\partial \nu} \|_\partial \|u\|_\partial + |\lambda|^{1/2} C \int_\Omega (|\lambda|^{1/2} |u|) |\nabla u| \d x \\
    &\leq C \|\frac{\partial u}{\partial \nu} \|_\partial \|u\|_\partial + |\lambda|^{1/2} C \big( \int_\Omega |\lambda| |u|^2 \big)^{1/2} \big( \int_\Omega (|\nabla u|^2 \d x \big)^{1/2} \\
    &\leq C \|\frac{\partial u}{\partial \nu} \|_\partial \||\lambda|^{1/2} u\|_\partial.
  \end{align*}
  Note that for the last estimate we also used the fact that $|\lambda| \geq \tau$ helps us to bound $\|u\|_\partial$ by $C |\lambda|^{1/2} \|u\|_\partial$.
  Rearranging terms in the last estimate, we now derive
  \begin{align}
    \label{eq:lambda12u}
    \| |\lambda|^{1/2} u\|_\partial \leq C \| \frac{\partial u}{\partial \nu} \|_\partial.
  \end{align}
  Estimate \eqref{eq:rellich2} follows directly from \eqref{eq:partOfRellich2} in combination with \eqref{eq:lambda12u} and this concludes our proof.
\end{proof}

Shen proved that under reasonable assumptions a theorem similar to \ref{thm:rellich} also holds for exterior domains.
%It is important to note that in the case $d = 2$ solutions $u$ that are given as a single layer potential do not fulfill the stated requirements on the decay.

\begin{thm}
  \label{thm:rellichExterior}
  Let $\lambda \in \Sigma_\theta$ and $|\lambda| \geq \tau$, where $\tau \in (0,1)$.
  Let $(u,\phi)$ be a solution of the Stokes resolvent Problem in $\Omega_- = \R^d \setminus \overline\Omega$.
  Suppose additionally that $(\nabla u)^*$, $(\phi)^* \in \Ell^2(\partial\Omega)$ and that $\nabla u$, $\phi$ have nontangential limits almost everywhere on $\partial\Omega$.
  Furthermore let for $|x| \to \infty$
  \begin{align*}
    |\phi(x)| + |\nabla u(x)| = O(|x|^{1 - d}) \quad\text{and}\quad 
    u(x) = \begin{cases} O(|x|^{2 - d}) &\quad\text{if } d \geq 3 \\ o(1) &\quad\text{if } d = 2. \end{cases}
  \end{align*}
  Then
  \begin{align}
    \label{eq:rellich1ext}
    \|\nabla u\|_\partial + \|\phi\|_\partial
    \leq C \big\{ \|\nabla_{\mathrm{tan}} u\|_\partial + |\lambda|^{1/2} \|u\|_\partial + |\lambda| \|u \cdot n\|_{\HH^{-1}(\partial\Omega)} \big\}
  \end{align}
  and
  \begin{align}
    \label{eq:rellich2ext}
    \|\nabla u\|_\partial + |\lambda|^{1/2} \|u\|_\partial + |\lambda| \|u \cdot n\|_{\HH^{-1}(\partial\Omega)} + \|\phi\|_\partial
    \leq C \|\frac{\partial u}{\partial \nu} \|_\partial,
  \end{align}
  where $C$ depends only on $d$, $\tau$, $\theta$ and the Lipschitz character of $\Omega$.
\end{thm}

\chapter{Solving the $\Ell^2$-Dirichlet Problem}

This section is all about the application of the method of layer potentials to solve the $\Ell^2$ Dirichlet problem for the Stokes resolvent system.
Furthermore we will establish a uniform $\Ell^p$ estimate for the nontangential-maximal-function which will be important for the proof of our central theorem.

For the remainder of this chapter let $\Omega$ always denote a bounded Lipschitz domain in $\R^d$, $d \geq 3$ with connected boundary.
We will use $\Ell^2_n(\partial\Omega)$ to denote the function space
\begin{align*}
  \Ell^2_n(\partial\Omega) \coloneqq \Big\{ f \in \Ell^2(\partial\Omega; \C^d) \colon \int_{\partial\Omega} f \cdot n \d \sigma = 0 \Big\},
\end{align*}
and $\Ell_0^2(\partial\Omega; \C^d)$  to denote the function space of $\Ell^2$ functions with mean value zero.
As before $\| \cdot \|_\partial$ stands for the norm of $\Ell^2(\partial\Omega)$.

We will first derive bounds on the inverse operator of $(1/2) I + \K_\lambda$ from Chapter 3.

\begin{lem}
  Let $\lambda \in \Sigma_\theta$ and $|\lambda| \geq \tau$, where $\tau \in (0,1)$.
  Suppose that $|\partial\Omega| = 1$.
  Then $(1/2)I + \K_\lambda$ is an isomorphism on $\Ell^2(\partial\Omega; \C^d)$ and
  \begin{align}
    \label{eq:inverseEstimate}
    \| f\|_\partial \leq C \|\big( (1/2) I + \K_\lambda \big) f \|_\partial \quad\text{for any } f \in \Ell^2(\partial\Omega; \C^d),
  \end{align}
  where $C$ depends only on $d$, $\theta$, $\tau$ and the Lipschitz character of $\Omega$.
\end{lem}

\begin{proof}
  We start with $f \in \Ell^2(\partial\Omega; \C^d)$ and the corresponding single layer potentials $u = \slp_\lambda(f)$ and $\phi = \slp_\Phi(f)$ given by \eqref{eq:defSingleLayer} and \eqref{eq:defSingleLayerPressure}.
  We saw in Chapter 3 that $(u,\phi)$ solves the Stokes resolvent problem in $\R^d \setminus \partial\Omega$ and got from Lemma \ref{lem:nontantentialMaximalFunctions} with $p = 2$ for the nontangential maximal functions that $(\nabla u)^*$, $(\phi)^* \in \Ell^2(\partial\Omega)$.
  We furthermore saw in Lemma \ref{lem:traceFormulas} that $\nabla u$ and $\phi$ have nontangential limits almost everywhere on $\partial\Omega$. 
  Finally in Theorem \ref{thm:jumpConditions} we saw that $\nabla_{\mathrm{tan}} u_+ = \nabla_{\mathrm{tan}} u_-$ and derived the jump condition $\big(\frac{\partial u}{\partial \nu}\big)_\pm = (\pm (1/2) I + \K_\lambda) f$.

  Our next step will be to show the estimate
  \begin{align}
    \label{eq:negNablaPhi}
    \| \nabla u_- \|_\partial + \| \phi_- \|_\partial \leq C \| \big( \frac{\partial u}{\partial \nu} \big) \|_\partial .
  \end{align}
  Assuming that \eqref{eq:negNablaPhi} holds we can prove \eqref{eq:inverseEstimate}:
  Set $f = \big(\frac{\partial u}{\partial \nu} \big)_+ -  \big( \frac{\partial u}{\partial \nu} \big)_-$.
  Then this gives with the definition of the conormal derivative and estimate \eqref{eq:negNablaPhi} that
  \begin{align*}
    \| f\|_\partial 
    &\leq  \| \big( \frac{\partial u}{\partial \nu} \big)_+ \|_\partial + \| \big( \frac{\partial u}{\partial \nu} \big)_- \|_\partial  \\
    &\leq \| \big( \frac{\partial u}{\partial \nu} \big)_+  \|_\partial + \| \big( \frac{\partial u}{\partial n} \big)_- \|_\partial + \| \phi_- n \|_\partial  \\
    &\leq C \| \big( \frac{\partial u}{\partial \nu} \big)_+ \|_\partial
    = C \| (1/2)I + K_\lambda) f\|_\partial.
  \end{align*}

  In order to prove \eqref{eq:negNablaPhi}, note that since $|u(x)| + |\nabla u(x)| = O(|x|^{-N})$ for all $N > 0$ and $\phi(x) = O(|x|^{1 - d})$ as $|x| \to \infty$ we can use Theorem \ref{thm:rellichExterior} to derive
  \begin{align}
    \|\nabla u_-\|_\partial + \|\phi_-\|_\partial
    &\leq C \Big\{ \|\nabla_{\mathrm{tan}} u_- \|_\partial + |\lambda|^{1/2} \| u_-\|_\partial + |\lambda| \| n \cdot u_-\|_{\HH^{-1}(\partial\Omega)} \Big\} \nonumber\\
    \label{eq:nablauminus}
    &= C \Big\{ \|\nabla_{\mathrm{tan}} u_+ \|_\partial + |\lambda|^{1/2} \| u_+ \|_\partial + |\lambda| \|n \cdot u_+ \|_{\HH^{-1}(\partial\Omega)} \Big\},
  \end{align}
  where we used the fact that $u_+ = u_-$ and $\nabla_{\mathrm{tan}} u_+ = \nabla_{\mathrm{tan}} u_-$ on $\partial\Omega$.
  Inequality \eqref{eq:rellich2} of Theorem \ref{thm:rellich} now allows us to estimate the right hand side of \eqref{nablauminus} by $C \|(\frac{\partial u}{\partial \nu} )_+ \|_\partial$ and thus the desired estimate \eqref{eq:neqNablaPhi} follows.

  Let's now work on the invertibility of $(1/2) I + \K_\lambda$. In the case $\lambda = 0$, Fabes, Kenig and Verchota showed in \cite{fabesKenigVerchota} that $(1/2) I + \K_0$ as an operator on $\Ell^2(\partial\Omega; \R^d)$ has a one dimensional null space and as range the space $\Ell_0^2(\partial\Omega; \R^d)$.
  Thus $(1/2) I + \K_0$ has Fredholm index $0$.
  This remains true if we replace $\Ell^2(\partial\Omega; \R^d)$ by $\Ell^2(\partial\Omega; \C^d)$ as this just corresponds to a complexification of the vector space and the operator. 
  Since the operator $\K_\lambda - \K_0$ is compact on $\Ell^2(\partial\Omega; \C^d)$ (see Toksdorf \cite{tolksdorfDiss}) we deduce that the operator
  \begin{align*}
    (1/2)I + \K_\lambda = (1/2)I + \K_0 + (\K_\lambda - \K_0)
  \end{align*}
  has the Fredholm index zero as well for all $\lambda \in \Sigma_\theta$.
  Now inequality \eqref{eq:inverseEstimate} gives that $(1/2)I + \K_\lambda$ is injective and thus the Fredholm index of zero implies that it is also surjective and hence an isomorphism.
\end{proof}

The next lemma works with the counterpart of $(1/2)I + \K_\lambda$.

\begin{lem}
  Let $\lambda \in \Sigma_\theta$.
  Then $-(1/2)I + \K_\lambda$ is a Fredholm operator on $\Ell^2(\partial\Omega; \C^d)$ with index zero and
  \begin{align}
    \label{eq:inverseEstimate2}
    \| f\|_\partial \leq C \| \big( - (1/2) I + \K_\lambda \big) f \|_\partial \quad\text{for all } f \in \Ell^2_n(\partial\Omega).
  \end{align}
\end{lem}

\begin{proof}
  In the case $\lambda = 0$, Fabes Kenig and Verchota proved in \cite{fabesKenigVerchota} that the Fredholm index of the operator $-(1/2)I + \K_0$ on $\Ell^2(\partial\Omega; \R^d)$ is zero and estimate \eqref{eq:inverseEstimate2} holds.
  As in the previous proof, this still remains true if we complexify the operator making it a Fredholm operator with index zero on  $\Ell^2(\partial\Omega; \C^d)$.
  Since $\K_\lambda - \K_0$ is compact on $\Ell^2(\partial\Omega; \C^d)$ and the Fredholm index remains unchanges under compact perturbations, we know that the Fredholm index of $-(1/2)I + \K_\lambda$ on $\Ell^2(\partial\Omega; \C^d)$ is zero for all $\lambda \in \Sigma_\theta$.
  This proves the first claim of the lemma.

  Now let $\tau < \frac{1}{2 \diam(\Omega)^2 + 1}$ and $|\lambda| < \tau$.
  Then
  \begin{align*}
    \| (\K_\lambda - \K_0 f\|_\partial \leq C |\lambda|^{1/2} \|f\|_\partial.
  \end{align*}
  In order to prove this inequality we once again apply Young's inequality, i.e. we start by estimating
  \begin{align*}
    \| (\K_\lambda - \K_0 f\|_\partial \leq \sup_{p \in \partial\Omega} \| \nabla_x \{ \Gamma(p - \cdot; \lambda) - \Gamma(p - \cdot; 0) \} \|_{\Ell^1(\partial\Omega)} \| f\|_{\Ell^2(\partial\Omega)}.
  \end{align*}
  In the next step we prove that for $p \in \partial\Omega$ the integral over the gradients of $\Gamma$ can be estimated independent of $p$.
  This is straightforward using Lemma \ref{lem:central} as Corollary \ref{cor:differenceFundamentalSolutionStokes} gives us
  \begin{align*}
    &\int_{\partial\Omega} | \nabla_x \{ \Gamma(p - y; \lambda) - \Gamma(p - y; 0) \} | \d\sigma(y)\\
    &\quad\leq C |\lambda|^{1/2} \int_{\partial\Omega} \frac{1}{|p - y|^{d - 2}} \d \sigma(y) \\
    &\quad= |\lambda|^{1/2} C \int_{\partial\Omega \cap \BB(p, r_0/4)} \frac{1}{|p - y|^{d - 2}} \d\sigma(y) + |\lambda|^{1/2} C \int_{\partial\Omega \setminus \BB(p, r_0/4)} \frac{1}{|p - y|^{d - 2}} \d\sigma(y) \\
    &\quad\leq C |\lambda|^{1/2} ( r_0/4 + 4^{2 - d} r_0^{d - 2} |\partial\Omega| ),
  \end{align*}
  where $r_0$ is the radius from the definition of Lipschitz domains. 
  Note that by the choice of $\tau$ the estimate from Corollary \ref{cor:differenceFundamentalSolutionStokes} applies on the whole domain of integration.

  For $f \in \Ell^2_n(\partial\Omega)$ we can now estimate
  \begin{align*}
    \| f\|_\partial 
    &\leq C \|(-(1/2)I + \K_0) f \|_\partial  \\
    &\leq C \|(-(1/2)I + \K_\lambda)f \|_\partial + \|(\K_\lambda - \K_0)f\|_\partial \\
    &\leq C \|(-(1/2)I + \K_\lambda)f\|_\partial + C |\lambda|^{1/2} \|f\|_\partial,
  \end{align*}
  with a constant $C$ depending only on $d$, $\theta$ and the Lipschitz character of $\partial\Omega$.
  Choosing $\tau$ even smaller allows us to rearrange the terms in the above estimate such that estimate \eqref{eq:inverseEstimate2} holds for $\lambda \in \Sigma_\theta$ and $|\lambda| < \tau$, with $\tau$ depending on $d$, $\theta$ and the Lipschitz character of $\Omega$.

  Now leave $\tau$ fixed and consider the case $|\lambda| \geq \tau$.
  This case will be handled using the Rellich estimates from Section 4.
  We use the facts that for $\nabla_{\mathrm{tan}} u$ and $u$ the inner and outer nontangential limits coincide and apply Theorems \ref{thm:rellich} and \ref{thm:rellichExterior} to conclude that
  \begin{align*}
    &\|\nabla u_+ \|_\partial + \| \phi_+ - \dashint_{\partial\Omega} \phi_+ \| \\
    &\quad\leq C \{ \|( \nabla_{\mathrm{tan}} u)_+ \|_\partial + |\lambda|^{1/2} \|u_+\|_\partial + |\lambda| \| u_+\cdot n\|_{\HH^{-1}(\partial\Omega)} \} \\
    &\quad = C \{ \|( \nabla_{\mathrm{tan}} u)_- \|_\partial + |\lambda|^{1/2} \|u_-\|_\partial + |\lambda| \| u_-\cdot n\|_{\HH^{-1}(\partial\Omega)} \} \\
    &\quad\leq C \| \big(\frac{\partial u}{\partial \nu} \big)_- \|_\partial.
  \end{align*}
  We can now use this inequality to estimate $\| \big( \frac{\partial u}{\partial \nu} \big)_+ \|_\partial$ since
  \begin{align*}
    \| \big( \frac{\partial u}{\partial \nu} \big)_+ \|_\partial
    &\leq \|\big(\frac{\partial u}{\partial n} \big)_+ \|_\partial + C \| \phi_+ \|_\partial \\
    &\leq C \| (\nabla u)_+ \|_\partial + C \|\phi_+ - \dashint_{\partial\Omega} \phi_+ \d\sigma \|_\partial + C \big| \dashint_{\partial\Omega} \phi_+ \d \sigma \big| \\
    &\leq C \| \big(\frac{\partial u}{\partial \nu} \big)_- \|_\partial + C \big| \int_{\partial\Omega} \phi_+ \d\sigma \big|
  \end{align*}
  Considering the jump relation \eqref{eq:nontangentialConormalDerivative} and the previous estimate we get that
  \begin{align}
    \|f \|_\partial 
    &\leq \|\big( \frac{\partial u}{\partial \nu} \big)_+ \|_\partial + \| \big( \frac{\partial u}{\partial \nu} \big)_- \|_\partial  \nonumber\\
    &\leq C \| \big( \frac{\partial u}{\partial \nu} \big)_- \|_\partial + C \big| \int_{\partial\Omega} \phi_+ \d\sigma \big| \nonumber\\
    \label{eq:estimatef}
    &\leq C \| (-(1/2)I + \K_\lambda) f \|_\partial + C \big| \int_{\partial\Omega} \phi_+ \d\sigma \big|.
  \end{align}
  We now are left with the term $\int_{\partial\Omega} \phi_+ \d\sigma$ that needs to be estimated.
  To this end, note that multiplying the conormal derivatives of $u$ by $n$ gives
  \begin{align*}
    \big( \frac{\partial u}{\partial \nu} \big)_+ \cdot n
    = \big( \frac{\partial u_i}{\partial x_j}\big)_+ n_i n_j - \phi_+
    = n_j \big( n_i \frac{\partial}{\partial x_j} - n_j \big( \frac{\partial}{\partial x_i} \big) u_i \big)_+ - \phi_+,
  \end{align*}
  where for the second equality we used that $\div(u) = 0$ in $\Omega$ and thus this also holds for the nontangential limit.
  This identity now implies
  \begin{align}
    \big| \int_{\partial\Omega} \phi_+ \d\sigma \big|
    &\leq \big| \int_{\partial\Omega} \big( \frac{\partial u}{\partial \nu} \big)_+ \cdot n \d\sigma \big|  + C \|\nabla_{\mathrm{tan}} u \|_{\partial} \nonumber\\
    &\leq \big| \int_{\partial\Omega} \big( \frac{\partial u}{\partial \nu} \big)_- \cdot n \d\sigma \big| + C \| \nabla_{\mathrm{tan}} u\|_\partial \nonumber\\
    \label{eq:estimatephiplus}
    &\leq C \| \big( \frac{\partial u}{\partial \nu} \big)_- \|_\partial,
  \end{align}
  where in the second step, we used the jump relation to exchange $\big(\frac{\partial u}{\partial \nu}\big)_+ \cdot n$ by $\big(\frac{\partial u}{\partial \nu} \big)_- + f \cdot n$ and then used the fact $f \in \Ell^2_n(\partial\Omega)$.
  The third step follows from Theorem \ref{thm:rellichExterior} considering that $\|\nabla_{\mathrm{tan}} u \|_\partial \leq C \|\nabla u \|_{\partial}$.
  Now extending estimate \eqref{eq:estimatef} by \eqref{eq:estimatephiplus} gives
  \begin{align*}
    \| f\|_\partial 
    \leq C \| \big( - (1/2) I + \K_\lambda) f \|_\partial + C \| \big(\frac{\partial u}{\partial \nu} \big)_- \|_\partial 
    \leq C \| \big( - (1/2) I + \K_\lambda) f\|_{\partial},
  \end{align*}
  where we used the jump relation \eqref{eq:nontangentialConormalDerivative} again.
  This proves estimate \eqref{eq:inverseEstimate2} in the case $|\lambda| \geq \tau$ and thus concludes the proof.
\end{proof}

In the following lemma we will show the uniqueness of solutions to the $\Ell^2$ Dirichlet problem to the Stokes resolvent system.

\begin{lem}
  Let $\lambda \in \Sigma_\theta$ and $(u,\phi)$ be a solution to the Stokes resolvent problem in $\Omega$.
  Furthermore suppose that the nontantential limit of $u$ exists almost everywhere on $\partial\Omega$ and that $(u)^* \in \Ell^2(\partial\Omega)$.
  Then 
  \begin{align}
    \label{eq:OmegaBoundaryEstimate}
    \int_\Omega |u|^2 \d x \leq C \int_{\partial\Omega} |u|^2 \d\sigma,
  \end{align}
  where $C$ depends only on $d$, $\theta$ and $\Omega$.
\end{lem}

\begin{proof}
  We use Verchota's approximation theorem \cite{verchotaDiss} and approximate $\Omega$ by a sequence of smooth domains with uniform Lipschitz characters from inside.
  As a consequence we may assume that $\Omega$ is smooth and $u$, $\phi$ are smoth in $\overline\Omega$..
  %The fact that $(u)^* \in \Ell^2(\partial\Omega)$ is necessary if we want to take the limit of the approximating sequence of domains, as it enables us to apply the dominated convergence theorem.
  Let $(w,\psi) \in \HH_0^1(\Omega; \C^d) \times \HH^1(\Omega)$ be a solution to the inhomogenous system
  \begin{align}
    \label{eq:inhomogenousStokes}
    \begin{cases}
      - \Delta w + \lambda w + \nabla \psi = \bar u &\text{ in } \Omega, \\
      \div(w) = 0 &\text{ in } \Omega.
    \end{cases}
  \end{align}
  It follows from testing \eqref{eq:inhomogenousStokes} against $u$ that
  \begin{align}
    \label{eq:testingInhomogenousStokes}
    \int_\Omega |u|^2 \d x
    &= \int_\Omega u \cdot \{ - \Delta w + \lambda w + \nabla \psi \} \d x.
    %&= - \int_{\partial\Omega} u \cdot \big\{ \frac{\partial w}{\partial n} - \psi n \big\} \d \sigma
  \end{align}
  Using one of Green's identities on the first summand and the fact that $u$ is the solution to the Stokes resolvent problem gives that
  \begin{align*}
    \int_\Omega -u \cdot \Delta w \d x
    &= \int_\Omega -w \cdot \Delta u \d x - \int_{\partial\Omega} u \cdot \frac{\partial w}{\partial n} \d \sigma, \\
    &= \int_\Omega w \cdot ( -\lambda u - \nabla \phi) \d x - \int_{\partial\Omega} u \cdot \frac{\partial w}{\partial n} \d \sigma \\
    &= \int_\Omega -\lambda w \cdot u \d x - \int_{\partial\Omega} u \cdot \frac{\partial w}{\partial n} \d \sigma, 
  \end{align*}
  where in the last step we used partial integration and the fact that $w$ vanishes on $\partial\Omega$ and is divergence free:
  \begin{align*}
    \int_\Omega w \cdot \nabla\phi \d x = -\int_\Omega \div(w) \phi \d x + \int_{\partial\Omega} \phi w \cdot n \d\sigma = 0.
  \end{align*}
  For the third summand in \eqref{eq:testingInhomogenousStokes} we do the same with the only difference that the second integral does not vanish.
  Putting everything together gives
  \begin{align*}
    \int_\Omega |u|^2 \d x 
    &= - \int_{\partial\Omega} u \cdot \big\{ \frac{\partial w}{\partial n} - \psi n \big\} \d\sigma \\
    &\leq \| u\|_\partial \{ \|\nabla w\|_\partial + \|\psi\|_\partial \}
  \end{align*}
  by the Cauchy-Schwartz inequality.
\end{proof}

\chapter{Derivation of Resolvent Estimates}

In this final chapter we will prove that the Stokes semigroup is analytic on $\Ell^p_\sigma(\Omega, \C^d)$ for bounded Lipschitz domains $\Omega \subseteq \R^d$, $d \geq 3$.

The first step will be to establich a weak reverse H\"older estimate for local solutions of the Stokes resolvent problem. 
We start with a similar result on Lipschitz cylinders.

\begin{lem}
  \label{lem:reverseHoelderCylinder}
  Let $\eta \colon \R^{d - 1} \to \R$ be a Lipschitz function.
  Furthermore, let $u \in \HH^1(D_\eta(r); \C^d)$ and $\phi \in \Ell^2(D_\eta(2r))$ solve the Stokes resolvent problem in $D_\eta(2r)$ with $u = 0$ on $I_\eta(2r)$ for some $0 < r < \infty$ and $\lambda \in \Sigma_\theta$.
  Let $p_d = \frac{2d}{d - 1}$.
  Then 
  \begin{align}
    \label{eq:reverseHoelderCylinder}
    \Big( \dashint_{D_\eta(r)} |u|^{p_d} \d x \Big)^{1/p_d} \leq C \Big( \dashint_{D_\eta(2r)} |u|^2 \d x \Big)^{1/2},
  \end{align}
  where $C$ only depends on $d$, $M$ and $\theta$.
\end{lem}

\begin{proof}
  Without loss of generality we rescale and assume that $r = 1$.
  Let $t \in (1,2)$. 
  We note that by \cite[Lemma 1.3.25]{tolksdorf} a Lipschitz cylinder is itself a Lipschitz domain.
  It is therefore admissible to apply Theorem \ref{eq:reverseTrace} to $u$ in $D_\eta(t)$ which yields
  \begin{align*}
    \Big( \int_{D_\eta(t)} |u|^{p_d} \d x \Big)^{2/p_d} \leq C \int_{\partial D_\eta(t)} |u|^2 \d \sigma,
  \end{align*}
  where $C$ depends only on $d$, $\theta$ and the Lipschitz character of $\Omega$. In particular $C$ does not depend on $t$.
  Since $u$ vanishes on $I(2)$ we have that
  \begin{align*}
    \Big( \int_{D_\eta(1)} |u|^p \d x \Big)^{2/p} \leq C \int_{\partial D_\eta(t) \setminus I(2)} |u|^2 \d \sigma.
  \end{align*}
  Applying the co-area formula to integrate both sides over the interval $(1,2)$ gives
  \begin{align*}
    \Big( \int_{D_\eta(1)} |u|^p \d x \Big)^{2/p} \leq C \int_{D_\eta(2)} |u|^2 \d x.
  \end{align*}
  Estimate \eqref{eq:reverseHoelderCylinder} now follows after dividing by $|D_\eta(1)|$.
\end{proof}

The next step is to extend the result to arbitrary Lipschitz domains. The following Lemma reduces the amount of work to a few special cases.

\begin{lem}[Tolksdorf]
  \label{lem:ballsforballs}
  Let $\Omega \subset \R^d$ be Lebesgue-measurable, $f, g \in \Ell^2(\Omega)$, $\alpha_2 > \alpha_1 > 1$, $p > 2$, $r > 0$ and $x_0 \in \R^d$ be such that $\BB(x_0, r) \cap \Omega \neq \emptyset$.
  If there exists $C > 0$ such that
  \begin{align*}
    &\Big( \frac{1}{s^d} \int_{\Omega \cap \BB(y, s)} |f|^p \d x \Big)^{1/p} \\
    &\qquad\leq C \Big\{ \Big( \frac{1}{s^d} \int_{\Omega \cap \alpha_1 \BB(y, s)} |f|^2 \d x \Big)^{1/2} + \sup_{B' \cap \BB(y,s)} \Big( \frac{1}{|B'|} \int_{\Omega \cap B'} |g|^2 \d x \Big)^{1/2} \Big\}
  \end{align*}
  holds for all balls $\BB(y,s)$ with $\BB(y,\alpha_2 s) \subset \BB(x_0, \alpha_2 r)$ and which are either centered on $\partial\Omega$ or satisfy $\alpha_2 \BB(y,s) \subset \Omega$, then for each $\alpha \in (1,\alpha_2)$ there exists a constant $C'$ such that
  \begin{align*}
    &\Big( \frac{1}{r^d} \int_{\Omega \cap \BB(x_0, r)} |f|^p \d x \Big)^{1/p} 
    \\
    &\qquad\leq C' \Big\{ \Big( \frac{1}{r^d} \int_{\Omega \cap \alpha \BB(x_0, r)} |f|^2 \d x \Big)^{1/2} + \sup_{B' \cap \BB(x_0,r)} \Big( \frac{1}{|B'|} \int_{\Omega \cap B'} |g|^2 \d x \Big)^{1/2} \Big\}.
  \end{align*}
\end{lem}

\begin{proof}
  A proof of this lemma was given by Tolksdorf \cite{tolksdorf2017}.
  We will give the proof for the sake of completeness.
\end{proof}

As of now, our toolbox comprises enough tools to prove that solutions to the Stokes resolvent system satisfy a weak reverse H\"older inequality.

\begin{lem}
  \label{lem:reverseHoelder}
  Let $x_0 \in \overline\Omega$ and $0 < 2r < r_0$ and set $\alpha_1 = \sqrt{d^2 10^2 (1 + M)^2 + 4}$ and $\alpha_2 = \alpha_1 + 1$.
  Let $u \in \HH^1(\BB(x_0,  \alpha_2 r) \cap \Omega; \C^d)$ and $\phi \in \Ell^2(\BB(x_0, \alpha_2 r) \cap \Omega)$ satisfy the Stokes resolvent system in $\BB(x_0, \alpha_2 r) \cap \Omega$. 
  If $\BB(x_0, \alpha_2 r) \cap \partial \Omega \neq \emptyset $, we additionally assume $u = 0$ on $\BB(x_0, \alpha_2 r) \cap \partial\Omega$.
  Then
 % Then for all balls $\BB(x_0, r)$ that are either centered on $\partial\Omega$ or satisfy $\alpha_2 \BB(x_0, r) \subseteq \Omega$ the estimate
  \begin{align}
    \label{eq:reverseHoelder}
    \Big( \dashint_{\BB(x_0, r) \cap \Omega} |u|^p \Big)^{1/p} \leq C \Big( \dashint_{\BB(x_0, 2 r) \cap \Omega} |u|^2 \Big)^{1/2}
  \end{align}
  holds, where $p = p_d$.
  Here, $C > 0$ only depends on $d$, $\theta$ and the Lipschitz character of $\Omega$.
\end{lem}

\begin{proof}
  %We first note that since estimate \eqref{eq:reverseHoelder} is a weak reverse H\"older inequality and thus possesses a self-improving property, see Giaquinta and Martinazzi \cite{giaquintaMartinazzi} or Giaquinta and Modica \cite{giaquintaModica}.
  %Consequently it suffices to prove \eqref{eq:reverseHoelder} for $p = p_d = \frac{2d}{d - 1}$.

  %We want to apply Lemma 4.2 from \cite{tolksdorRsec}.
  %Let $0 < s < r$ and $\tilde \BB{} = \BB(y,s)$ with $\alpha_2 \tilde \BB \subseteq \BB(x_0, \alpha_2 r)$.
  Due to Lemma \ref{lem:ballsforballs} it suffices to consider only two cases: (1) $x_0 \in \Omega$ with $\alpha_2 \BB(x_0,r) \subset \Omega$ and (2) $x_0 \in \partial\Omega$.

  Let $x_0 \in \Omega$ with $\alpha_2 \BB(x_0,r) \subset \Omega$. We may deploy the interior estimate \eqref{eq:interiorEstimateDoubleLayer} to derive that for all $x \in \BB(x_0,r)$
  \begin{align*}
    |u(x)|^p \leq C \Big( \dashint_{\BB(x,r)} |u(y)|^2 \d y \Big)^{p/2}
  \end{align*}
  which after integrating $x$ over $\BB(x_0, r)$ yields
  \begin{align*}
    \dashint_{\BB(x_0, r)} |u(x)|^p \d x \leq C \Big(\alpha_1^d  \dashint_{\BB(x_0, \alpha_1 r)} |u(z)|^2 \d z \Big)^{p/2},
  \end{align*}
  where we also used the fact that $\alpha_1 > 2$.

  If $x_0 \in \partial\Omega$, then by Lemma \ref{lem:reverseHoelderCylinder}
  \begin{align*}
    \Big( \frac{1}{r^d} \int_{\BB(x_0, r) \cap \Omega} |u|^p \Big)^{1/p}
    &\leq \Big( \frac{1}{r^d} \int_{D_{\eta_{x_0}}(r)} |u|^p \Big)^{1/p} \\
    &\leq C \Big( \frac{1}{r^d} \int_{D_{\eta_{x_0}}(2 r)} |u|^p \Big)^{1/p} \\
    &\leq C \Big( \frac{1}{r^d} \int_{\BB(x_0, \alpha_1 r) \cap \Omega} |u|^2 \Big)^{1/2}.
  \end{align*}
  Now the claim follows readily from an application of Lemma \ref{lem:ballsforballs} with $\alpha = 2 \in (1, \alpha_2)$.
\end{proof}

  We note that estimate \eqref{eq:reverseHoelder} is a weak reverse H\"older inequality and thus possesses a self-improving property, see Giaquinta and Martinazzi \cite[Thm. 6.38]{giaquintaMartinazzi} or Giaquinta and Modica \cite[Prop. 5.1]{giaquintaModica}.

  \begin{prop}[Giaquinta, Modica]
    \label{prop:giaquinta}
    Let $\Omega \subset \R^d$ be open, $f \in \Ell^1_{\mathrm{loc}}(\Omega)$, $q > 1$, be a non-negative function.
    If there exist constants $b > 0, R_0 > 0$ such that
    \begin{align*}
      \Big( \frac{1}{r^d} \int_{\BB(x_0, r)} f^q \d x \Big)^{1/q} \leq \frac{b}{r^d} \int_{\BB(x_0, 2r)} f \d x
    \end{align*}
    for all $x_0 \in \Omega$ and $0 < r < \min\{ R_0, \operatorname{dist}(x_0, \partial\Omega)/2\}$.
    Then $f \in \Ell^{q + \varepsilon}_{\mathrm{loc}}(\Omega)$ for some $\varepsilon > 0$, depending only on $d$, $q$, and $b$ and there is a contant $\tilde C$ depending only on $d$, $q$, $\varepsilon$ and $b$ such that
    \begin{align*}
      \Big( \frac{1}{r^d} \int_{\BB(x_0, r)} f^{q + \varepsilon} \d x \Big)^{1/(q + \varepsilon)} \leq \tilde C \Big( \frac{1}{r^d} \int_{\BB(x_0, 2r)} f^q \d x \Big)^{1/q}
    \end{align*}
    for all $x_0 \in \Omega$ and $0 < r < \min\{R_0, \operatorname{dist}(x_0, \partial\Omega)/2\}$.
  \end{prop}

  \begin{rem}
    \label{rem:reverseHoelder}
    The self-improving property of reverse Hölder estimates can now be used to make the result of Lemma \ref{lem:reverseHoelder} a little bit better. 
    Let $0 < 2r < r_0$.
    We are aiming to apply Proposition \ref{prop:giaquinta} for $x_0 \in \overline\Omega$ on the open set $\Omega \cap \BB(x_0, \alpha_2 r)$, for $\alpha_2$ as in Lemma \ref{lem:reverseHoelder}.
    Let also $u$ be as in Lemma \ref{lem:reverseHoelder} and set $R_0 = r_0/2$. 
    Then for $f = |u|^{2} \chi_{\BB(x_0,\alpha_2 r) \cap \Omega}$ which can be considered as a partial extension of $u$ by $0$ to $\R^d$ and $q = p/2$, inequality \eqref{eq:reverseHoelder} reads
    \begin{align*}
      \Big( \frac{1}{r^d} \int_{\BB(x_0, r)} f^q \d x \Big)^{1/q}
      &= \Big( \frac{1}{r^d} \int_{\BB(x_0, r) \cap \Omega} |u|^p \d x \Big)^{2/p} \\
      &\leq C^2 \frac{1}{r^d} \int_{\BB(x_0, 2r) \cap \Omega} |u|^2 \d x
      = C^2 \frac{1}{r^d} \int_{\BB(x_0, 2r)} f \d x.
    \end{align*}
    Consequently Proposition \ref{prop:giaquinta} gives us that there exists some $\varepsilon > 0$ which depends only on $d$, $q$ and $C^2$ and a constant $\tilde C$ depending only on $d$, $q$, $\varepsilon$ and $C^2$ such that
    \begin{align*}
      \Big( \frac{1}{r^d} \int_{\BB(x_0, r/2) \cap \Omega} |u|^{p + \varepsilon'}\d x \Big)^{2/(p + \varepsilon ')} 
      &= \Big( \frac{1}{r^d} \int_{\BB(x_0, r/2)} f^{q + \varepsilon} \d x \Big)^{1/(q + \varepsilon)} \\
      &\leq \tilde C \Big( \frac{1}{r^d} \int_{\BB(x_0, r) \cap \Omega} |u|^p \d x \Big)^{1/p} \\
      &\leq \tilde C C \Big( \frac{1}{r^d} \int_{\BB(x_0,2r)\cap \Omega } |u|^2 \d x \Big)^{1/2}.
    \end{align*}
    Another application of Lemma \ref{lem:ballsforballs} gives us that for all $r < r_0/4$ it holds that
    \begin{align}
      \label{eq:improvedreverseHoelder}
      \Big( \frac{1}{r^d} \int_{\BB(x_0, r) \cap \Omega} |u|^{p + \varepsilon'} \d x \Big)^{2/(p + \varepsilon')}
      \leq C \Big(\frac{1}{r^d} \int_{\BB(x_0, 2r) \cap \Omega} |u|^2 \d x.  \Big)^{1/2}.
    \end{align}
  \end{rem}

  %Consequently it suffices to prove \eqref{eq:reverseHoelder} for $p = p_d = \frac{2d}{d - 1}$.

  %We want to apply Lemma 4.2 from \cite{tolksdorRsec}.
  %Let $0 < s < r$ and $\tilde \BB{} = \BB(y,s)$ with $\alpha_2 \tilde \BB \subseteq \BB(x_0, \alpha_2 r)$.

The following extrapolation theorem by Shen will be necessary in order to derive $\Ell^p$-bounds on the solution of the Stokes resolvent system, \cite[Thm. 3.3]{shenExtra}.
Note that a more recent result from Tolksdorf \cite{tolksdorf2017} generalizes this result to operators which are defined on spaces of Banach space valued functions.

\begin{thm}
  \label{thm:extrapolation}
  Let $T$ be a bounded sublinear operator on $\Ell^2(\Omega; \C^d)$, where $\Omega$ is a bounded Lipschitz domain in $\R^n$ and $\|T\|_{\Li(\Ell^2(\Omega; \C^d))} \leq C_0$.
  Let $p > 2$.
  Suppose that there exist constants $R_0 > 0$, $N > 1$ and $\alpha_2 > \alpha_1 > 1$ such that for any bounded measurable function $f$ with $\supp(f) \subseteq \Omega \setminus \alpha_2 B$,
  \begin{align*}
    \Big\{ \frac{1}{r^d} \int_{\Omega \cap B} |Tf|^p \d x \Big\}^{1/p}
    \leq N \Big\{ \Big( \frac{1}{r^d} \int_{\Omega \cap \alpha_1 B} |T f|^2 \d x \Big)^{1/2} + \sup_{B' \supset B} \Big( \frac{1}{|B'|} \int_{B'} |f|^p \d x \Big)^{1/p} \Big\},
  \end{align*}
  where $B = B(x_0, r)$ is a ball with $0 < r < R_0$ and either $x_0 \in \partial\Omega$ or $B(x_0, \alpha_2 r) \subset \Omega$.
  Then $T$ is bounded on $\Ell^q(\Omega; \C^d)$ for any $2 < q < p$.
  Moreover $\|T\|_{\Li(\Ell^q(\Omega; \C^d))}$ is bounded by a constant depending at most on $d$, $N$, $C_0$, $p$, $q$ and the Lipschitz character of $\Omega$.
\end{thm}

We are now in the position to prove Theorem \ref{thm:main}, the main theorem of this thesis. For this, the improved weak reverse H\"older inequality derived in Remark \ref{rem:reverseHoelder} will be the crucial ingredient as it enables us to apply the extrapolation theorem \ref{thm:extrapolation} to a suitable family of operators.

%\begin{thm}[Shen]
%  Let $\Omega$ be a bounded Lipschitz domain in $\R^d$, $d \geq 3$.
%  There exists $\varepsilon > 0$, depending only on $d$, $\theta$ and the Lipschitz character of $\Omega$, such that if $f \in \Ell^2(\Omega; \C^d) \cap \Ell^p(\Omega; \C^d)$ and 
%  \begin{align*}
%    \Big| \frac{1}{p} - \frac{1}{2} \Big| < \frac{1}{2d} + \varepsilon,
%  \end{align*}
%  then the unique solution $u$ to \eqref{eq:stokesResolventSystem} in $\HH_0^1(\Omega; \C^d)$ satisfies the estimate
%  \begin{align*}
%    \|u\|_{\Ell^p(\Omega; \C^d)} \leq \frac{C_p}{|\lambda| + 1} \|f\|_{\Ell^p(\Omega; \C^d)},
%  \end{align*}
%  where $C_p$ depends only on $d$, $p$, $\theta$ and the Lipschitz character of $\Omega$.
%\end{thm}
%\begin{thm}
%  Let $\Omega$ be a bounded Lipschitz domain in $\R^d$, $d \geq 3$.
%  Then there exists $\varepsilon > 0$, depending only on $d$ and the Lischitz character of $\Omega$ such that if
%  \begin{align*}
%    \frac{2d}{d + 1} - \varepsilon < p < \frac{2d}{d - 1} + \varepsilon,
%  \end{align*}
%  then $-A_p$ generates a bounded analytic semigroup in $\Ell^p_\sigma(\Omega)$.
%\end{thm}

\begin{proof}[Proof of Theorem \ref{thm:main}]
  Consider a family of scaled solution operators to the Stokes resolvent system \eqref{eq:stokesResolventSystem}, more precisely consider the family
  \begin{align*}
    T_\lambda \colon \Ell^2(\Omega; \C^d) \to \Ell^2(\Omega; \C^d), \quad f \mapsto (|\lambda| + 1) (A_2 + \lambda)^{-1} \PP_2 f,
  \end{align*}
  where $\lambda \in \Sigma_\theta$, $\theta \in (0, \pi/2)$.
  Let us first verify that $u \coloneqq (|\lambda| + 1)^{-1} T_\lambda(f)$ does indeed solve \eqref{eq:stokesResolventSystem}.
  First note that since $\PP_2 f \in \Ell^2_\sigma(\Omega)$ we know that by the mapping properties of the Stokes resolvent we have $u \in \HH^1_{0,\sigma}(\Omega)$ and
  \begin{align*}
    A_2 u + \lambda u = \PP_2 f.
  \end{align*}
  Therefore $u$ is a weak solution to 
  \begin{align*}
    -\Delta u + \lambda u = \PP_2 f.
  \end{align*}
  By the usual arguments (c.f. Chapter 1), there exists a pressure $\pi \in \Ell^2(\Omega)$ such that 
  \begin{align*}
    -\Delta u + \nabla \pi + \lambda u = f.
  \end{align*}
  Furthermore by testing \eqref{stokesResolventSystem} with $u$ we derive the estimate
  \begin{align*}
    \| T_\lambda(f) \|_{\Ell^2(\Omega; \C^d)} = (|\lambda| + 1) \|u\|_{\Ell^2(\Omega; \C^d)}
    \leq C_0 \|f\|_{\Ell^2(\Omega; \C^d)},
  \end{align*}
  where $C_0$ only depends on $d$, $\theta$ and the Lipschitz character of $\Omega$.
  Accordingly the family $T_\lambda$ is bounded on $\Ell^2(\Omega; C^d)$ and $C_0$ is a uniform bound on the operator norms $\|T_\lambda\|_{\Li(\Ell^2(\Omega; \C^d))}$.

  We will now show that the operators $T_\lambda$ fulfill the estimate in Theorem \ref{thm:extrapolation}, in order to deduce their $\Ell^p$-boundedness.
  To this end let $x_0 \in \overline \Omega$ and $0 < 4r < r_0$ such that $3 \BB(x_0, r) \subseteq \Omega$ or $\BB(x_0, r)$ is centered on $\partial\Omega$.
  Furthermore let $f \in \Ell^\infty(\Omega; \C^d)$ with support in $\Omega \setminus 3 \BB(x_0, r)$.
  By construction $(u,\pi)$ does not only solve \eqref{eq:stokesResolventProblem} in $\Omega$, the pair also solves the dirichlet problem
  \begin{align*}
    -\Delta u + \nabla \phi + \lambda u &= 0 \\
    \div(u)&= 0
  \end{align*}
  in $\Omega \cap 3 \BB(x_0, r)$ where $u = 0$ on $\partial\Omega \cap 3 \BB(x_0, r)$.
  Therefore Remark \ref{rem:reverseHoelder} and more precisely inequality \eqref{eq:improvedreverseHoelder} give that
  \begin{align*}
    \Big( \frac{1}{r^d} \int_{\Omega \cap \BB(x_0, r)} |u|^p \d x \Big)^{1/p} \leq C \Big( \frac{1}{r^d} \int_{\Omega \cap 2 \BB(x_0, r)} |u|^2 \d x \Big)^{1/2},
  \end{align*}
  where $p = p_d + \varepsilon$.
  Multiplying this inequality on both sides with $(|\lambda| + 1)$ gives
  \begin{align}
    \label{eq:tlambdaEstimate}
    \Big( \frac{1}{r^d} \int_{\Omega \cap \BB(x_0, r)} |T_\lambda(f)|^p \d x \Big)^{1/p} \leq C \Big( \frac{1}{r^d} \int_{\Omega \cap 2 \BB(x_0, r)} |T_\lambda(f)|^2 \d x \Big)^{1/2},
  \end{align}
  where $C$ depends only on $d$, $\theta$ and the Lipschitz character of $\Omega$.
  Now Shen's extrapolation theorem \ref{thm:extrapolation} gives that $T_\lambda$ is bounded on $\Ell^q(\Omega; \C^d)$ for all $2 < q < p_d + \varepsilon$ and that the operator norms $\|T_\lambda\|_{\Li(\Ell^q(\Omega; \C^d))}$ are uniformly bounded by a constant $C_q$ depending only on $d$, $\theta$, $q$ and the Lipschitz character of $\Omega$.

  In the next step of the proof we now study the relationship between the operator $T_\lambda$ and the resolvent of the Stokes operator $A_q$ on $\Ell^q_\sigma(\Omega)$ for $q \in (2, p_d + \varepsilon)$.
  To this end let $f \in \Ell^q_\sigma(\Omega)$.
  We already know that $u = (1 + |\lambda|)^{-1} T_\lambda(f) = (A_2 + \lambda)^{-1} \PP_2(f) \in \Ell^p_\sigma(\Omega) \cap \Dom(A_2)$  by the mapping properties of $T_\lambda(f)$. As $\Ell^q_\sigma(\Omega) \subset \Ell^2_\sigma(\Omega)$ we have furthermore that
  \begin{align*}
    \lambda u + A_2 u = f \in \Ell^q_\sigma(\Omega)
  \end{align*}
  and thus $A_2 u \in \Ell^q_\sigma(\Omega)$.
  Appealing to Definition \ref{defn:stokeslp} we showed that $u \in \Dom(A_p)$ and that $A_2u = A_q u$. 
  Therefore we have that
  \begin{align*}
    \lambda u + A_q u = f \in \Ell^p_\sigma(\Omega)
  \end{align*}
  By the uniqueness of $u$, which follows from the $\Ell^2$-theory of the Stokes resolvent problem, we have that $u = (\lambda + A_p)^{-1} f$.
  Hence estimate \ref{eq:tlambdaEstimate} gives
  \begin{align*}
    \|u\|_{\Ell^q(\Omega; \C^d)} = \|(\lambda + A_q)^{-1}f \|_{\Ell^q(\Omega; \C^d)}
    \leq \frac{C_q}{ 1 + |\lambda|} \|f\|_{\Ell^q(\Omega; \C^d)}.
  \end{align*}
  And thus $A_p$ is sectorial on $\Ell^p_\sigma(\Omega)$.
  If necessary we take $\varepsilon$ to be the minimum of the parameter $\varepsilon$ used in the first part of this proof and the one from Theorem \ref{thm:exHelmholtz}.
  In this case it can be shown, that the spaces $\Ell^q_\sigma(\Omega)$ are reflexive and that $\Ell^q_\sigma(\Omega)^* = \Ell^{q'}_\sigma(\Omega)$ where $q'$ denotes the dual exponent $q' = q (q - 1)^{-1}$.
  By abstract operator theory \cite{haase} we get that $A_q$ is indeed densely defined and that $A_q^* =  A_q'$.
  Therefore
  \begin{align*}
    \| (A_q + \lambda)^{-1} \|_{\Li(\Ell^q_\sigma(\Omega))}
    = \| (A_q + \lambda)^{-1}* \|_{\Li(\Ell^{q'}_\sigma(\Omega))}
    = \| (A_{q'} + \lambda)^{-1} \|_{\Li(\Ell^{q'}_\sigma(\Omega))}.
  \end{align*}
  Consequently also the operators $A_{q'}$ are sectorial, densely defined and closed.
  This completes the proof.
\end{proof}

\chapter*{Appendix}
\markboth{APPENDIX}{}
\addcontentsline{toc}{chapter}{Appendix}
\label{chap:app}

In this chapter, we will provide the missing calculations for the proofs of Lemma~\ref{lem:HelmholtzLaplaceDifference} and Theorem~\ref{thm:differenceFundamentalSolutionStokes}.
Therefore, we will build on the notations which were already established at the beginning of Chapter~\ref{chap:2}.

We first collect expressions for the derivatives of the fundamental solutions to the scalar Helmholtz equation and the Laplace equation in $d= 2$.
For the fundamental solution to the Laplace equation $G(x; 0) = -\frac{1}{2\pi} \log(|x|)$, we have the following expressions for the partial derivatives:
\begin{align*}
  \partial_\gamma G(x; 0) &= -\frac{1}{2\pi} \, \frac{x_\gamma}{|x|^2} \\[0.5em]
  %
  \partial_\alpha \partial_\gamma G(x; 0) &= -\frac{1}{2\pi}\, \frac{\delta_{\alpha\gamma}}{|x|^2} + \frac{1}{\pi} \, \frac{x_\alpha x_\gamma}{|x|^4} \\[0.5em]
  %
  \partial_\beta \partial_\alpha \partial_\gamma G(x; 0)
  &= \frac{1}{\pi} \, \frac{\delta_{\beta \gamma} x_\alpha + \delta_{\alpha \gamma} x_\beta + \delta_{\alpha\beta} x_\gamma}{|x|^4} - \frac{4}{\pi}\, \frac{x_\alpha x_\beta x_\gamma}{|x|^6}.
\end{align*}
The fundamental solution for the scalar Helmholtz equation is given via 
\begin{align*}
  G(x; \lambda) = \frac{\ii}{4}\, H_0^{(1)}(k|x|),
\end{align*}
see Section \ref{sec:hankel}.
In the following, we will always have $z =  k|x|$.  
Then, applications of the chain rule and the product rule of differentiation give
\begin{align*}
  %%
  \partial_\gamma G(x; \lambda) &= \frac{\ii}{4}\, k\;  \frac{x_\gamma}{|x|}\, \frac{\d{}}{\d z} H_0^{(1)}(z) \\[0.5em]
  %%
  \partial_\alpha \partial_\gamma G(x; \lambda) 
  %%
  &= \frac{\ii}{4} \, k^2 \;  \frac{x_\alpha x_\gamma}{|x|^2} \, \frac{\d{}^2}{\d z^2} H_0^{(1)}(z) + \frac{\ii}{4}\,  k \,\bigg(\frac{\delta_{\alpha\gamma}}{|x|} - \frac{x_\alpha x_\gamma}{|x|^3} \bigg)\, \frac{\d{}}{\d z} H_0^{(1)}(z) \\[0.5em]
  %%
  \partial_\beta \partial_\alpha \partial_\gamma G(x; \lambda)
  %%
  &= \frac{\ii}{4} \, k^3 \; \frac{x_\alpha x_\beta x_\gamma}{|x|^3} \, \frac{\d{}^3}{\d z^3} H_0^{(1)}(z) \\[0.5em]
  %
  &\quad + \frac{\ii}{4} \, k^2 \,\bigg(\frac{\delta_{\beta\gamma} x_\alpha + \delta_{\alpha \gamma} x_\beta + \delta_{\alpha \beta} x_\gamma}{|x|^2} - 3\, \frac{x_\alpha x_\beta x_\gamma}{|x|^4} \bigg)\, \frac{\d{}^2}{\d z^2} H_0^{(1)}(z) \\[0.5em]
  %
  &\quad + \frac{\ii}{4} \, k \, \bigg( 3\, \frac{x_\alpha x_\beta x_\gamma}{|x|^5} - \frac{\delta_{\beta\gamma} x_\alpha + \delta_{\alpha\gamma} x_\beta + \delta_{\alpha\beta} x_\gamma}{|x|^3} \bigg)\, \frac{\d{}}{\d z} H_0^{(1)}(z).
\end{align*}

\newpage
The series expansions for the Hankel function $H_0^{(1)}(z)$ read according to Lebedev \cite[Sec.\@~5.6]{lebedev} with the Digamma function $\psi$ and the Bessel functions of the first and second kind $J_0$ and $Y_0$, respectively:
\begin{align*}
  %
  \frac{\pi}{2 \ii} \, H_0^{(1)}(z)
  %
  &= \frac{\pi}{2\ii} \, J_0(z) + \frac{\pi}{2}\, Y_0(z) \\
  &= \mathlarger{\sum}_{l = 0}^\infty \, \frac{(-1)^l}{(l!)^2 \, 4^l} \,  z^{2l} \Big( -\frac{\ii \pi}{2} - \log(2) - \psi(l + 1) \Big) 
  +  \mathlarger{\sum}_{l = 0}^\infty \, \frac{(-1)^l}{(l!)^2 \, 4^l} \,  z^{2l} \log(z) \\
  %&= \frac{2 \ii}{\pi} \, \mathlarger{\sum}_{l = 0}^\infty \, \frac{(-1)^l}{(l!)^2 \, 4^l} \,  z^{2l} \bigg( -\frac{\ii \pi}{2} - \log(2) - \psi(l + 1) \bigg) + \frac{2 \ii}{\pi} \, \mathlarger{\sum}_{l = 0}^\infty \, \frac{(-1)^l}{(l!)^2 \, 4^l} \,  z^{2l} \log(z) \\
  &= \mathlarger{\sum}_{l = 0}^\infty \, a_l \,  z^{2l} \, C_l 
   + \mathlarger{\sum}_{l = 0}^\infty \, a_l  \,z^{2l} \log(z) \\[0.5em]
  %
  \frac{\pi}{2\ii} \, \frac{\d{}}{\d z} H_0^{(1)}(z)
  %
  &= \mathlarger{\sum}_{l = 1}^\infty \, a_l \,  (2l) \, z^{2l - 1} \, C_l 
  + \mathlarger{\sum}_{l = 1}^\infty \, a_l \, (2l)  \, z^{2l - 1} \log(z) \, 
  + \mathlarger{\sum}_{l = 0}^\infty \, a_l \, z^{2l - 1} \\
  &= \mathlarger{\sum}_{l = 1}^\infty \, b_l \, z^{2l - 1} \, C_l 
  + \mathlarger{\sum}_{l = 1}^\infty \, b_l \, z^{2l - 1} \log(z) \, 
  +  \mathlarger{\sum}_{l = 0}^\infty \, a_l \, z^{2l - 1} \\[0.5em]
  %
  \frac{\pi}{2\ii} \, \frac{\d{}^2}{\d z^2} H_0^{(1)}(z)
  %
  &= \mathlarger{\sum}_{l = 1}^\infty \, b_l \, (2l - 1) \, z^{2l - 2} \, C_l 
  + \mathlarger{\sum}_{l = 1}^\infty \, b_l \, (2l - 1)\, z^{2l - 2} \log(z) 
  + \mathlarger{\sum}_{l = 1}^\infty \, b_l \, z^{2l - 2}\\
  &\quad + \mathlarger{\sum}_{l = 0}^\infty \, a_l \, (2l - 1) \, z^{2l - 2} \\
  &=   \mathlarger{\sum}_{l = 1}^\infty c_l \, z^{2l - 2} \, C_l 
  +  \!\mathlarger{\sum}_{l = 1}^\infty c_l \, z^{2l - 2} \log(z) 
  +  \!\mathlarger{\sum}_{l = 1}^\infty b_l \, z^{2l - 2} 
  +  \!\mathlarger{\sum}_{l = 0}^\infty a_l \, (2l - 1) \, z^{2l - 2} \\[0.5em]
  %
  \frac{\pi}{2\ii} \, \frac{\d{}^3}{\d z^3} H_0^{(1)}(z)
  %
  &= \sum_{l = 2}^\infty c_l \, (2l - 2) \, z^{2l - 3} \, C_l 
  +  \mathlarger{\sum}_{l = 2}^\infty \, c_l \, (2l - 2) \, z^{2l - 3} \log(z) 
  +  \mathlarger{\sum}_{l = 1}^\infty \, c_l \, z^{2l - 3} \\
 &\quad  
  + \mathlarger{\sum}_{l = 2}^\infty \, b_l \, (2l - 2) \, z^{2l - 3} 
  + \mathlarger{\sum}_{l = 0}^\infty \, a_l \, (2l - 1) \, (2l - 2) \, z^{2l - 3} \\
  %%
  &= \mathlarger{\sum}_{l = 2}^\infty \, d_l \, z^{2l - 3} \, C_l 
  + \mathlarger{\sum}_{l = 2}^\infty \, d_l \, z^{2l - 3} \log(z) 
  + \mathlarger{\sum}_{l = 1}^\infty \, c_l \, z^{2l - 3} \\
 &\quad  + \mathlarger{\sum}_{l = 2}^\infty \, b_l \, (2l - 2) \, z^{2l - 3} 
  + \mathlarger{\sum}_{l = 0}^\infty \, a_l \, (2l - 1)\, (2l - 2) \, z^{2l - 3} ,
\end{align*}
where, in order to increase readability, we introduced the following coefficients:
\begin{align*}
  C_l &\coloneqq -\frac{\ii \pi}{2} - \log(2) - \psi(l + 1), \\
  a_l &\coloneqq \frac{(-1)^l}{(l!)^2 \, 4^l}, \quad
  b_l \coloneqq a_l \cdot  2l, \quad
  c_l \coloneqq b_l \cdot (2l - 1), \quad
  d_l \coloneqq c_l \cdot (2l - 2).
\end{align*}

\section*{A.1\quad Proof of Lemma \ref{lem:HelmholtzLaplaceDifference} for $d = 2$}
\markboth{APPENDIX}{A.1.\quad PROOF OF LEMMA \ref*{lem:HelmholtzLaplaceDifference} FOR $d = 2$}
\addcontentsline{toc}{section}{A.1\quad Proof of Lemma \ref*{lem:HelmholtzLaplaceDifference} for $d = 2$}
\label{sec:A1}

For $\lambda \in \Sigma_\theta$, $\theta \in (0, \pi/2)$, we need to show that for $|\lambda| |x|^2 \leq (1/2)$ the estimate
\begin{align*}
  \Big|\, \partial_\beta \partial_\alpha \partial_\gamma \big\{ G(x; \lambda) - G(x; 0) \big\} \, \Big|
  \leq C\, |\lambda| |x|^{-1}
\end{align*}
holds with a constant $C > 0$ that only depends on $\theta$.
The strategy will be to first estimate all resulting terms individually and to extract those which cannot be estimated in this way.
We will call the terms that cannot be estimated individually \emph{problematic}.
In the second step, we will show that all problematic terms, when added together, cancel which shows that the claimed estimate holds.

In order to make the next calculations better to digest, we decompose the third derivative of $G(\,\cdot\, ; \lambda)$ as follows:
\begin{align*}
  \partial_\beta \partial_\alpha \partial_\gamma \big\{G(x; \lambda)\big\}
  = A_1 + A_2 + A_3,
\end{align*}
where each $A_i$, $i = 1,\dots,3$, corresponds to the term involving the $i$th derivative of $H_0^{(1)}$.
Let us start with $A_1$. 
We have
\begin{align*}
  A_1 = - \frac{1}{2 \pi } k^3 \, \frac{x_\alpha x_\beta x_\gamma}{|x|^3}  \,
  %%
  &\cdot\,\bigg\{
     \; \mathlarger{\sum}_{l = 2}^\infty \; d_l \, (k|x|)^{2l - 3} \, C_l +  \mathlarger{\sum}_{l = 2}^\infty \; d_l \, (k|x|)^{2l - 3} \log(k|x|)  \\
  &\quad\,  +  \mathlarger{\sum}_{l = 1}^\infty \; c_l \, (k|x|)^{2l - 3} +  \mathlarger{\sum}_{l = 2}^\infty \; b_l\, (2l - 2) \, (k|x|)^{2l - 3} \\
  &\quad\, +  \mathlarger{\sum}_{l = 0}^\infty \; a_l \, (2l - 1)\, (2l - 2) \, (k|x|)^{2l - 3} \;
  \bigg\}.
  %%
\end{align*}
We see that, using the fact $|\lambda| |x|^2 \leq (1/2)$ to trade one $k$ in the prefactor for a constant times $|x|^{-1}$, the only problematic term in $A_1$ is the first element of the last sum
\begin{align}
  \label{eq:P1}
  \tag{P1}- \frac{1}{2 \pi } k^3\, \frac{x_\alpha x_\beta x_\gamma}{|x|^3}  \; \mathlarger{\sum}_{l = 0}^{0} \; a_l \, (2l - 1)\, (2l - 2) \, (k|x|)^{2l - 3} .
\end{align}
For $A_2$, we calculate
\begin{align*}
  A_2 &= - \frac{1}{2\pi} k^2\, \bigg(\, \frac{\delta_{\beta\gamma} x_\alpha + \delta_{\alpha \gamma} x_\beta + \delta_{\alpha \beta} x_\gamma}{|x|^2} - 3\, \frac{x_\alpha x_\beta x_\gamma}{|x|^4} \, \bigg) \\ 
  &\qquad \cdot 
  \bigg\{
    \; \mathlarger{\sum}_{l = 1}^\infty \; c_l \, (k|x|)^{2l - 2} \, C_l 
  + \; \mathlarger{\sum}_{l = 1}^\infty \; c_l \, (k|x|)^{2l - 2} \log(k|x|) \\
  &\qquad\quad\,+ \; \mathlarger{\sum}_{l = 1}^\infty \; b_l \, (k|x|)^{2l - 2} 
  + \; \mathlarger{\sum}_{l = 0}^\infty \; a_l\, (2l - 1) \, (k|x|)^{2l - 2} 
  \bigg\}.
\end{align*}
As the prefactor already behaves like $|\lambda| |x|^{-1}$, the problematic terms are given by
\begin{align}
  \label{eq:P2}
  \tag{P2}
  \begin{alignedat}{1}
  & - \frac{1}{2\pi} k^2\, \bigg(\, \frac{\delta_{\beta\gamma} x_\alpha + \delta_{\alpha \gamma} x_\beta + \delta_{\alpha \beta} x_\gamma}{|x|^2} - 3\, \frac{x_\alpha x_\beta x_\gamma}{|x|^4} \, \bigg) 
   \, \cdot \mathlarger{\sum}_{l = 1}^1 \; c_l \, (k|x|)^{2l - 2} \log(k|x|) \\
   &- \frac{1}{2\pi} k^2 \,\bigg(\, \frac{\delta_{\beta\gamma} x_\alpha + \delta_{\alpha \gamma} x_\beta + \delta_{\alpha \beta} x_\gamma}{|x|^2} - 3\, \frac{x_\alpha x_\beta x_\gamma}{|x|^4} \, \bigg) 
  \, \cdot \mathlarger{\sum}_{l = 0}^0 \; a_l\, (2l - 1) \, (k|x|)^{2l - 2} .
  \end{alignedat}
\end{align}
For the last component, we have the following identity:
\begin{align*}
  A_3 = 
  & - \frac{1}{2\pi} k \bigg( 3\, \frac{x_\alpha x_\beta x_\gamma}{|x|^5} - \frac{\delta_{\beta\gamma} x_\alpha + \delta_{\alpha\gamma} x_\beta + \delta_{\alpha\beta} x_\gamma}{|x|^3} \bigg)  \\
  &\,\cdot\, \bigg\{ 
  \; \mathlarger{\sum}_{l = 1}^\infty \; b_l \, (k|x|)^{2l - 1} \, C_l 
  + \mathlarger{\sum}_{l = 1}^\infty \; b_l \, (k|x|)^{2l - 1} \log(k|x|) 
  + \mathlarger{\sum}_{l = 0}^\infty \; a_l \, (k|x|)^{2l - 1} .
  \bigg\}
\end{align*}
In this case, the prefactor behaves like $\sqrt{|\lambda|} |x|^{-2}$.
Therefore, problematic terms only arise in the last two sums
\begin{align}
  \label{eq:P3}
  \tag{P3}
  \begin{alignedat}{1}
    &- \frac{1}{2\pi} k\, \bigg( \, 3\, \frac{x_\alpha x_\beta x_\gamma}{|x|^5} - \frac{\delta_{\beta\gamma} x_\alpha + \delta_{\alpha\gamma} x_\beta + \delta_{\alpha\beta} x_\gamma}{|x|^3} \, \bigg)
  \, \cdot \mathlarger{\sum}_{l = 1}^1 \; b_l \, (k|x|)^{2l - 1} \log(k|x|)  \\
    &- \frac{1}{2\pi} k\, \bigg( \, 3\, \frac{x_\alpha x_\beta x_\gamma}{|x|^5} - \frac{\delta_{\beta\gamma} x_\alpha + \delta_{\alpha\gamma} x_\beta + \delta_{\alpha\beta} x_\gamma}{|x|^3} \, \bigg)
  \, \cdot \mathlarger{\sum}_{l = 0}^0 \; a_l \, (k|x|)^{2l - 1} .
  \end{alignedat}
\end{align}
Note that we have
\begin{align*}
  a_0 = 1, \qquad c_1 = b_1 = - \frac{1}{2}.
\end{align*}
Therefore, we can already see that the logarithmic terms in the sum $\eqref{eq:P2} + \eqref{eq:P3}$ cancel.
Next, we observe that if we take the sum over the problematic terms \eqref{eq:P1}, \eqref{eq:P2} and \eqref{eq:P3} and subtract $\partial_\beta \partial_\alpha \partial_\gamma G(x; 0)$, the result is $0$ which is easily seen by grouping the terms with the same power of $|x|$:
\begin{align*}
  &\hspace{-2cm}\mathrm{(P1)} + \mathrm{(P2)} + \mathrm{(P3)} - \partial_\beta \partial_\alpha \partial_\gamma G(x; 0) \\[0.5em]
  = &- \frac{1}{\pi}\, \frac{x_\alpha x_\beta x_\gamma}{|x|^6} \\[0.5em]
  %
    %&+ \frac{1}{4\pi} k^2 \bigg(\frac{\delta_{\beta\gamma} x_\alpha + \delta_{\alpha \gamma} x_\beta + \delta_{\alpha \beta} x_\gamma}{|x|^2} - 3 \frac{x_\alpha x_\beta x_\gamma}{|x|^4} \bigg) \log(k|x|)  \\[0.5em]
    %
    &+ \frac{1}{2\pi} \, \bigg(\,\frac{\delta_{\beta\gamma} x_\alpha + \delta_{\alpha \gamma} x_\beta + \delta_{\alpha \beta} x_\gamma}{|x|^4} - 3\, \frac{x_\alpha x_\beta x_\gamma}{|x|^6} \, \bigg) \\[0.5em]
    %
    %&+ \frac{1}{4\pi} k^2 \bigg( 3\, \frac{x_\alpha x_\beta x_\gamma}{|x|^4} - \frac{\delta_{\beta\gamma} x_\alpha + \delta_{\alpha\gamma} x_\beta + \delta_{\alpha\beta} x_\gamma}{|x|^2} \bigg) \,  \log(k|x|)  \\[0.5em]
    &- \frac{1}{2\pi} \, \bigg( \, 3\, \frac{x_\alpha x_\beta x_\gamma}{|x|^6} - \frac{\delta_{\beta\gamma} x_\alpha + \delta_{\alpha\gamma} x_\beta + \delta_{\alpha\beta} x_\gamma}{|x|^4} \, \bigg)\\[0.5em]
    %
   &- \frac{1}{\pi}\, \frac{\delta_{\beta \gamma} x_\alpha + \delta_{\alpha \gamma} x_\beta + \delta_{\alpha\beta} x_\gamma}{|x|^4} 
     + \frac{4}{\pi}\, \frac{x_\alpha x_\beta x_\gamma}{|x|^6} 
     \quad=\quad 0 \,.
\end{align*}


%\newpage
\section*{A.2\quad Proof of Theorem \ref{thm:differenceFundamentalSolutionStokes} for $d = 2$}
\markboth{APPENDIX}{A.2.\quad PROOF OF THEOREM \ref*{thm:differenceFundamentalSolutionStokes} FOR $d = 2$}
\addcontentsline{toc}{section}{A.2\quad Proof of Theorem \ref*{thm:differenceFundamentalSolutionStokes} for $d = 2$}
\label{sec:A2}

In this section, we want to show that for all $\lambda \in \Sigma_\theta$, $\theta \in (0,\pi/2)$ and under the assumption that $|\lambda| |x|^2 \leq (1/2)$, we have
\begin{align*}
  \Big|\, \nabla_x \big\{ \Gamma(x; \lambda) - \Gamma(x; 0) \big\} \, \Big| \leq C\, |\lambda| |x| \, \big|\log(|\lambda| |x|^2)\,\big|\,,
\end{align*}
where $C > 0$ depends only on $\theta$.
The means and the strategy to prove this estimate are similar to the procedure in the previous section.
In addition to the derivatives that were calculated at the beginning of this appendix, we will furthermore need  the first partial derivative for the fundamental solution of the Stokes problem:
\begin{align*}
  \partial_\gamma \Gamma_{\alpha\beta} (x; 0) 
  = \frac{1}{4 \pi}\, \bigg(\, \frac{\delta_{\alpha\gamma} x_\beta + \delta_{\gamma\beta} x_\alpha  - \delta_{\alpha\beta} x_\gamma}{|x|^2}  \, \bigg) 
  - \frac{1}{2\pi}\, \frac{x_\alpha x_\beta x_\gamma}{|x|^4} .
\end{align*}
Now consider the difference
\begin{align*}
  &\partial_\gamma \Gamma_{\alpha\beta}(x; \lambda) - \partial_\gamma \Gamma_{\alpha\beta}(x; 0) \\
  &\qquad=\partial_\gamma G(x; \lambda) \,\delta_{\alpha\beta}
  + \frac{1}{k^2} \, \partial_\beta \partial_\alpha \partial_\gamma \Big\{ G(x; \lambda) - G(x; 0) \Big\} 
  - \partial_\gamma \Gamma_{\alpha\beta}(x; 0) \\
  &\qquad\eqqcolon B_1 + B_2 + B_3 \,,
\end{align*}
where we introduced the variables $B_i$, $i = 1,\dots,3$, for the sake of readability.
As in the previous section we will study the terms $B_i$ independently and extract the terms that do not indicate the desired behavior.
For $B_1$, we have
\begin{align*}
  B_1 = -\frac{1}{2\pi} k\;  \frac{\delta_{\alpha\beta} x_\gamma}{|x|}  \;
  \cdot \,\bigg\{ \;
  \mathlarger{\sum}_{l = 1}^\infty \; b_l \, (k|x|)^{2l - 1} \, C_l 
  &+ \mathlarger{\sum}_{l = 1}^\infty \; b_l \, (k|x|)^{2l - 1} \log(k|x|) \; \\
  &+  \mathlarger{\sum}_{l = 0}^\infty \; a_l \, (k|x|)^{2l - 1} \; \bigg\}.
\end{align*}
In this expression, we only detect one problematic term, namely
\begin{align}
  \label{eq:Q1}
  \tag{Q1}
  -\frac{1}{2\pi} k\;  \frac{\delta_{\alpha\beta} x_\gamma}{|x|}\, \mathlarger{\sum}_{l = 0}^0 \; a_l \, (k|x|)^{2l - 1} .
\end{align}
For the expression $B_2$, we have just 
\begin{align*}
  B_2 = k^{-2} \, \Big( \, A_1 + A_2 + A_3 - \partial_\gamma\partial_\alpha\partial_\beta \Big\{ G(x; 0)\Big\} \, \Big) \eqqcolon A_1' + A_2' + A_3'\,,
\end{align*}
with the variables $A_i$, $i = 1, \dots, 3$, from the previous section. 
We will now list the problematic terms for $A_i'$ where we will already take into account the cancellations from the previous section.
This will result on the one hand in additional terms and on the other hand in subsequent terms in the same sums compared to \eqref{eq:P1}, \eqref{eq:P2} and \eqref{eq:P3}.

For $A_1'$, we see that the following terms do not meet the desired behavior:
\begin{align}
  \label{eq:Q2}
  \tag{Q2}
  \begin{alignedat}{1}
  &- \frac{1}{2 \pi } k \, \frac{x_\alpha x_\beta x_\gamma}{|x|^3} \; \mathlarger{\sum}_{l = 1}^1 \; c_l \, (k|x|)^{2l - 3}. 
  %&- \frac{1}{2 \pi } k \frac{x_\alpha x_\beta x_\gamma}{|x|^3}  \; \mathlarger{\sum}_{l = 1}^{1} \; a_l (2l - 1) (2l - 2) \, (k|x|)^{2l - 3} 
  \end{alignedat}
\end{align}
For $A_2'$, we see that compared to $A_2$ every summand is problematic which after cancellation leads to:
\begin{align}
  \label{eq:Q3}
  \tag{Q3}
  \begin{alignedat}{1}
  &- \frac{1}{2\pi} \, \bigg(\,\frac{\delta_{\beta\gamma} x_\alpha + \delta_{\alpha \gamma} x_\beta + \delta_{\alpha \beta} x_\gamma}{|x|^2} - 3\, \frac{x_\alpha x_\beta x_\gamma}{|x|^4} \, \bigg) 
  \,\cdot \mathlarger{\sum}_{l = 1}^1 \; c_l \, (k|x|)^{2l - 2} \, C_l \\
  %
  %&- \frac{1}{2\pi} \bigg(\frac{\delta_{\beta\gamma} x_\alpha + \delta_{\alpha \gamma} x_\beta + \delta_{\alpha \beta} x_\gamma}{|x|^2} - 3 \frac{x_\alpha x_\beta x_\gamma}{|x|^4} \bigg) 
  %\cdot \mathlarger{\sum}_{l = 1}^1 \; c_l \, (k|x|)^{2l - 2} \log(k|x|) \\
  %
  &- \frac{1}{2\pi} \, \bigg(\, \frac{\delta_{\beta\gamma} x_\alpha + \delta_{\alpha \gamma} x_\beta + \delta_{\alpha \beta} x_\gamma}{|x|^2} - 3\, \frac{x_\alpha x_\beta x_\gamma}{|x|^4} \, \bigg) 
  \,\cdot \mathlarger{\sum}_{l = 1}^1 \; b_l \, (k|x|)^{2l - 2}  \\
  %
  &- \frac{1}{2\pi} \, \bigg(\,\frac{\delta_{\beta\gamma} x_\alpha + \delta_{\alpha \gamma} x_\beta + \delta_{\alpha \beta} x_\gamma}{|x|^2} - 3\, \frac{x_\alpha x_\beta x_\gamma}{|x|^4} \, \bigg) 
  \,\cdot \mathlarger{\sum}_{l = 1}^1 \; a_l (2l - 1) \, (k|x|)^{2l - 2} .
  \end{alignedat}
\end{align}
The same holds for the expression $A_3'$:
\begin{align}
  \label{eq:Q4}
  \tag{Q4}
  \begin{alignedat}{1}
    &- \frac{1}{2\pi} \, \frac{1}{k} \, \bigg(\, 3\, \frac{x_\alpha x_\beta x_\gamma}{|x|^5} - \frac{\delta_{\beta\gamma} x_\alpha + \delta_{\alpha\gamma} x_\beta + \delta_{\alpha\beta} x_\gamma}{|x|^3} \, \bigg)  
   \,\cdot \mathlarger{\sum}_{l = 1}^1 \; b_l \, (k|x|)^{2l - 1} \, C_l  \\
    %
    %&- \frac{1}{2\pi} \, \frac{1}{k} \Big( 3 \frac{x_\alpha x_\beta x_\gamma}{|x|^5} - \frac{\delta_{\beta\gamma} x_\alpha + \delta_{\alpha\gamma} x_\beta + \delta_{\alpha\beta} x_\gamma}{|x|^3} \Big)  
  %\mathlarger{\sum}_{l = 1}^1 \; b_l \, (k|x|)^{2l - 1} \log(k|x|) \\
    %
    &- \frac{1}{2\pi} \, \frac{1}{k} \, \bigg( \, 3\, \frac{x_\alpha x_\beta x_\gamma}{|x|^5} - \frac{\delta_{\beta\gamma} x_\alpha + \delta_{\alpha\gamma} x_\beta + \delta_{\alpha\beta} x_\gamma}{|x|^3} \, \bigg)  
  \,\cdot \mathlarger{\sum}_{l = 1}^1 \; a_l \, (k|x|)^{2l - 1} .
  \end{alignedat}
\end{align}
Now it is time to sum all problematic terms, expand the variables and subtract the term $B_3$. 
As before, all the terms will cancel which will then prove our initial claim. 
With
\begin{align*}
  a_0 = 1, \quad a_1 = -\frac{1}{4}, \quad c_1 = b_1 =  -\frac{1}{2},
\end{align*}
we see that already within the sum \eqref{eq:Q3} + \eqref{eq:Q4} a lot of terms cancel which leaves us with:
\begin{align*}
  \label{eq:Q33}
  \tag{Q3'}
  &- \frac{1}{2\pi} \, \bigg(\, \frac{\delta_{\beta\gamma} x_\alpha + \delta_{\alpha \gamma} x_\beta + \delta_{\alpha \beta} x_\gamma}{|x|^2} - 3\, \frac{x_\alpha x_\beta x_\gamma}{|x|^4} \, \bigg) 
  \, \cdot \mathlarger{\sum}_{l = 1}^1 \; b_l \, (k|x|)^{2l - 2}.
\end{align*}
Now let us consider the expression \eqref{eq:Q1} + \eqref{eq:Q2} + \eqref{eq:Q33} - $B_3$:
\begin{align*}
  &-\frac{1}{2\pi} \,  \frac{\delta_{\alpha\beta} x_\gamma}{|x|^2} 
  + \frac{1}{4 \pi }\, \frac{x_\alpha x_\beta x_\gamma}{|x|^4} 
  %
  + \frac{1}{4\pi} \, \bigg(\,\frac{\delta_{\beta\gamma} x_\alpha + \delta_{\alpha \gamma} x_\beta + \delta_{\alpha \beta} x_\gamma}{|x|^2} - 3\, \frac{x_\alpha x_\beta x_\gamma}{|x|^4} \, \bigg) 
   \\[0.5em]
 &\quad - \frac{1}{4 \pi} \, \bigg(\, \frac{\delta_{\alpha\gamma} x_\beta + \delta_{\gamma\beta} x_\alpha  - \delta_{\alpha\beta} x_\gamma}{|x|^2} \, \bigg) 
  + \frac{1}{2\pi}\,  \frac{x_\alpha x_\beta x_\gamma}{|x|^4}
  \quad = \quad 0\, .
\end{align*}


% Literaturverzeichnis (beginnt auf einer ungeraden Seite)
\cleardoublepage
\nocite{*}
%\bibliographystyle{geralpha}
\bibliographystyle{acm}
\bibliography{thesis}
\addcontentsline{toc}{chapter}{Bibliography}

      
% ggf. hier Tabelle mit Symbolen 
% (kann auch auf das Inhaltsverzeichnis folgen)
  
% Stichwortverzeichnis (beginnt auf einer ungeraden Seite)
%\cleardoublepage
%\printindex
%\addcontentsline{toc}{chapter}{Stichwortverzeichnis}
  
% Eidesstattliche Erklaerung.
\cleardoublepage 
% Keine Kopf-/Fu�zeile auf dieser Seite
\thispagestyle{empty}

%\vspace*{4cm}
\section*{Erkl\"arung zur Abschlussarbeit gem\"a\ss{} \S23 Abs. 7 APB der TU Darmstadt}\index{Erkl�rung}
\addcontentsline{toc}{chapter}{Erk\"arung zur Abschlussarbeit}

Hiermit versichere ich, Fabian Gabel, die vorliegende Master-Thesis ohne Hilfe Dritter und nur mit den angegebenen Quellen und Hilfsmitteln angefertigt zu haben. Alle Stellen, die Quellen entnommen wurden, sind als solche kenntlich gemacht worden. Diese Arbeit hat in gleicher oder \"ahnlicher Form noch keiner Pr\"ufungsbeh\"orde vorgelegen. 

Mir ist bekannt, dass im Falle eines Plagiats (\S38 Abs.2 APB) ein T\"auschungsversuch vorliegt, der dazu f\"uhrt, dass die Arbeit mit 5,0 bewertet und damit ein Pr\"ufungsversuch verbraucht wird. Abschlussarbeiten d\"urfen nur einmal wiederholt werden.

Bei der abgegebenen Thesis stimmen die schriftliche und die zur Archivierung eingereichte elektronische Fassung �berein.

%\section*{Thesis Statement pursuant to \S23 paragraph 7 of APB TU Darmstadt}
\section*{English translation for information purposes only:\\
Thesis Statement pursuant to � 22 paragraph 7 and � 23 paragraph 7 of APB TU Darmstadt}

I herewith formally declare that I, Muster Mustermann, have written the submitted thesis independently pursuant to � 22 paragraph 7 of APB TU Darmstadt. I did not use any outside support except for the quoted literature and other sources mentioned in the paper. I clearly marked and separately listed all of the literature and all of the other sources which I employed when producing this academic work, either literally or in content. This thesis has not been handed in or published before in the same or similar form.

I am aware, that in case of an attempt at deception based on plagiarism (�38 Abs. 2 APB), the thesis would be graded with 5,0 and counted as one failed examination attempt. The thesis may only be repeated once.

In the submitted thesis the written copies and the electronic version for archiving are pursuant to � 23 paragraph 7 of APB identical in content.

%I herewith formally declare that I, Fabian Gabel, have written the submitted thesis independently. I did not use any outside support except for the quoted literature and other sources mentioned in the paper. I clearly marked and separately listed all of the literature and all of the other sources which I employed when producing this academic work, either literally or in content. This thesis has not been handed in or published before in the same or similar form.

%I am aware, that in case of an attempt at deception based on plagiarism (\S38 Abs. 2 APB), the thesis would be graded with 5,0 and counted as one failed examination attempt. The thesis may only be repeated once.

%In the submitted thesis the written copies and the electronic version for archiving are identical in content.
\vspace{20pt}


\noindent
Ort/Place, Datum/Date\vspace{30pt}

% Unterschrift (handgeschrieben)

\noindent
Unterschrift des Autors/Signature of author

\end{document}
