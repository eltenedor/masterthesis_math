\chapter{Fundamentals}

\section{Lipschitz-Domains}

In this section we establish the fundamental notions regarding Lipschitz domains.

\begin{defn}
  Let $\Omega \subset \R^d$ be a bounded open connected set.
  We call $\Omega$ a \emph{bounded Lipschitz domain} if there exist $r_0, M > 0$ such that for all $x \in \partial\Omega$ there exists a function $\eta_x \colon \R^{d - 1} \to \R$ which is Lipschitz continuous and fulfills $\eta_x(0) = 0$ and $\|\nabla \eta_x \|_{\Ell^\infty(\R^{d - 1})} \leq M$, and a rotation $R_x \colon \R^d \to \R^d$ such that for all $0 < r \leq r_0$ 
  \begin{align*}
    R_x[ \Omega - \{x\} ] \cap D(r) &= D_{\eta_x}(r) \\
    R_x[\partial\Omega - \{x\}] \cap D(r) &= I_{\eta_x}(r),
  \end{align*}
  where
  \begin{align*}
    D(r) &\coloneqq \{ (x', x_d) \colon |x'| < r, |x_d| < 10 d (M + 1) r\} \\
    D_{\eta_x}(r) &\coloneqq \{ (x', x_d) \colon |x'| < r, \eta_x(x') < x_d < 10 d(M+ 1)r\} \\
    I_{\eta_x}(r) &\coloneqq \{ (x', x_d) \colon |x'| < r, \eta_x(x') = x_d \} .
  \end{align*}
\end{defn}

\subsection{Basic Definitions and Properties}

\subsection{Tangential and Nontangential Operators}


\section{The Stokes Operator}

In this section we will introduce the Stokes operator on $\Ell^2(\Omega)$ and $\Ell^p(\Omega)$ for general $p$ and establish a relation to the \emph{Dirichlet problem for the Stokes resolvent system}
\begin{align}
  -\Delta u + \nabla \phi + \lambda u &= f  \quad\text{in } \Omega, \nonumber\\
  \div u &= 0 \quad\text{in } \Omega, \label{eq:stokesResolventSystem} \\
  u &= 0 \quad\text{on } \partial\Omega, \nonumber
\end{align}
where $\lambda \in \Sigma_\theta \coloneqq \{ z \in \C \colon \lambda \neq 0 \text{ and } |\arg(z) | < \pi - \theta \}$ and $\theta \in (0, \pi/2)$.

We beginn by defining the relevant function spaces.
Let $\Omega \subseteq \R^d$, $d \geq 2$ be a bounded Lipschitz domain and $1 < p < \infty$. 
We define
\begin{align*}
  \CC_{c, \sigma}^\infty(\Omega) \coloneqq \{ \varphi \in \CC_c^\infty(\Omega; \C^d) \colon \div(\varphi) = 0 \}
\end{align*}
can serve as a suitable space of test functions.
We can now close this space in $\Ell^p(\Omega)$ and the Sobolev Space $\WW_{1,p}(\Omega)$ which gives
\begin{align*}
  \Ell_\sigma^p \coloneqq \overline{\CC_{c,\sigma}^\infty(\Omega)}_{\Ell^p}
\end{align*}
and
\begin{align*}
  \WW_{0,\sigma}^{1,p}(\Omega) \coloneqq \overline{\CC_{c,\sigma}^\infty(\Omega)}_{\WW^{1,p}}.
\end{align*}
If $p = 0$, we will use the symbol $\HH_{0,\sigma}^1(\Omega)$ to denote $\WW_{0,\sigma}^{1,2}(\Omega)$ to emphasize that this space is a Hilbert space.

In order to define the Stokes operator we introduce the following sesquilinear form
\begin{align*}
  a \colon \HH_{0,\sigma}^1(\Omega) \times \HH_{0,\sigma}^1(\Omega) \to \C, \quad (u,v) \mapsto \int_\Omega \nabla u \cdot \overline{\nabla v} \d x.
\end{align*}
Note that for $u \in \HH_{0,\sigma}^1(\Omega)$ the gradient $\nabla u$ is a matrix and an element of the space $\Ell^2(\Omega; \C^{d \times d})$.

\begin{defn}
  \label{defn:stokes}
  The \emph{Stokes operator} $A_2$ on $\Ell_\sigma^2(\Omega)$ is given via
  \begin{align*}
    \Dom(A_2) &\coloneqq \Big\{ u \in \HH_{0,\sigma}^1(\Omega) \colon \exists ! f \in \Ell^2_\sigma(\Omega) \text{ s.t. } \forall v \in \HH_{0,\sigma}^1(\Omega) \colon a(u,v) = \int_\Omega f \cdot \overline v \d x \Big\} \\
    A_2 u &\coloneqq f,
  \end{align*}
  where $u \in \Dom(A_2)$ and $f$ comes from the definition of the domain.
\end{defn}

The following theorem from Mitrea and Monniaux \cite[Thm 4.7]{mitreaMonniaux} shows that our definition of the Stokes operator and the one used in Shen's paper coincide. 
Another advantage of this characterization is the immediate link of the Stokes operator to the Stokes system.

\begin{thm}
  If $\Omega \subseteq \R^d$, $d \geq 2$ is a bounded Lipschitz domain and $A_2$ is the Stokes operator on $\Ell^2_\sigma(\Omega)$ then
  \begin{align*}
    \Dom(A_2) = \big\{ u \in H^1_{0,\sigma}(\Omega) \colon \exists \pi \in \Ell^2(\Omega) \text{ s.t. } -\Delta u + \nabla \pi \in \Ell^2_\sigma(\Omega) \big\},
  \end{align*}
  where the expression $\Delta u + \nabla \pi \in\Ell^2_\sigma(\Omega)$ needs to be understood in the distributional sense.
  For $u \in \Dom(A_2)$ and the corresponding pressure $\pi$ we have
  \begin{align*}
    A_2 u = -\Delta u + \nabla \pi.
  \end{align*}
\end{thm}

The following proposition summarizes some facts about the Stokes operator on $\Ell^2_\sigma(\Omega)$.

\begin{prop}
  Let $\Omega \subset \R^d$ be a bounded Lipschitz domain and $A_2$ the Stokes operator as in Definition \ref{defn:stokes}. Then we have
  \begin{enumerate}[a)]
    \item $A_2$ is closed with dense domain. Furthermore $0 \in \rho(A_2)$.
    \item $\sigma(A) \subset [0,\infty)$ and for all $\theta \in (0,\pi]$ there exists $C > 0$ such that
      \begin{align}
        \label{eq:resolventEstimateL2}
        \| \lambda(\lambda + A)^{-1} \|_{\Li(\Ell^2_\sigma)} \leq C, \quad\text{for all } \lambda \in \C \setminus \overline \Sigma_\theta.
      \end{align}
      In particular $-A_2$ generates a bounded analytic semigroup on $\Ell^2_\sigma(\Omega)$.
  \end{enumerate}
\end{prop}

With these results at hand we can now give a quick recap of the solution theory to \eqref{eq:stokesResolventSystem}.
Let $f \in \Ell^2_\sigma$ and $\lambda \in \Sigma_\theta$, $\theta \in (0, \pi/2)$.
By the previous Theorems we know that there exists a unique $u \in \Dom(A_2) \subseteq \HH^1_{0,\sigma}(\Omega)$ and $\pi \in \Ell^2(\Omega)$ such that
\begin{align*}
  -\Delta u + \nabla \pi + \lambda u = A_2 u = f.
\end{align*}
For general $f = \Ell^2(\Omega)$ we use the \emph{Helmholtz projection} $\PP_2$ to get
\begin{align*}
  \Delta u + \nabla \pi + \lambda u + (I - \PP_2) f = \PP_2 f + (I - \PP_2) f = f,
\end{align*}
where $u$ and $\pi$ now correspond to $\PP_2 f \in \Ell^2_\sigma(\Omega)$. On bounded Lipschitz domains the orthogonal complement to $\PP_2$ is characterized via
\begin{align*}
  \Ell^2_\sigma(\Omega)^\perp = \big\{ f \in \Ell^2(\Omega; \C^d) \colon f = \nabla \phi, \text{ for some } \phi \in \Ell^2(\Omega) \big\}.
\end{align*}
A proof of this fact can be found in the book of Sohr \cite[Lemma 2.5.3]{sohr}.
Using this result we find $g \in \Ell^2(\Omega)$ such that $\nabla g = (I - \PP_2)f$ in the sense of distributions and we see that
\begin{align*}
  -\Delta u + \nabla( \pi + g ) + \lambda u = f.
\end{align*}
Furthermore we may deduce from the resolvent estimate \eqref{eq:resolventEstimateL2} that the solution $u$ which apparently is not affected by the additional part $(I-\PP_2)f$ fulfills the inequality
\begin{align*}
  |\lambda|^{-1} \| u\|_{\Ell^2(\Omega; \C^d)} = |\lambda|^{-1} \|(A_2 + \lambda)^{-1} \PP_2 f \|_{\Ell^2(\Omega; \C^d)} \leq C \| f \|_{\Ell^2(\Omega; \C^d)}.
\end{align*}
