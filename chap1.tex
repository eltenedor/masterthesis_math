\chapter{Fundamentals}

\section{Lipschitz-Domains}

In this section we establish the fundamental notions regarding Lipschitz domains.

\begin{defn}
  Let $\Omega \subset \R^d$ be a bounded open connected set.
  We call $\Omega$ a \emph{bounded Lipschitz domain} if there exist $r_0, M > 0$ such that for all $x \in \partial\Omega$ there exists a function $\eta_x \colon \R^{d - 1} \to \R$ which is Lipschitz continuous and fulfills $\eta_x(0) = 0$ and $\|\nabla \eta_x \|_{\Ell^\infty(\R^{d - 1})} \leq M$, and a Rotation $R_x \colon \R^d \to \R^d$ such that for all $0 < r \leq r_0$ 
  \begin{align*}
    R_x[ \Omega - \{x\} ] \cap D(r) &= D_{\eta_x}(r) \\
    R_x[\partial\Omega - \{x\}] \cap D(r) &= I_{\eta_x}(r),
  \end{align*}
  where
  \begin{align*}
    D(r) &\coloneqq \{ (x', x_d) \colon |x'| < r, |x_d| < 10 d (M + 1) r\} \\
    D_{\eta_x}(r) &\coloneqq \{ (x', x_d) \colon |x'| < r, \eta_x(x') < x_d < 10 d(M+ 1)r\} \\
    I_{\eta_x}(r) &\coloneqq \{ (x', x_d) \colon |x'| < r, \eta_x(x') = x_d \} .
  \end{align*}
\end{defn}

\subsection{Basic Definitions and Properties}

\subsection{Tangential and Nontangential Operators}


\section{The Stokes Operator}

In this section we will introduce the Stokes operator on $\Ell^2$ and $\Ell^p$ for general $p$ and establish a relation to the Stokes equation. 
We beginn by defining the relevant function spaces.

Let $\Omega \subseteq \R^d$, $d \geq 2$ and $1 < p < \infty$. 
We define
\begin{align*}
  \CC_{c, \sigma}^\infty(\Omega) \coloneqq \{ \varphi \in \CC_c^\infty(\Omega; \C^d) \colon \div(\varphi) = 0 \}
\end{align*}
which will serve as a suitable space of test functions.
We can now close this space in $\Ell^p$ and the Sobolev Space $\WW_{1,p}$ which gives
\begin{align*}
  \Ell_\sigma^p \coloneqq \overline{\CC_{c,\sigma}^\infty(\Omega)}_{\Ell^p}
\end{align*}
and
\begin{align*}
  \WW_{0,\sigma}^{1,p}(\Omega) \coloneqq \overline{\CC_{c,\sigma}^\infty(\Omega)}_{\WW^{1,p}}.
\end{align*}
If $p = 0$ we will use the symbol $\HH_{0,\sigma}^1(\Omega)$ to denote $\WW_{0,\sigma}^{1,2}(\Omega)$ to emphasize that this space is a Hilbert space.

In order to define the Stokes operator we introduce the following sesquilinear form
\begin{align*}
  a \colon \HH_{0,\sigma}^1(\Omega) \times \HH_{0,\sigma}^1(\Omega) \to \C, \quad (u,v) \mapsto \int_\Omega \nabla u \cdot \overline{\nabla v} \d x.
\end{align*}
Note that for $u \in \HH_{0,\sigma}^1(\Omega)$ the Gradient $\nabla u$ is a Matrix and an element of $\Ell^2(\Omega; \C^{d \times d})$.

\begin{defn}
  The \emph{Stokes operator} $A_2$ on $\Ell_\sigma^2$ is given via
  \begin{align*}
    \Dom(A_2) &\coloneqq \Big\{ u \in \HH_{0,\sigma}^1(\Omega) \colon \exists ! f \in \Ell^2_\sigma(\Omega) \text{ s.t. } \forall v \in \HH_{0,\sigma}^1(\Omega) \colon a(u,v) = \int_\Omega f \cdot \overline v \d x \Big\} \\
    A_2 u &\coloneqq f, \quad  u \in \Dom(A).
  \end{align*}
\end{defn}


