\chapter{Derivation of Resolvent Estimates}

In this final chapter, we will prove that the Stokes semigroup is analytic on $\Ell^p_\sigma(\Omega)$ for bounded Lipschitz domains $\Omega \subseteq \R^d$, $d \geq 2$.

The first step will be to establish a weak reverse Hölder estimate for local solutions of the Stokes resolvent problem. 
We start with a similar result on Lipschitz cylinders.

\begin{lem}
  \label{lem:reverseHoelderCylinder}
  Let $\eta \colon \R^{d - 1} \to \R$ be a Lipschitz function.
  Furthermore, let the functions $u \in \HH^1(D_\eta(r); \C^d)$ and $\phi \in \Ell^2(D_\eta(2r))$ solve the Stokes resolvent problem in $D_\eta(2r)$ with $u = 0$ on $I_\eta(2r)$ for some $0 < r < \infty$ and $\lambda \in \Sigma_\theta$.
  Let $p_d = \frac{2d}{d - 1}$.
  Then,
  \begin{align}
    \label{eq:reverseHoelderCylinder}
    \bigg( \frac{1}{r^{d -1 }}\int_{D_\eta(r)} |u|^{p_d} \d x \bigg)^{1/p_d} \leq C \, \bigg( \frac{1}{r^{d - 1}} \int_{D_\eta(2r)} |u|^2 \d x \bigg)^{1/2},
  \end{align}
  where $C$ only depends on $d$, $M$ and $\theta$.
\end{lem}

\begin{proof}
  Without loss of generality, we rescale and assume that $r = 1$.
  Let $t \in (1,2)$. 
  We note that thanks to a thorough investigation carried out by Tolksdorf, see \cite[Lem.\@~1.3.25]{tolksdorf}, a Lipschitz cylinder is itself a Lipschitz domain.
  It is therefore admissible to apply Theorem~\ref{eq:reverseTrace} to $u$ in $D_\eta(t)$ which yields
  \begin{align*}
    \bigg( \int_{D_\eta(t)} |u|^{p_d} \d x \bigg)^{2/p_d} \leq C \int_{\partial D_\eta(t)} |u|^2 \d \sigma,
  \end{align*}
  where $C$ depends only on $d$, $\theta$ and the Lipschitz character of $\Omega$. In particular, $C$ does not depend on $t$.
  Since $u$ vanishes on $I(2)$, we have that
%  \begin{align*}
%    \bigg( \int_{D_\eta(1)} |u|^p \d x \bigg)^{2/p} \leq C \int_{\partial D_\eta(t) \setminus I(2)} |u|^2 \d \sigma.
%  \end{align*}
  \begin{align*}
    \bigg( \int_{D_\eta(1)} |u|^p \d x \bigg)^{2/p} 
    &\leq C\, \int_1^2 \Bigg\{  \int_{\partial D(t)} \! |u(y)|^2 \chi_{D_\eta(2)}(y) \d\sigma(y) +  \int_{I_\eta(2)} |u(y)|^2 \d \sigma(y) \Bigg\} \d t  \\
    &= C \hphantom{\Big\{} \int_1^2 \int_{\partial D(y)} \! |u(y)|^2 \chi_{D_\eta(2)}(y) \d\sigma(y) \d t.
  \end{align*}
  We define the function
  \begin{align*}
    f \colon \R^d \to \R, \quad x \mapsto t \quad\text{iff}\quad x \in \partial D(t).
  \end{align*}
  Since $x \in \partial D(t)$ and $y \in \partial D(s)$, $s,t > 0$, we have that
  \begin{align*}
    |f(x) - f(y)| = |t - s| \leq \operatorname{dist}(\partial D(t), \partial D(s)) \leq |x- y|
  \end{align*}
  and consequently the function $f$ is Lipschitz continuous with Lipschitz constant $1$.
  Applying the co-area formula to integrate both sides over the interval $(1,2)$ gives
  \begin{align*}
    \bigg( \int_{D_\eta(1)} |u|^p \d x \bigg)^{2/p} \leq C \int_{D_\eta(2)} |u|^2 \d x,
  \end{align*}
  see Theorem~\ref{thm:coarea}.
  Estimate \eqref{eq:reverseHoelderCylinder} now follows after dividing by $|D_\eta(1)|$.
\end{proof}

The next step is to extend the previous result to arbitrary Lipschitz domains. 
The following Lemma, which appeared in Tolksdorf \cite[Lem.\@~4.2]{tolksdorf2017}, reduces the amount of work that needs to be done to a few special cases.

\begin{lem}
  \label{lem:ballsforballs}
  Let $\Omega \subseteq \R^d$ be Lebesgue-measurable, $f, g \in \Ell^2(\Omega)$, $\alpha_2 > \alpha_1 > 1$, $p > 2$, $r > 0$ and $x_0 \in \R^d$ be such that $\BB(x_0,r) \cap \Omega \neq \emptyset$.
  If there exists $C > 0$ such that
  \begin{align*}
    &\bigg( \frac{1}{s^d} \int_{\Omega \cap \BB(y,\, s)} |f|^p \d x \bigg)^{1/p} \\
    &\qquad\leq C\, \bigg\{\, \bigg( \frac{1}{s^d} \int_{\Omega \cap \alpha_1 \BB(y,\, s)} |f|^2 \d x \bigg)^{1/2} + \sup_{B' \cap \BB(y,\,s)} \bigg( \frac{1}{|B'|} \int_{\Omega \cap B'} |g|^2 \d x \bigg)^{1/2} \, \bigg\}
  \end{align*}
  holds for all balls $\BB(y,s)$ with $\BB(y,\alpha_2 s) \subseteq \BB(x_0, \alpha_2 r)$ and which are either centered on $\partial\Omega$ or satisfy $\alpha_2 \BB(y,s) \subseteq \Omega$, then for each $\alpha \in (1,\alpha_2)$ there exists a constant $C' > 0$ such that
  \begin{align*}
    &\bigg( \frac{1}{r^d} \int_{\Omega \cap \BB(x_0,\, r)} |f|^p \d x \bigg)^{1/p} 
    \\
    &\qquad\leq C'\, \bigg\{\, \bigg( \frac{1}{r^d} \int_{\Omega \cap \alpha \BB(x_0,\, r)} |f|^2 \d x \bigg)^{1/2} + \sup_{B' \cap \BB(x_0,\,r)} \bigg( \frac{1}{|B'|} \int_{\Omega \cap B'} |g|^2 \d x \bigg)^{1/2} \,\bigg\}.
  \end{align*}
  This constant $C'$ only depends on $d$, $\alpha$, $\alpha_1$, $\alpha_2$, $p$ and $C$.
\end{lem}

%\begin{proof}
%  A proof of this lemma was given by Tolksdorf \cite{tolksdorf2017}.
%  We will give the proof for the sake of completeness.
%\end{proof}

As of now, our toolbox comprises enough tools to prove that solutions to the Stokes resolvent system satisfy a weak reverse H\"older inequality.

\begin{lem}
  \label{lem:reverseHoelder}
  Let $x_0 \in \overline\Omega$ and $0 < 2r < r_0$ and set $\alpha_1 = \sqrt{d^2 10^2 (1 + M)^2 + 4}$ and $\alpha_2 = \alpha_1 + 1$.
  Let $u \in \HH^1(\BB(x_0, \alpha_2 r) \cap \Omega; \C^d)$ and $\phi \in \Ell^2(\BB(x_0, \alpha_2 r) \cap \Omega)$ satisfy the Stokes resolvent system in $\BB(x_0, \alpha_2 r) \cap \Omega$. 
  If $\BB(x_0, \alpha_2 r) \cap \partial \Omega \neq \emptyset $, we additionally assume $u = 0$ on $\BB(x_0, \alpha_2 r) \cap \partial\Omega$.
  Then,
 % Then for all balls $\BB(x_0,\, r)$ that are either centered on $\partial\Omega$ or satisfy $\alpha_2 \BB(x_0,\, r) \subseteq \Omega$ the estimate
  \begin{align}
    \label{eq:reverseHoelder}
    \bigg( \frac{1}{r^{d}} \int_{\BB(x_0, r) \cap \Omega} |u|^p \bigg)^{1/p} \leq C \, \bigg( \frac{1}{r^{d}} \int_{\BB(x_0, 2 r) \cap \Omega} |u|^2 \bigg)^{1/2}
  \end{align}
  holds, where $p = p_d$.
  Here, $C > 0$ only depends on $d$, $\theta$ and the Lipschitz character of $\Omega$.
\end{lem}

\begin{proof}
  %We first note that since estimate \eqref{eq:reverseHoelder} is a weak reverse H\"older inequality and thus possesses a self-improving property, see Giaquinta and Martinazzi \cite{giaquintaMartinazzi} or Giaquinta and Modica \cite{giaquintaModica}.
  %Consequently it suffices to prove \eqref{eq:reverseHoelder} for $p = p_d = \frac{2d}{d - 1}$.

  %We want to apply Lemma 4.2 from \cite{tolksdorRsec}.
  %Let $0 < s < r$ and $\tilde \BB{} = \BB(y,\,s)$ with $\alpha_2 \tilde \BB \subseteq \BB(x_0,\, \alpha_2 r)$.
  Due to Lemma~\ref{lem:ballsforballs}, it suffices to consider only two cases: (1) $x_0 \in \Omega$ with $\alpha_2 \BB(x_0,r) \subseteq \Omega$ and (2) $x_0 \in \partial\Omega$.

  In order to prove (1), let $x_0 \in \Omega$ with $\alpha_2 \BB(x_0,r) \subseteq \Omega$. We may deploy the interior estimate \eqref{eq:interiorEstimateDoubleLayer} to derive that for all $x \in \BB(x_0,r)$ the estimate
  \begin{align*}
    |u(x)|^p \leq C\, \bigg( \frac{1}{r^{d}} \int_{\BB(x,\,r)} |u(y)|^2 \d y \bigg)^{p/2}
  \end{align*}
  holds which after integrating $x$ over $\BB(x_0, r)$ yields
  \begin{align*}
    \frac{1}{r^{d}}\int_{\BB(x_0,\, r)} |u(x)|^p \d x \leq C\, \bigg(  \frac{1}{r^{d-1}}\int_{\BB(x_0,\, \alpha_1 r)} |u(z)|^2 \d z \bigg)^{p/2},
  \end{align*}
  where we used the fact that $\alpha_1 > 2$. 
  Here, $C$ depends only on $d$ and $\theta$.

  For (2), note that if $x_0 \in \partial\Omega$, then by Lemma~\ref{lem:reverseHoelderCylinder} and Pythagoras' theorem we have
  \begin{align*}
    \bigg( \frac{1}{r^d} \int_{\BB(x_0,\, r) \cap \Omega} |u|^p \bigg)^{1/p}
    &\leq \bigg( \frac{1}{r^d} \int_{D_{\eta_{x_0}}(r)} |u|^p \bigg)^{1/p} \\[0.5em]
    &\leq C \, \bigg( \frac{1}{r^d} \int_{D_{\eta_{x_0}}(2 r)} |u|^p \bigg)^{1/p} \\[0.5em]
    &\leq C \, \bigg( \frac{1}{r^d} \int_{\BB(x_0,\, \alpha_1 r) \cap \Omega} |u|^2 \bigg)^{1/2}.
  \end{align*}
  Now, the claim follows readily from an application of Lemma~\ref{lem:ballsforballs} with the parameter $\alpha = 2 \in (1, \alpha_2)$.
\end{proof}

  We note that estimate \eqref{eq:reverseHoelder} is a weak reverse H\"older inequality and thus possesses a self-improving property, see Giaquinta and Martinazzi \cite[Thm.\@~6.38]{giaquintaMartinazzi} or Giaquinta and Modica \cite[Prop.\@~5.1]{giaquintaModica}.

  \begin{prop}[Giaquinta, Modica]
    \label{prop:giaquinta}
    Let $\Omega \subseteq \R^d$ be open, $f \in \Ell^1_{\mathrm{loc}}(\Omega)$, $q > 1$, be a non-negative function.
    If there exist constants $b > 0, R_0 > 0$ such that
    \begin{align*}
      \bigg( \frac{1}{r^d} \int_{\BB(x_0,\, r)} f^q \d x \bigg)^{1/q} \leq \frac{b}{r^d} \int_{\BB(x_0,\, 2r)} f \d x
    \end{align*}
    for all $x_0 \in \Omega$ and $0 < r < \min\big\{ R_0, \operatorname{dist}(x_0, \partial\Omega)/2\big\}$, then $f \in \Ell^{q + \varepsilon}_{\mathrm{loc}}(\Omega)$ for some $\varepsilon > 0$, depending only on $d$, $q$, and $b$ and there is a constant $\tilde C$ depending only on $d$, $q$, $\varepsilon$ and $b$ such that
    \begin{align*}
      \bigg( \frac{1}{r^d} \int_{\BB(x_0,\, r)} f^{q + \varepsilon} \d x \bigg)^{1/(q + \varepsilon)} \leq \tilde C \,\bigg( \frac{1}{r^d} \int_{\BB(x_0,\, 2r)} f^q \d x \bigg)^{1/q}
    \end{align*}
    for all $x_0 \in \Omega$ and $0 < r < \min\{R_0, \operatorname{dist}(x_0, \partial\Omega)/2\}$.
  \end{prop}

  \begin{rem}
    \label{rem:reverseHoelder}
    The self-improving property of reverse Hölder estimates can now be used to make the result of Lemma~\ref{lem:reverseHoelder} a little bit better. 
    Let $0 < 2r < r_0$.
    We are aiming to apply Proposition~\ref{prop:giaquinta} for $x_0 \in \overline\Omega$ on the open set $\Omega \cap \BB(x_0, \alpha_2 r)$, with $\alpha_2$ as in Lemma~\ref{lem:reverseHoelder}.
    Let also $u$ be as in Lemma~\ref{lem:reverseHoelder} and set $R_0 = r_0/2$. 
    Then, for $f = |u|^{2} \chi_{\BB(x_0,\,\alpha_2 r) \cap \Omega}$ which can be considered as a partial extension of $u$ by $0$ to $\R^d$ and $q = p/2$, inequality \eqref{eq:reverseHoelder} reads
    \begin{align*}
      \bigg( \frac{1}{r^d} \int_{\BB(x_0,\, r)} f^q \d x \bigg)^{1/q}
      &= \bigg( \frac{1}{r^d} \int_{\BB(x_0,\, r) \cap \Omega} |u|^p \d x \bigg)^{2/p} \\[0.5em]
      &\leq C^2 \frac{1}{r^d} \int_{\BB(x_0,\, 2r) \cap \Omega} |u|^2 \d x
      = C^2 \frac{1}{r^d} \int_{\BB(x_0,\, 2r)} f \d x.
    \end{align*}
    Consequently, Proposition~\ref{prop:giaquinta} gives us that there exists some $\varepsilon > 0$ which depends only on $d$, $q$ and $C^2$ and a constant $\tilde C >0$ depending only on $d$, $q$, $\varepsilon$ and $C^2$ such that
    \begin{align*}
      \bigg( \frac{1}{r^d} \int_{\BB(x_0,\, r/2) \cap \Omega} |u|^{p + \varepsilon'}\d x \bigg)^{2/(p + \varepsilon ')} 
      &= \bigg( \frac{1}{r^d} \int_{\BB(x_0,\, r/2)} f^{q + \varepsilon} \d x \bigg)^{1/(q + \varepsilon)} \\[0.5em]
      &\leq \tilde C\, \bigg( \frac{1}{r^d} \int_{\BB(x_0,\, r) \cap \Omega} |u|^p \d x \bigg)^{1/p} \\[0.5em]
      &\leq \tilde C C\, \bigg( \frac{1}{r^d} \int_{\BB(x_0,\,2r)\cap \Omega } |u|^2 \d x \bigg)^{1/2}.
    \end{align*}
    Another application of Lemma~\ref{lem:ballsforballs} gives us that for all $r < ( r_0/4 )$ it holds that
    \begin{align}
      \label{eq:improvedreverseHoelder}
      \bigg( \frac{1}{r^d} \int_{\BB(x_0,\, r) \cap \Omega} |u|^{p + \varepsilon'} \d x \bigg)^{2/(p + \varepsilon')}
      \leq C \, \bigg(\frac{1}{r^d} \int_{\BB(x_0,\, 2r) \cap \Omega} |u|^2 \d x.  \bigg)^{1/2}
    \end{align}
    and we have succeeded in improving our original estimate \eqref{eq:reverseHoelder}.
  \end{rem}

  %Consequently it suffices to prove \eqref{eq:reverseHoelder} for $p = p_d = \frac{2d}{d - 1}$.

  %We want to apply Lemma 4.2 from \cite{tolksdorRsec}.
  %Let $0 < s < r$ and $\tilde \BB{} = \BB(y,\,s)$ with $\alpha_2 \tilde \BB \subseteq \BB(x_0,\, \alpha_2 r)$.

The following extrapolation theorem by Shen \cite[Thm.\@~3.3]{shenExtra} will be necessary in order to derive $\Ell^p$ bounds on the solution of the Stokes resolvent system.
Essentially, this theorem states that if the non locality of an $\Ell^2$ bounded operator  $T$ can be quantified via a reverse Hölder estimate, then this operator extends to an $\Ell^p$ bounded operator for certain values $p$.
In this sense, this extrapolation theorem can also be considered a p-sensitive version of the famous Calder\'on-Zygmund Lemma.
Note that a more recent result from Tolksdorf \cite[Thm.\@~4.1]{tolksdorf2017} generalizes this result to operators which are defined on spaces of Banach space valued functions.

\begin{thm}
  \label{thm:extrapolation}
  Let $T$ be a bounded sublinear operator on $\Ell^2(\Omega; \C^d)$, where $\Omega$ is a bounded Lipschitz domain in $\R^n$ and $\|T\|_{\Li(\Ell^2(\Omega; \C^d))} \leq C_0$.
  Let $p > 2$.
  Suppose that there exist constants $R_0 > 0$, $N > 1$ and $\alpha_2 > \alpha_1 > 1$ such that for any bounded measurable function $f$ with $\supp(f) \subseteq \Omega \setminus \alpha_2 B$,
  \begin{align*}
    \Big\{ \frac{1}{r^d} \int_{\Omega \cap B} |Tf|^p \d x \Big\}^{1/p}
    \leq N \Big\{ \bigg( \frac{1}{r^d} \int_{\Omega \cap \alpha_1 B} |T f|^2 \d x \bigg)^{1/2} + \sup_{B' \supset B} \bigg( \frac{1}{|B'|} \int_{B'} |f|^p \d x \bigg)^{1/p} \Big\},
  \end{align*}
  where $B = B(x_0, r)$ is a ball with $0 < r < R_0$ and either $x_0 \in \partial\Omega$ or $B(x_0, \alpha_2 r) \subseteq \Omega$.
  Then, the restriction of $T$ to $\Ell^q(\Omega;\C^d)$ yields a bounded operator on $\Ell^q(\Omega; \C^d)$ for any $2 < q < p$.
  Moreover, the operator norm $\|T\|_{\Li(\Ell^q(\Omega; \C^d))}$ is bounded by a constant depending at most on $d$, $N$, $C_0$, $p$, $q$ and the Lipschitz character of $\Omega$.
\end{thm}

We are now in the position to prove Theorem~\ref{thm:main}, the main theorem of this thesis. 
For this, the improved weak reverse H\"older inequality derived in Remark~\ref{rem:reverseHoelder} will serve as the crucial ingredient, enabling us to apply the extrapolation theorem, Theorem~\ref{thm:extrapolation}, to a suitable family of operators.
As we want to prove resolvent estimates of the Stokes operator on $\Ell^p$, the family of operators under consideration will basically consist of resolvent operators.
To this end, note that from the last sentence of Theorem~\ref{thm:extrapolation} we get uniform bounds on our operator family on $\Ell^p$ provided that we start with a uniform bound $C_0$ on $\Ell^2$.
This aspect regarding the uniformity of the estimates and the $p$-sensitivity of the extrapolation theorem are the distinguished properties of this theorem compared to classic results from the Calder\'on-Zygmund theory of convolution operators, see Grafakos \cite[Sec.\@~5.3]{grafakos2014classical} or Stein \cite[Ch.\@~2]{stein}.

%\begin{thm}[Shen]
%  Let $\Omega$ be a bounded Lipschitz domain in $\R^d$, $d \geq 3$.
%  There exists $\varepsilon > 0$, depending only on $d$, $\theta$ and the Lipschitz character of $\Omega$, such that if $f \in \Ell^2(\Omega; \C^d) \cap \Ell^p(\Omega; \C^d)$ and 
%  \begin{align*}
%    \Big| \frac{1}{p} - \frac{1}{2} \Big| < \frac{1}{2d} + \varepsilon,
%  \end{align*}
%  then the unique solution $u$ to \eqref{eq:stokesResolventSystem} in $\HH_0^1(\Omega; \C^d)$ satisfies the estimate
%  \begin{align*}
%    \|u\|_{\Ell^p(\Omega; \C^d)} \leq \frac{C_p}{|\lambda| + 1} \|f\|_{\Ell^p(\Omega; \C^d)},
%  \end{align*}
%  where $C_p$ depends only on $d$, $p$, $\theta$ and the Lipschitz character of $\Omega$.
%\end{thm}
%\begin{thm}
%  Let $\Omega$ be a bounded Lipschitz domain in $\R^d$, $d \geq 3$.
%  Then there exists $\varepsilon > 0$, depending only on $d$ and the Lischitz character of $\Omega$ such that if
%  \begin{align*}
%    \frac{2d}{d + 1} - \varepsilon < p < \frac{2d}{d - 1} + \varepsilon,
%  \end{align*}
%  then $-A_p$ generates a bounded analytic semigroup in $\Ell^p_\sigma(\Omega)$.
%\end{thm}

\begin{proof}[Proof of Theorem~\ref{thm:main}]
  Consider a family of scaled solution operators to the Stokes resolvent system \eqref{eq:stokesResolventSystem}, more precisely consider the family
  \begin{align*}
    T_\lambda \colon \Ell^2(\Omega; \C^d) \to \Ell^2(\Omega; \C^d), \quad f \mapsto (|\lambda| + 1) (A_2 + \lambda)^{-1} \PP_2 f,
  \end{align*}
  where $\lambda \in \Sigma_\theta$, $\theta \in (0, \pi/2)$.
  Let us first verify that $u \coloneqq (|\lambda| + 1)^{-1} T_\lambda(f)$ does indeed solve \eqref{eq:stokesResolventSystem}.
  To this end, note that since $\PP_2 f \in \Ell^2_\sigma(\Omega)$ we know that by the mapping properties of the Stokes resolvent we have $u \in \HH^1_{0,\sigma}(\Omega)$ and
  \begin{align*}
    A_2 u + \lambda u = \PP_2 f.
  \end{align*}
  Therefore, $u$ is a weak solution to 
  \begin{align*}
    -\Delta u + \lambda u = \PP_2 f.
  \end{align*}
  By the usual arguments (c.f. Section~\ref{sec:stokesOperator}), there exists a pressure $\pi \in \Ell^2(\Omega)$ such that 
  \begin{align*}
    -\Delta u + \nabla \pi + \lambda u = f
  \end{align*}
  holds in the sense of distributions.
  Furthermore, by testing this identity with $u$ and then using the Poincar\'e inequality together with Lemma~\ref{lem:laxMilgramIneq}, we derive the resolvent-type estimate
  \begin{align*}
    \| T_\lambda(f) \|_{\Ell^2(\Omega; \C^d)} = (|\lambda| + 1)\, \|u\|_{\Ell^2(\Omega; \C^d)}
    \leq C_0 \, \|f\|_{\Ell^2(\Omega; \C^d)},
  \end{align*}
  where $C_0$ only depends on $d$, $\theta$ and the Lipschitz character of $\Omega$.
  Accordingly, the family $T_\lambda$ is bounded on $\Ell^2(\Omega; \C^d)$ and $C_0$ is a uniform bound on the operator norms $\|T_\lambda\|_{\Li(\Ell^2(\Omega; \C^d))}$.

  We will now show that the operators $T_\lambda$ fulfill the estimate in Theorem~\ref{thm:extrapolation}, in order to deduce their $\Ell^p$ boundedness.
  To this end, let $x_0 \in \overline \Omega$ and $0 < 4r < r_0$ such that $3 \BB(x_0, r) \subseteq \Omega$ or $\BB(x_0, r)$ is centered on $\partial\Omega$.
  Furthermore, let $f \in \Ell^\infty(\Omega; \C^d)$ with support in $\Omega \setminus 3 \BB(x_0, r)$.
  By construction, $(u,\pi)$ does not only solve \eqref{eq:stokesResolventProblem} in $\Omega$, the pair also solves the Dirichlet problem
  \begin{align*}
    -\Delta u + \nabla \phi + \lambda u &= 0 \\
    \div(u)&= 0
  \end{align*}
  in $\Omega \cap 3 \BB(x_0, r)$ where $u = 0$ on $\partial\Omega \cap 3 \BB(x_0, r)$.
  Therefore, Remark~\ref{rem:reverseHoelder} and more precisely inequality \eqref{eq:improvedreverseHoelder} give that
  \begin{align*}
    \bigg( \frac{1}{r^d} \int_{\Omega \cap \BB(x_0,\, r)} |u|^p \d x \bigg)^{1/p} 
    \leq C\, \bigg( \frac{1}{r^d} \int_{\Omega \cap 2 \BB(x_0,\, r)} |u|^2 \d x \bigg)^{1/2},
  \end{align*}
  where $p = p_d + \varepsilon$.
  Multiplying this inequality on both sides with $(|\lambda| + 1)$ gives
  \begin{align}
    \label{eq:tlambdaEstimate}
    \bigg( \frac{1}{r^d} \int_{\Omega \cap \BB(x_0,\, r)} |T_\lambda(f)|^p \d x \bigg)^{1/p} 
    \leq C\, \bigg( \frac{1}{r^d} \int_{\Omega \cap 2 \BB(x_0,\, r)} |T_\lambda(f)|^2 \d x \bigg)^{1/2},
  \end{align}
  where $C$ depends only on $d$, $\theta$ and the Lipschitz character of $\Omega$.
  Now, Shen's extrapolation theorem, Theorem~\ref{thm:extrapolation}, gives that $T_\lambda$ is bounded on $\Ell^q(\Omega; \C^d)$ for all $2 < q < p_d + \varepsilon$ and that the operator norms $\|T_\lambda\|_{\Li(\Ell^q(\Omega; \C^d))}$ are uniformly bounded by a constant $C_q$ depending only on $d$, $\theta$, $q$ and the Lipschitz character of $\Omega$.

  In the next step of the proof, we study the relationship between the operator $T_\lambda$ and the resolvent of the Stokes operator $A_q$ on $\Ell^q_\sigma(\Omega)$ for $q \in (2, p_d + \varepsilon)$.
  To this end, let $f \in \Ell^q_\sigma(\Omega)$.
  We already know that $u = (1 + |\lambda|)^{-1} T_\lambda(f) = (A_2 + \lambda)^{-1} \PP_2(f) \in \Ell^q_\sigma(\Omega) \cap \Dom(A_2)$  by the mapping properties of $T_\lambda(f)$.
  As $\Ell^q_\sigma(\Omega) \subseteq \Ell^2_\sigma(\Omega)$, we have furthermore that
  \begin{align*}
    \lambda u + A_2 u = f \in \Ell^q_\sigma(\Omega)
  \end{align*}
  and thus $A_2 u \in \Ell^q_\sigma(\Omega)$.
  Appealing to Definition~\ref{defn:stokeslp}, we showed that $u \in \Dom(A_q)$ and that $A_2u = A_q u$. 
  Therefore, we have that
  \begin{align*}
    \lambda u + A_q u = f \in \Ell^q_\sigma(\Omega)
  \end{align*}
  By the uniqueness of $u$, which follows from the $\Ell^2$ theory of the Stokes resolvent problem, we have that $u = (\lambda + A_q)^{-1} f$.
  Hence, estimate~\eqref{eq:tlambdaEstimate} gives
  \begin{align*}
    \|u\|_{\Ell^q(\Omega; \C^d)} = \|(\lambda + A_q)^{-1}f \|_{\Ell^q(\Omega; \C^d)}
    \leq \frac{C_q}{ 1 + |\lambda|} \|f\|_{\Ell^q(\Omega; \C^d)}
  \end{align*}
  and thus $A_q$ is sectorial on $\Ell^q_\sigma(\Omega)$.
  If necessary, we take $\varepsilon$ to be the minimum of the parameter $\varepsilon$ used in the first part of this proof and the one from Theorem~\ref{thm:stokesOperatorLp}.
  It was also shown in Chapter~\ref{chap:1} that the spaces $\Ell^q_\sigma(\Omega)$ are reflexive and that $\Ell^q_\sigma(\Omega)^* = \Ell^{q'}_\sigma(\Omega)$ where $q'$ denotes the dual exponent $q' = q (q - 1)^{-1}$, see Lemma~\ref{lem:duality}.
  By a general result about sectorial operators on reflexive Banach spaces, which can be found in Haase's book \cite[Prop.\@~2.1.1]{haase}, we get that $A_q$ is indeed densely defined and that $A_q^* =  A_{q'}$.
  Therefore,
  \begin{align*}
    \| (A_q + \lambda)^{-1} \|_{\Li(\Ell^q_\sigma(\Omega))}
    = \| \big(A_q + \lambda)^{-1}\big)* \|_{\Li(\Ell^{q'}_\sigma(\Omega))}
    = \| (A_{q'} + \lambda)^{-1} \|_{\Li(\Ell^{q'}_\sigma(\Omega))}.
  \end{align*}
  Consequently, also the operators $A_{q'}$ are sectorial, densely defined and closed.
  This completes the proof.
\end{proof}
